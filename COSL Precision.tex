\documentclass[10pt]{article}%
\usepackage{longtable}
\usepackage{hyperref}
\usepackage{bbding}
\usepackage{xcolor}
\usepackage{grbbridge}
\setboolean{spellten}{true}
% \setboolean{leadingspace}{true}
% \setboolean{betweencards}{true}
\usepackage{url}
\newcommand{\gap}{\vspace{\baselineskip}}
\newcommand{\hcp}{\textsc{hcp}}
\newcommand{\sq}{\textsc{sq}}
\newcommand{\ltc}{\textsc{ltc}}

\begin{document}

\title{COSL Precision Bidding System}
\author{Sudhir}
\date{v1.0, November 2020}
\maketitle

\tableofcontents

\section{Opening Bids}
All strong hands are opened \cl{1} which is forcing for one round. In
general, a major suit opening shows 5+-cards and the higher ranking
suit is opened with suits of equal length. A no-trump opening shows a
balanced hand with a possible 5-card minor.

A \di{1} bid is limited and denies a 5-card major or a good quality
5-card club suit. Therefore, it is possible that opener has only a
doubleton diamond holding a hand such as \hhand{AQJT,KQ,76,J7642} or
\hhand{AKT9,AK98,432,32}. In the first case, \cl{2} is not possible
because of the weak suit and \nt{1} is unattractive in both cases (in
the latter case, the \di{1} bid allows an immediate search for a major
suit fit). A 5-card major could be suppressed in exceptional cases
such as \hhand{AKQT5,AJ,JT9432,} when \di{1} should be followed by a
jump shift in \sp{} to show the fine 6-5 hand.

\begin{longtable}{ p{2.5cm}p{8.5cm} }
  \hline
  \cl{1} & 16+\hcp\ and any distribution,
           \underline{forcing}. \hyperlink{1c}{\HandCuffRight} \\
  \di{1} & 11-15\hcp, at least 2 cards in \di{}, no 5-card major and
           less than 6 clubs. \hyperlink{1d}{\HandCuffRight} \\
  \he{1}, \sp{1} & 11-15\hcp, 5-cards in suit bid. Any other opening bid
                   below \he{2} (except \cl{1}) denies a 5-card
                   major. \hyperlink{1major}{\HandCuffRight} \\
  \nt{1} & 13-15\hcp, balanced. 6-3-2-2 / 5-4-2-2 in a minor ok with
           stoppers in both
           doubletons. \hyperlink{1nt}{\HandCuffRight} \\
  \cl{2} & 11-15\hcp, Either a good 5-card club suit with 4-card major
           or a 6+-card club suit (7\sq\ hand). \hyperlink{2c}{\HandCuffRight} \\
  \di{2} & 11-15\hcp, Short \di{} in a 3=4=1=5, 4=3=1=5, 4=4=1=4 or
           4=4=0=5 shape, \underline{forcing}. \hyperlink{2d}{\HandCuffRight} \\
  \he{2}, \sp{2} & 8-10\hcp, 6+-card suit (\sq\ of 8 when vulnerable
                   and 7 non-vulnerable). With 6 or less losers, open
                   \sp{1}/\he{1}/\di{1}. \hyperlink{2major}{\HandCuffRight}
  \\
  \nt{2} & 6-12\hcp, 5-5 or better in minors, limited
           strength. \hyperlink{2nt}{\HandCuffRight} \\
  \cl{3}, \di{3}, \he{3}, \sp{3} & 8-10\hcp, 7+-card suit (\sq\ of 9
                                   when vulnerable and 8 non-vulnerable). Apply rule of
                                   2/3/4. \hyperlink{3preempt}{\HandCuffRight}
  \\
  \nt{3} & $\approx$10\hcp, solid 7+-card minor suit. Any side suit has
           limited strength. \hyperlink{3nt}{\HandCuffRight} \\
  \hline
\end{longtable}

\section{Responses to \cl{1}}

\hypertarget{1c}
The responses to \cl{1} can be negative, constructive or positive. All
positive responses are forcing to game unless both hands are minimum
and no fit is found. 

\begin{longtable}{ p{2.5cm}p{8.5cm} }
  \hline
  \multicolumn{2}{l}{\emph{\underline{Negative Response}}} \\
  \di{1} & 0-7\hcp, \underline{forcing}. \hyperlink{1c1d}{\HandCuffRight} \\
  \multicolumn{2}{l}{\emph{\underline{Constructive Responses}}} \\
  \he{2}, \sp{2} & 4-7\hcp, 6+-card suit, little strength
                   outside. \hyperlink{1c2major}{\HandCuffRight} \\
  \multicolumn{2}{l}{\emph{\underline{Positive Responses (forcing to game})}} \\
  \he{1}, \sp{1}, \cl{2}, \di{2} & 8+\hcp, 5+-cards in suit, can stop
                                   short of game if no fit.
                                   \hyperlink{1csuit}{\HandCuffRight}\\
  \nt{1} & 8-13\hcp, no five-card suit. \hyperlink{1c1nt}{\HandCuffRight} \\
  \nt{2} & 14+\hcp, balanced hand. \hyperlink{1c2nt}{\HandCuffRight} \\
  {\color{blue}\cl{3}} & {\color{blue}8-10\hcp, 4-4-4-1 shape with a black singleton. \di{3} is a
           relay and \he{3}/\sp{3} would show \cl{}/\sp{} singleton
           respectively. A rebid in the singleton suit is a \emph{control
           asking bid}.} \hyperlink{controlask}{\HandCuffRight} \\
  {\color{blue}\di{3}} & {\color{blue}8-10\hcp, 4-4-4-1 shape with a red singleton. \he{3} is a
           relay and \sp{3}/\nt{3} show a \di{}/\he{} singleton
           respectively. A rebid in the singleton suit is a \emph{control
           asking bid}.} \hyperlink{controlask}{\HandCuffRight} \\
  \he{3}, \sp{3}, \cl{4}, \di{4} & 11+\hcp, singleton in suit above
                                   the one bid. {\color{blue}A rebid in the
                                   singleton suit is a \emph{control
                                   asking bid}.} \hyperlink{controlask}{\HandCuffRight} \\
  \hline
\end{longtable}

\subsection{Handling intervention over \cl{1}}

If an opponent doubles \cl{1}, the responses other than \di{1} remain
the same. However, the additional possible responses of \emph{Pass}
and \emph{Redouble} are used to provide more granular
information. Over an opponent's overcall in a suit at the 1-level (or
a \nt{1} overcall), different responses are used as in the table
below.

\begin{longtable}{ p{2.5cm}p{8.5cm} }
  \hline
  \multicolumn{2}{l}{\emph{\underline{After \cl{1}--Double}}} \\
  \emph{Pass} & 0-4\hcp. \\
  \di{1} & 5-7\hcp, \underline{forcing}. \\
  \emph{Redouble} & 8+\hcp, 4-4 in the major suits. \\
  \emph{Others} & Same as over \cl{1} without intervention. \\
  \multicolumn{2}{l}{\emph{\underline{After \cl{1}--\di{1}/\he{1}/\sp{1}}}} \\
  \emph{Pass} & 0-4\hcp. \\
  \emph{New suit} & 5-8\hcp, 5+-card suit. \\
  \emph{Jump in suit} & 8-10\hcp, 6+-card suit. \\
  \nt{1} & 9-11\hcp\ with stopper in opponent's suit. \\
  \nt{2} & 12-14\hcp\ with one or two stoppers. \\
  \emph{Double} & 5-8\hcp\ or 9+\hcp, no 5-card suit,
                  \underline{takeout}. Cue bid on next round clarifies hand as
                  9+\hcp. \\
  \emph{Cue Bid} & 9+\hcp, \underline{game forcing}. \\
  \multicolumn{2}{l}{\emph{\underline{After \cl{1}--\nt{1}}}} \\
  \emph{Pass} & 0-4\hcp. \\
  \emph{New suit} & 5-8\hcp, 5+-card suit. \\
  \emph{Double} & 5+\hcp. \\
  \multicolumn{2}{l}{\emph{{\color{blue}\underline{At 2-level}}}} \\
  \emph{Pass} & 0-4\hcp\ or 9+\hcp\ with strength in opponent's
                suit. \\
  \emph{New suit} & 5-8\hcp, 5+-card suit. \\
  \emph{Double} & 5+\hcp. \\
  \multicolumn{2}{l}{\emph{{\color{blue}\underline{At 3-level}}}} \\
  \emph{Double} & 5+\hcp, takeout. \\
  \multicolumn{2}{l}{\emph{{\color{blue}\underline{At 4-level}}}} \\
  \emph{Double} & Weak hand. \\
  \emph{Pass} & \underline{Forcing pass}, ask opener to take action. \\
  \hline
\end{longtable}

\subsection{Bidding after a negative response \cl{1}-\di{1}}

\hypertarget{1c1d}
Opener rebids \nt{1}, \nt{2} or \he{3} with balanced hands (16-18,
19-21 or 25+\hcp\ respectively), minimum suit bids with 16-21\hcp\ and
a jump in a suit with 22+\hcp. Further bidding is largely natural.

\begin{longtable}{ p{2.5cm}p{8.5cm} }
  \hline
  \nt{1} & 16-18\hcp, balanced hand, no 5-card
           major. Responder's rebids are: \\
         & \begin{tabular}{ll}
             \emph{Pass} & 0-4\hcp. \\
             \cl{2} & 5-7\hcp, \emph{Stayman}
                      \hyperlink{stayman}{\HandCuffRight}
                      with at least one 4-card major. \\
             \di{2} & 0-7\hcp, transfer to \he{2}. \\
             \he{2} & 0-7\hcp, transfer to \sp{2}. \\
             \sp{2} & 0-7\hcp, transfer to \cl{3}. \\
             \nt{2} & Weak with both minors or \di{}. \\
             {\color{blue}\cl{4}} & {\color{blue}\emph{Gerber} ace-asking.} \hyperlink{gerber}{\HandCuffRight} \\
             {\color{blue}\di{4}, \he{4}} & {\color{blue}\emph{Texas} transfers to \he{4}/\sp{4}
                              respectively.} \\
           \end{tabular} \\
  \nt{2} & 19-21\hcp, balanced hand, no 5-card
           major. Responder's rebids are: \\
         & \begin{tabular}{ll}
             \emph{Pass} & 0-3\hcp. \\
             \cl{3} & 4-7\hcp, \emph{Stayman}---see\ref{stayman}. \\
             \di{3}, \he{3} & $<8$\hcp, transfer to \he{3}/\sp{3} respectively. \\
             \sp{3} & Transfer to minors. \\
             \nt{3} & 4-5\hcp, sign-off. \\
             \cl{4} & \emph{Gerber} ace-asking. \hyperlink{gerber}{\HandCuffRight} \\
             \di{4}, \he{4} & \emph{Texas} transfers to \he{4}/\sp{4}
                              respectively. \\
           \end{tabular} \\
  \he{1}, \sp{1} & 4+-card suit, non-forcing. Responder's rebids are: \\
         & \begin{tabular}{lp{6cm}}
             \emph{Pass} & 0-3\hcp. \\
             \sp{1} & 4-7\hcp, 4+-cards, may have three \he{}. \\
             \nt{1} & 5-7\hcp, no 5-card suit, no 4-card spade after
                      \he{1}. May have 3-card support. \\
             \cl{2}, \di{2} & 5-7\hcp, 5-card suit, denies 3-card support. \\
             \emph{Single raise} & 2-4\hcp\ with 3+-card support. \\
             \emph{Double raise} & 5-7\hcp\ with 4+-card support \\
             {\color{blue}\emph{Game raise}} & {\color{blue}8+\hcp, god hand, 4+-card support.} \\
           \end{tabular} \\
  \cl{2}, \di{2} & 5+-card suit, may have a 4-card major,
                   non-forcing. Responder's rebids are: \\
         & \begin{tabular}{lp{6cm}}
             \emph{Pass} & 0-3\hcp. \\
             \he{2}, \sp{2} & 4-7\hcp, 5+-cards. \\
             \nt{2} & 5-7\hcp, balanced hand, no 5-card suit. \\
             {\color{blue}\di{2}}, \cl{3} & 5-7\hcp, 6+-card suit. \\
             {\color{blue}\emph{Single raise}} & {\color{blue}3+-card support, non-forcing.} \\
             \he{3}, \sp{3} & 5-7\hcp\ \emph{Splinter}, 4+ \cl{}/\di{}. \\
           \end{tabular} \\
  \he{2}, \sp{2}, \cl{3}, \di{3} & 22+\hcp, 5-card suit. Responder's rebids are: \\
         & \begin{tabular}{lp{6cm}}
             {\color{blue}\emph{Minimum \nt{}}} & {\color{blue}0-3\hcp, minimum, no support.} \\
             \emph{Raise} & 0-3\hcp, minimum, 3+-card support. \\
             \emph{Jump raise} & 4-7\hcp, maximum, 3+-card support, no
                                 specific values in other suits. \\
             \emph{New suit} & 4-7\hcp, values in suit, does not deny
                               support for partner's suit. \underline{forcing}. \\
             {\color{blue}\emph{Jump \nt{}}} & {\color{blue}4-7\hcp\ maximum, spread values, no support.} \\
           \end{tabular} \\
  \he{3} & 25+\hcp, balanced hand, \underline{forcing}. Responder's rebids are: \\
         & \begin{tabular}{p{1.5cm}p{6cm}}
             \sp{3} & 0-4\hcp, puppet to \nt{3}. Opener rebids \nt{3} after
                      which \cl{4}, \di{4}, \he{4}, \sp{4} will show 6+-cards and
                      5-6\hcp. \\
             \cl{4}, \di{4}, \he{4}, \sp{4} & 5-6\hcp, 5+-cards. Opener bids
                                              one above suit (\di{4},
                                              \he{4}, \sp{4}, \nt{4}) to show
                                              fit and \emph{RKCB}.
                                              \hyperlink{blackwood}{\HandCuffRight} \\
             \nt{4} & 5-6\hcp, no 5-card suit, quantitative. \\
           \end{tabular} \\
  \nt{3} & Running suit, to play. \\
  \hline
\end{longtable}

\subsection{Bidding after a constructive response \cl{1}-\he{2}/\sp{2}}

\hypertarget{1c2major}
Opener needs to decide the best contract and if there is no chance for
game or slam, he should pass with a tolerance for responder's suit.

\begin{longtable}{ p{2.5cm}p{8.5cm} }
  \hline
  \emph{Pass} & Game unlikely. \\
  \nt{3}, \he{4}, \sp{4} & Raise to game is a sign-off. \\
  \emph{New suit} & \underline{One-round force}. Responder's rebids are:\\
              & \begin{tabular}{p{1.8cm}p{5cm}}
                  \emph{Raise} & 3+-card support (or \emph{Qx}). \\
                  \emph{Rebid \he{}/\sp{}} & Minimum, no support. \\
                  \emph{Cue bid under} \nt{3} & Singleton or void in suit bid. \\
                \end{tabular} \\
  \nt{2} & Support for suit, \underline{game force}. Responders's
           rebids are:\\
              & \begin{tabular}{p{1.8cm}p{5cm}}
                  \emph{Rebid} \he{}/\sp{} & Minimum. \\
                  \emph{New suit} & Singleton or void in bid suit. \\
                \end{tabular} \\
  {\color{blue}\nt{3}} & {\color{blue}AQ or KQ in suit.} \\
  {\color{blue}\nt{4}} & {\color{blue}Blackwood---see \ref{blackwood}.} \\
  \hline
\end{longtable}

\subsection{Bidding after a positive no-trump response \cl{1}--1NT/2NT}

\hypertarget{1c1nt} A \cl{2} rebid after a response of \nt{1} is
\emph{Transfer Stayman} and responder rebids as below:

\begin{longtable}{ p{2.5cm}p{8.5cm} }
  \hline
  \di{2} & 8-10\hcp, 4 card \he{}, may have 4 card
           \sp{}. Opener's rebids are: \\
         & \begin{tabular}{lp{7cm}}
             \he{2} & \he{} fit assured; bid \nt{2} with 4-3-3-3 else second
                      suit at 3-level. \\
             \sp{2} & 4-card \sp{}, no 4-card \he{}. \\
             \nt{2} & No 4-card major. \\
           \end{tabular} \\
  \he{2} & 8-10\hcp, 4 card \sp{}, denies 4-card \he{}. \\
         & \begin{tabular}{lp{7cm}}
             \sp{2} & Spade fit confirmed, relay; responder bids
                      \nt{2} with 4-3-3-3 else second
                      suit at 3-level. \\
             \nt{2} & No 4-card \sp{}, may have 4-card \he{}. \\
           \end{tabular} \\
  \sp{2} & 8-10\hcp, no 4 card major. Opener then bids \nt{2} to ask
           for a further description. Responder's rebids are: \\
         & \begin{tabular}{lp{7cm}}
             \cl{3}, \di{3} & 4-3-3-3 with 4-card \cl{}/\di{}. \\
             \he{3} & 4-4 in minors with 3 \he{}. \\
             \sp{3} & 4-4 in minors with 3 \sp{}. \\
             \nt{3} & 5-card minor. \\
           \end{tabular} \\
  \nt{2} & 11-13\hcp, 4-3-3-3 shape (4-card minor). \cl{3} by opener
           is then a relay asking responder to bid his suit or \nt{3}
           with clubs. \\
  \cl{3} & 11-13\hcp, 4-4-3-2 shape with 4 clubs. Opener bids \di{3}
           as a relay and responder bids \he{3} with \sp{}, \sp{3}
           with \he{} and \nt{3} with \di{}.\\
  \di{3} & 11-13\hcp, 4-4-3-2 shape with \di{} and \he{}. \\
  \he{3} & 11-13\hcp, 4-4-3-2 shape with \he{} and \sp{}. \\
  \sp{3} & 11-13\hcp, 4-4-3-2 shape with \sp{} and \di{}. \\
  \nt{3} & 11-13\hcp, 5-card minor suit. \cl{4} by opener is then a
           relay asking responder to bid his suit. \\
  \di{2}, \he{2}, \sp{2}, \cl{3} & Good suit asking responder for
                                   strength and fit---show in steps: \\
         & \begin{tabular}{ll}
             \emph{1 step} & 8-10\hcp\ with no fit. \\
             \emph{2 steps} & 8-10\hcp\ with 3+-card support. \\
             \emph{3 steps} & 11-13\hcp\ with no fit. \\
             \emph{4 steps} & 11-13\hcp\ with 3+-card support. \\
           \end{tabular} \\
  \hline
\end{longtable}

\gap

Any new suit bid by the \cl{1} opener that is not a relay and is under
the level of game is a \emph{control asking bid} and responses
are in steps as below:

\begin{longtable}{ p{2.5cm}p{8.5cm}  }
  \multicolumn{2}{l}{\emph{Responses to control asking bid}} \\
  \hline
  \emph{1 step} & None of the top three honours. \\
  \emph{2 steps} & One of the top three honours. \\
  \emph{3 steps} & Two of the top three honours. \\
  \emph{4 steps} & AKQ of suit. \\
  \hline
\end{longtable}

\gap

A raise in no-trumps by opener shows some slam interest but a direct raise to game is a sign-off.

\begin{longtable}{ p{2.5cm}p{8.5cm}  }
  \multicolumn{2}{l}{\emph{Raise in no-trumps after \cl{1}--\nt{1}}} \\
  \hline
  \nt{2} & 5+-card club suit, natural bidding until game or slam is
           reached. \\
  \nt{3} & Minimum \cl{1} hand - sign-off (to play). \\
  \hline
\end{longtable}

\gap

\hypertarget{1c2nt}
Over a \nt{2} response to \cl{1}, the bidding is simpler due to the
lack of bidding room at this level.

\begin{longtable}{ p{2.5cm}p{8.5cm} }
  \multicolumn{2}{l}{\emph{Opener's rebids after \cl{1}--\nt{2}}} \\
  \hline
  \cl{3} & \emph{Baron}: asking responder to show 4-card suits upwards
           (\nt{3} after \cl{3} shows 4-3-3-3 in \cl{}). \\
  \di{3}, \he{3}, \sp{3}, \nt{3} & 5+-card suit (\nt{3} shows \cl{}).
                                   Subsequent bidding is natural until
                                   game or slam is reached. \\
  \cl{4} & Shows 6+-card \cl{} suit. \\
  \hline
\end{longtable}

\subsection{Bidding after a positive suit response \cl{1}--\he{1}/\sp{1}/\cl{2}/\di{2}}

\hypertarget{1csuit}
Opener may use the $\beta$ \emph{control asking bid} with a fit in
responder's suit or bid a new suit or no-trumps with no fit. With no
chance of slam, the principle of fast arrival should be used to bid
the appropriate game contract.

\begin{longtable}{ p{2.5cm}p{8.5cm} }
  \multicolumn{2}{l}{\emph{Opener's rebids after a positive major suit response} \cl{1}--\he{1}/\sp{1}}\\
  \hline
  \emph{New suit} & 5+-card suit, denies 3-card support for responder's
                    suit. Subsequent bids are natural to find the correct game
                    contract. \underline{Forcing for one
                    round}. Responses are: \\
                  & \begin{tabular}{ll}
                      \emph{New suit} & 4-card suit. \\
                      \emph{Raise} & 3+-card support. \\
                      \emph{Rebid original} & 6+-card suit, semi-solid
                                              if minor. \\
                      \emph{Minimum NT} & 5-3-3-2 shape, values in
                                          unbid suits. \\
                      \end{tabular} \\
  \nt{1} & $\beta$ \emph{control asking bid}. \hyperlink{controlask}{\HandCuffRight} \\
  \emph{Single raise} & Suit fit but not a great hand since $\beta$
                                  wasn't used. \\
  \nt{2} & $\beta$ \hyperlink{controlask}{\HandCuffRight} over
           \cl{2}/\di{2}; Natural, no 5-card suit over \he{1}/\sp{1}. \\
  \sp{3}, \cl{4}, \di{4}, \he{4} & \emph{Splinter} bid 4-card fit
                                   promised. \\
  \nt{4} & \emph{RKC Blackwood}---see \ref{blackwood}. \\
  \hline
\end{longtable}

Bidding after a \cl{2} or \di{2} response is only slightly different
from the above.

\begin{longtable}{ p{2.5cm}p{8.5cm}  }
  \multicolumn{2}{l}{\emph{Opener's rebids after a positive minor suit response} \cl{1}--\cl{2}/\di{2}}\\
  \hline
  \emph{New suit} & 5+-card suit, denies 3-card support for responder's
                    suit. Subsequent bids are natural to find the correct game
                    contract. \\
  \nt{2} & No 5-card suit and no slam interest. \\
  \emph{Raise to} \cl{3}/\di{3} & $\beta$ \emph{control asking bid}. \\
  \nt{3} & To play (by principle of fast arrival). \\
  \hline
\end{longtable}

\gap

\section{Responses to \di{1}}

\hypertarget{1d}
Even though a \di{1} opening may be made on a hand with a doubleton
diamond, it is non-forcing and partner can pass with a weak
hand. Opener must clarify his shape at the earliest if no 4-card major
fit is found.

\begin{longtable}{ p{2.5cm}p{8.5cm}  }
  \hline
  \emph{Pass} & 0-7\hcp. \\
  \he{1} & 8-15\hcp, 4-card \he{}, may have 4-card \sp{},
           \underline{forcing} for one round. Opener rebids: \\
              & \begin{tabular}{p{1.5cm}p{7cm}}
                  \sp{1} & 11-14\hcp, 4-card \sp{}, no 4-card
                           \he{}. Responder rebids: \\
                         & \begin{tabular}{ll}
                             \nt{1} & Sign-off. \\
                             \cl{2} & 4th-suit-\underline{forcing}
                                      (``do something clever''). \\
                             \sp{2} & 8-9\hcp, 4-card \sp{}. \\
                           \end{tabular} \\
                  \nt{1} & 11-14\hcp, balanced, denies 4-card fit. Responder can
                           rebid \\
                         & \begin{tabular}{ll}
                             \cl{2} & New minor---\underline{one-round force}. \\
                             \di{2} & 8-9\hcp, 4-card \sp{}. \\
                           \end{tabular} \\
                  \cl{2} & Unbalanced, 5-4 in minors, no 4-card
                           major. Responder can rebid \\
                         & \begin{tabular}{ll}
                             \di{2} & Weak hand, to play. \\
                             \he{2} & 6-card \he{} suit, to play. \\
                             \cl{3} & 8-9\hcp, keep bidding alive. \\
                             \nt{3} & To play. \\
                           \end{tabular} \\
                  \di{2} & 6+-card \di{} suit, non-forcing. \\
                  \he{2} & 3-4 card support, if 3-card promises a singleton
                           in a side suit. \\
                  \sp{2} & 14-15\hcp, 5+-\di{} and 4+-\sp{}. \\
                  \nt{2} & 14-15\hcp, 5-4-2-2 shape with \sp{}/\cl{}
                           stopper. \\
                  \emph{Double raise} & 15\hcp, 4-card support. \\
                  \di{3} & 15\hcp, 6+-card \di{} suit, no 4-card major. \\
                  \cl{3} & 15\hcp, unbalanced, at least 5-5 in minors, no
                           4-card major. \\
                \end{tabular} \\
  \sp{1} &  8-15\hcp, 4-card \sp{}, denies 4-card \he{},
           \underline{forcing} for one round. Opener's rebids have the
           same structure as over \he{1}. \\
  \nt{1} & 8-10\hcp, balanced, no 4-card major. \\
  \cl{2}, \di{2} & 11-15\hcp, 4+-card suit, no 4-card major,
                   \underline{forcing}. Opener's rebids are: \\
              & \begin{tabular}{ll}
                  \he{2} & 11-14\hcp, \he{} stopper, no \sp{} stopper. \\
                  \sp{2} & 11-14\hcp, \sp{} stopper, no \he{} stopper. \\
                  \nt{2} & 11-14\hcp, stopper in both majors. \\
                  \di{2}, \di{3} & No stopper in majors, genuine \di{}
                                   suit. \\
                  \he{3} & 15\hcp, \underline{game force}, \he{} stopper, no
                           \sp{} stopper. \\
                  \sp{3} & 15\hcp, \underline{game force}, \sp{} stopper, no
                           \he{} stopper. \\
                  \nt{3} & 15\hcp, stopper in both majors. \\
                  \cl{3} & No stopper in majors. \\
                \end{tabular} \\
  \he{2}, \sp{2} & Weak jump shift, 5+-card suit,
                   non-forcing. Opener's rebids are: \\
              & \begin{tabular}{p{1.5cm}p{7cm}}
                  \nt{2} & 11-12\hcp, no 4-card major. \\
                  \cl{3} & Both minors asking responder to decide between
                           \cl{3}, \di{3} and \nt{3}. \\
                \end{tabular} \\
  {\color{blue}\nt{2}} & {\color{blue}15+\hcp, balanced, no 4-card major.} \\
  \di{3} & Up to 10\hcp, preemptive, 5+-cards in \di{}. \\
  \he{3}, \sp{3} & 6-9\hcp, 7+-card suit, invitational to game with
                   fit. \\
  \nt{3} & 13-14\hcp, balanced, no 4-card major. \\
  \di{4} & Preemptive. \\
  \hline
\end{longtable}

\subsection{Intervention over \di{1}}

If opponent doubles \di{1}, the responses are:

\begin{longtable}{p{2.5cm}p{8.5cm} }
  \hline
  \emph{Pass} & 0-4\hcp\ or 9-10\hcp. \\
  \emph{New suit} & 5-8\hcp. \\
  \nt{1} & 6-8\hcp, balanced. \\
  \di{2}, \di{3} & $<8$\hcp, 4+-card support, preemptive. \\
  \emph{Redouble} & 11+\hcp. \\
  \hline
\end{longtable}

After an overcall by opponent up to the \sp{2} level, the responses are:

\begin{longtable}{p{2.5cm}p{8.5cm} }
  \hline
  \emph{Double} & 8-10\hcp, 4+-cards in other major, negative. \\
  \emph{New suit} & 5-card suit if major, 4-card otherwise. \\
  \di{2} & 6-9\hcp, support for major. \\
  \di{3} & 10-11\hcp. \\
  \nt{1} & 8-10\hcp, stopper in opponent's suit, balanced. \\
  \nt{2} & 11-13\hcp, stopper in opponent's suit, balanced. \\
  \hline
\end{longtable}

\section{Responses to \he{1} or \sp{1}}

\hypertarget{1major}
Responses to a major opening, include \emph{Bergen} raises, \emph{Splinter}
bids, weak jump shifts, new-minor forcing and 4th-suit forcing
approaches.

\begin{longtable}{ p{2.5cm}p{8.5cm}  }
  \hline
  \emph{Pass} & 0-7\hcp\ and poor support. \\
  \emph{Single raise} & 7-10\hcp\ with 3-card support, constructive. \\
  \emph{Double raise} & 0-6\hcp\ with 4-card support (preemptive
                        \emph{Bergen} raise  \hyperlink{bergen}{\HandCuffRight}). \\
  \cl{3} & 6-8\hcp\ with 4-card support (\emph{Bergen} raise \hyperlink{bergen}{\HandCuffRight}). \\
  \di{3} & 9-11\hcp\ with 4-card support (\emph{Bergen} raise  \hyperlink{bergen}{\HandCuffRight}). \\
  \emph{Game raise} & 0-7\hcp\ with 5-card support, to play. If opener
                      bids a new suit, it is a cue bid ace and slam try. \\
  \sp{1} & See below for \sp{1} over \he{1}. \\
  \nt{1} & 8-15\hcp, balanced hand with mild support for opener's suit
           or unbalanced hand with insufficient \hcp\ to justify a
           2-over-1 response. \underline{Forcing}---Opener's
           rebids are: \\
              & \begin{tabular}{ll}
                  \multicolumn{2}{l}{\emph{\underline{With 11-13\hcp}}} \\
                  \cl{2}, \di{2}, \he{2} & 11-13\hcp, 4-card (or
                                           3-card better minor) suit. \\
                  \emph{Rebid of suit} & 11-13\hcp, 6-card suit. \\
                  \multicolumn{2}{l}{\emph{\underline{With a maximum 14-15\hcp}}} \\
                  \emph{Jump rebid of suit} & 14-15\hcp, 6-card solid suit. \\
                  \nt{2} & 5-3-3-2 distribution. \\
                  \emph{Jump in new suit} & 5-5 distribution. \\
                \end{tabular} \\
  \cl{2}, \di{2},
  \he{2} & 8-15\hcp, 4-card minor or 5-card heart
           suit (after \sp{1}). Opener's rebids are:\\
              & \begin{tabular}{ll}
                  \he{2} (after \sp{1}) & Natural, 4-card+ suit. \\
                  \sp{2} (after \he{1}) & 14-15\hcp, reverse. \underline{forcing}. \\
                  \emph{Rebid own suit} & Minimum, not necessarily a 6-carder. \\
                  \nt{2} & 11-13\hcp, stoppers in unbid suits \\
                  \emph{Raise} & 11-13\hcp, good support, non-forcing. \\
                  \emph{Jump in new suit} & 14-15\hcp, good support, control in bid
                                            suit. \\
                  \emph{Jump in own suit} & 14-15\hcp, very good 6-card suit. \\
                  \nt{3} & 15\hcp, stoppers in unbid suits. \\
                \end{tabular} \\
  \nt{2} & 13+\hcp, \underline{forcing to game}, \emph{Jacoby}
           \nt{2}. See \ref{jacoby2nt} for responses. \\
  \emph{Double jump-shift} & \emph{Splinter} bid, singleton or void in bid suit,
                             4+card support. \underline{Forcing to game}. \\
  \nt{3} & 14-15\hcp, usually 4-card support for opener's major,
           responder lacks a void or singleton, minimum 4 controls. \\
  \hline
\end{longtable}

A response of \sp{1} over \he{1} shows 8-15\hcp\ with a 4-card
suit and is forcing for one round. Opener's rebids are:

\begin{longtable}{ p{2.5cm}p{8.5cm}  }
  \hline
  \nt{1} & 11-13\hcp, minimum. \\
  \cl{2}, \di{2} & 11-15\hcp, non-forcing. \\
  \sp{2} & 11-13\hcp, 4-card support. \\
  \nt{2} & 14-15\hcp, balanced, stoppers in both minors. \\
  \cl{3}, \di{3} & 13-15\hcp, 5-card suit. \\
  \he{3} & 13-15\hcp, 6+-card suit. \\
  \sp{3} & 13-15\hcp, 4-card spade support. \\
  \nt{3} & To play with running suit. \\
  \cl{4}, \di{4} & \emph{Splinter} bids, \underline{game forcing}. \\
  \he{4} & To play---distributional hand. \\
  \sp{4} & To play---maximum hand with 13-15\hcp\ and
           distributional. \\
  \nt{4} & Ace-asking with agreement in \sp{}. \\
  \hline
\end{longtable}

\subsection{Intervention over a major suit opening}

If an opponent doubles, responder can bid:

\begin{longtable}{ p{2.5cm}p{8.5cm}  }
  \hline
  \emph{Pass} & 0-4\hcp\ or 9-10\hcp. \\
  \emph{Raise} & $<8$\hcp, preemptive, 3+-card support following the
                 Law of Total Tricks. \\
  \emph{New suit} & 5-8\hcp, 5+-card suit. \\
  \nt{1} & 6-8\hcp, balanced. \\
  \nt{2} & \emph{Jacoby} \nt{2}---see \ref{jacoby2nt} for responses. \\
  \emph{Redouble} & 12+\hcp, support for opener's suit. \\
  \hline
\end{longtable}

If an opponent overcalls:

\begin{longtable}{ p{2.5cm}p{8.5cm}  }
  \hline
  \emph{Pass} & 0-7\hcp\ or 8+\hcp and waiting for penalty if opener reopens
                with a double. \\
  \emph{Double} & 7-9\hcp, 4-card suit in other major. \\
  \nt{1} & 8-10\hcp\ with stopper in opponent's suit. \\
  \nt{2} & 11-12\hcp\ with stopper in opponent's suit. \\
  \emph{Cue bid} & 15+\hcp with singleton or void in opponent's
                   suit. \underline{Forcing}. \\
  \hline
\end{longtable}

\section{Responses to 1NT}

\hypertarget{1nt}
A \nt{1} opening shows a 13-15\hcp\ balanced hand without a 5-card
major holding but could be a 6-3-2-2 or 5-4-2-2 hand with a long minor
suit in which case opener must hold stoppers in both
doubletons. Responses are as below:

\begin{longtable}{ p{2.5cm}p{8.5cm}  }
  \hline
  \emph{Pass} & 0-7\hcp, poor support. \\
  \cl{2} & 8-11\hcp, \emph{Stayman}---see \ref{stayman} for
           responses. \\
  \di{2}, \he{2} & \emph{Jacoby} transfers to \he{} and \sp{}
                   respectively. See \ref{jacoby} for rebids. \\
  \sp{2} & 8+\hcp---Minor suit \emph{Stayman}, denies 4-card major and asks
           opener for a 4-card minor. Shows a minor 2-suiter (5-4 or
           better). \\
  \emph{3 of suit} & Good suit, \underline{game forcing}. \\
  \cl{4} & \emph{Gerber} ace-asking. \hyperlink{gerber}{\HandCuffRight} \\
  \di{4}, \he{4} & \emph{Texas} transfers to \he{4} and \sp{4}
                   respectively. Denies slam values. \\
  \nt{4} & 17-18\hcp, balanced, quantitative.\\
  \nt{5} & 22-23\hcp, balanced, \emph{Grand Slam Force}. \\
  \nt{6} & 19-21\hcp, balanced. \\
  \hline
\end{longtable}

\subsection{Intervention after 1NT}

If an opponent doubles \nt{1}, responder can bid:

\begin{longtable}{ p{2.5cm}p{8.5cm}  }
  \hline
  \emph{Pass} & Weak or 6-11\hcp, balanced. \\
  \emph{Redouble} & 5-card suit, asking partner to bid \cl{2} and then
                    pass or rectify. \\
  \cl{2} & 4-card suit (or good 3-card clubs). \\
  \di{2} & Short in clubs. \\
  \he{2} & Tolerance for majors (at least 4-3). \\
  \sp{2} & 12+\hcp, \underline{forcing}. \\
  \cl{3}, \di{3}, \he{3}, \sp{3} & 6+-card suit, invitational. \\
  \hline
\end{longtable}

After an opponent's overcall, responder has the following choices:

\begin{longtable}{ p{2.5cm}p{8.5cm}  }
  \hline
  \emph{Double} & Penalty double. \\
  \nt{2} & \emph{Lebensohl}---partner must bid \cl{3}. \\
  \emph{Suit at 2-level} & 0-6\hcp, sign-off. \\
  \emph{Suit at 3-level above overcall} & \underline{Forcing to
                                          game}. \\
  \nt{3} & \emph{Lebensohl}, stopper in opponent's suit. \\
  \hline
\end{longtable}

\section{Responses to \cl{2}}

\hypertarget{2c}
Since a \cl{2} opening may be either a 6+-card suit or a 5-carder with
a 4-card major, the \di{2} response is conventional to ask opener to
clarify his hand.

\begin{longtable}{ p{2.5cm}p{8.5cm} }
  \hline
  \emph{Pass} & 0-7\hcp, poor support. \\
  \di{2} & 11+\hcp, conventional and \underline{forcing} (with club fit,
           may be made with only 8\hcp). Opener's rebids are: \\
              & \begin{tabular}{lp{5cm}}
                  \he{2}, \sp{2} & 11-13\hcp, 4-card suit. \\
                  \nt{2} & 11-13\hcp, 6-3-2-2 balanced hand with stoppers in two
                           suits. Responder then bids \di{3} to enquire about stoppers and
                           opener's rebids are: \\
                                 & \begin{tabular}{ll}
                                     \he{3} & \he{} and \di{} stoppers. \\
                                     \sp{3} & \sp{} and \di{} stoppers. \\
                                     \nt{3} & \he{} and \sp{} stoppers. \\
                                   \end{tabular} \\
                  \cl{3} & 11-13\hcp, 6 clubs with 1 outside stopper. Responder bids
                           \di{3} to enquire about the stopper and opener's rebids are:\\
                                 & \begin{tabular}{ll}
                                     \he{3} & \he{} stopper. \\
                                     \sp{3} & \sp{} stopper. \\
                                     \nt{3} & \di{} stopper.  \\
                                     \di{3} & 5-card suit (6-5 in \cl{} and \di{}) \\
                                   \end{tabular} \\
                  \nt{3} & 14-15\hcp, 6-3-3-2 balanced hand, good club suit. \\
                  \he{3}, \sp{3} & 14-15\hcp, 4-card suit. \\
                \end{tabular} \\
  \di{3}, \he{3}, \sp{3} & 16+\hcp, good 5-card suit. Opener rebids are: \\
              & \begin{tabular}{lp{5cm}}
                  \nt{3} & Less than 2-card support. \\
                  \emph{Raise} & Minimum, 3-card support. \\
                  \emph{New suit} & Maximum, 3+-card support, cue bid ace, slam try.  \\
                \end{tabular} \\
  \he{2}, \sp{2} & 8-10\hcp, non-forcing, invitational. Opener may
                   pass with a minimum and mild support. \\
  \nt{2} & 10-11\hcp, invitation to \nt{3}. \\
  \cl{3} & 8-10\hcp, 3+-card support, no 5-card major,
           non-forcing. Opener may pass with a minimum or bid \nt{3} with a
           maximum. \\
  \hline
\end{longtable}

\subsection{Intervention over \cl{2}}

\begin{longtable}{ p{2.5cm}p{8.5cm} }
  \hline
  \emph{Negative double} & Through \sp{3}. \\
  \emph{Redouble} & 10+\hcp. \\
  \emph{Cue bid} & 12+\hcp, singleton or void in opponent's suit. \\
  \hline
\end{longtable}

\section{Responses to \di{2}}

\hypertarget{2d}
A \di{2} opening describes a three suited hand with shortness
in diamonds and the responder can place the contract fairly easily in
most cases. The only positive response is \nt{2} which is forcing
to game.

\begin{longtable}{ p{2.5cm}p{8.5cm} }
  \hline
  \emph{Pass} & 6+ diamonds, no interest in other suits or bidding higher. \\
  \he{2}, \sp{2}, \cl{3} & Natural, sign-off. \\
  \nt{2} & 11+\hcp, artificial \underline{game-force} asks opener to
           further describe his hand. Opener's rebids are: \\
              & \begin{tabular}{ll}
                  \cl{3}, \di{3}  & 3=1=4=5 or 4=3=1=5 shape respectively. \\
                  \he{3} & 11-13\hcp\ and 4=4=1=4 shape. \\
                  \sp{3} & 14-15\hcp\ and 4=4=1=4 shape. \\
                  \nt{3} & 14-15\hcp, 4=4=1=4 shape and \di{} A or K. \\
                  \cl{4} & 11-13\hcp, 4=4=0=5 shape. \\
                  \di{4} & 14-15\hcp, 4=4=0=5 shape. \\
                \end{tabular} \\
  \di{3} & 6+ \di{} suit, invitation to \nt{3}. \\
  \he{3}, \sp{3} & 7-9\hcp, preemptive, 5+-card suit. \\
  \hline
\end{longtable}

\subsection{Intervention over \di{2}}

If opponents double \di{2}, responder can either \emph{Pass} if he
wants to play in diamonds or \emph{Redouble} asking partner to bid a
major.

\section{Responses to \he{2} or \sp{2}}

\hypertarget{2major}
Opener shows a 6+-card major with 8-10\hcp\ and a good suit with a
minimum suit quality of 8 when vulnerable or 7 when non-vulnerable.

\begin{longtable}{ p{2.5cm}p{8.5cm} }
  \hline
  \emph{Pass} & No game, no fit. \\
  \emph{Raise to 3 or 4} & Natural, sign-off. \\
  \emph{New suit} & Natural, \underline{forcing}. Opener's rebids are: \\
              & \begin{tabular}{ll}
                  \emph{Raise} & 3-card support. \\
                  \emph{Rebid own suit} & Forcing. \\
                \end{tabular} \\
  \nt{2} & \underline{Forcing}. Opener bids suit with singleton or void or rebids
           his own suit. \\
  \hline
\end{longtable}

\section{Responses to 2NT}

\hypertarget{2nt}
Opener is showing 5-5 in the minors with 6-12 points and responder's
bids are:

\begin{longtable}{ p{2.5cm}p{8.5cm} }
  \hline
  \emph{Pass} & No game, no fit. \\
  \cl{3}, \di{3} & Interception bid. Opener should bid \he{3}/\sp{3}
                   with a strong hand with less than five losers. \\
  \hline
\end{longtable}

\section{Responses to \cl{3}/\di{3}/\he{3}/\sp{3}}

\hypertarget{3preempt}
After a preemptive 3-bid by opener, responder's rebids are:

\begin{longtable}{p{2.5cm}p{8.5cm}}
  \hline
  \emph{Raise} & Preemptive. \\
  \emph{New suit} & Forcing for one round. \\
  \emph{Others} & Natural. \\
  \hline
\end{longtable}

\section{Responses to 3NT}

\hypertarget{3nt}
Responses to the gambling \nt{3} are:

\begin{longtable}{p{2.5cm}p{8.5cm}}
  \hline
  \cl{4} & Asks opener to pass or bid \di{4} if that is his suit. \\
  \di{4} & Asks opener to bid his short suit(s). \\
  \nt{4} & Quantitative, asking opener if he has an 8+-card suit. \\
  \hline
\end{longtable}

\section{Competitive Bidding}

\subsection{Overcalls}

At the one level, overcall with a 5-card suit and 8-15\hcp. At the
two-level, overcall with at least 11-15\hcp. Responses to a suit
overcall are:

\begin{longtable}{p{2.5cm}p{8.5cm}}
  \hline
  \emph{Pass} & $<8$\hcp. \\
  \emph{Raise} & 9-10\hcp\ with 3-card support. \\
  \nt{1} & 9-10\hcp, balanced with a stopper in opponent's suit. \\
  \hline
\end{longtable}

A \nt{1} overcall should be 13-15\hcp with a stopper in the opponent's
suit.

A jump suit overcall is a weak preemptive bid with a 6+-card suit.

A cue bid is \emph{Michael's} showing a 2-suiter in the highest unbid
suit and another.

\subsection{Doubles}

A direct double over opponent's opening is either 13-15\hcp\ (takeout)
or a power double with 16+\hcp.

A \emph{takeout} double over opponent's minor opening bid usually
promises a 4-card suit in both majors. Over a major suit opening, it
promise a 4-card suit in the other major. A response is requested even
with a blank hand unless the other opponent bids.

\begin{longtable}{p{2.5cm}p{8.5cm}}
  \hline
  \emph{Pass} & Long and solid holding in opponent's suit \\
  \emph{1-level} & 5-8\hcp. \\
  \emph{2-level} & 7-11\hcp. \\
  \emph{3-level} & 9-13\hcp. \\
  \nt{1} & 8-10\hcp\ with stopper in opponent's suit. \\
  \emph{Jump} & 9+\hcp. \\
  \hline
\end{longtable}

If the doubler rebids or raises, it indicates a power hand of 16+\hcp.

\gap

Doubles of an opponent's overcall are \emph{negative} doubles
indicating a lack of a biddable 5-card suit and no fit with partner's
bid suit. Interest in one of the unbid suits is strongly indicated.

At the one-level, the \emph{negative} double shows 8-15\hcp; at the
two-level, it shows 11-15\hcp. Over an opponent's overcall in a major,
it promises a 4-card holding in the other major.

\gap

A double in a competitive auction is a \emph{responsive} double
indicating 3-card support in partner's suit.

\gap

A double of a slam contract is a \emph{Lightner} double requesting an
unusual lead from partner.

\section{Gadgets and Conventions}

\subsection{Stayman Convention}
\label{stayman}
The \emph{Stayman} convention is used to find a 4-4 major suit fit
after a \nt{1} opening by bidding \cl{2}. Opener responds with one of:

\begin{longtable}{p{2.5cm}p{8.5cm}}
  \hline
  \di{2} & No four card major. \\
  \he{2} & 4-card heart suit, may have 4-card spades. \\
  \sp{2} & 4-card spade suit, no 4-card heart suit. \\
  \hline
\end{longtable}

\subsection{Jacoby Transfers}
\label{jacoby}

After a \nt{1} opening, responder bids \di{2} with a 5-card or better
heart suit and \he{2} with spades. Opener will bid \he{2} or \sp{2} so
that the strong hand becomes declarer. Responder's rebids are:

\begin{longtable}{p{2.5cm}p{8.5cm}}
  \hline
  \emph{Pass} & A weak hand with 5+-card \he{} or \sp{}. \\
  \sp{2} & Invitational with 5-5 in the majors after
           \nt{1}--\di{2}--\he{2}. \\
  \nt{2} & Balanced or semi-balanced hand with 5-card \he{} or
           \sp{}. Invitational---partner can pass or sign-off in 3 of major or bid \nt{3}. \\
  \cl{3}, \di{3} & 4-card suit in addition to 5-card major,
                   \underline{game forcing}.. \\
  \emph{Raise} & 6-card suit, invitational. \\
  \he{3} & (After \sp{2}) 5-5 in the majors with slam
           interest. Stronger than an immediate jump to \he{4}. \\
  \sp{3} & (After \he{2}) Singleton or void with slam interest. \\
  \nt{3} & Balanced or semi-balanced hand. Partner can pass or correct
           to 4 of major. \\
  \he{4} & (After \sp{2}) 5-5 in majors with no slam interest. Partner
           can pass or correct to \sp{4}. \\
  \emph{Double raise} & 6+-card major, sign-off. \\
  \nt{4} & Quantitative, inviting slam in major or no-trumps. \\
  \hline
\end{longtable}

\subsection{Jacoby 2NT}
\label{jacoby2nt}

A \nt{2} response over an opening of \he{1} or \sp{1} is conventional
and shows 13+\hcp\ with 4+-card support of partner's suit. It is
\underline{forcing to game}.

Opener's rebids would be:

\begin{longtable}{p{5cm}p{6cm}}
  \hline
  \emph{New suit} & Singleton or void in suit bid. \\
  \emph{Rebid of suit at 3-level} & Maximum strength hand. \\
  \emph{Jump shift} & Good 5+-card side suit. \\
  \emph{Game in original suit} & Minimum opening, sign-off. \\
  \nt{3} & 12-13 \hcp, medium strength hand. \\
  \hline
\end{longtable}


\subsection{Gerber Convention}
\label{gerber}

An immediate response of \cl{4} to any no-trump bid (or overcall) is
\emph{Gerber}. A jump rebid of \cl{4} in response to a natural
no-trump bid is \emph{Gerber} as also when a trump suit has not been
identified and no-trumps has been rebid. Gerber should not be used
holding a void.

Opener shows number of aces in steps as follows:

\begin{longtable}{p{2.5cm}p{8.5cm}}
  \hline
  \di{4} & Zero or Four aces. \\
  \he{4} & One ace. \\
  \sp{4} & Two aces. \\
  \nt{4} & Three aces. \\
  \hline
\end{longtable}

\subsection{RKC Blackwood}
\label{blackwood}

A \emph{Roman Key Card Blackwood} bid of \nt{4} is used to enquire
about the number of key cards (any ace or the trump suit king) in partner's hand.

\emph{Blackwood} should not be used when you have a void or two
fast losers.

The responses to the bid of \nt{4} are in steps and differ depending
on whether opponents have bid over \nt{4}. Note that the 3rd and 4th
steps show only 2 key cards if opponents have intervened since the 5
key card response is shown by step 2 in such cases.

\begin{longtable}{p{3.2cm}|p{1.2cm}p{3.2cm}p{3.2cm}}
  \emph{Holding} & \emph{Silent} & \emph{Double (ROPI)} & \emph{Overcall (DOPI)} \\
  \hline
  \emph{1/4 key cards} & \cl{5} & \emph{Redouble} & \emph{Double} \\
  \emph{0/3 key cards} & \di{5} & \emph{Pass} (0/3/5 key cards) & \emph{Pass} (0/3/5 key cards) \\
  \emph{2/5 key cards
  without queen of trumps} & \he{5} & \cl{5} (two key cards) & \emph{Cheapest suit} (two key cards)
  \\
  \emph{2/5 key cards
  and queen of trumps} & \sp{5} & \di{5} (two key cards) & \emph{Second-cheapest suit} (two key cards)\\
  
  \hline
\end{longtable}

When holding a void, after a trump suit is agreed, jumping to another
suit at the 4 or 5 level in the void suit initiates a \emph{key card
  exclusion} asking bid. Partner shows his key cards \emph{excluding} any
in the void suit in steps.

\begin{longtable}{p{2.5cm}p{8.5cm}}
  \hline
  \emph{1 step} & 1 or 4 key cards. \\
  \emph{2 steps} & 0 or 3 key cards. \\
  \emph{3 steps} & 2 or 5 key cards without trump Q. \\
  \emph{4 steps} & 2 or 5 key cards with trump Q. \\
  \hline
\end{longtable}

\subsection{$\beta$ Control Asking Bid}
\label{controlask}

A $\beta$ control asking bid can occur either after a \nt{1} rebid by
the \cl{1} opener over a positive suit response or by cue bidding a
singleton suit after a positive response of \cl{3}, \di{3}, \he{3},
\sp{3}, \cl{4} or \di{4} over \cl{1}.

The number of controls held (\emph{A=2}, \emph{K=1}) are shown in steps as below:

\begin{longtable}{ p{2.5cm}p{8.5cm} }
  \hline
  \emph{1 step} & 0-2 controls.
                  A relay bid by opener in the cheapest suit over the
                  1-step response will then ask for clarification and
                  again the responses are in steps: \\
                & \begin{tabular}{ll}
                    \emph{1 step} & No controls. \\
                    \emph{2 steps} & 1 control. \\
                    \emph{3 steps} & 2 controls. \\  
                  \end{tabular} \\
  \emph{2 steps} & 3 controls. \\
  \emph{3 steps} & 4 controls. \\
  \emph{4 steps} & 5 controls. \\
  \hline
\end{longtable}

\subsection{Bergen Raises}

\hypertarget{bergen}
After a \he{1} or \sp{1} opening, responses of \cl{3}, \di{3}, \he{3}
and \sp{3} show different types of 4-card support. The mnemonic
\emph{CLAP} (Constructive, Limited and Preemptive) helps to remember
the order of the bids.

\begin{longtable}{p{2.5cm}p{8.5cm}}
  \hline
  \emph{\he{1}--\cl{3}} & Constructive, 7-10\hcp, 4-card \he{}. \\
  \emph{\he{1}--\di{3}} & Limited, 10-12\hcp, 4-card \he{}. \\
  \emph{\he{1}--\he{3}} & Preemptive, 0-6\hcp, 4-card \he{}. \\
  \emph{\sp{1}--\cl{3}} & Constructive, 7-10\hcp, 4-card \sp{}. \\
  \emph{\sp{1}--\di{3}} & Limited, 10-12\hcp, 4-card \sp{}. \\
  \emph{\sp{1}--\he{3}} & Strong, 12+\hcp, 4-card \sp{}, undisclosed singleton/void. \\
  \emph{\sp{1}--\sp{3}} & Preemptive, 7-10\hcp, 4-card \sp{}. \\
  \hline
\end{longtable}

\section{Miscellaneous}

\emph{High-card Points} (\hcp) are assigned as follows---Ace: 4, King:
3, Queen: 2 and Jack: 1.  Once a trump-fit has been found,
distribution points can be assigned---Void: 3, Singleton: 2,
Doubleton: 1. Singleton honours should be counted only once (either
\hcp\ or shortness).

\gap

\emph{Suit Quality} (\sq) is calculated as suit length plus number of
honours in the suit. The Jack or Ten should be counted only if a
higher ranking honour is held. For example, a holding of K-J-9-5-4
would have 7\sq\ but J-10-9-5-4 would have 5\sq.

For an overcall, the \sq\ should equal or exceed the number of tricks
bid (e.g., \sq\ of 8 for a 2-level overcall).

When preempting, the \sq\ should equal the level of preempt when
vulnerable and can be one less when non-vulnerable.

\gap

The \emph{Losing Trick Count} (\ltc) is used only once a trump suit
has been established. Count losers only in the top three cards of the
suit holding---there are never more than 3 losers in a suit. With
three or more cards, A/K/Q are not losers but any lower card is a
loser. With two cards, only A or K are not losers.

Add your and partner's loser count and subtract from 24 to estimate
the number of tricks that can be won.  You can estimate your partner's
\ltc\ as follows:

\begin{tabular}{p{2.5cm}p{8.5cm}}
  \emph{\hcp{}} & \emph{Expected Losers} \\
  \hline
  \emph{7-9} & 8-9 losers (9) \\
  \emph{10-12} & 7-8 losers (8) \\
  \emph{13-15} & 6-7 losers (7) \\
  \emph{16-18} & 5-6 losers (6) \\
  \emph{19-21} & 4-5 losers (5) \\
  \emph{22+} & 4 losers or less \\
  \hline
\end{tabular}

\end{document}
