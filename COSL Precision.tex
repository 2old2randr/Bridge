\documentclass[a4paper,article,oneside]{memoir}
\counterwithout{section}{chapter}
\setsecnumdepth{subsection}
\maxtocdepth{subsection}
\usepackage{microtype}
\usepackage{longtable}
\usepackage{hyperref}
\usepackage{bbding}
\usepackage{xcolor}
\usepackage{grbbridge}
\setboolean{spellten}{true}
\usepackage{url}
\newcommand{\gap}{\vspace{\baselineskip}}
\newcommand{\hcp}{\textsc{hcp}}
\newcommand{\sq}{\textsc{sq}}
\newcommand{\ltc}{\textsc{ltc}}
\newcommand{\forcing}[1]{\fbox{forcing#1}}

% Colours for changes
\definecolor{v1color}{rgb}{0.01,0.28,1.0} % Blue
\definecolor{v2color}{rgb}{0.47,0.27,0.23} % Bole
\newcommand{\vone}[1]{{\color{v1color}#1}}
\newcommand{\vtwo}[1]{{\color{v2color}#1}}

\begin{document}

\title{COSL Precision Bidding System}
\author{Sudhir}
\date{v2.1, December 2020}
\maketitle

\tableofcontents

\pagebreak

\section{Opening Bids}

All strong hands (\vone{with one exception}\footnote{Balanced 22-23
  point hands are opened \nt{2}.}) are opened \cl{1} which is forcing
for one round. In general, a major suit opening shows $5^+$-cards and
the higher ranking suit is opened with suits of equal length. A
no-trump opening shows a balanced hand with a possible 5-card minor. A
\di{1} opening normally shows a 3-card holding but could sometimes be
made with a doubleton when bidding \nt{1} or \cl{2} is unattractive
e.g., \hhand{AQJT,KQ,76,J7642} or \hhand{AKT9,AK98,432,32}.

\begin{longtable}{ p{1.5cm}p{9.5cm} }
  \hline
  \cl{1} & \vtwo{$16^+$\hcp\ (unbalanced) or $17^+$\hcp\ (balanced)
           \forcing. Hands with a powerful $6^+$-card suit that can play
           opposite a singleton and have 15\hcp\ with a void or
           singleton should also be opened with \cl{1}, e.g.,
           \hhand{AQJT98,8,KQ7,QJT}.}\hyperlink{1c}{\HandCuffRight} \\
  \di{1} & 11-15\hcp, at least 2 cards in \di{}, no 5-card major and
           less than 6 clubs.\hyperlink{1d}{\HandCuffRight} \\
  \he{1}/\sp{} & 11-15\hcp, $5^+$-cards in suit bid.\hyperlink{1major}{\HandCuffRight} \\
  \nt{1} & \vtwo{14-16\hcp\ in $1^{st}$/$2^{nd}$ positions and
           15-17\hcp\ in $3^{rd}$/$4^{th}$ position}, balanced. May
           have a five-card minor (even a 5-4-2-2 distribution with
           a five-card minor is acceptable with stoppers in the
           doubletons).\hyperlink{1nt}{\HandCuffRight} \\
  \cl{2} & 11-15\hcp, $6^+$-card club suit (7\sq\ hand), \vtwo{may have a
           4 or 5-card major}.\hyperlink{2c}{\HandCuffRight} \\
  \di{2} & 11-15\hcp, 5-4-3-1, 4-4-4-1 or 5-4-4-0 shape with short \di{}
           (5-card suit if present is \cl{}), \forcing{}\hyperlink{2d}{\HandCuffRight} \\
  \he{2}/\sp{} & \vtwo{5-10\hcp}, $6^+$-card suit (\sq\ of 8 when vulnerable
                 and 7 non-vulnerable). With 6 or less losers, open
                 \sp{1}/\he{}/\di{}.\hyperlink{2major}{\HandCuffRight} \\
  \vone{\nt{2}}& \vone{22-23\hcp, balanced hand, \vtwo{may have a
                 5-card major}.\hyperlink{2nt}{\HandCuffRight}} \\
  \emph{3 of suit} & Pre-emptive, $<10$\hcp, $7^+$-card suit (\sq\ of 9
                     when vulnerable and 8 non-vulnerable). Apply rule of
                     2/3/4.\hyperlink{3preempt}{\HandCuffRight} \\
  \nt{3} & ``Gambling'', solid $7^+$-card minor suit \vtwo{with nothing
           outside in $1^{st}$ or $2^{nd}$ position. In third or
           fourth position,} any side suit has
           limited strength.\hyperlink{3nt}{\HandCuffRight} \\
  \vtwo{\cl{4}/\di{}} & \vtwo{Pre-emptive, $8^+$-card suit.} \\
  \hline
\end{longtable}

\section{Responses to \cl{1}}

\hypertarget{1c}
The responses to \cl{1} can be negative, constructive or positive. All
positive responses are forcing to game unless both hands are minimum
and no fit is found. There is no response to \cl{1} above \di{4}.

\begin{longtable}{ p{1.5cm}p{9.5cm} }
  \hline
  \multicolumn{2}{l}{\emph{\underline{Negative Response}}} \\
  \di{1} & 0-7\hcp, \forcing. \vtwo{Note that with an ace and a king
           (3 controls) either in the same suit or two different
           $4^+$-card suits, a positive response should be
           made.}\hyperlink{1c1d}{\HandCuffRight} \\
  \multicolumn{2}{l}{\emph{\underline{Constructive Responses}}} \\
  \he{2}/\sp{} & 4-7\hcp, $6^+$-card suit \vone{with two of the top four honours
                 but not \emph{AK}}.\hyperlink{1c2major}{\HandCuffRight} \\
  \multicolumn{2}{l}{\emph{\underline{Positive Responses}} \forcing{ to game}} \\
  \he{1}/\sp{}, \cl{2}/\di{} & $8^+$\hcp, $5^+$-cards in suit.
                               \vtwo{There are no restrictions on suit
                               quality}.\hyperlink{1csuit}{\HandCuffRight}\\
  \nt{1} & 8-13\hcp, balanced hand with no five-card
           suit.\hyperlink{1c1nt}{\HandCuffRight} \\
  \nt{2} & $14^+$\hcp, balanced hand.
           \vtwo{\forcing{ to \nt{4}}}\hyperlink{1c2nt}{\HandCuffRight} \\
  \vtwo{\sp{3}} & \vtwo{A solid 7 or 8 card suit (with or without side
                  controls) that will play for no losers
                  opposite a singleton, e.g., \emph{AKQJxxx} or
                  \emph{AKQxxxxx}.\hyperlink{1cl3sp}{\HandCuffRight}} \\
  \multicolumn{2}{l}{\emph{\underline{Unusual Positive---Three-suited hands}}
                     \forcing{ to game}} \\
  \vtwo{\cl{3}} & \vtwo{8-11\hcp\ or $<4$ controls, 4-4-4-1 shape with a
                  black singleton (\sp{} or
                  \cl{})}.\hyperlink{unusualpositive}{\HandCuffRight} \\
  \vtwo{\di{3}} & \vtwo{8-11\hcp\ or $<4$ controls, 4-4-4-1 shape with a
                  red singleton (\he{} or
                  \di{})}.\hyperlink{unusualpositive}{\HandCuffRight} \\
  \vtwo{\he{3}, \nt{3},
  \cl{4}/\di{}} & \vtwo{$12^+$\hcp\ \underline{and} $4^+$-controls,
                  4-4-4-1 shape with singleton in suit above the one
                  bid (\sp{}, \cl{}, \di{}, \he{}
                  respectively).}\hyperlink{unusualpositive}{\HandCuffRight} \\
  \hline
\end{longtable}

\subsection{Bidding after a negative response \cl{1}--\di{1}}

\hypertarget{1c1d}
Opener rebids 1, 2 or \nt{3} with balanced hands (17-19, 20-21 or
24-26\hcp\ respectively), a non-jump in a suit with 16-21\hcp\ and a
jump in a suit with powerful hands ($22^+$\hcp). The jump rebid may be
made with a lower point-count with greater playing strength.

\vtwo{\hyperlink{ex1c1d}{Bidding Examples}}

\begin{longtable}{ p{1.5cm}p{9.5cm} }
  \hline
  \multicolumn{2}{l}{\emph{\underline{Balanced Hands}}} \\
  \nt{1} & \vtwo{17-19\hcp}, balanced hand, no 5-card
           major. Responder's rebids are: \\
         & \begin{tabular}{lp{7.5cm}}
             \emph{Pass} & 0-5\hcp, no major suit to escape to. \\
             \cl{2} & \vtwo{6-7\hcp}, \emph{Stayman}.\hyperlink{stayman}{\HandCuffRight} \\
             \di{2}/\he{} & 0-7\hcp, transfer to \he{2}/\sp{2} respectively.
                            \vtwo{Responder will invite with 7\hcp\ and pass
                            with 0-6\hcp\ unless opener
                            \emph{super-accepts}.\hyperlink{superaccept}{\HandCuffRight}} \\
             \vone{\di{4}/\he{}} & \vone{\emph{Texas}
                                   transfers to \he{4}/\sp{}
                                   respectively.} \\
           \end{tabular} \\
  \nt{2} & \vtwo{20-21\hcp, balanced hand, may have a 5-card
           major}. Responder's rebids are: \\
         & \begin{tabular}{lp{7.5cm}}
             \emph{Pass} & 0-3\hcp. \\
             \cl{3} & 4-7\hcp, \vtwo{\emph{Puppet Stayman} asking for
                      5-card majors if any}.\hyperlink{puppetstayman}{\HandCuffRight} \\
             \di{3}/\he{} & Weak, transfer to \he{3}/\sp{} respectively. \\
             \nt{3} & 4-5\hcp, sign-off. \\
             \di{4}/\he{} & \emph{Texas} transfers to \he{4}/\sp{}
                            respectively. \\
           \end{tabular} \\
  \vtwo{\nt{3}} & \vtwo{24-26\hcp, balanced hand, may have a 5-card major}.
                  Responder's rebids are: \\
         & \begin{tabular}{p{1.5cm}p{6.5cm}}
             \emph{Pass} & 0-4\hcp, balanced. \\
             \emph{4 of suit} & 5-7\hcp, $5^+$-cards. Opener bids one
                                above suit (\di{4}/\he{4}/\sp{} or \nt{4}) to show
                                fit and start \emph{Roman Key-Card
                                Blackwood}.\hyperlink{blackwood}{\HandCuffRight} \\
             \nt{4} & 5-7\hcp, no 5-card suit, quantitative. \\
           \end{tabular} \\
  \multicolumn{2}{l}{\emph{\underline{Unbalanced Hands}}} \\
  \he{1}/\sp{} & \vtwo{5$^+$-card suit, non-forcing. Can be only four cards
                 if opener started with a 4-4-4-1 shape (with a
                 singleton minor, opener will rebid \he{1}).} \\
         & \begin{tabular}{lp{6.7cm}}
             \emph{Pass} & \vtwo{0-4\hcp, especially when balanced.} \\
             \sp{1} & 4-7\hcp, $4^+$-cards, may have three hearts. It is
                      \vtwo{important to bid the spades before supporting
                      hearts since opener may have bid \he{1} holding
                      a 4-4-4-1 distribution with both majors.}\\
             \nt{1} & 5-7\hcp, no 5-card suit, no 4-card spade after
                      \he{1}. May have 3-card support. \vtwo{This bid should
                      be avoided as far as possible to prevent the
                      strong hand from coming down.} \\
             \cl{2}/\di{} & 5-7\hcp, 5-card suit, denies 3-card support. \\
             \emph{Single raise} & \vtwo{4-5\hcp} with $3^+$-card support. \\
             \emph{Double raise} & \vtwo{6-7\hcp\ with $3^+$-card support.} \\
             \vtwo{\emph{Jump shift}} & \vtwo{\emph{Splinter} with $4^+$-card
                                        support showing slam interest. E.g., bid
                                        \cl{4} when holding \hhand{JT98,93,AJT987,5}
                                        after opener rebids \sp{1}. A splinter of \cl{3}
                                        would be slightly weaker showing a game-going hand.} \\
           \end{tabular} \\
  \cl{2}/\di{} & $5^+$-card suit, may have a 4-card major,
                 non-forcing. \vtwo{Responses have the same structure
                 as over \he{1} and \sp{1}.} \\
  \he{2}/\sp{} & \vtwo{Powerful hand with $22^+$\hcp\ and $5^+$-card
                 suit, equivalent of a Standard \cl{2} bid}.
                 Responder's rebids are: \\
         & \begin{tabular}{lp{6.7cm}}
             \vtwo{\nt{2}} & \vtwo{0-3\hcp, minimum, no support.} \\
             \vtwo{\nt{3}} & \vtwo{4-7\hcp\ maximum, spread values, no support.} \\
             \emph{Raise} & 0-3\hcp, minimum, $3^+$-card support. \\
             \emph{Jump raise} & 4-7\hcp, maximum, $3^+$-card support, no
                                 specific values in other suits. \\
             \emph{New suit} & 4-7\hcp, values in suit, does not deny
                               support for partner's suit. \forcing{} \\
           \end{tabular} \\
  \vtwo{\cl{3}/\di{}} & \vtwo{Very strong unbalanced hand with a long
                        minor and good playing strength that is too
                        strong for \cl{2}/\di{}. E.g., \hhand{A,AK,KQJT876,QJ5}
                        or \hhand{KQJ5,6,A,AKQT964}.} \forcing{} \\
  \vtwo{\he{3}/\sp{}} & \vtwo{Extremely powerful hand with a solid suit and
                        at least nine tricks. This bid sets trumps and
                        asks responder to cue bid an ace or
                        void. E.g., \hhand{AKQJT98,4,KJ3,AK} or
                        \hhand{65,AKQT7543,AKJ,} where a slam is on if
                        responder can cue bid.} \forcing{} Responder bids: \\
         & \begin{tabular}{lp{6.7cm}}
             \vtwo{\emph{Suit}} & \vtwo{First-round control---ace or void in suit.} \\
             \vtwo{\nt{3}} & \vtwo{No first round control but has a
                             king or singleton in a non-trump suit.
                             Opener rebids \cl{4} to ask which suit.} \\
             \vtwo{\emph{Game raise}} & \vtwo{Denies ace, king, singleton or void.} \\
          \end{tabular} \\
  \hline
\end{longtable}

\subsection{Bidding after a constructive response \cl{1}--\he{2}/\sp{2}}

\hypertarget{1c2major}
Since responder is showing a strictly limited hand with a long suit,
the opener needs to decide the best contract. If there is no chance
for game or slam, he should pass with a tolerance for responder's
suit.

\begin{longtable}{ p{2cm}p{9cm} }
  \hline
  \emph{Pass} & Game unlikely. \\
  \he{4}/\sp{} & Raise to game is a sign-off. \\
  \emph{New suit} & Natural, \forcing. Responder's rebids are:\\
              & \begin{tabular}{p{1.8cm}p{6.3cm}}
                  \emph{Raise} & $3^+$-card support (or \emph{Qx}). \\
                  \emph{Rebid \he{}/\sp{}} & Minimum, no support. \\
                  \vone{{\nt{3}}} & \vone{Maximum, no support.} \\
                  \emph{Cue bid under} \nt{3} & \vone{Maximum
                                                with support,}
                                                singleton or void
                                                in suit bid. \\
                \end{tabular} \\
  \nt{2} & Support for suit, \vtwo{asking for shortness} \forcing{ to
           game.} Responders's rebids are:\\
              & \begin{tabular}{p{1.8cm}p{5cm}}
                  \emph{Rebid} \he{}/\sp{} & Minimum. \\
                  \emph{New suit} & Singleton or void in bid suit. \\
                \end{tabular} \\
  \vtwo{\nt{3}} & \vtwo{\emph{AQ} or \emph{KQ} in suit.} \\
  \vone{\nt{4}} & \vone{\emph{Roman Key-Card
                  Blackwood}.\hyperlink{blackwood}{\HandCuffRight}} \\
  \hline
\end{longtable}

\subsection{Bidding after a positive No-Trump response \cl{1}--1NT/2NT}

\hypertarget{1c1nt}
Responder is showing a balanced hand with 8-13\hcp\ (\nt{1}) or
$14^+$\hcp\ (\nt{2}). \vtwo{Opener can either (a) bid his own suit at
  the 2-level showing a 5-carder, (b) bid his own suit at the 3-level
  showing a very strong hand with slam interest, (c) bid \cl{2}
  (\emph{Stayman}) or (d) raise no-trumps.}

\vtwo{\hyperlink{ex1cnt}{Bidding Examples}}

\subsubsection{\cl{1}--1NT--\cl{2}}

A \cl{2} rebid after a response of \nt{1} is \emph{Transfer Stayman}
(other rebids are shown subsequently) and responder rebids as below:

\begin{longtable}{ p{1.5cm}p{9.5cm} }
  \hline
  \di{2} & 8-10\hcp, 4 card \he{}, may have 4 card
           \sp{}. Opener's rebids are: \\
         & \begin{tabular}{lp{7cm}}
             \he{2} & \he{} fit assured; responder should then bid
                      \nt{2} with 4-3-3-3 or a second suit at 3-level. \\
             \sp{2} & 4-card \sp{}, no 4-card \he{}. \\
             \nt{2} & No 4-card major. \\
           \end{tabular} \\
  \he{2} & 8-10\hcp, 4 card \sp{}, denies 4-card \he{}. \\
         & \begin{tabular}{lp{7cm}}
             \sp{2} & Spade fit confirmed, relay; responder bids
                      \nt{2} with 4-3-3-3 else second
                      suit at 3-level. \\
             \nt{2} & No 4-card \sp{}, may have 4-card \he{}. \\
           \end{tabular} \\
  \sp{2} & 8-10\hcp, no 4 card major. Opener then bids \nt{2} to ask
           for a further description. Responder's rebids are: \\
         & \begin{tabular}{lp{6cm}}
             \cl{3}/\di{} & 4-3-3-3 with 4-card \cl{}/\di{}. \\
             \he{3} & 4-4 in minors with three hearts. \\
             \sp{3} & 4-4 in minors with three spades. \\
             \nt{3} & 5-card minor. \\
           \end{tabular} \\
  \nt{2} & 11-13\hcp, 4-3-3-3 shape with a 4-card minor. \cl{3} by opener
           is then a relay asking responder to bid his suit or \nt{3}
           with clubs. \\
  \cl{3} & 11-13\hcp, 4-4-3-2 shape with 4 clubs. Opener bids \di{3}
           as a relay and responder bids \he{3} with spades, \sp{3}
           with hearts and \nt{3} with diamonds.\\
  \di{3} & 11-13\hcp, 4-4-3-2 shape with \di{} and \he{}. \\
  \he{3} & 11-13\hcp, 4-4-3-2 shape with \he{} and \sp{}. \\
  \sp{3} & 11-13\hcp, 4-4-3-2 shape with \sp{} and \di{}. \\
  \nt{3} & 11-13\hcp, \vtwo{poor} 5-card minor suit. \cl{4} by opener is then a
           relay asking responder to bid his suit. \\
  \hline
\end{longtable}

\subsubsection{Suit bid after \cl{1}--1NT}

A bid of \di{2}/\he{}/\sp{} or \vtwo{\nt{2} (for \cl{}s)} over a
response of \nt{1} shows a $5^+$-card suit and asks responder to show
strength and fit in steps as below. \vone{This is a support asking bid
  and in this context, a minimum is 8-10\hcp\, a maximum is 11-13\hcp\
  and support is \emph{Hxx}, \emph{xxxx} or better.}

\begin{longtable}{p{1.5cm}p{9.5cm}}
  \hline
  \emph{1 step} & Minimum and no support. \\
  \emph{2 steps} & Minimum with support. \\
  \emph{3 steps} & Maximum and no support. \\
  \emph{4 steps} & Maximum with support. \\
  \hline
\end{longtable}

\vtwo{A jump rebid in a suit, e.g., \cl{1}--\nt{1}--\cl{3} shows a
  very strong hand with slam interest, sets trumps and asks for
  responder's holding in that suit.  Responses are in steps as below:}

\begin{longtable}{ p{1.5cm}p{9.5cm}}
  \hline
  \vtwo{\emph{1 step}} & \vtwo{Two or three spot cards.} \\
  \vtwo{\emph{2 steps}} & \vtwo{Doubleton honour.} \\
  \vtwo{\emph{3 steps}} & \vtwo{Tripleton honour.} \\
  \vtwo{\emph{4 steps}} & \vtwo{Two honours doubleton.} \\
  \vtwo{\emph{5 steps}} & \vtwo{Two honours tripleton.} \\
  \vtwo{\emph{6 steps}} & \vtwo{Four card support.} \\
  \hline
\end{longtable}

\subsubsection{No-Trump raise after \cl{1}--1NT}

\vtwo{
  Since a raise to \nt{2} shows a club suit (see above), there are only
  two possible raises in no-trumps---\nt{3} and \nt{4}.
}

\begin{longtable}{ p{1.5cm}p{9.5cm}}
  \hline
  \vtwo{\nt{3}} & \vtwo{Minimum balanced hand with no four-card
                  major or interest in slam.} \\
  \vtwo{\nt{4}} & \vtwo{Quantitative raise with a balanced hand and no
                  four-card major inviting slam.} \\
  \hline
\end{longtable}

\subsubsection{Bidding after \cl{1}--2NT}

\vtwo{
  \hypertarget{1c2nt}
  A \nt{2} response shows $14^+$\hcp\ and immediately puts the
  partnership in slam range.  It is, therefore, forcing to
  \nt{4}. Responses are:
}

\begin{longtable}{ p{1.5cm}p{9.5cm} }
  \hline
  \cl{3} & \emph{Baron}: asking responder to show 4-card suits upwards
           (\nt{3} after \cl{3} shows 4-3-3-3 with four cards in \cl{}). \\
  \di{3}/\he{}/\sp{} or \vtwo{\cl{4}} & $5^+$-card suit. Subsequent bidding is natural. \\
  \vtwo{\nt{3}} & \vtwo{Asks responder to clarify his point range as follows:} \\
            & \begin{tabular}{lp{6.5cm}}
                   \vtwo{\cl{4}} & \vtwo{14-15\hcp.} \\
                   \vtwo{\di{4}} & \vtwo{16-17\hcp.} \\
                   \vtwo{\he{4}} & \vtwo{18-19\hcp.} \\
                   \vtwo{\sp{4}} & \vtwo{20-21\hcp.} \\
                   \vtwo{\nt{4}} & \vtwo{$22^+$\hcp.} \\
               \end{tabular} \\
  \hline
\end{longtable}

\subsection{Bidding after a positive suit response \cl{1}--\he{1}/\sp{1}/\cl{2}/\di{2}}

\vtwo{
  \hypertarget{1csuit}
  Opener rebids no-trumps with a balanced hand. With support for
  responder's suit he has the option of initiating a series of
  \emph{asking bids}.\footnote{As a rule of thumb, asking bids should
    not be used if two of the outside suits are missing first-round
    controls. This is because once asking bids are triggered, there is
    no way to return to natural bidding.} With an unbalanced hand and no
  support for responder's suit, opener bids his suit and further bidding
  is natural. All bids short of game are forcing.

  With a 4-4-4-1 distribution, if responder bids the singleton suit, opener
  should rebid no-trumps. E.g., \nt{1} over \he{1} or \nt{2} over
  \di{2}. If responder rebids his suit, opener should rebid \nt{}.
  Partner should cater to this possibility and insist on his suit as
  trumps only with a $6^+$-card suit.

\hyperlink{ex1suit}{Bidding Examples}
}

\begin{longtable}{ p{2.5cm}p{8.5cm} }
  \hline
  \emph{New suit} & $5^+$-card suit, denies 3-card support for responder's
                    suit. Subsequent bids are natural to find the correct game
                    contract. Responses are: \\
                  & \begin{tabular}{lp{5.7cm}}
                      \emph{New suit} & 4-card suit. \\
                      \emph{Raise} & $3^+$-card support. \\
                      \emph{Rebid suit} & $6^+$-card suit, semi-solid
                                          if minor. \\
                      \emph{Lowest NT} & 5-3-3-2 shape, values in
                                         unbid suits. \\
                    \end{tabular} \\
  \nt{1} & \vtwo{17-19\hcp, balanced (\nt{2} over \cl{2}/\di{2}).
           No 5-card major, may have 3-card support (to show shape first).} \\
  \nt{2} & \vtwo{20-21\hcp, balanced (\nt{3} over \cl{2}/\di{2}).
           No 5-card major, may have 3-card support.} \\
  \emph{Single raise} & \vtwo{$\gamma$-\emph{trump asking bid}---shows
                        an extremely powerful hand with distinct slam
                        possibilities.\hyperlink{gamma}{\HandCuffRight}
                        Any further new suits bid by opener after the $\gamma$
                        response will be $\epsilon$-\emph{control
                        asking bids}.\hyperlink{epsilon}{\HandCuffRight}} \\
  \vtwo{\emph{Double raise}} & \vtwo{Minimum balanced hand, 4-card fit
                               with good controls. Avoids $\gamma$ sequences.} \\
  \vtwo{\emph{Game raise}} & \vtwo{Minimum balanced hand, 4-card fit
                             with poor controls.} \\
  \sp{3}, \cl{4}/\di{}/\he{} & \emph{Splinter} bid 4-card fit promised. \\
  \nt{4} & \emph{Roman Key-Card Blackwood}.\hyperlink{blackwood}{\HandCuffRight} \\
  \hline
\end{longtable}

\vtwo{
  In general, over a minor suit positive response, an \nt{} bid
  by opener is preferred if it is likely that the final contract will
  be \nt{3}. This will ensure that the strong hand is
  declarer. Similarly, with a 5-card minor suit, opener should
  consider rebidding \nt{} rather than his suit since, in most cases,
  \nt{3} is preferable to 5 in a minor.
}

\subsection{Bidding after a \sp{3} response}

\vtwo{
  \hypertarget{1cl3sp}
  A \sp{3} response places responder with a minimum 7-card suit headed
  by \emph{AKQ} with or without outside controls. The suit should be
  obvious on most occasions. Opener's rebids are:
}

\begin{longtable}{ p{2.5cm}p{8.5cm} }
  \hline
  \nt{3} & To play. Responder should pass unless he has many
           outside controls. \\
  \cl{4} & $\beta$-ask for \emph{outside controls}---responses
           are in the lower (0-3) scale. Any following suit bid
           that is not the trump suit is an $\epsilon$ control ask. \\
  \di{4} & When opener cannot identify the suit---asks responder
           to bid his suit. Diamonds are indicated by a \nt{4}
           response. A subsequent suit bid would be an $\epsilon$-ask. \\
  \he{4}/\sp{} & $5^+$-card suit, to play. Responder should pass
                 with 3-card support or doubleton honour. \\
  \hline
\end{longtable}

\vtwo{\hyperlink{ex1c3s}{Bidding Examples}}

\subsection{Bidding after an \emph{Unusual Positive} response}

\vtwo{
  \hypertarget{unusualpositive}
  An unusual positive response shows a 4-4-4-1 distribution. With less
  than 4 controls (typically, 8-13\hcp), the singleton is not shown
  directly---\cl{3} is bid with a black singleton and \di{3} is bid
  with a red singleton. With more than 4 controls (typically
  $12^+$\hcp), the singleton is immediately known since responder bids
  the the suit below the singleton.

  After \cl{3} or \di{3}, opener bids the next higher suit to ask
  responder to clarify where his singleton lies.  Responder bids one
  step above the relay to show the lower ranking suit and two steps
  above to show the higher ranking suit.
}

\begin{longtable}{ p{5cm}p{6cm}  }
  \multicolumn{2}{l}{\emph{Possible sequences after an unusual positive}}\\
  \hline
  \cl{1}--\cl{3}--\di{3}--\he{3} & 4-4-4-1 (\cl{} singleton) \\
  \cl{1}--\cl{3}--\di{3}--\sp{3} & 1-4-4-4 (\sp{} singleton) \\
  \cl{1}--\di{3}--\he{3}--\sp{3} & 4-4-1-4 (\di{} singleton) \\
  \cl{1}--\di{3}--\he{3}--\nt{3} & 4-1-4-4 (\he{} singleton) \\
  \hline
\end{longtable}

\vtwo{
  Once the singleton is known, a bid by opener in the singleton
  suit is $\beta$ and asks responder for the number of controls held
  (the lower scale is used after \cl{3}/\di{} and the upper scale is
  used after the stronger
  responses).\hyperlink{controlask}{\HandCuffRight}

\hyperlink{ex1c3c}{Bidding Examples}
}

\subsection{Handling intervention over \cl{1}}

If an opponent doubles \cl{1}, the responses other than \di{1} remain
the same. However, the additional possible responses of \emph{Pass}
and \emph{Redouble} are used to provide more granular
information. Over an opponent's overcall in a suit at the 1-level (or
a \nt{1} overcall), different responses are used as in the table
below.

\begin{longtable}{ p{1.5cm}p{9.5cm} }
  \hline
  \multicolumn{2}{l}{\emph{\underline{After \cl{1}--Double}}} \\
  \emph{Pass} & 0-4\hcp. \\
  \di{1} & 5-7\hcp, \forcing{} \\
  \emph{Redouble} & $8^+$\hcp, 4-4 in the major suits. \\
  \emph{Others} & Same as over \cl{1} without intervention. \\
  \multicolumn{2}{l}{\emph{\underline{After \cl{1}--\di{1}/\he{}/\sp{}}}} \\
  \emph{Pass} & 0-4\hcp. \\
  \emph{New suit} & 5-8\hcp, $5^+$-card suit. \\
  \emph{Jump in new suit} & 8-10\hcp, $6^+$-card suit. \\
  \nt{1} & 9-11\hcp\ with stopper in opponent's suit. \\
  \nt{2} & 12-14\hcp\ with one or two stoppers. \\
  \emph{Double} & 5-8\hcp\ or $9^+$\hcp, no 5-card suit,
                  \underline{takeout}. Cue bid on next round clarifies hand as
                  $9^+$\hcp. \\
  \emph{Cue Bid} & $9^+$\hcp, \forcing{ to game} \\
  \multicolumn{2}{l}{\emph{\underline{After \cl{1}--\nt{1}}}} \\
  \emph{Pass} & 0-4\hcp. \\
  \emph{New suit} & 5-8\hcp, $5^+$-card suit. \\
  \emph{Double} & $5^+$\hcp. \\
  \multicolumn{2}{l}{\emph{\vone{\underline{At 2-level}}}} \\
  \vone{\emph{Pass}} & \vone{0-4\hcp\ or $9^+$\hcp\
                       with strength in opponent's suit.} \\
  \vone{\emph{New suit}} & \vone{5-8\hcp, $5^+$-card suit.} \\
  \vone{\emph{Double}} & \vone{$5^+$\hcp.} \\
  \multicolumn{2}{l}{\emph{\vone{\underline{At 3-level}}}} \\
  \vone{\emph{Double}} & \vone{$5^+$\hcp, takeout.} \\
  \multicolumn{2}{l}{\emph{\vone{\underline{At 4-level}}}} \\
  \vone{\emph{Double}} & \vone{Weak hand.} \\
  \vone{\emph{Pass}} & \vone{\underline{\emph{\forcing{ pass,}}}
                       ask opener to take action.} \\
  \hline
\end{longtable}

\section{Responses to \di{1}}

\hypertarget{1d}
Although a \di{1} opening may be made on a doubleton
diamond, it is non-forcing and partner can pass with a weak
hand. The first priority is to find a 4-4 fit in the majors if there is one.

\begin{longtable}{ p{1.5cm}p{9.5cm}  }
  \hline
  \emph{Pass} & 0-7\hcp\ \vtwo{and usually, no 4-card major}. \\
  \multicolumn{2}{l}{\emph{\underline{Balanced Hands}}} \\
  \nt{1} & 8-10\hcp, balanced, no 4-card major. \\
  \vtwo{\nt{2}} & \vtwo{11-12\hcp, balanced, no 4-card major.
                  Could be a 4-3-3-3 shape with a weak four card major
                  and tenaces that would play better as declarer
                  in \nt{}. E.g., \hhand{AQT,T642,QT9,KT7} or
                  \hhand{9843,KJT,AQ7,JT5}} \\
  \nt{3} & \vtwo{13-15}\hcp, balanced, no 4-card major. \\
  \multicolumn{2}{l}{\emph{\underline{New Suit}}} \\
  \he{1}/\sp{} & \vtwo{Usually\footnote{Sometimes, with favourable
                 vulnerabilty and an extremely weak hand, a tactical
                 bid may be made to stop opponent's game. E.g.,
                 holding \hhand{754,J852,985,654}, you could bid
                 \he{1} planning to pass any rebid by opener.}
                 $6^+$\hcp}, $4^+$-card suit. \sp{1} denies four cards
                 in hearts but responder will bid \he{1} with both majors.
                 \forcing{} Opener rebids: \\
              & \begin{tabular}{p{1.5cm}p{7cm}}
                  \sp{1} & 4-card \sp{}, denies 4-card
                           \he{}. Responder rebids: \\
                         & \begin{tabular}{lp{5.5cm}}
                             \nt{1} & Sign-off. \\
                             \cl{2} & 4th-suit-\forcing{}
                                      (``do something clever''). \\
                             \sp{2} & 8-9\hcp, 4-card \sp{}. \\
                           \end{tabular} \\
                  \nt{1} & 11-14\hcp, balanced, denies 4-card fit. Can be 3-3-2-5 shape.
                           Responder can rebid \\
                         & \begin{tabular}{ll}
                             \cl{2} & New minor---\forcing{} \\
                             \di{2} & 8-9\hcp, 4-card \sp{} (after initial \he{1}). \\
                           \end{tabular} \\
                  \cl{2} & Unbalanced, usually 5-4 in minors and no 4-card
                           major. Responder can rebid \\
                         & \begin{tabular}{lp{5.2cm}}
                             \di{2} & Weak hand, to play. \\
                             \he{2}/\sp{} & $6^+$-card suit, to play. \\
                             \cl{3} & 8-9\hcp, keep bidding alive. \\
                             \nt{3} & To play. \\
                           \end{tabular} \\
                  \di{2} & 6+-card \di{} suit, non-forcing. \\
                  \he{2}/\sp{} & Raise of responder's suit shows 3-4 card
                                 support (if 3-card, it promises a singleton
                                 in a side suit).

                                 \vtwo{A \emph{reverse} (bidding the
                                 other major)} shows 14-15\hcp, $6^+$-\di{}
                                 and $4^+$-cards in the suit bid. \\
                  \nt{2} & 14-15\hcp, \vtwo{good diamonds and}
                           stoppers in the unbid major and clubs. \\
                  \he{3}/\sp{} & A double raise shows a maximum
                                 14-15\hcp, 4-card support \vtwo{and
                                 distributional values}. \\
                  \di{3} & 14-15\hcp, $6^+$-card \di{} suit, no 4-card major. \\
                  \cl{3} & 14-15\hcp, at least 5-5 in minors with
                           points concentrated in the two suits. \\
                \end{tabular} \\
  \cl{2} & \vtwo{$10^+$\hcp, $5^+$-card suit}, no 4-card major,
                 \forcing. Opener's rebids are: \\
              & \begin{tabular}{lp{7cm}}
                  \he{2} & 11-14\hcp, \he{} stopper, no \sp{} stopper. \\
                  \sp{2} & 11-14\hcp, \sp{} stopper, no \he{} stopper. \\
                  \nt{2} & 11-14\hcp, stopper in both majors. \\
                  \di{2}/\di{3} & No stopper in majors, genuine \di{}
                                  suit. \\
                  \he{3} & 15\hcp, \forcing{ to game} \he{} stopper, no
                           \sp{} stopper. \\
                  \sp{3} & 15\hcp, \forcing{ to game} \sp{} stopper, no
                           \he{} stopper.  \\
                  \nt{3} & 15\hcp, stopper in both majors. \\
                  \cl{3} & No stopper in majors. \\
                \end{tabular} \\
  \multicolumn{2}{l}{\emph{\underline{Inverted Raises with Diamond Support}}} \\
  \vtwo{\di{2}} & \vtwo{10-12\hcp, $5^+$-card diamond suit.
                  \forcing{ to \nt{2} or \di{3}}} \\
  \di{3} & $<10$\hcp, $5^+$-cards in \di{}, \vtwo{usually with
           a singleton or void}. \\
  \di{4} & Preemptive, \vtwo{with more shape and trumps than
           for \di{3}, i.e., 6 or 7-card diamond suit}. \\
  \multicolumn{2}{l}{\emph{\underline{Unbalanced Hands}}} \\
  \he{2}/\sp{} & Weak jump shift \vtwo{(0-7\hcp) with a long suit},
                 non-forcing. Opener's rebids are: \\
              & \begin{tabular}{p{1.1cm}p{7cm}}
                  \nt{2} & 11-12\hcp, no 4-card major. \\
                  \cl{3} & Shows both minors and asks responder to
                           choose between \cl{3}, \di{3} and \nt{3}. \\
                \end{tabular} \\
  \vtwo{\cl{3}} & \vtwo{Invitational, with a long club suit.
                  E.g., \hhand{Q4,75,97,AQJ8654}} \\
  \he{3}/\sp{}/\cl{4} & \vtwo{\emph{Splinter} bid with $5^+$-card
                        diamond support and no 4-card major.} \\
  \vtwo{\he{4}/\sp{}} & \vtwo{Single-suited hand with $7^+$-cards
                        and no slam interest.} \\
  \hline
\end{longtable}

\subsection{Intervention over \di{1}}

If opponent doubles \di{1}, the responses are:
\begin{longtable}{p{1.5cm}p{9.5cm} }
  \hline
  \emph{Pass} & 0-4\hcp\ or 9-10\hcp. \\
  \emph{New suit} & 5-8\hcp. \\
  \nt{1} & 6-8\hcp, balanced. \\
  \di{2}, \di{3} & $<$8\hcp, 4+-card support, preemptive. \\
  \emph{Redouble} & $11^+$\hcp. \\
  \hline
\end{longtable}

After an overcall by opponent up to the \sp{2} level, the responses
are:
\begin{longtable}{p{1.5cm}p{9.5cm} }
  \hline
  \emph{Double} & 8-10\hcp, $4^+$-cards in other major, negative. \\
  \emph{New suit} & 5-card suit if major, 4-card otherwise. \\
  \di{2} & 6-9\hcp, support for major. \\
  \di{3} & 10-11\hcp. \\
  \nt{1} & 8-10\hcp, stopper in opponent's suit, balanced. \\
  \nt{2} & 11-13\hcp, stopper in opponent's suit, balanced. \\
  \hline
\end{longtable}

\section{Responses to \he{1} or \sp{1}}

\hypertarget{1major}
Responses to a major opening include \emph{Bergen} raises,
\emph{Splinter} bids, \vtwo{a forcing \nt{1} and 2/1 game force}.

\begin{longtable}{ p{1.5cm}p{9.5cm}  }
  \hline
  \emph{Pass} & 0-7\hcp\ and poor support. \\
  \multicolumn{2}{l}{\emph{\underline{Direct and Bergen Raises}}} \\
  \emph{Single raise} & 7-10\hcp\ with 3-card support, constructive. \\
  \emph{Double raise} & 0-6\hcp\ with 4-card support (preemptive
                        \emph{Bergen} raise).\hyperlink{bergen}{\HandCuffRight} \\
  \cl{3} & \vtwo{7-10\hcp} with 4-card support (constructive
           \emph{Bergen} raise).\hyperlink{bergen}{\HandCuffRight} \\
  \di{3} & \vtwo{10-12\hcp} with 4-card support (limited \emph{Bergen}
           raise).\hyperlink{bergen}{\HandCuffRight} \\
  \emph{Game raise} & \vtwo{Wide variety of hands where responder is fairly sure there is
  no play for slam. E.g., raise to \sp{4} with \hhand{K65,AQ,K82,J9876}.} \\
  \multicolumn{2}{l}{\emph{\underline{One-Over-One Response}}} \\
  \sp{1} & $8^+$\hcp, $4^+$-card suit, \forcing. See below for detailed treatment. \\
  \nt{1} & 8-15\hcp, balanced hand with mild support for opener's suit
           or unbalanced hand with insufficient \hcp\ to justify a
           2-over-1 response. \forcing{} Opener's
           rebids are: \\
              & \begin{tabular}{p{2cm}p{6.5cm}}
                  \multicolumn{2}{l}{\emph{\underline{With 11-13\hcp}}} \\
                  \cl{2}/\di{}/\he{} & 11-13\hcp, 4-card suit (or
                                       3-card better minor). \\
                  \emph{Rebid of suit} & 11-13\hcp, 6-card suit. \\
                  \multicolumn{2}{l}{\emph{\underline{With a maximum 14-15\hcp}}} \\
                  \emph{Jump rebid of suit} & 14-15\hcp, 6-card solid suit.
                                              \vtwo{The jump rebid should be made on the basis of
                                              playing strength rather than \hcp.} \\
                  \nt{2} & 5-3-3-2 distribution. \\
                  \emph{Jump in new suit} & 5-5 distribution. \\
                \end{tabular} \\
              & \vtwo{A \emph{reverse}, e.g., \he{1}--\nt{1}--\sp{2}
                would show shape rather than \hcp\ (typically 14-15) and
                indicate a 6-5 distribution (or 6-4 with a very strong
                spade holding such as \emph{AKQx})} \\
  \multicolumn{2}{l}{\emph{\underline{Two-Over-One Game Force}}} \\
  \cl{2}/\di{}/\he{} & \vtwo{$12^+$\hcp}, $4^+$-card minor or 5-card heart
                       suit (over \sp{1}). \vtwo{Unless the suit is
                       rebid at the three level, all 2-over-1
                       responses are \forcing{ to game.}}
                       Opener's rebids are:\\
              & \begin{tabular}{p{2cm}p{6.5cm}}
                  \he{2} (following \sp{1}) & Natural, $4^+$-card suit. \\
                  \sp{2} (following \he{1}) & 14-15\hcp, reverse. \\
                  \emph{Rebid suit} & Minimum, not necessarily a 6-carder. \\
                  \nt{2} & 11-13\hcp, stoppers in unbid suits \\
                  \emph{Raise} & 11-13\hcp, good support, non-forcing. \\
                  \emph{Jump in new suit} & 14-15\hcp, good support, control in bid
                                            suit. \\
                  \emph{Jump rebid of suit} & 14-15\hcp, very good 6-card suit.
                                              \vtwo{The jump rebid should be made
                                              on the basis of playing strength
                                              rather than \hcp.} \\
                  \nt{3} & 15\hcp, stoppers in unbid suits. \\
                \end{tabular} \\
  \nt{2} & \vtwo{$12^+$\hcp}, 4-card support, \forcing{ to game,} \emph{Jacoby
           \nt{2}}. See \ref{jacoby2nt} for responses. \\
  \multicolumn{2}{l}{\emph{\underline{Other Bids at 3-level and Above}}} \\
  \emph{Double jump shift} & \emph{Splinter} bid, singleton or void in bid suit,
                             $4^+$-card support. \forcing{ to game} \\
  \nt{3} & 14-15\hcp, usually 4-card support for opener's major,
           responder lacks a void or singleton, minimum 4 controls. \\
  \hline
\end{longtable}

A response of \sp{1} over \he{1} shows $8^+$\hcp\ with a 4-card suit
and is forcing for one round. Opener's rebids are:

\begin{longtable}{ p{1.5cm}p{9.5cm}  }
  \hline
  \nt{1} & 11-13\hcp, minimum. \\
  \cl{2}/\di{} & 11-15\hcp, $4^+$-card suit, non-forcing. \\
  \sp{2} & 11-13\hcp, 4-card support. \\
  \nt{2} & 14-15\hcp, balanced, stoppers in both minors. \\
  \cl{3}/\di{} & 13-15\hcp, 5-card suit. \\
  \he{3} & 13-15\hcp, $6^+$-card suit. \\
  \sp{3} & 13-15\hcp, 4-card spade support. \\
  \nt{3} & To play with running suit. \\
  \cl{4}/\di{} & \emph{Splinter} bids, \forcing{ to game} \\
  \he{4} & To play---distributional hand. \\
  \sp{4} & To play---maximum hand with 13-15\hcp\ and
           distributional values. \\
  \nt{4} & \emph{Roman Key-Card Blackwood}\hyperlink{blackwood}{\HandCuffRight}
           with agreement in \sp{}. \\
  \hline
\end{longtable}

\subsection{Intervention over a major suit opening}

If an opponent doubles, responder can bid:
\begin{longtable}{ p{1.5cm}p{9.5cm}  }
  \hline
  \emph{Pass} & 0-4\hcp\ or 9-10\hcp. \\
  \emph{Raise} & $<$8\hcp, preemptive, $3^+$-card support following the
                 \emph{Law of Total Tricks}. \\
  \emph{New suit} & 5-8\hcp, $5^+$-card suit. \\
  \nt{1} & 6-8\hcp, balanced. \\
  \nt{2} & \emph{Jacoby \nt{2}}---see \ref{jacoby2nt} for responses. \\
  \emph{Redouble} & $12^+$\hcp, support for opener's suit. \\
  \hline
\end{longtable}

If an opponent overcalls:
\begin{longtable}{ p{1.5cm}p{9.5cm}  }
  \hline
  \emph{Pass} & 0-7\hcp\ or $8^+$\hcp\ and waiting for penalty if opener reopens
                with a double. \\
  \emph{Double} & 7-9\hcp, 4-card suit in other major. \\
  \nt{1} & 8-10\hcp\ with stopper in opponent's suit. \\
  \nt{2} & 11-12\hcp\ with stopper in opponent's suit. \\
  \emph{Cue bid} & $15^+$\hcp\ with singleton or void in opponent's
                   suit. \forcing{} \\
  \hline
\end{longtable}

\section{Responses to 1NT}

\hypertarget{1nt}
A \nt{1} opening shows a \vtwo{14-16\hcp\ balanced hand (15-17\hcp\ in
  $3^{rd}$/$4^{th}$ seat)} without a 5-card major holding but could be
a 6-3-2-2 or 5-4-2-2 hand with a long minor suit in which case opener
must hold stoppers in both doubletons. Responses are as below:

\begin{longtable}{ p{1.5cm}p{9.5cm}  }
  \hline
  \emph{Pass} & 0-7\hcp, poor support. \\
  \cl{2} & \vtwo{10-11\hcp}, \emph{Stayman}.\hyperlink{stayman}{\HandCuffRight} \\
  \di{2}/\he{} & \emph{Jacoby} transfers to \he{} and \sp{}
                 respectively. See \ref{jacoby} for rebids. \\
  \sp{2} & $8^+$\hcp---Minor suit \emph{Stayman}, denies 4-card major and asks
           opener for a 4-card minor. Shows a minor 2-suiter (5-4 or
           better). \\
  \emph{3 of suit} & Good suit, \forcing{ to game} \\
  \cl{4} & \emph{Gerber} ace-asking.\hyperlink{gerber}{\HandCuffRight} \\
  \di{4}/\he{} & \emph{Texas} transfers to \he{4} and \sp{4}
                 respectively. Denies slam values. \\
  \nt{4} & 17-18\hcp, balanced, quantitative.\\
  \nt{5} & 22-23\hcp, balanced, \emph{Grand Slam Force}. \\
  \nt{6} & 19-21\hcp, balanced. \\
  \hline
\end{longtable}

\subsection{Intervention after 1NT}

If an opponent doubles \nt{1}, responder can bid:
\begin{longtable}{ p{1.5cm}p{9.5cm}  }
  \hline
  \emph{Pass} & Weak or 6-11\hcp, balanced. \\
  \emph{Redouble} & 5-card suit, asking partner to bid \cl{2} and then
                    pass or rectify. \\
  \cl{2} & 4-card suit (or good 3-card clubs). \\
  \di{2} & Short in clubs. \\
  \he{2} & Tolerance for majors (at least 4-3). \\
  \sp{2} & $12^+$\hcp, \forcing{} \\
  \emph{3 of suit} & $6^+$-card suit, invitational. \\
  \hline
\end{longtable}

After an opponent's overcall, responder has the following choices:
\begin{longtable}{p{3cm}p{8cm}  }
  \hline
  \emph{Double} & Penalty double. \\
  \emph{Suit at 2-level} & 0-6\hcp, natural and non-forcing. \\
  \nt{2} & \emph{Lebensohl}---partner must bid \cl{3}. See \ref{lebensohl}. \\
  \vone{\emph{Suit at 3-level}} & Natural, \forcing{ to game} \\
  \vone{\emph{Cue bid}} & \vone{Asks opener to bid a 4-card major
                          if he has one, denies a stopper in opponent's suit.} \\
  \nt{3} & \emph{Lebensohl}---denies stopper in opponent's suit. See \ref{lebensohl}. \\
  \hline
\end{longtable}

\section{Responses to \cl{2}}

\hypertarget{2c}
Bidding over a \cl{2} opening (showing a $6^+$-card club suit) is
largely natural except for the conventional \di{2} response that
asks opener to further describe his hand.

\begin{longtable}{ p{1.5cm}p{9.5cm} }
  \hline
  \emph{Pass} & 0-7\hcp, poor hand. \\
  \di{2} & 11+\hcp, conventional and \forcing{} (with club fit,
           may be made with only 8\hcp). Opener's rebids are: \\
              & \begin{tabular}{lp{7.5cm}}
                  \he{2}/\sp{} & 11-13\hcp, 4-card suit. \\
                  \nt{2} & 11-13\hcp, 6-3-2-2 balanced hand with stoppers in two
                           suits. Responder then bids \di{3} to enquire about stoppers and
                           opener's rebids are: \\
                               & \begin{tabular}{ll}
                                   \he{3} & \he{} and \di{} stoppers. \\
                                   \sp{3} & \sp{} and \di{} stoppers. \\
                                   \nt{3} & \he{} and \sp{} stoppers. \\
                                 \end{tabular} \\
                  \cl{3} & 11-13\hcp, 6 clubs with 1 outside stopper. Responder bids
                           \di{3} to enquire about the stopper and opener's rebids are:\\
                               & \begin{tabular}{lp{4.5cm}}
                                   \he{3} & \he{} stopper. \\
                                   \sp{3} & \sp{} stopper. \\
                                   \nt{3} & \di{} stopper.  \\
                                   \di{4} & 5-card suit (6-5 in \cl{} and \di{}) \\
                                 \end{tabular} \\
                  \nt{3} & 14-15\hcp, 6-3-3-2 balanced hand, good club suit. \\
                  \he{3}/\sp{} & 14-15\hcp, 4-card suit. \\
                \end{tabular} \\
  \he{2}/\sp{} & 8-10\hcp, non-forcing, invitational. Opener may
                 pass with a minimum and mild support. \\
  \nt{2} & 10-11\hcp, invitation to \nt{3}. Any rebid other than
                 \cl{3} (sign-off) by opener commits to game.\\
  \cl{3} & \vtwo{Pre-emptive raise based on a club fit. Not forward-going.} \\
  \di{3}/\he{}/\sp{} & \vtwo{$6^+$-card suit with game-forcing values ($12^+$\hcp)}.
                       Opener's rebids are: \\
              & \begin{tabular}{lp{7cm}}
                  \nt{3} & Less than 2-card support. \\
                  \emph{Raise} & Minimum, 3-card support. \\
                  \emph{New suit} & Maximum, 3+-card support, cue bid ace.  \\
                \end{tabular} \\
  \vtwo{\nt{3}} & \vtwo{13-15\hcp, balanced hand with stoppers in the
                  unbid suits and no interest in the majors.} \\
  \vtwo{\cl{4}} & \vtwo{Pre-emptive raise with extra club support or
                  distributional values as compared to a raise to \cl{3}.} \\
  \vtwo{\he{4}/\sp{}} & \vtwo{Very long suit with no interest in slam, to play.} \\
  \hline
\end{longtable}

\subsection{Intervention over \cl{2}}

\begin{longtable}{ p{2.5cm}p{8.5cm} }
  \hline
  \emph{Negative double} & Through \sp{3}. \\
  \emph{Redouble} & 10+\hcp. \\
  \emph{Cue bid} & 12+\hcp, singleton or void in opponent's suit. \\
  \hline
\end{longtable}

\section{Responses to \di{2}}

\hypertarget{2d}
A \di{2} opening describes a three suited hand with shortness in
diamonds and the responder can place the contract fairly easily in
most cases. The only positive response is \nt{2} which is forcing to
game.

\begin{longtable}{ p{2cm}p{9cm} }
  \hline
  \emph{Pass} & 6+ diamonds, no interest in other suits or bidding higher. \\
  \he{2}/\sp{}, \cl{3} & Natural, sign-off. \\
  \nt{2} & 11+\hcp, artificial \forcing{ to game} asks opener to
           further describe his hand. Opener's rebids are: \\
              & \begin{tabular}{lp{6.5cm}}
                  \cl{3}/\di{}  & 3=4=1=5 or 4=3=1=5 shape respectively. \\
                  \he{3} & 11-13\hcp\ and 4=4=1=4 shape. \\
                  \sp{3} & 14-15\hcp\ and 4=4=1=4 shape. \\
                  \nt{3} & 14-15\hcp, 4=4=1=4 shape and \di{}A or \di{}K. \\
                  \cl{4} & 11-13\hcp, 4=4=0=5 shape. \\
                  \di{4} & 14-15\hcp, 4=4=0=5 shape. \\
                \end{tabular} \\
              & \vtwo{A further rebid in diamonds (the singleton suit)
                by opener would be $\beta$ asking for
                controls.\hyperlink{controlask}{\HandCuffRight}} \\
  \di{3} & 6+-card \di{} suit, invitation to \nt{3}. \\
  \he{3}/\sp{} & 7-9\hcp, preemptive, $5^+$-card suit. \\
  \vone{\he{4}/\sp{}, \cl{5}} & \vone{Sign-off, to play.} \\
  \hline
\end{longtable}

\subsection{Intervention over \di{2}}

If opponents double \di{2}, responder can either \emph{Pass} if he
wants to play in diamonds or \emph{Redouble} asking partner to bid a
major.

\section{Responses to \he{2} or \sp{2}}

\hypertarget{2major}
Opener shows a 6+-card major with 8-10\hcp\ and a good suit with a
minimum suit quality of 8 when vulnerable or 7 when non-vulnerable.

\begin{longtable}{ p{2.5cm}p{8.5cm} }
  \hline
  \emph{Pass} & No game, no fit. \\
  \emph{Raise to 3 or 4} & Natural, sign-off. \\
  \emph{New suit} & Natural, \forcing. Opener's rebids are: \\
              & \begin{tabular}{ll}
                  \emph{Raise} & 3-card support. \\
                  \emph{Rebid own suit} & \forcing{} \\
                \end{tabular} \\
  \nt{2} & \forcing{} Opener is asked to bid suit where he
           has a singleton or void or rebid his own suit lacking either. \\
  \hline
\end{longtable}

\section{Responses to 2NT}

\vone{
  Opener is showing a balanced hand with 22-23\hcp\ and no 5-card
  suit---this is the only $16^+$\hcp\ hand that is not opened with
  \cl{1}. Responses are:
}

\begin{longtable}{ p{1.5cm}p{9.5cm} }
  \hline
  \vone{\emph{Pass}} & \vone{Very weak, no suit to escape to.} \\
  \vone{\di{3}/\he{}} & \vone{Weak hand that doesn't want to play in no-trumps.
                        \emph{Flint} convention---asks for a transfer to
                        \he{}/\sp{} respectively.} \\
  \vone{\di{4}/\he{}} & \vone{Transfer to \he{4}/\sp{} respectively.
                        $6^+$-card suit, to play.} \\
  \vone{\nt{3}} & \vone{To play.} \\
  \vone{\cl{3}} & \vone{\emph{Smolen Stayman}---asking for majors.
                  Game-going with slam interest. Responses are:} \\
                     & \begin{tabular}{lp{7.7cm}}
                         \vone{\di{3}} & \vone{No 4-card
                                         major---responder can now rebid:} \\
                                       & \begin{tabular}{lp{6.2cm}}
                                           \vone{\he{3}} & \vone{4-card \he{} and 5-card \sp{}
                                                           (allows for 5-3 fit).} \\
                                           \vone{\sp{3}} & \vone{4-card \sp{} and 5-card \he{}
                                                           (allows for 5-3 fit).} \\
                                           \vone{\nt{3}} & \vone{To play.} \\
                                           \vone{\cl{4}} & \vone{Transfer to diamonds.} \\
                                         \end{tabular} \\
                         \vone{\he{3}/\sp{}} & \vone{4-card major.} \\
                         \vone{\nt{3}} & \vone{4 cards in both majors.
                                         Responder bids:} \\
                                       & \begin{tabular}{lp{5.5cm}}
                                           \vone{\emph{Pass}} & \vone{To play.} \\
                                           \vone{\cl{4}/\di{}/\he{}} & \vone{Transfer to
                                                                       \di{}/\he{}/\sp{} respectively.} \\
                                         \end{tabular} \\
                       \end{tabular} \\
  \vone{\sp{3}} & \vone{Transfer to clubs with a
                  correction to diamonds. \emph{Minor suit slam try}.} \\
  \hline
\end{longtable}


\section{Responses to \cl{3}/\di{3}/\he{3}/\sp{3}}

\hypertarget{3preempt}
After a preemptive opening of three of a suit, responder's rebids are:

\begin{longtable}{p{1.5cm}p{9.5cm}}
  \hline
  \emph{Raise} & Preemptive. \\
  \emph{New suit} & Natural, \forcing{} \\
  \emph{Others} & Natural. \\
  \hline
\end{longtable}

\section{Responses to 3NT}

\hypertarget{3nt}
Responses to the gambling \nt{3} are:

\begin{longtable}{p{1.5cm}p{9.5cm}}
  \hline
  \cl{4} & Escape---asks opener to pass or bid \di{4} if that is his suit. \\
  \di{4} & Asks opener to bid his short suit(s). \\
  \nt{4} & Quantitative, asking opener if he has an $8^+$-card suit. \\
  \hline
\end{longtable}

\section{Competitive Bidding}

\subsection{Overcalls}

At the one level, overcall with a 5-card suit and 8-15\hcp. At the
two-level, overcall with at least 11-15\hcp. Responses to a suit
overcall are:

\begin{longtable}{p{1.5cm}p{9.5cm}}
  \hline
  \emph{Pass} & $<$8\hcp. \\
  \emph{Raise} & 9-10\hcp\ with 3-card support. \\
  \nt{1} & 9-10\hcp, balanced with a stopper in opponent's suit. \\
  \hline
\end{longtable}

A \nt{1} overcall should be 13-15\hcp\ with a stopper in the opponent's
suit.

A jump suit overcall is a weak preemptive bid with a $6^+$-card suit.

A cue bid is \emph{Michael's} showing a 2-suiter in the highest unbid
suit and another.

\subsection{Doubles}

A direct double over opponent's opening is either 13-15\hcp\ (takeout)
or a power double with $16^+$\hcp.

A \emph{takeout} double over opponent's minor opening bid usually
promises a 4-card suit in both majors. Over a major suit opening, it
promise a 4-card suit in the other major. A response is requested even
with a blank hand unless the other opponent bids.

\begin{longtable}{p{1.5cm}p{9.5cm}}
  \hline
  \emph{Pass} & Long and solid holding in opponent's suit \\
  \emph{1-level} & 5-8\hcp. \\
  \emph{2-level} & 7-11\hcp. \\
  \emph{3-level} & 9-13\hcp. \\
  \nt{1} & 8-10\hcp\ with stopper in opponent's suit. \\
  \emph{Jump} & $9^+$\hcp. \\
  \hline
\end{longtable}

If the doubler rebids or raises, it indicates a power hand of $16^+$\hcp.

\gap

Doubles of an opponent's overcall are \emph{negative} doubles
indicating a lack of a biddable 5-card suit and no fit with partner's
bid suit. Interest in one of the unbid suits is strongly indicated.

At the one-level, the \emph{negative} double shows 8-15\hcp; at the
two-level, it shows 11-15\hcp. Over an opponent's overcall in a major,
it promises a 4-card holding in the other major.

\gap

A double in a competitive auction is a \emph{responsive} double
indicating 3-card support in partner's suit.

\gap

A double of a slam contract is a \emph{Lightner} double requesting an
unusual lead from partner.

\section{Gadgets and Conventions}

\subsection{Stayman Convention}
\label{stayman}
\hypertarget{stayman}
The \emph{Stayman} convention is used to find a 4-4 major suit fit
after a \nt{1} opening by bidding \cl{2}. Opener responds with one of:

\begin{longtable}{p{1.5cm}p{9.5cm}}
  \hline
  \di{2} & No four card major. \\
  \he{2} & 4-card heart suit, may have 4-card spades. \\
  \sp{2} & 4-card spade suit, no 4-card heart suit. \\
  \hline
\end{longtable}

\subsubsection{Puppet Stayman}

\vtwo{
  \hypertarget{puppetstayman}
  Used over a \nt{2} bid that may be made with a 5-card major, \cl{3}
  is a conventional bid that endeavours to find a major suit
  fit. Responses are:
}

\begin{longtable}{p{1.5cm}p{9.5cm}}
  \hline
  \di{3} & No five card major but at least one four card major.
           Responder with one four card major should bid
           the major that he \emph{does not have} and if there is
           a 4-4 fit, opener will bid it. \\
  \he{3} & 5-card heart suit. \\
  \sp{3} & 5-card spade suit. \\
  \nt{3} & No 4 or 5-card major. Responder can bid \cl{4} or \di{4}
           to transfer to \he{4} or \sp{4} respectively when he has
           a six-card major. \\
  \hline
\end{longtable}

\subsection{Roman Key-Card Blackwood}
\label{blackwood}
\hypertarget{blackwood}
A \emph{Roman Key-Card Blackwood} bid of \nt{4} is used to enquire
about the number of key cards (any ace or the trump suit king) in
partner's hand.

\emph{RKCB} should not be used when you have a void or two fast
losers.

The responses to the bid of \nt{4} are in steps and differ depending
on whether opponents have bid over \nt{4}. Note that the 3rd and 4th
steps show only 2 key cards if opponents have intervened since the 5
key card response is shown by step 2 in such cases.

\begin{longtable}{p{3.2cm}|p{1cm}p{3cm}p{3.2cm}}
  \emph{Holding} & \emph{Silent} & \emph{Double (R0P1)} & \emph{Overcall (D0P1)} \\
  \hline
  \emph{1/4 key cards} & \cl{5} & \emph{Redouble} & \emph{Double} \\
  \emph{0/3 key cards} & \di{5} & \emph{Pass} (0/3/5 key cards) & \emph{Pass} (0/3/5 key cards) \\
  \emph{2/5 key cards
  without queen of trumps} & \he{5} & \cl{5} (two key cards) & \emph{Cheapest suit} (two key cards)
  \\
  \emph{2/5 key cards
  and queen of trumps} & \sp{5} & \di{5} (two key cards) & \emph{Second-cheapest suit} (two key cards)\\
  
  \hline
\end{longtable}

When holding a void, after a trump suit is agreed, jumping to another
suit at the 4 or 5 level in the void suit initiates a \emph{key card
  exclusion} asking bid. Partner shows his key cards \emph{excluding}
any in the void suit in steps.

\begin{longtable}{p{1.5cm}p{9.5cm}}
  \hline
  \emph{1 step} & 1 or 4 key cards. \\
  \emph{2 steps} & 0 or 3 key cards. \\
  \emph{3 steps} & 2 or 5 key cards without trump Q. \\
  \emph{4 steps} & 2 or 5 key cards with trump Q. \\
  \hline
\end{longtable}

\subsection{Gerber Ace-asking Convention}
\label{gerber}
\hypertarget{gerber}
An immediate response of \cl{4} to any no-trump bid (or overcall) is
the \emph{Gerber} ace-asking convention. A jump rebid of \cl{4} in
response to a natural no-trump bid is \emph{Gerber} as also when a
trump suit has not been identified and no-trumps has been
rebid. Gerber should not be used holding a void.

Opener shows number of aces in steps as follows:

\begin{longtable}{p{1.5cm}p{9.5cm}}
  \hline
  \di{4} & Zero or four aces. \\
  \he{4} & One ace. \\
  \sp{4} & Two aces. \\
  \nt{4} & Three aces. \\
  \hline
\end{longtable}

\subsection{$\beta$ Control Asking Bid}

\hypertarget{controlask}
A $\beta$ control asking bid can occur either after a \nt{1} rebid by
the \cl{1} opener over a positive suit response or by cue bidding a
singleton suit after a positive response of \cl{3}, \di{3}, \he{3},
\sp{3}, \cl{4} or \di{4} over \cl{1}.

The number of controls held (\emph{A=2}, \emph{K=1}) are shown in
steps as below:

\begin{longtable}{ p{1.5cm}p{9.5cm} }
  \hline
  \emph{1 step} & 0-2 controls.
                  A relay bid by opener in the cheapest suit over the
                  1-step response will then ask for clarification and
                  again the responses are in steps: \\
                & \begin{tabular}{ll}
                    \emph{1 step} & No controls. \\
                    \emph{2 steps} & 1 control. \\
                    \emph{3 steps} & 2 controls. \\  
                  \end{tabular} \\
  \emph{2 steps} & 3 controls. \\
  \emph{3 steps} & 4 controls. \\
  \emph{4 steps} & 5 controls. \\
  \hline
\end{longtable}

\vtwo{
  When responder has already shown controls as less than four or
  at least four, e.g., after an unusual positive, a modified scale of
  responses is used as follows:
}

\begin{longtable}{ p{3cm}p{4cm}p{4cm}}
  \vtwo{\emph{Known to have}}  & \vtwo{\emph{0-3 controls}} & \vtwo{\emph{$4^+$ controls}} \\
  \hline
  \vtwo{\emph{1 step}} & \vtwo{No controls.} & \vtwo{4 controls.} \\
  \vtwo{\emph{2 steps}} & \vtwo{1 control.} & \vtwo{5 controls.} \\
  \vtwo{\emph{3 steps}} & \vtwo{2 controls.} & \vtwo{6 controls.} \\
  \vtwo{\emph{4 steps}} & \vtwo{3 controls.} & \vtwo{7 controls.} \\
  \vtwo{\emph{5 steps}} &            & \vtwo{8 controls.} \\
  \hline
\end{longtable}

\subsection{$\gamma$ Trump Asking Bid}

\vtwo{
  \hypertarget{gamma}  
  The $\gamma$-asking bid is used to find out the quality of
  responder's suit. It is typically used to find out whether the suit
  will provide enough tricks in no-trumps or as a trump suit.
}

\begin{longtable}{p{1.5cm}p{9.5cm}}
  \multicolumn{2}{l}{\vtwo{\emph{Responses to a $\gamma$-suit asking bid}}} \\
  \hline
  \vtwo{\emph{1 step}} & \vtwo{No top honour, $5^+$-card suit.} \\
  \vtwo{\emph{2 steps}} & \vtwo{5-card suit, 1 honour.} \\
  \vtwo{\emph{3 steps}} & \vtwo{5-card suit, 2 honours.} \\
  \vtwo{\emph{4 steps}} & \vtwo{6-card suit, 1 honour.} \\
  \vtwo{\emph{5 steps}} & \vtwo{6-card suit, 2 honours.} \\
  \vtwo{\emph{6 steps}} & \vtwo{AKQ of suit, $5^+$-card suit.} \\
  \hline
\end{longtable}

\vtwo{The $\gamma$ bid may be repeated to get clarification on the
  first response.}

\begin{longtable}{p{3cm}p{8cm}}
  \vtwo{\emph{Honours shown}} & \vtwo{\emph{Responses to repeat $\gamma$-ask}} \\
  \hline
  \vtwo{\emph{Zero} or \emph{AKQ}} & \begin{tabular}{lp{6cm}}
                             \vtwo{1 step} & \vtwo{$7^+$-card suit} \\
                             \vtwo{2 steps} & \vtwo{6-card suit} \\
                             \vtwo{3 steps} & \vtwo{5-card suit} \\
                          \end{tabular} \\
  \hline
  \vtwo{\emph{One}} & \begin{tabular}{lp{6cm}}
                             \vtwo{1 step} & \vtwo{\emph{Ace}} \\
                             \vtwo{2 steps} & \vtwo{\emph{King}} \\
                             \vtwo{3 steps} & \vtwo{\emph{Queen}} \\
                          \end{tabular} \\
  \hline
  \vtwo{\emph{Two}} & \begin{tabular}{lp{6cm}}
                             \vtwo{1 step} & \vtwo{\emph{AK}} \\
                             \vtwo{2 steps} & \vtwo{\emph{AQ}} \\
                             \vtwo{3 steps} & \vtwo{\emph{KQ}} \\
                          \end{tabular} \\
  \hline
\end{longtable}

\subsection{$\epsilon$ Control Asking Bid}

\vtwo{
  \hypertarget{epsilon}
  The $\epsilon$-asking bid is used to find out what controls the
  responder holds in a specific suit. It follows a $\gamma$ asking bid
  and terminates only when trumps or no-trumps is bid, i.e., any other
  suit bid is an $\epsilon$-ask in that suit.  Responses are in
  steps:
}

\begin{longtable}{p{1.5cm}p{9.5cm}}
  \multicolumn{2}{l}{\vtwo{\emph{Responses to a $epsilon$-suit asking bid}}} \\
  \hline
  \vtwo{\emph{1 step}} & \vtwo{No control---\emph{Jxx} or worse.} \\
  \vtwo{\emph{2 steps}} & \vtwo{Third round control---\emph{Q} or doubleton.} \\
  \vtwo{\emph{3 steps}} & \vtwo{Second round control---\emph{K} or singleton.} \\
  \vtwo{\emph{4 steps}} & \vtwo{First round control---\emph{A} or void.} \\
  \vtwo{\emph{5 steps}} & \vtwo{\emph{AK} or \emph{AQ}.} \\
  \hline
\end{longtable}

\vtwo{
  The $\epsilon$-ask can be repeated if it is important to know
  whether the control is based on shortness or strength. The response
  is again in steps---the first step showing that the previous
  response was based on \emph{shortness} and the second step showing
  \emph{strength}.

  If the first $\epsilon$ bid is at the level of \cl{5} or higher, a
  compressed scale of responses is used where the first step shows no
  control, second step shows second-round control and the third step
  shows first-round control.
}

\subsection{Jacoby Transfers}
\label{jacoby}
\hypertarget{jacoby}
After a \nt{1} opening, responder bids \di{2} with a 5-card or better
heart suit and \he{2} with spades. Opener will bid \he{2} or \sp{2} so
that the strong hand becomes declarer. Responder's rebids are:

\begin{longtable}{p{2.5cm}p{8.5cm}}
  \hline
  \emph{Pass} & A weak hand with $5^+$-card \he{} or \sp{}. \\
  \sp{2} & Invitational with 5-5 in the majors after
           \nt{1}--\di{2}--\he{2}. \\
  \nt{2} & Balanced or semi-balanced hand with 5-card \he{} or
           \sp{}. Invitational---partner can pass or sign-off in
           3 of a major or bid \nt{3}. \\
  \cl{3}/\di{} & 4-card suit in addition to 5-card major,
                 \forcing{ to game} \\
  \emph{Raise} & 6-card suit, invitational. \\
  \he{3} & (After \sp{2}) 5-5 in the majors with slam
           interest. Stronger than an immediate jump to \he{4}. \\
  \sp{3} & (After \he{2}) Singleton or void with slam interest. \\
  \nt{3} & Balanced or semi-balanced hand. Partner can pass or correct
           to 4 of major. \\
  \he{4} & (After \sp{2}) 5-5 in majors with no slam interest. Partner
           can pass or correct to \sp{4}. \\
  \emph{Double raise} & $6^+$-card major, sign-off. \\
  \nt{4} & Quantitative, inviting slam in major or no-trumps. \\
  \hline
\end{longtable}

\vtwo{
\hypertarget{superaccept}
Opener could \emph{super-accept} the transfer with a $4^+$-card
holding in the transfer suit. In this case, he can break the transfer
and show any doubletons.  For example, after \nt{1}--\he{2} (transfer
to \sp{2}), opener with a 4-card spade suit could bid:
}

\begin{longtable}{p{2.5cm}p{8.5cm}}
  \hline
  \vtwo{\nt{2}} & \vtwo{4=3=3=3 shape.} \\
  \vtwo{\cl{3}} & \vtwo{4=x=x=2 (doubleton club).} \\
  \vtwo{\di{3}} & \vtwo{4=x=2=x (doubleton diamond).} \\
  \vtwo{\he{3}} & \vtwo{4=2=x=x (doubleton heart).} \\
  \vtwo{\sp{3}} & \vtwo{4 spades, any other distribution.} \\
  \hline
\end{longtable}

\subsection{Jacoby 2NT}
\label{jacoby2nt}
\hypertarget{jacoby2nt}
A \nt{2} response over an opening of \he{1} or \sp{1} is conventional
and shows $13^+$\hcp\ with $4^+$-card support of partner's suit. It is
\forcing{ to game.} Opener's rebids are:

\begin{longtable}{p{3.5cm}p{7.5cm}}
  \hline
  \emph{New suit} & Singleton or void in suit bid. \\
  \emph{Rebid of suit at 3-level} & Maximum strength hand. \\
  \emph{Jump shift} & Good $5^+$-card side suit. \\
  \emph{Game in original suit} & Minimum opening, sign-off. \\
  \nt{3} & 12-13 \hcp, medium strength hand. \\
  \hline
\end{longtable}


\subsection{Bergen Raises}

\hypertarget{bergen}
After a \he{1} or \sp{1} opening, responses of \cl{3}, \di{3}, \he{3}
and \sp{3} show different types of 4-card support. The mnemonic
\emph{CLAP} (Constructive, Limited and Preemptive) helps to remember
the order of the bids.

\begin{longtable}{p{1.5cm}p{9.5cm}}
  \hline
  \emph{\he{1}--\cl{3}} & Constructive, 7-10\hcp, 4-card \he{}. \\
  \emph{\he{1}--\di{3}} & Limited, 10-12\hcp, 4-card \he{}. \\
  \emph{\he{1}--\he{3}} & Preemptive, 0-6\hcp, 4-card \he{}. \\
  \emph{\sp{1}--\cl{3}} & Constructive, 7-10\hcp, 4-card \sp{}. \\
  \emph{\sp{1}--\di{3}} & Limited, 10-12\hcp, 4-card \sp{}. \\
  \emph{\sp{1}--\he{3}} & \emph{Spare bid}---used to show a strong
                          $12^+$\hcp\ hand with 4-card support and an
                          undisclosed singleton/void. \\
  \emph{\sp{1}--\sp{3}} & Preemptive, 0-6\hcp, 4-card \sp{}. \\
  \hline
\end{longtable}

\subsection{Lebensohl}
\label{lebensohl}
\hypertarget{lebensohl}
The \emph{Lebensohl} convention is used by responder after an opponent
overcalls partner's opening \nt{1} bid in order to compete further
without necessarily committing to game. It is initiated after the
right-hand opponent makes a suit overcall at the two-level.

\begin{longtable}{p{2.5cm}p{8.5cm}}
  \hline
  \emph{2 in new suit} & Natural and non-forcing. \\
  \nt{2} & A puppet bid requiring opener to bid \cl{3}. After opener's
           forced \cl{3} bid, \\
                       & \begin{tabular}{p{8cm}}
                           --- 3 of a lower ranking suit than
                           overcaller's is natural, to play. \\
                           --- 3 of a higher ranking suit than
                           overcaller's is natural and invitational. \\
                           --- 3 of the opponent's suit is artificial
                           asking opener to bid a 4-card major and
                           showing a stopper in opponent's suit. \\
                           --- 3NT is natural, to play, and shows a
                           stopper in the opponent's suit. \\
                         \end{tabular} \\
  \emph{3 in new suit} & Natural, \forcing{ to game} \\
  \emph{Cue bid} & Artificial---asks opener to bid a 4-card major and
                   denies a stopper in opponent's suit. \\
  \nt{3} & Natural, to play, and denies a stopper in opponent's
           suit. \\
  \hline
\end{longtable}

\section{Miscellaneous}

\emph{High-card Points} (\hcp) are assigned as follows---Ace=4,
King=3, Queen=2 and Jack=1.  Once a trump-fit has been found,
distribution points can be assigned---Void=3, Singleton=2,
Doubleton=1.

Singleton honours should be counted only once (either \hcp\ or
shortness).

\gap

\emph{Suit Quality} (\sq) is calculated as suit length plus number of
honours in the suit. The Jack or Ten should be counted only if a
higher ranking honour is held. For example, a holding of K-J-9-5-4
would have 7\sq\ but J-10-9-5-4 would have 5\sq.

For an overcall, the \sq\ should equal or exceed the number of tricks
bid (e.g., \sq\ of 8 for a 2-level overcall).

When preempting, the \sq\ should equal the level of preempt when
vulnerable and can be one less when non-vulnerable.

\gap

The \emph{Losing Trick Count} (\ltc) is used only once a trump suit
has been established. Count losers only in the top three cards of the
suit holding---there are never more than 3 losers in a suit. With
three or more cards, A/K/Q are not losers but any lower card is a
loser. With two cards, only A or K are not losers.

Add your and partner's loser count and subtract from 24 to estimate
the number of tricks that can be won.  You can estimate your partner's
\ltc\ as follows:

\begin{tabular}{rp{3cm}}
  \emph{\hcp{}} & \emph{Expected Losers} \\
  \hline
  \emph{7-9} & 8-9 losers (9) \\
  \emph{10-12} & 7-8 losers (8) \\
  \emph{13-15} & 6-7 losers (7) \\
  \emph{16-18} & 5-6 losers (6) \\
  \emph{19-21} & 4-5 losers (5) \\
  \emph{$22^+$} & 4 losers or less \\
  \hline
\end{tabular}

\section{Bidding Examples}
\setboolean{betweencards}{true} % spaces between cards in hand diagrams
\setboolean{leadingspace}{true}

\subsection{Negative Response over \cl{1}}

\hypertarget{ex1c1d}
After a \di{1} response, there is no temptation to get too high on
misfitting hands. For example,

\vhand[West]{4,AK954,AJ4,KQT9}\vhand[East]{KJT753,62,753,54}
\auction{1c,p,1d,p,1h,p,1s(1),p,2c(2),p,2s(3)}\\ (1) 4-7\hcp, 4+-card
suit. \\ (2) Shows minimum with second 4-card suit (implies 5
hearts). \\ (3) 6-card suit, no fit.

Opener shows discipline and passes recognising misfit and no chance
for game.

\gap
A \di{1} response does not rule out game. With a 2-suited hand,
it is easy to find a game contract when the fit is in the second bid
suit.

\vhand[West]{AK752,AQT43,A5,2}\vhand[East]{4,K852,9642,J754}
\auction{1c,p,1d,p,1s,p,1n(1),p,2h,p,3h,p,4h} \\ (1) As he has already
limited his hand, East is not afraid to improve the contract. After
that, all goes smoothly.

Suppose the lead against \he{4} is a low diamond. The best technique
for declarer is to win with the Ace, cash \sp{}A and ruff a spade with
a low trump. Then he plays a club to establish communication between
the two hands.

The opponents will probably continue diamonds. West ruffs the third
round and leads another low spade, ruffing with the \he{}8. If, at
worst, spades are 5-2 and South overruffs, declarer retains the
possibility of ruffing the other spade loser with the \he{}K. The
contract will fail only against very unlucky distribution.

\gap
With a powerful hand, opener would jump rebid his suit and responder
would know there is a game or slam on if he is in the upper range. For
example,

\vhand[West]{AKJ8753,A,K72,AQ}\vhand[East]{642,J73,AJ54,865}
\auction{1c,p,1d,p,2s(1),p,3d(2),p,3s(3),p,4s,p,4n,p,5c(4),p,6s} \\
(1) 22+\hcp, 5-card suit. \\ (2) 4-7\hcp, showing side-suit before
showing fit in spades. \\ (3) After the new suit bid at the 3-level,
opener knows he will not be left in \sp{3}. \\ (4) One key card.

After a heart lead and assuming trumps don't break worse than 2-1, the
contract can be made without the club finesse by playing \emph{A} and
\emph{K} of diamonds followed by a low diamond towards the
\emph{J}. This works whenever diamonds break 3-3, \di{}\emph{Q} is
held by North or is a doubleton with South.

\gap
With a balanced hand, opener will rebid \nt{}. For example,

\vhand[West]{K63,KJT,A862,AK3}\vhand[East]{AJ742,754,J95,T4}
\auction{1c,p,1d,p,1n(1),p,2h(2),p,2s,p,2n(3),p,3n(4)} \\
(1) 17-19\hcp, balanced. \\ (2) Weak transfer to \sp{2}. \\ (3)
Balanced hand, invitational. \\ (4) With 18\hcp\ and three cards in
spades, East tries for game in no trumps.

North leads the \he{}3, South plays the Queen and West the King. After
this favourable opening, West can afford to make a safety play in
spades. He plays King and another, North following suit with low
cards. To make absolutely sure of four tricks, even when North holds
Q10xx, declarer ducks in dummy. He makes game with four spades, two
hearts and three top tricks in the minors.

\gap
Even with a weak two-suiter, Precision enables finding slams with
relatively low point counts following a negative response. For
example,

\vhand[West]{A854,AK943,AKJ8,}\vhand[East]{6,J87652,Q976,64}
\auction{1c,p,1d,p,1h,p,3s(1),p,4c,p,4h(2),p,6h} \\ (1) Splinter
showing 4-card or better support in hearts and a singleton or void
in spades. \\ (2) Responder could conceivably also bid \he{5}
with the 6-card suit.

With a combined total of 22 points, although 13 tricks are available
if the opening lead is not ruffed, most pairs will probably stop in
\he{6} after the splinter bid using a sequence similar to the one
above.

\subsection{Positive Response in a Suit after \cl{1}}

\hypertarget{ex1csuit}
Using Precision, game is always reached after a positive response to a
\cl{1} opening.  The partnership will have a minimum of 24 points if
opener is unbalanced (16 vs 8) or 25 points if he is balanced (17 vs
8). This works well in practice, for example:

\vhand[West]{AKJ86,64,KQT9,K6}\vhand[East]{Q92,875,A543,Q94}
\auction{1c,p,1n,p,2s,p,3s,p,4s} \\
A dull 16\hcp\ \cl{1} opening against an equally dull 8\hcp\ but still
\sp{4} is an odds-on favourite to make.

\gap
Game contracts can be reached on smaller point counts if there are
distributional features. For example,

\vhand[West]{AKJT96,A,QJT9,65}\vhand[East]{Q82,965,K743,743}
\auction{1c(1),p,1d,p,1s,p,3s(2),p,4s} \\ (1) A strong 15\hcp\ with a
good suit should be opened with \cl{1}.\\ (2) As he has already
limited his hand, East is not shy in raising partner's suit with 5\hcp
and inviting game.

As compared to the previous deal, this is a 15\hcp\ vs 5\hcp\ hand
that may be passed out after \sp{1} in standard systems. However, the
game contract is virtually lay-down.

\gap
With a balanced hand, opener will rebid \nt{} over a positive suit
response.  Even with 3-card support for partner's suit, it is
sometimes correct to first bid \nt{} and only later raise partner's
suit. For example, with \hhand{AJT,KQT9,QJ4,KJ7}, if responder bids
\sp{1}, it is correct to rebid \nt{1} showing a balanced minimum
before raising spades. However, with a slightly different hand such as
\hhand{AJT7,KQT,QJ4,KJ7}, the rebid could be \sp{3} or \sp{4} showing
a minimum hand, probably balanced, with 4-card support.

Alternative sequences showing support have slightly different
meanings.  For example, whereas the sequence \cl{1}--\sp{1}--\sp{4}
would show a minimum hand with poor controls, the sequence
\cl{1}--\sp{1}--\nt{1}--any--\sp{4} would show a balanced minimum with
good controls.

The intermediate \nt{1} rebid can also be made when you want to find
out if responder has a distributional hand. For example, when holding
\hhand{AK87,A753,KQ4,A6}, after partner's positive response of \sp{1},
rebid \nt{1} and if partner rebids \cl{2} (four-card suit), you may
have very good play for \sp{7} if partner is holding something like
\hhand{QJ543,82,A8,K954}.  However, you need to know about the four
clubs first.

\gap
With a distributional hand where you have strong support for partner's
suit and the only question for slam is whether his suit has good
quality, \emph{asking bids} ($\gamma$ and $\epsilon$) can be used to
good effect. For example,

\vhand[West]{QJ632,5,AKQ8,KJ9}\vhand[East]{AKT54,987,T4,A53}
\auction{1c,p,1s,p,2s(1),p,3h(2),p,4c(3),p,4n(4),p,5h(3),p,5s(5),p,6s} \\
(1) $\gamma$ trump asking bid (possible slam if trumps are strong). \\
(2) 2 honours, 5-card suit. \\ (3) $\epsilon$ control asking bid. \\
(4) Ace or void. \\ (5) No control.

With a sure loser in hearts, opener stops in the small slam.

\subsection{Positive No-Trump Response to \cl{1}}

\hypertarget{ex1cnt}
With both majors, it is sometimes correct to use \emph{Stayman} even
when holding a 5-card suit.  For example, holding
\hhand{AKQ64,KQ87,A5,95}, it is better to bid \cl{2} over a \nt{1}
response rather than bidding \sp{2}. If responder holds something like
\hhand{JT2,AJ94,543,Q43}, he will certainly raise spades after \sp{2}
and the 4-4 heart fit will not be discovered. In this case although
there are 10 tricks in spades and 11 in hearts (given normal breaks),
sometimes the difference may be 10 tricks in the 4-4 fit versus 9 in
the 5-3 fit.

Similarly, with \hhand{3,AKQ7,AQ,KQJT98}, bid \emph{Stayman}. If
partner bids \di{2} (four hearts), you will bid \he{2} and later ask
for aces. If partner has two aces, you can confidently bid the grand
slam or the small slam if he shows only one ace. If partner holds
something like \hhand{AQ6,JT86,J76,543}, \he{6} from the strong side
is best, while \cl{6} will depend on the diamond finesse.

\subsection{\sp{3} response to \cl{1}}

\hypertarget{ex1c3s}
Opener can place the contract fairly easily given responder's solid
suit and use asking bids to decide if a slam is on. For example,

\vhand[West]{4,AT987,A4,AKQ87}\vhand[East]{AKQJ987,3,K7,T96}
\auction{1c,p,3s(1),p,4c(2),p,4h(3),p,7s(4)} \\
(1) Solid suit. Opener can tell that it is spades by looking
at his own hand. \\ (2) $\beta$-ask for outside controls. \\
(3) One outside control (\di{} or \he{} king). \\ (4) 13 tricks
are on top.

\subsection{Unusual Positive Response to \cl{1}}

\hypertarget{ex1c3c}
If responder bids an \emph{unusual positive}, slam is most likely on
the cards and with the right cards, grand slams can be reached on very
low point counts.

\vhand[West]{AKQ876,976,AK43,}\vhand[East]{J543,A,T987,AK43}
\auction{1c,p,4d(1),p,4h(2),p,4n(3),p,7s(4)} \\ (1) 4-1-4-4,
$4^+$-controls, $12^+$\hcp \\ (2) $\beta$ asking for controls \\
(3) 5 controls (2 steps) \\ (4) Partner must have two aces and
\cl{}\emph{K}, 13 tricks are visible.

Barring horrendous breaks and a ruff on the opening lead, this
27-point grand slam is lay-down.

\end{document}
