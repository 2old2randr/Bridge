\documentclass[a4paper,article,oneside]{memoir}
\counterwithout{section}{chapter}
\setsecnumdepth{subsection}
\maxtocdepth{subsection}
\usepackage{microtype}
\usepackage{longtable}
\usepackage{nicefrac}
\usepackage{hyperref}
\usepackage{marginnote}
\usepackage{wasysym}
\usepackage{xcolor}
\usepackage{grbbridge}
\setboolean{spellten}{true}
\usepackage{url}
\newcommand{\gap}{\vspace{\baselineskip}}
\newcommand{\hcp}{\textsc{hcp}}
\newcommand{\sq}{\textsc{sq}}
\newcommand{\ltc}{\textsc{ltc}}
\newcommand{\orf}[1]{\textcolor{blue}{#1$\dagger$}} % One round force
\newcommand{\gf}[1]{\textcolor{red}{#1$\ddagger$}} % Game force
\newcommand{\excp}[1]{\textcolor{magenta}{#1}} % Exception

\begin{document}

\title{COSL Precision Bidding System}
\author{Sudhir Shenoy}
\date{v2.6, 5 May 2021}
\maketitle

\tableofcontents

\gap

\emph{Note}: In the text, bids that are forcing to game are in
\textcolor{red}{red} text and marked with a double dagger ($\ddagger$)
symbol. Bids that are forcing for at least one round are in
\textcolor{blue}{blue} text and marked with a dagger ($\dagger$)
symbol. Exceptions or unintuitive bids are \excp{highlighted}.

\pagebreak

\section{Opening bids}

All strong hands (with one exception\footnote{Balanced 22-23 point
  hands are opened \nt{2}.}) are opened \cl{1} which is forcing for
one round.

In general, a major suit opening shows $5^+$-cards and the higher
ranking suit is opened with suits of equal length. A no-trump opening
shows a balanced hand with a possible 5-card minor. A club suit can be
opened with \cl{2} with six or more cards in the suit.

A \di{1} opening would normally be made with at least a 3-card holding
but could sometimes be made with a doubleton when bidding \nt{1} or
\cl{2} is not attractive e.g., \hhand{AQJT,KQ,76,J7642} (only 5 clubs)
or \hhand{AKT9,AK98,32,432} (both majors, two suits unstopped). Three
suited hands with a singleton or void in diamonds are opened with an
artificial bid of \di{2}.

All opening bids from \di{1} through \di{2} are made with between 11
and 15 \emph{high card points} (\hcp). The strictly limited nature of
these openings means that, in general, partner is not forced to
respond with less than 8\hcp.

\begin{longtable}{>{\raggedright}p{1.5cm}p{9.5cm}}
  \hline
  \orf{\cl{1}} & $16^+$\hcp\ (unbalanced) or $17^+$\hcp\
                 (balanced). Hands with a powerful $6^+$-card suit
                 that can play opposite a singleton and have 15\hcp\
                 with a void or singleton should also be opened with
                 \cl{1}, e.g., \hhand{AQJT98,8,KQ7,QJT}.
                 \hyperlink{1c}{$\Rightarrow$} \\
  \di{1} & 11-15\hcp, $3^+$-cards in \di{} (\excp{could be 2}), no
           5-card major and less than 6 clubs.
           \hyperlink{1d}{$\Rightarrow$} \\
  \he{1},
  \sp{1} & 11-15\hcp, $5^+$-cards in suit bid.
           \hyperlink{1major}{$\Rightarrow$} \\
  \nt{1} & 14-16\hcp\ in $1^{st}$/$2^{nd}$ positions and
           \excp{15-17\hcp\ in $3^{rd}$/$4^{th}$ position},
           balanced. May have a five-card minor (even a 5-4-2-2
           distribution with a five-card minor is acceptable with
           stoppers in the doubletons).
           \hyperlink{1nt}{$\Rightarrow$} \\ 
  \cl{2} & 11-15\hcp, $6^+$-card club suit (7\sq\ hand), may have a 4
           or 5-card major. \hyperlink{2c}{$\Rightarrow$} \\
  \orf{\di{2}} & 11-15\hcp, 5-4-3-1, 4-4-4-1 or 5-4-4-0 shape with
                 short \di{} (5-card suit if present is
                 \cl{}). \hyperlink{2d}{$\Rightarrow$} \\
  \he{2},
  \sp{2} & 6-10\hcp, exactly six cards in suit with two of the top
           three or three of the top five honours.
           \hyperlink{2major}{$\Rightarrow$} \\
  \nt{2} & 22-23\hcp, balanced hand, no 5-card
           suit. \hyperlink{2nt}{$\Rightarrow$} \\
  \sp{3},
  \he{3},
  \di{3},
  \cl{3} & Preemptive, 0-10\hcp, $7^+$-card suit (\sq\ of 9 when
           vulnerable and 8 non-vulnerable). Apply rule of 2/3/4.
           \hyperlink{3preempt}{$\Rightarrow$} \\
  \nt{3} & ``Gambling'', solid $7^+$-card minor suit (\emph{AKQ} or
           better) with \emph{no outside ace or
           king}. \hyperlink{3nt}{$\Rightarrow$}  \\
  \gf{\cl{4}},
  \gf{\di{4}} & \emph{Namyats}---long semi-solid major suit (usually
                 $8^+$-cards) with 8 or more tricks. Used in $1^{st}$
                 or $2^{nd}$ seat only. Constructive.
                 \hyperlink{namyats}{$\Rightarrow$} \\
  \he{4},
  \sp{4} & Preemptive with $7\nicefrac{1}{2}$ tricks. \\
  \hline
\end{longtable}

As in \emph{Standard American}, a \nt{1} response over an opening of
one of a major is forcing for one round. Like the \emph{2/1 Game
  Force} system a two-over-one response is mostly forcing to game.

In most cases, the bidding in \emph{Precision} does not change with
vulnerability or seat position.

\pagebreak

\section{Responses to \cl{1}}

\hypertarget{1c} The responses to \cl{1} can be negative, constructive
or positive. All positive responses are forcing to game, i.e., holding
a \cl{1} hand, you want to be in a game contract opposite any random
8\hcp\ hand.

\begin{longtable}{>{\raggedright}p{2cm}p{9cm}}
  \multicolumn{2}{l}{\emph{\underline{Negative response}}} \\
  \orf{\di{1}} & 0-7\hcp. \excp{Note that with an ace and a king (3
                 controls) either in the same suit or two different
                 $4^+$-card suits, a positive response should be
                 made}. \hyperlink{1c1d}{$\Rightarrow$} \\
  \multicolumn{2}{l}{\emph{\underline{Constructive responses}}} \\
  \he{2},
  \sp{} & 4-7\hcp, $6^+$-card suit with two of the top four honours
          but \excp{not both \emph{A} and \emph{K} when a positive
          suit response should be made}.
          \hyperlink{1c2major}{$\Rightarrow$} \\
  \multicolumn{2}{l}{\emph{\underline{Positive responses---forcing to game}}} \\
  \gf{\he{1}},
  \gf{\sp{1}},
  \gf{\cl{2}},
  \gf{\di{2}} & $8^+$\hcp, $5^+$-cards in suit. There are no
                restrictions on suit quality.
                \hyperlink{1csuit}{$\Rightarrow$} \\
  \gf{\nt{1}} & 8-13\hcp, balanced hand with no five-card
                suit. \hyperlink{1c1nt}{$\Rightarrow$} \\
  \gf{\nt{2}} & $14^+$\hcp, balanced hand, \underline{forcing to
                \nt{4}}. \hyperlink{1c2nt}{$\Rightarrow$} \\
  \gf{\sp{3}} & A solid 7 or 8 card suit (with or without side
                controls) that will play for no losers opposite a
                singleton, e.g., \emph{AKQJxxx} or
                \emph{AKQxxxxx}. \hyperlink{1cl3sp}{$\Rightarrow$} \\
  \multicolumn{2}{l}{\emph{\underline{Unusual positive---three-suited hands}}} \\
  \gf{\cl{3}} & 8-11\hcp\ or less than four controls, 4-4-4-1 shape
                with a black singleton (spades or clubs).
                \hyperlink{unusualpositive}{$\Rightarrow$} \\
  \gf{\di{3}} & 8-11\hcp\ or less than four controls, 4-4-4-1 shape
                with a red singleton (hearts or diamonds).
                \hyperlink{unusualpositive}{$\Rightarrow$} \\
  \gf{\he{3}},
  \gf{\nt{3}},
  \gf{\cl{4}},
  \gf{\di{4}} & $12^+$\hcp\ \underline{and} $4^+$-controls, 4-4-4-1
                shape with singleton in suit above the one bid (\sp{},
                \cl{}, \di{}, \he{} respectively).
                \hyperlink{unusualpositive}{$\Rightarrow$} \\
  \hline
\end{longtable}

\subsection{Bidding after a negative response}

\hypertarget{1c1d} Opener rebids no-trumps with balanced hands
(\nt{1}: 17-19\hcp, \nt{2}: 20-21\hcp, \nt{3}: 24-26\hcp), a 5-card
suit with 16-21\hcp\ and jumps in a 5-card suit with powerful hands
($22^+$\hcp). The jump rebid in a suit may be made with a lower
point-count given greater playing strength.

\hyperlink{ex1c1d}{Examples of bidding after a negative response.$\Rightarrow$}

\begin{longtable}{ p{1.5cm}p{9.5cm}}
  \hline
  \multicolumn{2}{l}{\emph{\underline{Balanced hands}}} \\
  \nt{1} & 17-19\hcp, balanced, no 5-card major (\excp{18-19\hcp\ in
           $3^{rd}$/$4^{th}$ position}). Responder's rebids are: \\
         & \begin{tabular}{lp{7cm}}
             \emph{Pass} & 0-5\hcp, no major suit to escape to. \\
             \orf{\cl{2}} & 6-7\hcp, \emph{Stayman}.
                      \hyperlink{stayman}{$\Rightarrow$} \\
             \orf{\di{2}},
             \orf{\he{2}} & 0-7\hcp, transfer to \he{2} and \sp{2}
                      respectively. Responder will invite with 7\hcp\
                      and pass with 0-6\hcp\ unless opener
                      \emph{super-accepts}.
                      \hyperlink{superaccept}{$\Rightarrow$} \\
             \orf{\di{4}},
             \orf{\he{4}} & \emph{Texas} transfers to \he{4} and
                            \sp{4} respectively. \\
           \end{tabular} \\
  \nt{2} & 20-21\hcp, balanced, \emph{may have a 5-card
           major}. Responder's rebids are: \\
         & \begin{tabular}{lp{7cm}}
             \emph{Pass} & 0-3\hcp. \\
             \orf{\cl{3}} & 4-7\hcp, \emph{Puppet Stayman} asking for
                            5-card majors if any.
                      \hyperlink{puppetstayman}{$\Rightarrow$} \\
             \orf{\di{3}},
             \orf{\he{3}} & Weak, transfer to \he{3} and \sp{3}
                            respectively. \\
             \nt{3} & 4-5\hcp, sign-off. \\
             \orf{\di{4}},
             \orf{\he{4}} & Transfer to \he{4} and
                            \sp{4}---sign-off. \\
           \end{tabular} \\
  \nt{3} & 24-26\hcp, balanced hand, \emph{may have a 5-card
           major}. Responder's rebids are: \\
         & \begin{tabular}{p{1.5cm}p{6.5cm}}
             \emph{Pass} & 0-4\hcp, balanced. \\
             \emph{4 of
             suit} & 5-7\hcp, $5^+$-cards. Opener bids one above suit
                     (\di{4}, \he{4}, \sp{4} or \nt{4}) to show fit
                     and start \emph{Roman key-card Blackwood}.
                     \hyperlink{blackwood}{$\Rightarrow$} \\ 
             \nt{4} & 5-7\hcp, no 5-card suit, quantitative. \\
           \end{tabular} \\
  \multicolumn{2}{l}{\emph{\underline{Unbalanced hands}}} \\
  \he{1},
  \sp{1} & 5$^+$-card suit, non-forcing. Can be only four cards if
           opener started with a 4-4-4-1 shape (with a singleton
           minor, opener will rebid \he{1}). Responder's rebids are: \\ 
         & \begin{tabular}{lp{6.7cm}}
             \emph{Pass} & 0-4\hcp, especially when balanced. \\
             \sp{1} & 4-7\hcp, $4^+$-cards, may have three hearts. It
                      is important to bid the spades before supporting
                      hearts since opener may have bid \he{1} holding
                      a 4-4-4-1 distribution with both majors. \\
             \nt{1} & 5-7\hcp, no 5-card suit, no 4-card spade after
                      \he{1}. May have 3-card support. This bid should
                      be avoided as far as possible to prevent the
                      strong hand from coming down. \\
             \cl{2},
             \di{2} & 5-7\hcp, 5-card suit, denies 3-card support. \\
             \emph{Single raise} & 4-5\hcp\ with $3^+$-card support. \\
             \emph{Double raise} & 6-7\hcp\ with $3^+$-card support. \\
             \gf{\emph{Jump shift}} & \emph{Splinter} with $4^+$-card
                                      support showing slam
                                      interest. E.g., bid \cl{4} when
                                      holding \hhand{JT98,93,AJT987,5}
                                      after opener rebids \sp{1}. A
                                      splinter of \cl{3} would be
                                      slightly weaker showing a
                                      game-going hand. \\
           \end{tabular} \\
  \cl{2},
  \di{2} & $5^+$-card suit, may have a 4-card major,
           non-forcing. Responses have the same structure as over
           \he{1} and \sp{1}. \\
  \orf{\he{2}},
  \orf{\sp{2}} & Powerful hand with $22^+$\hcp\ and $5^+$-card suit,
                 equivalent of a Standard \cl{2} bid. Responder's
                 rebids are: \\
         & \begin{tabular}{lp{6.7cm}}
             \nt{2} & 0-3\hcp, minimum, no support. \\
             \nt{3} & 4-7\hcp\ maximum, spread values, no support. \\
             \emph{Raise} & 0-3\hcp, minimum, $3^+$-card support. \\
             \emph{Game raise} & 4-7\hcp, maximum, $3^+$-card support,
                                 no specific values in other suits. \\
             \orf{\emph{New suit}} & 4-7\hcp, values in suit, does not
                                     deny support for partner's
                                     suit. \\
           \end{tabular} \\
  \orf{\cl{3}},
  \orf{\di{3}} & Very strong unbalanced hand with a long minor and
                 good playing strength that is too strong for either
                 \cl{2} or \di{2}. E.g., \hhand{A,AK,KQJT876,QJ5} or
                 \hhand{KQJ5,6,A,AKQT964}. \\
  \gf{\he{3}},
  \gf{\sp{3}} & Extremely powerful hand with a solid suit and at least
                nine tricks. This bid sets trumps and asks responder
                to cue-bid an ace or void. E.g.,
                \hhand{AKQJT98,4,KJ3,AK} or \hhand{65,AKQT7543,AKJ,}
                where a slam is on if responder can cue-bid. Responder
                bids: \\
         & \begin{tabular}{lp{6.7cm}}
             \orf{\emph{New suit}} & First-round control---ace or void
                                     in suit. \\
             \orf{\nt{3}} & No first round control but has a king or
                      singleton in a non-trump suit. Opener rebids
                      \cl{4} to ask which suit. \\
             \emph{Game raise} & Denies ace, king, singleton or void. \\
           \end{tabular} \\
  \hline
\end{longtable}

\subsection{Bidding after a constructive response}

\hypertarget{1c2major} Since responder is showing a strictly limited
hand of 4-7\hcp\ with a long suit, the opener needs to decide on the best
contract. If there is no chance for game or slam, he should pass with
a tolerance for responder's suit.

\begin{longtable}{ p{2cm}p{9cm}}
  \hline
  \emph{Pass} & Game unlikely. \\
  \he{4}, \sp{4} & Sign-off, to play. \\
  \orf{\emph{New suit}} & Natural, $5^+$-card suit. Responder's rebids
                          are: \\
              & \begin{tabular}{>{\raggedright}p{1.8cm}p{6.3cm}}
                  \emph{Raise} & $3^+$-card support (or \emph{Qx}). \\
                  \emph{Rebid suit} & Minimum, no support. \\
                  \nt{3} & Maximum, no support. \\
                  \orf{\emph{New
                  suit}} & Maximum with support, singleton or void in
                           suit bid. \\
                \end{tabular} \\
  \gf{\nt{2}} & Showing support for suit and asking for
                shortness. Responders's rebids are: \\
              & \begin{tabular}{p{1.8cm}p{5cm}}
                  \emph{Rebid suit} & Minimum, no singleton or
                                      void. \\
                  \emph{New suit} & Singleton or void in bid suit. \\
                \end{tabular} \\
  \nt{3} & \emph{AQ} or \emph{KQ} in suit. \\
  \orf{\cl{4}} & \emph{Roman key-card} ask \excp{with modified responses}
                 since responder cannot have more than 2 key cards.
                 \hyperlink{blackwoodmod}{$\Rightarrow$} \\
  \hline
\end{longtable}

\subsection{Bidding after a positive no-trump response}

\hypertarget{1c1nt} Responder is showing a balanced hand with
8-13\hcp\ (\nt{1}) or $14^+$\hcp\ (\nt{2}). After \nt{1}, opener can
either (a) bid his own suit at the 2-level showing a 5-carder, (b) bid
his own suit at the 3-level showing a very strong hand with slam
interest, (c) bid \cl{2} (\emph{Transfer Stayman}) or (d) raise
no-trumps.

\gf{\emph{All bidding sequences are forcing to game}}.

\hyperlink{ex1cnt}{Examples of bidding after a positive no-trump response.$\Rightarrow$}

\subsubsection{\cl{1}--1NT--\cl{2}}

\cl{2} over a response of \nt{1} is \emph{Transfer Stayman} and
responses are as below:

\begin{longtable}{ p{1.5cm}p{9.5cm}}
  \hline
  \di{2} & 8-10\hcp, 4 card \he{}, may have 4 card \sp{}. Opener's
           rebids are: \\
         & \begin{tabular}{lp{7cm}}
             \he{2} & Relay affirming fit in hearts---responder should
                      bid \nt{2} with 4-3-3-3 or a second suit at
                      3-level. \\
             \sp{2} & 4-card \sp{}, no 4-card \he{}. \\
             \nt{2} & No 4-card major. \\
           \end{tabular} \\
  \he{2} & 8-10\hcp, 4 card \sp{}, denies 4-card \he{}.
           Opener rebids: \\
         & \begin{tabular}{lp{7cm}}
             \sp{2} & Relay affirming fit in spades---responder bids
                      \nt{2} with 4-3-3-3 else second suit at
                      3-level. \\
             \nt{2} & No 4-card \sp{}, may have 4-card \he{}. \\
           \end{tabular} \\
  \sp{2} & 8-10\hcp, no 4 card major. Opener then bids \nt{2} to ask
           for a further description. Responder's rebids are: \\
         & \begin{tabular}{lp{6cm}}
             \cl{3}
             \di{3} & 4-3-3-3 with 4-card suit. \\
             \he{3} & 4-4 in minors with three hearts. \\
             \sp{3} & 4-4 in minors with three spades. \\
             \nt{3} & 5-card minor. \\
           \end{tabular} \\
  \nt{2} & 11-13\hcp, 4-3-3-3 shape. \cl{3} by opener is then a relay
           asking responder to bid his 4-card suit (\nt{3} with
           clubs). \\
  \cl{3} & 11-13\hcp, 4-4-3-2 shape with 4 clubs. Opener bids \di{3}
           as a relay and responder bids \he{3} with spades, \sp{3}
           with hearts and \nt{3} with diamonds.\\
  \di{3} & 11-13\hcp, 4-4-3-2 shape with \di{} and \he{}. \\
  \he{3} & 11-13\hcp, 4-4-3-2 shape with \he{} and \sp{}. \\
  \sp{3} & 11-13\hcp, 4-4-3-2 shape with \sp{} and \di{}. \\
  \nt{3} & 11-13\hcp, 5-card minor suit. \cl{4} by opener is then a
           relay asking responder to bid his suit. \\
  \hline
\end{longtable}

\subsubsection{Suit bid after \cl{1}--1NT}

When opener has a possible trump suit, he bids it asking responder to
show shape and point range. With a very strong hand and a good suit,
he can jump in the suit setting trumps and asking partner for his
holding in that suit.

\begin{longtable}{>{\raggedright}p{1.5cm}p{9.5cm}}
  \hline
  \di{2},
  \he{2},
  \sp{2},
  \nt{2} & $5^+$-card suit, \emph{support-asking bid} (\nt{2} shows
           clubs). Responder's rebids are (minimum = 8-10\hcp, maximum
           = 11-13\hcp, support = \emph{Hxx}, \emph{xxxx} or
           better): \\
         & \begin{tabular}{ll}
             \emph{1 step} & Minimum and no support. \\
             \emph{2 steps} & Minimum with support. \\
             \emph{3 steps} & Maximum and no support. \\
             \emph{4 steps} & Maximum with support. \\
           \end{tabular} \\
  \cl{3},
  \di{3},
  \he{3},
  \sp{3} & Very strong hand with slam interest, sets
                trumps and asks for responder's holding in
                the suit bid. Responses are in steps: \\
         & \begin{tabular}{ll}
             \emph{1 step} & Two or three spot cards. \\
             \emph{2 steps} & Doubleton honour. \\
             \emph{3 steps} & Tripleton honour. \\
             \emph{4 steps} & Two honours doubleton. \\
             \emph{5 steps} & Two honours tripleton. \\
             \emph{6 steps} & Four card support. \\
           \end{tabular} \\
  \hline
\end{longtable}

\underline{Any suit bid} after a support-asking bid is an $\epsilon$
control-asking bid in that suit. \hyperlink{epsilon}{$\Rightarrow$}

\subsubsection{No-Trump raise after \cl{1}--1NT}

Since a raise to \nt{2} shows a club suit (see above), there are only
two possible raises in no-trumps---\nt{3} and \nt{4}.

\begin{longtable}{ p{1.5cm}p{9.5cm}}
  \hline
  \nt{3} & Minimum balanced hand with no four-card major nor interest
           in slam. \\
  \nt{4} & Quantitative raise with a balanced hand and no four-card
           major inviting slam. \\
  \hline
\end{longtable}

\subsubsection{Bidding after \cl{1}--2NT}

\hypertarget{1c2nt} A \nt{2} response shows $14^+$\hcp\ and
immediately puts the partnership in slam range.  It is, therefore,
\underline{forcing to \nt{4}}. Responses are:

\begin{longtable}{ p{2.5cm}p{8.5cm}}
  \hline
  \cl{3} & \emph{Baron}---asking responder to show 4-card suits
           upwards (\nt{3} would show 4-3-3-3 with four clubs). \\
  \di{3},
  \he{3},
  \sp{3},
  \cl{4} & $5^+$-card suit. Subsequent bidding is natural. \\
  \nt{3} & Asks responder to clarify his point range as follows: \\
         & \begin{tabular}{lp{6.5cm}}
             \cl{4} & 14-15\hcp. \\
             \di{4} & 16-17\hcp. \\
             \he{4} & 18-19\hcp. \\
             \sp{4} & 20-21\hcp. \\
             \nt{4} & $22^+$\hcp. \\
           \end{tabular} \\
  \hline
\end{longtable}

\subsection{Bidding after a positive suit response}

\hypertarget{1csuit} Opener rebids no-trumps with a balanced
hand. With support for responder's suit he has the option of
initiating a series of \emph{asking bids}.\footnote{As a rule of
  thumb, asking bids should not be used if two of the outside suits
  are missing first-round controls. This is because once asking bids
  are triggered, there is no way to return to natural bidding.} With
an unbalanced hand and no support for responder's suit, opener bids
his suit and further bidding is natural.

With a 4-4-4-1 distribution, if responder bids the singleton suit,
opener should rebid no-trumps. E.g., \nt{1} over \he{1} or \nt{2} over
\di{2}. If responder rebids his suit, opener should rebid \nt{}.
Partner should cater to this possibility and insist on his suit as
trumps only with a $6^+$-card suit.

\gf{\emph{All bids short of game are forcing}}.

\hyperlink{ex1suit}{Examples of bidding after a positive response in a suit.$\Rightarrow$}

\begin{longtable}{ p{2.5cm}p{8.5cm}}
  \hline
  \emph{New suit} & $5^+$-card suit, denies 3-card support for
                    responder's suit. Subsequent bids are natural to
                    find the correct game contract. Responder's rebids are: \\
        & \begin{tabular}{lp{5.5cm}}
            \emph{New suit} & $4^+$-card suit. \\
            \emph{Raise} & $3^+$-card support. \\
            \emph{Rebid suit} & $6^+$-card suit, semi-solid if minor. \\
            \emph{No-trumps} & 5-3-3-2 shape, values in unbid suits. \\
          \end{tabular} \\
  \nt{1} & 17-19\hcp, balanced (\nt{2} over \cl{2} or \di{2}). No
           5-card major, may have 3-card support (shows shape first).

           After \he{1} or \sp{1}, a \emph{jump rebid} of the suit
           shows a semi-solid $6^+$-card suit. A \emph{jump shift}
           would show a 5-5 two-suited limited hand---typically
           \emph{KQxxx} in both suits with nothing outside. \\
  \nt{2} & 20-21\hcp, balanced (\nt{3} over \cl{2} or \di{2}). No
           5-card major, may have 3-card support. \\
  \emph{Single raise} & $\gamma$ \emph{trump-asking bid}---shows an
                        extremely powerful hand with distinct slam
                        possibilities. \hyperlink{gamma}{$\Rightarrow$}

                        Any further new suits bid by opener after the
                        $\gamma$ response will be $\epsilon$
                        \emph{suit control-asking
                        bids}. \hyperlink{epsilon}{$\Rightarrow$} \\
  \emph{Double raise} & Minimum balanced hand, 4-card support with
                        good controls. Avoids $\gamma$ sequences.  \\
  \emph{Game raise} & Minimum balanced hand, 4-card fit with poor
                      controls. \\
  \sp{3},
  \cl{4},
  \di{4},
  \he{4} & \emph{Splinter} bid with $4^+$-card support and a singleton
           or void in the bid suit. \\
  \nt{4} & \emph{Roman key-card Blackwood}.
           \hyperlink{blackwood}{$\Rightarrow$} \\
  \hline
\end{longtable}

In general, over a minor suit positive response, a rebid in no-trumps
by opener is preferred if it is likely that the final contract will be
\nt{3}. This will ensure that the strong hand is declarer. Similarly,
with a 5-card minor suit, opener should consider rebidding no-trumps
rather than his suit since, in most cases, \nt{3} is preferable to
five of a minor.

\subsection{Bidding after a \sp{3} response}

\hypertarget{1cl3sp} A \sp{3} response places responder with a solid
$7^+$-card suit headed by \emph{AKQ} with or without outside
controls. The suit should be obvious on most occasions.

\gf{\emph{All bidding sequences are forcing to game}}.

Opener's rebids are:
\begin{longtable}{p{1cm}p{10cm}}
  \hline
  \nt{3} & To play. Responder should pass unless he has many outside
           controls. \\
  \cl{4} & $\beta$-ask for \excp{\emph{outside} controls---responses are in
           the lower (0-3) scale}. \hyperlink{beta}{$\Rightarrow$}

           Any following non-trump suit bid is an
           $\epsilon$ control ask. \hyperlink{epsilon}{$\Rightarrow$} \\
  \di{4} & Asks responder to bid his suit (diamonds are indicated by a
           \nt{4} response).

           A subsequent bid in a new suit would be an
           $\epsilon$-ask. \hyperlink{epsilon}{$\Rightarrow$} \\
  \he{4},
  \sp{4} & $5^+$-card suit, to play. Responder should pass with 3-card
           support or doubleton honour. \\
  \hline
\end{longtable}

\hyperlink{ex1c3s}{Examples of bidding after a \sp{3} response.$\Rightarrow$}

\subsection{Bidding after an \emph{unusual positive} response}

\hypertarget{unusualpositive} An \emph{unusual positive} response
shows a 4-4-4-1 distribution. With less than 4 controls (typically,
8-13\hcp), the singleton is not shown directly---\cl{3} is bid with a
black singleton and \di{3} is bid with a red singleton. With four or more
controls (typically $12^+$\hcp), the singleton is immediately known
since responder bids the the suit below the singleton.

After \cl{3} or \di{3}, opener bids the next higher suit to ask
responder to clarify where his singleton lies.  Responder bids one
step above the relay to show the lower ranking suit and two steps
above to show the higher ranking suit.

\gf{\emph{All bidding sequences are forcing to game}}.

\begin{longtable}{p{3cm}p{2cm}|p{3cm}p{2cm}}
  \multicolumn{4}{l}{\emph{Possible sequences after an unusual positive of \cl{3}/\di{3}}}\\
  \hline
  \cl{1}--\cl{3}--\di{3}--\he{3} & 4-4-4-1 (\cl{}) & \cl{1}--\di{3}--\he{3}--\sp{3} & 4-4-1-4 (\di{}) \\
  \cl{1}--\cl{3}--\di{3}--\sp{3} & 1-4-4-4 (\sp{}) & \cl{1}--\di{3}--\he{3}--\nt{3} & 4-1-4-4 (\he{}) \\
  \hline
\end{longtable}

Once the singleton is known, opener can bid the singleton suit to
initiate $\beta$ and ask for the number of controls held (\excp{the lower
scale is used after \cl{3} or \di{3} and the upper scale is used after
the stronger responses}). \hyperlink{beta}{$\Rightarrow$}

\hyperlink{ex1c3c}{Examples of bidding after an unusual positive.$\Rightarrow$}

\subsection{Handling intervention over \cl{1}}

Over a double of \cl{1}, the additional bids of \emph{Redouble} and
\emph{Pass} are used to provide more granular information in case of a negative
response. When the double is
conventional (e.g., shows both majors), the bidding is the same except
that a bid of \nt{1} would also confirm stoppers in both majors. All
other bids retain their normal meaning.

\begin{longtable}{ p{1.5cm}p{9.5cm}}
  \hline
  \multicolumn{2}{l}{\emph{\underline{After \cl{1}--(Double)}}} \\
  \emph{Pass} & 0-4\hcp. \\
  \orf{\di{1}} & 5-7\hcp, artificial. \\
  \gf{\emph{Redouble}} & $8^+$\hcp, usually balanced. \\
  \gf{\nt{1}} & Normal 8-13\hcp, but if the double shows a two-suited
                hand, shows stoppers in both implied suits. \\
  \emph{Others} & Same as over \cl{1} without intervention. \\
  \hline
\end{longtable}

After an overcall in a suit at the one-level, \emph{any suit or
  no-trump bid} is a positive response forcing to game. A \emph{trap
  pass} can be made when responder wants to double for penalties---in
this case, he will pass a re-opening double by opener.

When opponents overcall with \nt{1}, the responses are different
depending on whether the overcall is a genuine strong hand or is
conventional showing a two-suiter (the \emph{unusual no-trump}). In
the latter case, the \emph{unusual over unusual} approach repurposes
the \cl{2} and \di{2} bids to show a game-going hand with a major
suit.

\begin{longtable}{>{\raggedright}p{2.5cm}p{8.5cm}}
  \hline
  \multicolumn{2}{l}{\emph{\underline{After a one-level suit overcall \cl{1}--(\di{1}/\he{1}/\sp{1})}}} \\
  \emph{Pass} & 0-4\hcp\ or a \emph{trap pass}. \\
  \orf{\emph{Double}} & 5-8\hcp\ unbalanced or $5^+$\hcp, balanced. \\
  \gf{\emph{Suit}} & Natural, $8^+$\hcp, $5^+$-card suit. \\
  \gf{\emph{Jump to \cl{3}, \di{3}}} & Unusual positive with 4-4-4-1. \\
  \gf{\emph{Cue-bid}} & $8^+$\hcp, balanced hand with \emph{no
                        stopper} in opponent's suit. \\
  \gf{\nt{1}},
  \gf{\nt{2}} & Usual meaning and promises a stopper in opponent's
                suit. \\
  \multicolumn{2}{l}{\emph{\underline{After an artificial no-trump overcall \cl{1}--(\nt{1}) (showing minors)}}} \\
  \emph{Double} & Modest high-card points, suitable for penalising one
                  of opponent's suits, usually no 5-card major. \\
  \gf{\cl{2}},
  \gf{\di{2}} & $8^+$\hcp, $5^+$-card heart or spade suit respectively. \\
              & These two bids are the so-called \emph{unusual over
                unusual} responses in which cue-bids of known suits
                correspond to forcing bids in the unbid suits. \\
  \he{2},
  \sp{2} & Natural, non-forcing. \\
  \multicolumn{2}{l}{\emph{\underline{After a strong no-trump overcall \cl{1}--(\nt{1})}}} \\
  \emph{Pass} & 0-4\hcp. \\
  \emph{Double} & $5^+$\hcp, balanced---for penalties. \\
  \emph{Suit} & 5-8\hcp, $5^+$-card suit. \\
  \hline
\end{longtable}

Over higher level overcalls, bidding is largely natural but responses
over an \emph{unusual \nt{2}} are still \emph{unusual over unusual}.

\begin{longtable}{ p{1.5cm}p{9.5cm}}
  \hline
  \multicolumn{2}{l}{\emph{\underline{After a suit overcall at 2-level}}} \\
  \emph{Double} & 6-8\hcp, any shape. Any suit rebid by opener would
                  be a one-round force. \\
  \gf{\emph{Suit}} & Natural and forcing to game. Note that a jump to
                     \di{3} over \cl{2} would be an unusual
                     positive. \\
  \gf{\nt{2}} & 8-10 or $14^+$\hcp, with stopper in opponent's suit. \\
  \gf{\nt{3}} & 11-13\hcp\ with stopper in opponent's suit. \\
  \gf{\emph{Cue-bid}} & Values to be in game but no clear-cut
                        action---no long suit, no stopper in
                        overcaller's suit. \\
  \multicolumn{2}{l}{\emph{\underline{After an overcall of \nt{2} (unusual no-trump)}}} \\
  \emph{Double} & Penalty oriented, usually no 5-card major. \\
  \gf{\cl{3}},
  \gf{\di{3}} & $8^+$\hcp, $5^+$-card heart or spade suit respectively (\emph{unusual/unusual}). \\  
  \multicolumn{2}{l}{\emph{\underline{After an overcall at 3-level}}} \\
  \emph{Double} & Balanced hand with $8^+$\hcp. \\
  \gf{\emph{Suit}} & Positive, natural, game forcing. \\
  \nt{3} & 8-11\hcp\ with stoppers. \\
  \multicolumn{2}{l}{\emph{\underline{After an overcall at 4-level}}} \\
  \emph{Double} & Shows values---support for partner if he bids and
                  provides defence if he passes. \\
  \emph{Suit} & Natural. \\
  \hline
\end{longtable}

\subsubsection{Intervention after a negative response}

If the intervention occurs after partner's negative response of \di{1}, e.g.,
\cl{1}--(\emph{Pass})--\di{1}--(\emph{RHO doubles / bids}), opener should rebid
as follows:

\begin{longtable}{ p{1.5cm}p{9.5cm}}
  \hline
  \multicolumn{2}{l}{\emph{\underline{After \cl{1}--(Pass)--\di{1}--(Double) (usually showing diamonds)}}} \\
  \emph{Pass} & Balanced minimum (no 5-card suit). \\
  \nt{1} & Upper end of range with diamond stopper. \\
  \emph{Suit} & Same meaning as without the double. \\
  \multicolumn{2}{l}{\emph{\underline{After a 1-level suit overcall \cl{1}--(Pass)--\di{1}--(\he{1}/\sp{1})}}} \\
  \emph{Pass} & Balanced minimum (no 5-card suit). \\
  \orf{\emph{Double}} & For takeout with support for other suits. \\
  \emph{Suit} & Natural, at least 5-cards, non-forcing. \\
  \nt{1} & Upper end of the range with stopper. \\
  \nt{2} & Same as \nt{2} without interference but promises
           stopper. \\
  \orf{\emph{Cue-bid}} & Strong hand, lacking stopper in overcalled
                         suit. \\
  \multicolumn{2}{l}{\emph{\underline{After a no-trump overcall \cl{1}--(Pass)--\di{1}--(\nt{1}) (showing minors)}}} \\
  \emph{Pass} & Balanced minimum (no 5-card suit). \\
  \emph{Double} & Penalty oriented. \\
  \orf{\cl{2}} & Heart suit with extra values (\emph{unusual/unusual}). \\
  \orf{\di{2}} & Spade suit with extra values (\emph{unusual/unusual}). \\
  \he{2},
  \sp{2} & Natural, non-forcing. \\
  \nt{2} & Upper-end of the \nt{1} rebid range with stoppers in both
           minors. \\
  \multicolumn{2}{l}{\emph{\underline{After an intervention above 1-level}}} \\
  \emph{Pass} & Balanced minimum (no 5-card suit). \\
  \emph{Others} & A little extra weight as compared to without the
                  intervention.

                  \emph{Unusual/unusual} applies over a \nt{2}
                  overcall. \\
  \hline
\end{longtable}

\hyperlink{ex1cintervene}{Examples of bidding after opponents intervene.$\Rightarrow$}

\pagebreak

\section{Responses to \di{1}}

\hypertarget{1d} A \di{1} opening guarantees 11-15\hcp\ and at least
three diamonds although it could occasionally be a
doubleton. Regardless, the bid is not forcing and partner can pass
with a poor hand (0-7\hcp\ and no 4-card major).

When opener does not have a genuine diamond suit, it is either because
(a) he is interested in the major suits but does not have a
five-carder, (b) he holds a club suit but cannot bid \cl{2} because it
is less than six cards or (c) he has a balanced hand that cannot be
opened with \nt{1}, i.e., 11-13\hcp. The opener's first rebid will
clarify which type of hand he holds (bid or raise a major suit with
(a), bid clubs with (b), or bid no-trumps or a minor with
(c)). Opener's rebids will also classify his point range into a
minimum (11-13\hcp) or maximum (14-15\hcp).

The first priority for both partners is to establish a 4-4 major suit
fit if it exists so a bid by either partner that skips a major suit
implies that he does not hold four cards in the major. If a major suit
fit is found, a real diamond suit holding in opener's hand may never
be mentioned.

Responder will always bid either \sp{1} ($4^+$-card spades, less than
four cards in hearts) or \he{1} ($4^+$-card hearts, may also have
$4^+$-card spades) if he has a four-card major so any other bid
 \emph{denies four cards in either major suit}. As a consequence,
\emph{Precision} does \emph{not} have the concept of \emph{$4^{th}$-suit
forcing} that is found in standard systems and a simple rebid by responder
in a new suit (which may be a six-card suit) is always non-forcing. However,
any new suit bid at the three-level (with or without a jump) is forcing.

\subsection{Responder has a 4-card major}

Opener's rebids after a response of \he{1}
(usually\footnote{Sometimes, with favourable vulnerabilty and an
  extremely weak hand, a tactical bid may be made to interfere with
  opponent's game. E.g., holding \hhand{754,J852,985,654}, you could
  bid \he{1} planning to pass any rebid by opener.} $6^+$\hcp,
$4^+$-card suit, may also have a 4-card spade suit) are as follows:
\begin{longtable}{>{\raggedright}p{0.75cm}p{10.25cm}}
  \hline
  \orf{\sp{1}} & 4-card \sp{}, denies four cards in hearts. Responder
                 rebids: \\
         & \begin{tabular}{>{\raggedright}p{2cm}p{7.25cm}}
             \sp{2} & 8-9\hcp, 4-card spades. \\
             \sp{3} & 10-11\hcp, 4-card spades,
                      invitational. \\
             \sp{4} & Weak hand with long trumps or strong hand with no
                           interest in slam. \\
             \nt{1} & Minimum hand, sign-off. \\
             \nt{2} & 10-12\hcp, balanced hand, invitational. \\
             \cl{2},
             \di{2},
             \he{2} & Minimum hand, $5^+$-cards, attempt to find a
                      better part-score, \excp{non-forcing}. \\
             \gf{\cl{3}} & Could be a 3-card suit. Opener should show
                           3-card heart support with \he{3}. \\ 
             \orf{\di{3}} & $5^+$-card suit, strong hand,
                            game-going. \\
             \he{3} & Jump rebid indicates a $6^+$-card suit with
                           10-12\hcp. Invitational. \\
             \gf{\emph{Double jump
             shift}} & \emph{Splinter} bid with singleton or void and
                       spade support. \\
           \end{tabular} \\
  \nt{1} & 11-13\hcp, balanced, denies 4-card major. Can be
           3-3-2-5 shape. Responder can rebid: \\
         & \begin{tabular}{>{\raggedright}p{2cm}p{7.25cm}}
             \nt{2} & 11-12\hcp, balanced hand, invitational. \\
             \nt{3} & $13^+$\hcp, balanced hand. \\
             \cl{2}, \di{2}, \he{2} & Minimum with $5^+$-card suit,
                                      retreat from no-trumps. \\
             \orf{\sp{2}} & \emph{Reverse}, at least 4-4 in the major
                            suits, strong hand. \\
             \orf{\di{3}} & $5^+$-card diamonds, strong hand. \\
             \gf{\emph{Jump
             shift}} & Could be a 3-card suit. Intermediate bid to
                       decide between \nt{3} and \he{4} (if opener
                       shows delayed support with \he{3}). \\
           \end{tabular} \\
  \cl{2} & Unbalanced, usually 5-4 in minors and no 4-card
           major. Responder's rebids: \\
               & \begin{tabular}{>{\raggedright}p{2cm}p{7.25cm}}
                   \di{2} & Weak hand, to play, preference for diamonds. \\
                   \he{2} & $6^+$-card suit, to play. \\
                   \orf{\sp{2}} & \emph{Reverse}, 5-4 in the major
                                  suits, strong hand. \\
                   \cl{3} & 10-12\hcp, at least 3-card support for
                            clubs. \\
                   \orf{\di{3}} & $5^+$-card suit, strong hand, invitational. \\
                   \nt{3} & To play. \\
                   \gf{\emph{Double jump
                   shift}} & \emph{Splinter}---singleton or void in
                             suit bid and club support. \\
                 \end{tabular} \\
  \di{2} & $6^+$-card diamond suit, no 4-card major, non-forcing. \\
  \he{2} & Single raise---4-card heart suit, 11-13\hcp. Could be a
           3-card suit with a singleton elsewhere when no other bid is
           available. E.g., \hhand{2,AK9,J9832,AT92}. \\
  \he{3} & Jump raise---4-card heart suit, 14-15\hcp. \\
  \sp{2} & \emph{Reverse}---14-15\hcp, $6^+$-cards in diamonds and
           $4^+$-cards in spades. \\
  \nt{2} & 14-15\hcp, good diamonds and stoppers in the unbid major
           and clubs. \\
  \cl{3} & 14-15\hcp, at least 5-5 in minors with points concentrated
           in the two suits. \\
  \di{3} & 14-15\hcp, $6^+$-card diamond suit, no 4-card major. \\
  \hline
\end{longtable}

The bidding after a response of \sp{1} ($6^+$\hcp, $4^+$-card spade
suit, less than four hearts) is similar to that after a response of
\he{1} but is repeated here for convenience. Opener's rebids are:
\begin{longtable}{>{\raggedright}p{0.75cm}p{10.25cm}}
  \hline
  \sp{2} & Single raise---4-card spade suit, 11-13\hcp. Could be a
           3-card suit with a singleton elsewhere. \\
  \sp{3} & Jump raise---4-card spade suit, 14-15\hcp. \\
  \nt{1} & 11-13\hcp, balanced, may have 4-card hearts. Responder can
           rebid: \\
         & \begin{tabular}{>{\raggedright}p{2cm}p{7.25cm}}
             \nt{2} & 11-12\hcp, balanced hand, invitational. \\
             \nt{3} & $13^+$\hcp, balanced hand. \\
             \cl{2},
             \di{2},
             \sp{2} & $5^+$-card suit, minimum, retreat from
                      no-trumps. \\
             \gf{\emph{Jump
             shift}} & Could be a 3-card suit. Intermediate bid to
                       decide between \nt{3} and \sp{4} (if opener
                       shows delayed support with \sp{3}). \\
           \end{tabular} \\
  \cl{2} & Unbalanced, usually 5-4 in minors and no 4-card
           major. Responder's rebids: \\
         & \begin{tabular}{>{\raggedright}p{2cm}p{7.25cm}}
             \di{2} & Weak hand, to play. \\
             \sp{2} & $6^+$-card suit, to play. \\
             \cl{3} & 10-12\hcp, at least 3-card support for clubs. \\
             \orf{\di{3}} & $5^+$-card suit, strong hand. \\
             \nt{3} & To play. \\
             \gf{\emph{Double jump
             shift}} & \emph{Splinter}---singleton or void in suit bid
                       and club support. \\
           \end{tabular} \\
  \di{2} & $6^+$-card diamond suit, no 4-card major, non-forcing. \\
  \nt{2} & 14-15\hcp, good diamonds and stoppers in the unbid major
           and clubs. \\
  \cl{3} & 14-15\hcp, at least 5-5 in minors with points concentrated
           in the two suits. \\
  \di{3} & 14-15\hcp, $6^+$-card diamond suit, no 4-card major. \\
  \hline
\end{longtable}

\subsection{Responder does not have a 4-card major}

When responder does not have a four card major, he has the following
bids:
\begin{longtable}{>{\raggedright}p{1.5cm}p{9.5cm}}
  \multicolumn{2}{l}{\emph{\underline{Balanced hands}}} \\
  \nt{1} & 8-10\hcp, usually balanced hand. \\
  \nt{2} & 11-12\hcp, balanced. Could be a 4-3-3-3 shape with a weak
           four card major and tenaces that would play better as
           declarer in \nt{}. E.g., \hhand{AQT,T642,QT9,KT7} or
           \hhand{9843,KJT,AQ7,JT5} \\
  \nt{3} & 13-15\hcp, balanced. \\
  \multicolumn{2}{l}{\emph{\underline{Support for diamonds (inverted raises)}}} \\
  \orf{\di{2}} & $11^+$\hcp, $5^+$-card diamond suit,
                 \underline{forcing to \nt{2} or \di{3}}. \\
  \di{3} & 0-10\hcp, $5^+$-cards in \di{}, usually with a singleton or
           void. \\
  \di{4} & Preemptive, with more shape and trumps than for \di{3},
           i.e., 6 or 7-card diamond suit. \\
  \multicolumn{2}{l}{\emph{\underline{Unbalanced hands}}} \\
  \orf{\cl{2}} & $10^+$\hcp, usually $5^+$-card suit but could be four
                 cards. Opener's rebids are: \\
         & \begin{tabular}{lp{7cm}}
             \he{2} & 11-$14^-$\hcp, \he{} stopper, no \sp{}
                      stopper. \\
             \sp{2} & 11-$14^-$\hcp, \sp{} stopper, no \he{}
                      stopper. \\
             \nt{2} & 11-$14^-$\hcp, stopper in both majors. \\
             \di{2},
             \di{3} & No stopper in majors, genuine \di{} suit. \\
             \gf{\he{3}} & $14^+$-15\hcp\ with a heart stopper and no
                           spade stopper. \\
             \gf{\sp{3}} & $14^+$-15\hcp\ with a spade stopper and no
                           heart stopper. \\
             \nt{3} & $14^+$-15\hcp\ with stoppers in both majors. \\
             \cl{3} & No stopper in majors. \\
           \end{tabular} \\
  \he{2},
  \sp{2} & Weak jump shift (0-7\hcp) with a long suit,
           non-forcing. Opener's rebids are: \\
         & \begin{tabular}{p{1.1cm}p{7cm}}
             \nt{2} & 11-12\hcp, no 4-card major. \\
             \cl{3} & Shows both minors and asks responder to choose
                      between \cl{3}, \di{3} and \nt{3}. \\
           \end{tabular} \\
  \cl{3} & Invitational, with a long club suit. E.g.,
           \hhand{Q4,75,97,AQJ8654} \\
  \orf{\he{3}},
  \orf{\sp{3}},
  \orf{\cl{4}} & \emph{Splinter} bid with $5^+$-card diamond
                 support. \\
  \he{4},
  \sp{4} & Single-suited hand with $7^+$-cards and no slam
           interest. \\
  \hline
\end{longtable}

\hyperlink{ex1d}{Examples of bidding after a \di{1} opening.$\Rightarrow$}

\subsection{Intervention after a \di{1} opening}

If opponent doubles \di{1}, most responses are the same except that
raises in diamonds are not inverted. Since opener often uses \di{1} as
a vehicle to discover a 4-4 major fit, responder, lacking the strength
to redouble should bid a four-card major if he has one.
\begin{longtable}{p{2.5cm}p{8.5cm}}
  \hline
  \emph{Pass} & 0-4\hcp\ or a ``\emph{trap pass}''. \\
  \emph{New suit} & 4-9\hcp, $4^+$-cards, no longer forcing. \\
  \nt{1} & 6-9\hcp, no 4-card major. \\
  \di{2}, \di{3} &  0-10\hcp\ with length in diamonds since opener may
                   be short. \\
  \orf{\nt{2}} & $10^+$\hcp, $5^+$-card diamond suit, replaces \di{2}
                 ``inverted raise''. \\
  \emph{Redouble} & $10^+$\hcp. No specific distribution, may be for
                    penalties. \\
  \emph{Jump shift} & Weak, usually $7^+$-card suit without interest
                      in playing in another suit or \nt{}. \\
  \hline
\end{longtable}

After an overcall by opponent up to the \sp{2} level, the responses
are:
\begin{longtable}{p{2.5cm}p{8.5cm}}
  \hline
  \emph{Pass} & Poor hand with nothing to bid or ``\emph{trap
                pass}''. \\
  \emph{Double} & $8^+$\hcp, negative double.
                  \hyperlink{negative}{$\Rightarrow$} \\
  \orf{\emph{New suit}} & $10^+$\hcp, natural---if at two-level,
                          values for two-over-one response. \\
  \emph{Raise} & 6-10\hcp, $5^+$-card diamonds (no longer
                 ``inverted''). \\
  \orf{\emph{Jump raise}} & $11^+$\hcp, $5^+$-card diamonds (no longer
                            weak). \\
  \nt{1} & 8-10\hcp, stopper in opponent's suit, balanced. \\
  \nt{2} & 11-12\hcp, stopper in opponent's suit, balanced. \\
  \nt{3} & $12^+$-15\hcp, stopper in opponent's suit, to play. \\
  \gf{\emph{Cue-bid}} & $13^+$\hcp\ with slam interest. A passed hand
                        may cue-bid with $10^+$\hcp\ as a game try. \\
  \orf{\emph{Jump
  Cue-bid}} & After a one-level overcall, a \emph{jump} cue-bid at the
              three level asks opener to bid \nt{3} with a stopper in
              opponent's suit. \\
  \hline
\end{longtable}

\pagebreak

\section{Responses to \he{1} or \sp{1}}

\hypertarget{1major} Responses to a major opening include
\emph{Bergen} raises, \emph{splinter} bids, a forcing \nt{1} and 2/1
game force.

\begin{longtable}{>{\raggedright}p{2cm}p{9.5cm}}
  \hline
  \emph{Pass} & 0-7\hcp\ and poor support. \\
  \multicolumn{2}{l}{\emph{\underline{Raises with support}}} \\
  \emph{Single raise} & 7-10\hcp\ with 3-card support,
                        constructive. \\
  \emph{Jump raise} & 0-6\hcp\ with 4-card support (preemptive
                      \emph{Bergen} raise).
                      \hyperlink{bergen}{$\Rightarrow$} \\
  \gf{\nt{2}} & $12^+$\hcp, 4-card support, \emph{Jacoby \nt{2}}.
                \hyperlink{jacoby2nt}{$\Rightarrow$}. \\
  \cl{3} & 7-10\hcp\ with 4-card support (constructive \emph{Bergen}
           raise). \hyperlink{bergen}{$\Rightarrow$} \\
  \di{3} & 10-12\hcp\ with 4-card support (limited \emph{Bergen}
           raise).
           \hyperlink{bergen}{$\Rightarrow$} \\
  \emph{Game raise} & Wide variety of hands where responder is fairly
                      sure there is no play for slam. E.g., raise to
                      \sp{4} with \hhand{K65,AQ,K82,J9876} or with
                      \hhand{98732,A5,Q,T9743}. \\
  \multicolumn{2}{l}{\emph{\underline{One-over-one response}}} \\
  \orf{\sp{1}} & $8^+$\hcp, $4^+$-card suit. See below for detailed
                 treatment. \\
  \orf{\nt{1}} & 8-15\hcp, balanced hand with mild support for
                 opener's suit or unbalanced hand with insufficient
                 \hcp\ to justify a 2-over-1 response. Opener's rebids
                 are: \\
              & \begin{tabular}{>{\raggedright}p{2cm}p{6cm}}
                  \multicolumn{2}{l}{\emph{\underline{With a minimum 11-13\hcp}}} \\
                  \cl{2},
                  \di{2},
                  \he{2} & 11-13\hcp, 4-card suit (or 3-card better
                           minor). \\
                  \emph{Rebid suit} & 11-13\hcp, 6-card suit. \\
                \end{tabular} \\
              & \begin{tabular}{>{\raggedright}p{2cm}p{6.5cm}}
                  \multicolumn{2}{l}{\emph{\underline{With a maximum 14-15\hcp}}} \\
                  \emph{Jump rebid
                  suit} & 14-15\hcp, 6-card solid suit. The jump rebid
                          should be made on the basis of playing
                          strength rather than \hcp. \\
                  \nt{2} & 5-3-3-2 distribution. \\
                  \emph{Jump in new suit} & 5-5 distribution. \\
                \end{tabular} \\
              & A \emph{reverse}, e.g., \he{1}--\nt{1}--\sp{2} would
                show shape rather than \hcp\ (typically 14-15) and
                indicate a 6-5 distribution (or 6-4 with a very strong
                spade holding such as \emph{AKQx}) \\
  \multicolumn{2}{l}{\emph{\underline{Two-over-one game force}}} \\
  \gf{\cl{2}},
  \gf{\di{2}},
  \gf{\he{2}} & $12^+$\hcp, $4^+$-card minor or 5-card heart suit
                (over \sp{1}). Unless the suit is rebid at the three
                level, all 2-over-1 responses are to
                game. \marginnote{A 2-over-1 response by a
                \emph{passed hand} indicates \excp{8-10\hcp\ with a good
                $5^+$-card suit and is non-forcing}.} Opener's rebids
                are: \\
              & \begin{tabular}{>{\raggedright}p{2.5cm}p{6cm}}
                  \he{2} (after \sp{1}) & Natural, $4^+$-card suit. \\
                  \sp{2} (after \he{1}) & 14-15\hcp, reverse. \\
                  \emph{Rebid suit} & Minimum, not necessarily a
                                      6-carder. \\
                  \emph{New Suit} & Natural, second suit. \\
                  \nt{2} & 11-13\hcp, stoppers in unbid suits \\
                  \emph{Raise} & 11-13\hcp, good support,
                                 non-forcing. \\
                  \emph{Jump in
                  new suit} & 14-15\hcp, good support, control in bid
                              suit. \\
                  \emph{Jump
                  rebid suit} & 14-15\hcp, very good 6-card suit. The
                                jump rebid should be made on the basis
                                of playing strength rather than
                                \hcp. \\
                  \nt{3} & 15\hcp, stoppers in unbid suits. \\
                \end{tabular} \\
  \multicolumn{2}{l}{\emph{\underline{Other bids at 3-level and above}}} \\
  \gf{\emph{Double
  jump shift}} & \emph{Splinter} bid, singleton or void in bid suit,
                 $4^+$-card support, slam interest. \\
  \orf{\nt{3}} & 14-15\hcp, usually 4-card support for opener's major,
                 responder lacks a void or singleton, minimum 4
                 controls. \\
  \hline
\end{longtable}

A response of \sp{1} over \he{1} shows $8^+$\hcp\ with a 4-card suit
and is \orf{forcing for one round}. Opener's rebids are:

\begin{longtable}{p{1.5cm}p{9.5cm}}
  \hline
  \multicolumn{2}{l}{\emph{\underline{Balanced hands}}} \\
  \nt{1} & 11-13\hcp, minimum. \\
  \nt{2} & 14-15\hcp, balanced, stoppers in both minors. \\
  \nt{3} & To play with running suit. \\
  \multicolumn{2}{l}{\emph{\underline{Unbalanced hands}}} \\
  \cl{2},
  \di{2} & 11-15\hcp, $4^+$-card suit, non-forcing. \\
  \sp{2} & 11-13\hcp, 4-card support. \\
  \cl{3},
  \di{3} & 13-15\hcp, 5-card suit. \\
  \he{3} & 13-15\hcp, $6^+$-card heart suit. \\
  \sp{3} & 13-15\hcp\ with $4^+$-card spade support. \\
  \he{4} & To play---distributional hand. \\
  \sp{4} & To play---maximum 13-15\hcp\ and distributional values. \\
  \multicolumn{2}{l}{\emph{\underline{Slam tries}}} \\
  \gf{\cl{4}},
  \gf{\di{4}} & \emph{Splinter} bids, good spade support (\emph{Qxx},
                \emph{xxxx} or better), slam interest. \\
  \nt{4} & \emph{Roman key-card Blackwood} with spades as trumps.
           \hyperlink{blackwood}{$\Rightarrow$} \\
  \hline
\end{longtable}

\hyperlink{ex1h}{Examples of bidding after a major suit opening.$\Rightarrow$}

\subsection{Intervention after a major suit opening}

After an opponent's double, responder can bid:
\begin{longtable}{>{\raggedright}p{2.5cm}p{8.5cm}}
  \hline
  \emph{Pass} & 0-4\hcp\ or a ``\emph{trap pass}''. \\
  \emph{Raise},
  \emph{Jump raise},
  \emph{Game raise} & 0-10\hcp, preemptive, $3^+$-card support
                      ($4^+$-cards for jumps) following the \emph{Law
                      of Total Tricks}. \\
  \emph{New suit} & 4-9\hcp, $5^+$-card suit, non-forcing. \\
  \emph{Jump shift} & Weak, usually $7^+$-card suit without interest
                      in playing in another suit or \nt{}. \\
  \nt{1} & 6-11\hcp, balanced. \\
  \gf{\nt{2}} & $12^+$\hcp, $4^+$-card support, \emph{Jacoby \nt{2}}.
                \hyperlink{jacoby2nt}{$\Rightarrow$}. \\
  \emph{Redouble} & $12^+$\hcp, no specific distribution, may be for
                    penalties. \\
  \hline
\end{longtable}

After an opponent's overcall, responder's bids are:
\begin{longtable}{ p{2.5cm}p{8.5cm}}
  \hline
  \emph{Pass} & Poor hand of 0-7\hcp\ or ``\emph{trap pass}''. \\
  \emph{Double} & $8^+$\hcp, negative double $4^+$ cards in unbid
                  major. \hyperlink{negative}{$\Rightarrow$} \\
  \emph{Raise} & Competitive, typically 6-10\hcp, $3^+$-card
                 support. \\
  \emph{Jump
  raise} & 11-12\hcp, limit raise (replaces \emph{Bergen} raise of
           \di{3}). \\
  \orf{\emph{New suit}} & $10^+$\hcp, $5^+$-card suit (values for a
                          2-over-1 response). \\
  \nt{1} & 8-10\hcp\ with stopper in opponent's suit. \\
  \nt{2} & 11-12\hcp\ with stopper in opponent's suit. \\
  \nt{3} & $12^+$-15\hcp, stopper in opponent's suit, to play. \\
  \gf{\emph{Cue-bid}} & $13^+$\hcp, slam interest. A passed hand may
                        cue-bid with $10^+$\hcp\ as a game try.

                        After a \sp{1} overcall over \he{1}, a
                        \emph{jump} cue-bid of \sp{3} asks opener to
                        bid \nt{3} with a stopper in spades. \\
  \hline
\end{longtable}

\pagebreak

\section{Responses to 1NT}

\hypertarget{1nt} A \nt{1} opening shows a 14-16\hcp\ balanced hand
(\excp{15-17\hcp\ in $3^{rd}$ or $4^{th}$ seat}) without a 5-card major suit
(a 5-card minor is possible). Simple arithmetic will almost always
tell responder how high to place the contract. E.g., 0-8\hcp: 1NT or 2
of suit, 9-10\hcp: game invite, 11-17: game force, 18+: slam. The
response structure below is applicable when \nt{1} is opened in
$1^{st}$ or $2^{nd}$ seat since otherwise, responder is already
limited to less than $11$\hcp.

\begin{longtable}{ p{1.5cm}p{9.5cm}}
  \hline
  \multicolumn{2}{l}{\emph{\underline{Balanced hands---Quantitative raises}}} \\
  \emph{Pass} & 0-8\hcp, balanced hand (or 5-card minor). \\
  \nt{2} & 9-10\hcp, balanced hand---invitation to \nt{3}. \\
  \nt{3} & 11-15\hcp, balanced hand, sign-off. \\
  \nt{4} & 16-17\hcp, balanced hand, invites slam with a maximum. \\
  \orf{\nt{5}} & 20-22\hcp, balanced---partner to choose between
                 \nt{6} and \nt{7}. \\
  \nt{6} & 18-19\hcp, balanced, sign-off. \\
  \multicolumn{2}{l}{\emph{\underline{Unbalanced hands}}} \\  
  \orf{\cl{2}} & \emph{Stayman}. \hyperlink{stayman}{$\Rightarrow$}
                 This can be used on both weak and strong hands. \\
              & \begin{tabular}{>{\raggedright}p{2.5cm}p{6cm}}
                  \emph{Weak hand with
                  both majors} & When \he{2} or \sp{2} is preferable
                                 to \nt{1}, responder bids \cl{2} and
                                 passes any major suit response. If
                                 opener denies majors with \di{2},
                                 responder rebids \he{2} asking opener
                                 to pass or correct to \sp{2}. \\
                \end{tabular} \\
              & \begin{tabular}{>{\raggedright}p{2.5cm}p{6cm}}
                  \emph{Invitational or
                  game-forcing hands} & To find a 4-4 major suit fit
                                        (implies one 4-card
                                        major). With 5-4 in the majors
                                        and a game-going hand, if
                                        opener denies majors with
                                        \di{2}, responder jumps in the
                                        4-card suit to
                                        \gf{\he{3}}/\gf{\sp{3}} which
                                        shows 5-cards in the other
                                        major (\emph{Smolen}
                                        convention). Opener can then
                                        bid \nt{3} with a doubleton or
                                        \he{4}/\sp{4} with 3-card
                                        support. \\
                \end{tabular} \\
  \orf{\di{2}},
  \orf{\he{2}} & \emph{Jacoby} transfers to \he{2} and \sp{2}
                 respectively. May be made with both strong and weak
                 hands. \hyperlink{jacoby}{$\Rightarrow$} \\
  \orf{\sp{2}} & Minor suit \emph{Stayman}---shows a minor 2-suiter
                 (5-4 or better) and asks opener to bid his 4-card
                 minor. This is used with either very weak hands or
                 strong slam-invitational hands. With intermediate
                 hands, you will want to play \nt{3} so asking for
                 minors is not helpful. Opener responds: \\
              & \begin{tabular}{lp{7.5cm}}
                  \cl{3},
                  \di{3} & 4-card suit. Responder will pass with a
                           weak hand or bid on with slam interest. \\
                  \orf{\nt{2}} & No four card minor. Responder rebids \cl{3}
                           with a weak hand asking opener to pass or
                           correct to \di{3}. With a strong hand,
                           responder bids \gf{\he{3}} or \gf{\sp{3}}
                           to show a singleton. \\
                \end{tabular} \\
  \gf{\emph{3 of suit}} & $10^+$\hcp, good $5^+$-card suit. \\
  \gf{\di{4}},
  \gf{\he{4}} & \emph{Texas} transfers to \he{4} and \sp{4}
                respectively. Denies slam values. \\
  \multicolumn{2}{l}{\emph{\underline{Others}}} \\
  \gf{\cl{4}} & \emph{Gerber} ace-asking.
                \hyperlink{gerber}{$\Rightarrow$} \\
  \hline
\end{longtable}

\subsection{Intervention after opening 1NT}

If an opponent doubles \nt{1} for penalties, we use the \emph{Meckwell
  Escapes} to run to a safer contract. This does not work very well if responder has a
4-3-3-3 distribution and in this case, responder can either treat his
four-card suit as a five-card suit, or treat his four-card suit
and his best 3-card suit as a two-suiter and pray that the opponents
elect to introduce a suit of their own.
\begin{longtable}{ p{1.5cm}p{9.5cm}}
  \hline
  \orf{\emph{Pass}} & Forcing, shows either clubs, diamonds or both
                      majors. Opener must rebid \cl{2} after which
                      responder rebids: \\
                    & \begin{tabular}{lp{7.5cm}}
                        \emph{Pass} & 5-card club suit, escape
                                      completed. \\
                        \di{2} & 5-card diamond suit. \\
                        \orf{\he{2}} & 4-4 in the majors, opener
                                       passes or corrects to
                                       \sp{2}. \\
              \end{tabular} \\
  \emph{Redouble} & Natural, non-forcing. \\
  \cl{2} & Clubs and another suit, 4-4 or better. Opener passes or
           corrects to \di{2} asking for the higher suit. \\
  \di{2} & Diamonds and a higher ranking suit, 4-4 or better. Opener
           passes or corrects to \he{2} asking for the higher suit. \\
  \he{2},
  \sp{2} & Natural, 5-card suit. \\
  \emph{3 of suit} & $6^+$-card suit, invitational. \\
  \hline
\end{longtable}


After an opponent's overcall, responder has the following choices:
\begin{longtable}{p{3cm}p{8cm}}
  \hline
  \emph{Double} & Penalty double. \\
  \emph{Suit at 2-level} & 0-6\hcp, natural and non-forcing. \\
  \orf{\nt{2}} & \emph{lebensohl}---partner must bid \cl{3}.
                 \hyperlink{lebensohl:nt}{$\Rightarrow$} \\
  \gf{\emph{Suit at 3-level}} & Natural, forcing to game. \\
  \orf{\emph{Cue-bid}} & Asks opener to bid a 4-card major if he has
                         one, denies a stopper in opponent's suit. \\
  \nt{3} & \emph{lebensohl}---denies stopper in opponent's suit.
           \hyperlink{lebensohl:nt}{$\Rightarrow$} \\
  \hline
\end{longtable}

\pagebreak

\section{Responses to \cl{2}}

\hypertarget{2c} Bidding after a \cl{2} opening (11-15\hcp, $6^+$-card
club suit) is largely natural except for the conventional \di{2}
response that asks opener to further describe his hand. Note that a
direct \excp{\di{4} (not \nt{4}) is used to trigger key card asking}.

\begin{longtable}{ p{1.5cm}p{9.5cm}}
  \hline
  \emph{Pass} & 0-7\hcp, poor hand. \\
  \multicolumn{2}{l}{\emph{\underline{Weak responses}}} \\
  \cl{3} & Preemptive raise based on a club fit. Not
           forward-going. \\
  \cl{4} & Preemptive raise with extra club support or distributional
           values as compared to a raise to \cl{3}. \\
  \he{2},
  \sp{2} & 8-10\hcp, non-forcing, invitational. Opener may pass with a
           minimum and mild support. \\
  \nt{2} & 10-11\hcp, invitation to \nt{3}. Any rebid other than
           \cl{3} (sign-off) by opener commits to game.\\
  \he{4},
  \sp{4} & Very long suit with no interest in slam, to play. \\
  \multicolumn{2}{l}{\emph{\underline{Strong responses}}} \\
  \orf{\di{2}} & 11+\hcp, conventional (with club fit, may be made
                 with only 8\hcp). Opener's rebids are: \\
              & \begin{tabular}{lp{7cm}}
                  \he{2},
                  \sp{2} & 11-13\hcp, 4-card suit. \\
                  \nt{2} & 11-13\hcp, 6-3-2-2 balanced hand with
                           stoppers in two suits. Responder then bids
                           \di{3} to enquire about stoppers and
                           opener's rebids are: \\
                         & \begin{tabular}{ll}
                             \he{3} & \he{} and \di{} stoppers. \\
                             \sp{3} & \sp{} and \di{} stoppers. \\
                             \nt{3} & \he{} and \sp{} stoppers. \\
                           \end{tabular} \\
                  \cl{3} & 11-13\hcp, 6 clubs with 1 outside
                           stopper. Responder bids \di{3} to enquire
                           about the stopper and opener's rebids
                           are: \\
                         & \begin{tabular}{lp{4.5cm}}
                             \he{3} & \he{} stopper. \\
                             \sp{3} & \sp{} stopper. \\
                             \nt{3} & \di{} stopper.  \\
                             \di{4} & 5-card suit (6-5 in clubs and
                                      diamonds) \\
                           \end{tabular} \\
                  \nt{3} & 14-15\hcp, 6-3-3-2 balanced hand, good club
                           suit. \\
                  \gf{\he{3}},
                  \gf{\sp{3}} & 14-15\hcp, 4-card suit. \\
                \end{tabular} \\
  \gf{\di{3}},
  \gf{\he{3}},
  \gf{\sp{3}} & $6^+$-card suit with game-forcing values
                ($12^+$\hcp). Opener's rebids are: \\
              & \begin{tabular}{lp{7cm}}
                  \nt{3} & Less than 2-card support. \\
                  \emph{Raise} & Minimum, 3-card support. \\
                  \gf{\emph{New
                  suit}} & Maximum, 3+-card club support, ace or void
                           in suit bid. \\
                \end{tabular} \\
  \nt{3} & 13-15\hcp, balanced hand with stoppers in the unbid suits
           and no interest in the majors. \\
  \gf{\di{4}} & Ace-asking, \emph{Roman key-card Blackwood}
                 \hyperlink{blackwood}{$\Rightarrow$} \\
  \hline
\end{longtable}

\hyperlink{ex2c}{Examples of bidding after a \cl{2} opening.$\Rightarrow$}

\subsection{Intervention after a \cl{2} opening}

After an opponent's \emph{double}, all bids carry their normal
meaning. The extra bid of \emph{Redouble} shows $10^+$\hcp\ and is
penalty oriented.

Over an opponent's overcall, raises are natural. Others:
\begin{longtable}{ p{2.5cm}p{8.5cm}}
  \hline
  \orf{\emph{Double}} & $8^+$\hcp, negative double.
                        \hyperlink{negative}{$\Rightarrow$} \\
  \gf{\emph{Cue-bid}} & 12+\hcp, singleton or void in opponent's
                        suit. \\
  \orf{\emph{New suit}} & Any new suit bid is natural and forcing. \\
  \hline
\end{longtable}

\pagebreak

\section{Responses to \di{2}}

\hypertarget{2d} A \di{2} opening describes a three suited hand with
shortness in diamonds and the responder can place the contract fairly
easily in most cases. The only positive responses are the artificial
bids of \nt{2} or \di{3} (both forcing to game).

\begin{longtable}{ p{2cm}p{9cm}}
  \hline
  \emph{Pass} & 6+ diamonds, no interest in other suits or bidding
                higher. \\
  \he{2},
  \sp{2},
  \cl{3} & Natural, sign-off. \\
  \gf{\nt{2}} & 11+\hcp, artificial, asks opener to further describe
                his hand. Opener's rebids are: \\
              & \begin{tabular}{lp{6.5cm}}
                  \cl{3},
                  \di{3}  & 3=4=1=5 or 4=3=1=5 shape respectively. \\
                  \he{3} & 11-13\hcp\ and 4=4=1=4 shape. \\
                  \sp{3} & 14-15\hcp\ and 4=4=1=4 shape. \\
                  \nt{3} & 14-15\hcp, 4=4=1=4 shape and \di{}\emph{A}
                           or \di{}\emph{K}. \\
                  \cl{4} & 11-13\hcp, 4=4=0=5 shape. \\
                  \di{4} & 14-15\hcp, 4=4=0=5 shape. \\
                \end{tabular} \\
              & A further rebid in diamonds (the singleton suit) by
                opener would be $\beta$ asking for controls.
                \hyperlink{beta}{$\Rightarrow$} \\
  \orf{\di{3}} & $\beta$ control-asking bid. \\
  \he{3},
  \sp{3} & 7-9\hcp, preemptive, $5^+$-card suit. \\
  \he{4},
  \sp{4},
  \cl{5} & Sign-off, to play. \\
  \hline
\end{longtable}

\hyperlink{ex2d}{Examples of bidding after a \di{2} opening.$\Rightarrow$}

\subsection{Intervention after a \di{2} opening}

When opponents double \di{2} (possibly showing values in diamonds):
\begin{longtable}{p{2.5cm}p{8.5cm}}
  \hline
  \orf{\emph{Pass}} & Waiting action, forcing. Partner will
                      \emph{redouble} with a minimum hand or bid
                      \he{2} or \sp{2} (four card suit) with a 3-4-1-5
                      shape. \\
  \emph{Redouble} & Desire to play in diamonds with a long
                    suit. Prepared to punish the opponents when they
                    run from the redouble. \\
  \emph{Others} & Same meaning as without the double. \\
  \hline
\end{longtable}

Responses after an opponent's overcall:
\begin{longtable}{p{2.5cm}p{8.5cm}}
  \hline
  \emph{Double} & For penalties. \\
  \gf{\nt{2}} & Same as without overcall. \\
  \emph{Others} & Natural, non-forcing. \\
  \hline
\end{longtable}

\pagebreak

\section{Responses to \he{2} or \sp{2}}

\hypertarget{2major} Opener shows exactly six cards in the suit bid
with two of the top three honours (or three of the top five) and
6-10\hcp. It denies four cards in the unbid major.

Without two-card support, it is generally correct to pass unless
holding $16^+$\hcp\ or a very good suit of one's own. The general
system of responses is called \emph{RONF} (Raise is the Only
Non-Forcing bid) and any new suit bid is forcing asking partner to
raise with 3-card support. A raise is a sign-off and opener should not
bid again.

A \nt{2} response asks opener to show a ``feature'' (an outside
stopper, i.e., an ace or king). If there is a suit fit, the feature
will help in deciding whether to bid game with an invitational
hand. Likewise, without a suit fit, it could help in deciding whether
to bid \nt{3}.

Summary of responses:
\begin{longtable}{>{\raggedright}p{2cm}p{9cm}}
  \hline
  \emph{Pass} & No game, no fit. \\
  \emph{Raise} & 6-9\hcp, 3-card support, sign-off. \\
  \emph{Game raise} & Variety of hands, sign-off. E.g., raise to
                      \he{4} with either \hhand{AKJ3,Q5,4,A76432} or
                      \hhand{4,KJ43,KJ743,T87}. \\
  \orf{\emph{New
  suit}} & $13^+$\hcp, $5^+$-card suit, asking for 3-card
           support. E.g., with \hhand{AKJ943,2,AQT,T95}, bid \sp{2}
           over \he{2}. Opener's rebids are: \\
              & \begin{tabular}{ll}
                  \emph{Raise} & $3^+$-card support. \\
                  \emph{Rebid suit} & No support. \\
                \end{tabular} \\
  \orf{\nt{2}} & Either $16^+$\hcp\ or $13^+$\hcp\ with 3-card
                 support. Opener's rebids are: \\
              & \begin{tabular}{lp{6.5cm}}
                  \emph{Rebid suit} & No outside stoppers. \\
                  \emph{New suit} & Stopper in suit bid. \\
                  \nt{3} & Very good suit headed by \emph{AQJ},
                           \emph{AKJ}, or exceptionally the
                           \emph{AKQ}. \\
                \end{tabular} \\
  \nt{3} & $16^+$\hcp, all other suits stopped. Normally indicates a
           solid minor with a singleton or void in opener's
           suit. E.g., bid \nt{3} with \hhand{K4,2,AKQJ876,K76} over
           an opening \he{2}. \\
  \gf{\cl{4}} & \emph{Roman key-card} ask with modified responses
                since opener cannot have 3 or 4 key cards.
                \hyperlink{blackwoodmod}{$\Rightarrow$} \\
  \hline
\end{longtable}

\pagebreak

\section{Responses to 2NT}

\hypertarget{2nt} Opener is showing a balanced hand with
22-23\hcp. Note that the rebids after a \cl{3} response differ from
those after \cl{1}--\di{1}--\nt{2} because here, responder is not
limited to 0-7\hcp\ and slam is a possibility. Responses are:

\begin{longtable}{ p{1.5cm}p{9.5cm}}
  \hline
  \emph{Pass} & Very weak, no suit to escape to. \\
  \nt{3} & To play. \\
  \orf{\cl{3}} & \emph{Smolen Stayman}---asking for majors (implies at
                 least one four-card major). Game-going with slam
                 interest. Responses are: \\
              & \begin{tabular}{lp{7.5cm}}
                  \orf{\di{3}} & No 4-card major. Responder's rebids: \\
                         & \begin{tabular}{lp{6cm}}
                             \he{3} & 4-card \he{} and 5-card \sp{}
                                      (allows for 5-3 fit). \\
                             \sp{3} & 4-card \sp{} and 5-card \he{}
                                      (allows for 5-3 fit). \\
                             \nt{3} & To play (no five-card major). \\
                           \end{tabular} \\
                  \he{3},
                  \sp{3} & 4-card major. \\
                  \nt{3} & 4 cards in both majors. Responder bids: \\
                         & \begin{tabular}{lp{5.5cm}}
                             \emph{Pass} & To play. \\
                             \di{4},
                             \he{4} & Transfer to \he{4} or \sp{4}
                                      respectively. \\
                           \end{tabular} \\
                \end{tabular} \\
  \orf{\di{3}},
  \orf{\he{3}} & Transfer to \he{3} or \sp{3} respectively.

                 With 5-5 in the majors, responder bids \sp{3} after a
                 transfer to hearts. With 5-4 (four hearts), he bids
                 \he{4} after a transfer to spades. Bidding \nt{3}
                 asks opener to pass or bid four of the major holding
                 3-card support. \\
  \orf{\sp{3}} & Minor suit slam try. Opener bids \cl{4} (corrected to
                 \di{4} if needed). \\
  \orf{\cl{4}} & Minor two-suiter. \di{4} sets diamonds, any other bid
                 agrees clubs. \\
  \orf{\di{4}},
  \orf{\he{4}} & Transfer to \he{4} or \sp{4} respectively. $6^+$-card
                 suit, to play. \\
  \hline
\end{longtable}

\pagebreak

\section{Responses to 3-level Preempts}

\hypertarget{3preempt} A preempt at the 3-level is made with less than
10\hcp\ and at least a 7-card suit with a suit quality of 8 \excp{(or
  9 when vulnerable)} and obeying the rule of 2/3/4.  In general, it
is expected that opener will not have a four-card major and have two
or more good honours in his suit with not more than one useful honour
in a side suit. Unless partner makes a forcing response, the opener is
not expected to bid again.

Responses are based on playing strength (additional tricks that
responder can provide) rather than high card points:
\begin{longtable}{p{1.5cm}p{9.5cm}}
  \hline
  \emph{Pass} & No support, no suit of one's own, at best can provide
                2 or 3 tricks. \\
  \emph{Raise} & Could be preemptive with 3-card support or with good
                 cards in side suits that can provide four tricks or
                 more. \\
  \orf{\emph{New suit}} & $6^+$-card suit, good playing strength since
                          this forces opener to respond. Opener should
                          raise the suit with two or three trumps. \\
  \nt{3} & Natural, good controls in side suits, good chance of taking
           6-7 tricks in opener's suit even with the known problems of
           entries in dummy. \\
  \hline
\end{longtable}

\section{Responses to 3NT}

\hypertarget{3nt} The ``gambling'' \nt{3} bid shows a solid $7^+$-card
minor suit (\emph{AKQ} or better) and no outside ace or
king. Responses are:

\begin{longtable}{p{1.5cm}p{9.5cm}}
  \hline
  \emph{Pass} & To play, stoppers in side suits. \\
  \cl{4} & Escape---asks opener to pass or bid \di{4} if that is his
           suit. \\
  \orf{\di{4}} & Asks opener to bid a singleton or void if he has
                 one. Responses: \\
              & \begin{tabular}{lp{6cm}}
                  \orf{\he{4}},
                  \orf{\sp{4}} & Singleton or void in the bid suit. \\
                  \orf{\nt{4}} & Singleton or void in the other
                                 minor. \\
                  \cl{5},
                  \di{5} & Shows that minor and denies a singleton or
                           void. \\
                \end{tabular} \\
  \he{4},
  \sp{4} & Natural, to play. \\
  \nt{4} & Quantitative, asking opener to bid \nt{6} with extra length
           or an extra trick outside (such as \emph{Qxx}). \\
  \cl{5} & Sign-off and weak. Opener should correct to \di{5} if that
           is his suit. \\
  \di{5} & Also a sign-off but responder indicates he knows opener's
           suit is diamonds and it would be advantageous to play from
           his side. \\
  \orf{\nt{5}} & Grand Slam try showing no losers outside the trump
                 suit but indicating a void in opener's suit. If
                 opener is completely solid (e.g., \emph{AKQJ} to
                 seven card), he bids 7 otherwise he bids 6. \\
  \cl{6} & Asks opener to pass or correct to \di{6}. This could be a
           tactical bid. \\
  \hline
\end{longtable}

\hyperlink{ex3nt}{Examples of bidding after a \nt{3} opening.$\Rightarrow$}

\pagebreak

\section{Responses to Namyats}

\hypertarget{namyats} The \emph{Namyats} convention (\emph{Stayman}
spelt backwards) shows a long (normally eight cards) semi-solid (not
missing both ace and king) major suit with eight or more playing
tricks and playing strength concentrated in the trump suit. The bid is
meant to be constructive rather than preemptive and allows you to
distinguish hands that are close to game versus purely preemptive
openings.

\cl{4} shows hearts and \di{4} shows spades. With a weaker hand, the
preemptive bids of \he{4} or \sp{4} would be used instead.

Responses are:
\begin{longtable}{p{1.5cm}p{9.5cm}}
  \hline
  \he{4},
  \sp{4} & Bidding game in partner's suit is a sign-off with no slam
           interest (responder wants to be declarer). \\
  \orf{\di{4}},
  \orf{\he{4}} & The next step above opener's bid is a relay asking
                 him to bid his suit (\he{} or \sp{}). After opener
                 bids his suit, any new suit bid by responder is a
                 cue-bid. Responder passes if he does not have slam
                 interest but wants his hand to be dummy. \\
  \orf{\nt{4}} & \emph{Roman key-card Blackwood}.
                 \hyperlink{blackwood}{$\Rightarrow$} \\
  \orf{\emph{Suit}} & $\epsilon$ suit control-asking bid. A compressed
                      scale of responses is used (note that the steps
                      skip over \nt{} because that has a special
                      meaning): \\
               & \begin{tabular}{lp{6.5cm}}
                   \emph{Cheapest
                   \nt{}} & Guarded king in suit. When you have ample
                            tricks elsewhere, this allows you to
                            declare an \nt{} contract from the correct
                            side. \\
                   \emph{1 step} & No first or second round
                                   control. \\
                   \emph{2 steps} & Second round control---any
                                    singleton. \\
                   \emph{3 steps} & First round control---void or
                                    ace. \\
                   \multicolumn{2}{p{9cm}}{A repeat ask in the same
                   suit asks for third round control, i.e., a
                   doubleton or guarded queen. Responses
                   are---$1^{st}$ step: no control, $2^{nd}$ step:
                   doubleton and $3^{rd}$ step: guarded queen.} \\
                 \end{tabular} \\
  \hline
\end{longtable}

\hyperlink{namyats}{Examples of bidding after a \emph{Namyats} opening.$\Rightarrow$}

\pagebreak

\section{Competitive Bidding}

In general, overcalls in a suit are made with a hand of 8-15\hcp\ and
a good suit---the higher the level of the overcall, the stronger the
hand in terms of playing strength (longer suit / higher point
range). Jump overcalls are always weak or preemptive. An overcall in
no-trumps, however, shows the equivalent of an opening no-trump bid
\emph{with a stopper} in the opponent's suit.

\emph{Takeout doubles} are made with an opening hand or better
($12^+$\hcp) and must have tolerance for all the unbid suits. The only
exception is when you have a powerful hand of $16^+$\hcp\ when you
should double for takeout and bid your suit over partner's forced
response. A double of \nt{1} is \underline{not} a takeout double (you
cannot have tolerance for all four suits) and has special meanings
described below.

\subsection{Overcalls}

Overcalls over an opponent's opening of one of a suit carry the
following meanings:
\begin{longtable}{>{\raggedright}p{2.5cm}p{8.5cm}}
  \hline
  \emph{Non-jump
  overcall} & 8-15\hcp, good $5^+$-card suit (11-15\hcp, $6^+$-card
              suit at 2-level). Opener's responses are: \\
            & \begin{tabular}{p{2cm}p{5.5cm}}
                \multicolumn{2}{l}{\emph{\underline{With support for partner's suit}}} \\
                \emph{Single raise} & 8-9\hcp, $3^+$-card support. \\
                \emph{Jump raise} & Weak, 5-8\hcp, $4^+$-card support,
                                    preemptive. \\
                \emph{Game raise} & Hand good enough for game or weak
                                    and preemptive with 5-card
                                    support. \\
                \orf{\emph{Cue-bid}} & $9^+$\hcp, $4^+$-card support
                                       (or $10^+$\hcp\ with 3-card
                                       support).

                                       If the overcaller shows a
                                       minimum by rebidding his suit,
                                       responder can pass with
                                       9-11\hcp, raise with 12-14\hcp\
                                       and cue-bid again with
                                       $15^+$\hcp. \\
              \end{tabular} \\
            & \begin{tabular}{p{2cm}p{5.5cm}}
                \multicolumn{2}{l}{\emph{\underline{Without support for partner's suit}}} \\
                \emph{New suit} & $5^+$-card suit, non-forcing. \\
                \emph{Jump in suit} & 11-13\hcp, $6^+$-card suit,
                                      non-forcing. \\
                \nt{1} & 8-11\hcp\ with stopper in opponent's suit. \\
                \nt{2} & 12-15\hcp\ with stopper in opponent's
                         suit. \\
                \nt{3} & $16^+$\hcp\ with stopper in opponent's
                         suit. \\
                \orf{\emph{Cue-bid}} & Strong hand with interest in game.

                                       To distinguish from the case
                                       with support for partner's
                                       suit, responder will rebid
                                       \nt{} or jump in a new suit on
                                       his next bid. \\
              \end{tabular} \\
  \emph{Jump
  overcall} & 0-10\hcp, preemptive with $6^+$-card suit (7-card at
              3-level). \\
  \nt{1} & 14-16\hcp, stopper in opponent's suit. Responses (including
           \emph{Stayman} and transfers) are the same as over a \nt{1}
           opening. Stronger balanced hands should open with a takeout
           double followed by an NT bid. \\
  \emph{Cue-bid} & \emph{Michael's cue-bid} showing a two-suiter in
                   the highest ranking unbid suit and another.
                   \hyperlink{michaels}{$\Rightarrow$} \\
  \emph{Jump
  to \nt{2}} &  \emph{Unusual no-trump} showing a two-suiter in the two
            lowest ranking unbid suits.
            \hyperlink{unusualnt}{$\Rightarrow$} \\
  \hline
\end{longtable}

\subsection{Doubles}

A double of opponent's opening of one of a suit is either 12-15\hcp\
with tolerance for any unbid suit or a power double with $16^+$\hcp. A
double of an opponent's minor opening bid usually shows 4-4 in the
majors and over a major suit opening, it normally promises a 4-card
suit in the other major. A response must be made even with a blank
hand unless the other opponent bids.

Responder's bids:
\begin{longtable}{>{\raggedright}p{2.5cm}p{8.5cm}}
  \hline
  \emph{Suit} & 0-8\hcp, may be only four cards (forced
                response). Doubler should pass with a minimum, raise
                and invite with support and 15-17\hcp\ or bid a new
                suit or no-trumps with $18^+$\hcp. \\
  \emph{Free bid
  or jump in suit} & 9-11\hcp. Doubler should compete with 12-14\hcp,
                     force game with 15-17\hcp\ and bid a new suit or
                     no-trumps with $18^+$\hcp. \\
  \gf{\emph{Cue-bid}} & Strong hand, forcing to game. \\
  \nt{1} & 6-9\hcp, stopper in opponent's suit, non-forcing. \\
  \nt{2} & 10-11\hcp, stopper in opponent's suit, invitational. \\
  \nt{3} & Values for game, no interest in slam. Doubler can check
           aces and try for slam with $18^+$\hcp. \\
  \hline
\end{longtable}

\subsubsection{Doubles following a weak two opening}

Over a double of a weak two opening (or when \emph{RHO} raises
opener's suit to the two level in an auction such as
(\he{1})--\emph{Double}--(\he{2})--???) the responses are:
\begin{longtable}{>{\raggedright}p{4cm}p{7cm}}
  \hline
  \emph{Pass} & (Only when \emph{RHO} has bid) Weak, 0-7\hcp. \\
  \emph{2 of suit} & Weak, 0-7\hcp. \\
  \orf{\nt{2}} or
  \orf{\emph{3 of
  a suit}} & \emph{lebensohl} convention.
             \hyperlink{lebensohl:weak}{$\Rightarrow$} \\
  \hline
\end{longtable}

\subsubsection{Negative doubles}

A negative double after an opponent's overcall of one of a major
implies possession of four cards in the other major and sufficient
values to justify a response. Responder is also expected to have mild
support either for the unbid minor or for partner's suit. At the two
or three-level, it implies that the hand is playable in either of the
unbid suits and possibly even partner's suit at the level he will be
forced to bid. Sometimes, responder may make a negative double with a
single-suited hand when he has insufficient high-card strength to make
a forcing bid in his suit.

Examples:
\begin{longtable}{l|p{5cm}>{\raggedright\arraybackslash}p{3cm}}
  \emph{Bidding} & \emph{Sample hand} & \\
  \hline
  \di{1}--(\cl{2})--\emph{Double}
                 & Reasonable major suit holdings and strength to play
                   at two-level.
                                      & \hhand{KJT5,QJT7,QT9,64}  \\
  \sp{1}--(\he{2})--\emph{Double}
                 & Reasonable minor suit holdings and sufficient
                   values to play at 3-level.
                                      & \hhand{JT,85,QJT43,AJT9} \\
  \di{1}--(\sp{1})--\emph{Double}
                 & Four hearts with at least 8\hcp.
                                      & \hhand{87,KQT3,JT4,QT92} \\
  \di{1}--(\sp{1})--\emph{Double}
                 & \he{2} cannot be bid since it would imply values
                   for a two-over-one response. The solution is to use
                   a negative double and rebid hearts.
                                      & \hhand{87,AQJT95,T96,JT} \\
  \hline
\end{longtable}

Note that in the sequence, \di{1}--(\he{1}), responder should bid
\sp{1} holding four spades rather than double.

\hypertarget{negative} Openers rebids after partner's negative double
are as follows:
\begin{longtable}{p{2.5cm}p{8.5cm}}
  \hline
  \emph{Pass} & For penalties (see below). \\
  \multicolumn{2}{l}{\emph{\underline{With a minimum hand (11-13\hcp)}}} \\
  \emph{New suit} & Shows fit in responder's implied suit (especially
                    if a major). \\
  \emph{Rebid own suit} & Shows $6^+$-card suit (or a strong 5-card
                          suit). \\
  \emph{Cheapest \nt{}} & No suit worth bidding, stopper in opponent's
                          suit. \\
  \multicolumn{2}{l}{\emph{\underline{With a maximum hand (14-15\hcp)}}} \\
  \emph{Jump shift} & Invitational but non-forcing. \\
  \emph{Jump rebid} & Shows a good $6^+$-card suit, invitational. \\
  \emph{Jump in \nt{}} & Stopper in opponent's suit, ready source of
                         tricks. \\
  \orf{\emph{Cue-bid}} & Shortness in opponent's suit and by
                         inference, support for doubler's suits. \\
  \hline
\end{longtable}

Opener can also \emph{pass} for penalties after a negative double
although this is rare since it implies opponents have bid opener's
good suit. For example, after \di{1}--(\cl{2})--\emph{Double}, opener
could pass with \hhand{5,A76,AKJ3,QT942} or \hhand{983,Q,AK865,KQJ9}
(but not with \hhand{QT65,9,AKT,KT32} where the correct bid would be
to support one of partner's implied suits with \sp{2}).

\subsubsection{Responsive and Lightner doubles}

A double in a competitive auction is a \emph{responsive double}
indicating 3-card support in partner's suit.

A double of a slam contract is a \emph{Lightner double} requesting an
unusual lead from partner.

\subsection{Defence against a no-trump opening}

Doubles and overcalls over a \nt{1} opening have different meanings
depending on whether the opponents are playing a strong no-trump or a
weak no-trump. For our purposes, a strong no-trump is one whose point
range includes $15^+$\hcp.

\subsubsection{After a strong no-trump opening}

Over a strong no-trump opening, we use the \emph{Meckwell}
convention. This set of responses is also used when intervening after
having passed once (since a penalty double is no longer meaningful).

\begin{longtable}{ p{1.5cm}p{9.5cm}}
  \hline
  \emph{Double} & Single-suited minor or major two-suiter (5-4 or
                  better but normally 5-5). Responder can then bid: \\
                & \begin{tabular}{lp{7cm}}
                    \orf{\cl{2}} & Asks partner to clarify his hand
                                   (rebids \he{2} with major
                                   two-suiter, else \di{2} or
                                   \cl{3}). \\
                    \emph{Pass} & Good hand, for penalties. \\
                    \emph{Suit} & Good suit, natural. \\
                  \end{tabular} \\
  \cl{2},
  \di{2} & Two-suiter---the suit bid and a major suit. Responder can
           bid: \\
                & \begin{tabular}{p{1.5cm}p{7cm}}
                    \emph{Pass} & To play. \\
                    \he{2} & Ask partner to pass or correct to
                             \sp{2}. \\
                    \emph{New suit} & Natural. \\
                    \emph{Raise} & Preemptive, weak. \\
                    \orf{\nt{2}} & Game interest, see below. \\
                  \end{tabular} \\
  \he{2},
  \sp{2} & Natural, $6+$-card suit or an excellent 5-card suit. \\
  \nt{2} & Minor two-suiter or a big hand. Partner should bid his
           better minor. With a big hand, bidding continues
           naturally. \\
  \hline
\end{longtable}

\nt{2} after \cl{2}, \di{2} (or after \emph{Double}--(\emph{Pass})--\cl{2}--(\emph{Pass})--\he{2}) shows
game interest and partner rebids as follows:
\begin{longtable}{>{\raggedright}p{1.5cm}p{9.5cm}}
  \hline
  \multicolumn{2}{l}{\emph{\underline{After \cl{2}}}} \\
  \cl{3} & Minimum hand. \\
  \di{3},
  \he{3},
  \sp{3} & Maximum hand, second suit. \\
  \multicolumn{2}{l}{\emph{\underline{After \di{2}}}} \\
  \cl{3} & Minimum hand, diamonds and hearts. \\
  \di{3} & Minimum hand, diamonds and spades. \\
  \he{3},
  \sp{3} & Maximum hand, second suit. \\
  \multicolumn{2}{l}{\emph{\underline{After (\nt{1})--Double--\cl{2}--\he{2}--\nt{2}}}} \\
  \cl{3} & Minimum, hearts longer than spades. \\
  \di{3} & Minimum, spades longer than hearts. \\
  \he{3} & Maximum, hearts longer than spades. \\
  \sp{3} & Maximum, spades longer than hearts. \\
  \hline
\end{longtable}

\subsubsection{After a weak no-trump opening}

The \emph{Cappelletti} system is employed over a weak no-trump. As
compared to the \emph{Meckwell} convention, this has the advantage of
retaining a double for penalties.

\begin{longtable}{ p{1.5cm}p{9.5cm}}
  \hline
  \emph{Double} & Strong hand (normally equal strength), for
                  penalties. \\
  \orf{\cl{2}} & Any one suited hand ($6^+$-cards). Responder bids
                 \di{2} to ask partner to name his suit. \\
  \orf{\di{2}} & Two-suiter---both majors. \\
  \he{2},
  \sp{2} & Two-suiter---hearts/spades and one minor. \nt{2} asks
                 for the minor suit. \\
  \orf{\nt{2}} & Two-suiter---both minors. \\
  \emph{3 of suit} & Intermediate hands with 13-16\hcp\ and $6^+$-card
                     suit. \\
  \hline
\end{longtable}

After a \cl{2} overcall, any bid other than \di{2} is natural and
non-forcing. Responder may pass \cl{2} with $6^+$ clubs and a void
elsewhere (likely to be partner's suit $\smiley$).

After (\nt{1})--\di{2}--\emph{Pass}--???, responder can jump in a
major suit to invite game or pass with $6^+$ diamonds and a disaster
in the majors.

\pagebreak

\section{Gadgets and Conventions}

\subsection{$\beta$ control-asking bid}

\hypertarget{beta} A $\beta$ control-asking bid can occur either after
a \nt{1} rebid by the \cl{1} opener over a positive suit response or
by cue-bidding a singleton suit after a positive response of \cl{3},
\di{3}, \he{3}, \sp{3}, \cl{4} or \di{4} over \cl{1}.

The number of controls held (\emph{A=2}, \emph{K=1}) are shown in
steps as below:

\begin{longtable}{ p{1.5cm}p{9.5cm}}
  \hline
  \emph{1 step} & 0-2 controls. A relay bid by opener in the cheapest
                  suit over the 1-step response will then ask for
                  clarification with responses in steps: \\
                & \begin{tabular}{ll}
                    \emph{1 step} & No controls. \\
                    \emph{2 steps} & 1 control. \\
                    \emph{3 steps} & 2 controls. \\
                  \end{tabular} \\
  \emph{2 steps} & 3 controls. \\
  \emph{3 steps} & 4 controls. \\
  \emph{4 steps} & 5 controls. \\
  \hline
\end{longtable}

When responder has already shown controls as less than four or at
least four, e.g., after an unusual positive, a modified scale of
responses is used as follows:

\begin{longtable}{ p{3cm}p{4cm}p{4cm}}
  \emph{Known to have}  & \emph{0-3 controls} & \emph{$4^+$ controls} \\
  \hline
  \emph{1 step}  & No controls. & 4 controls. \\
  \emph{2 steps} & 1 control.   & 5 controls. \\
  \emph{3 steps} & 2 controls.  & 6 controls. \\
  \emph{4 steps} & 3 controls.  & 7 controls. \\
  \emph{5 steps} &              & 8 controls. \\
  \hline
\end{longtable}

\subsection{$\gamma$ trump-asking bid}

\hypertarget{gamma} The $\gamma$ trump-asking bid is used to determine
the quality of responder's suit, i.e., whether the suit will provide
enough tricks in no-trumps or as a trump suit.

\begin{longtable}{p{1.5cm}p{9.5cm}}
  \multicolumn{2}{l}{\emph{Responses to a $\gamma$ trump-asking bid}} \\
  \hline
  \emph{1 step} & No top honour, $5^+$-card suit. \\
  \emph{2 steps} & 5-card suit, 1 honour. \\
  \emph{3 steps} & 5-card suit, 2 honours. \\
  \emph{4 steps} & 6-card suit, 1 honour. \\
  \emph{5 steps} & 6-card suit, 2 honours. \\
  \emph{6 steps} & AKQ of suit, $5^+$-card suit. \\
  \hline
\end{longtable}

The $\gamma$ bid may be repeated to get clarification on the first
response.

\begin{longtable}{p{4cm}p{7cm}}
  \emph{Honours already shown} & \emph{Responses to repeat $\gamma$-ask} \\
  \hline
  \emph{Zero} or
  \emph{AKQ} & \begin{tabular}{lp{5cm}}
                 1 step & $7^+$-card suit \\
                 2 steps & 6-card suit \\
                 3 steps & 5-card suit \\
               \end{tabular} \\
  \hline
  \emph{One} & \begin{tabular}{lp{5cm}}
                 1 step & \emph{Ace} \\
                 2 steps & \emph{King} \\
                 3 steps & \emph{Queen} \\
               \end{tabular} \\
  \hline
  \emph{Two} & \begin{tabular}{lp{5cm}}
                 1 step & \emph{AK} \\
                 2 steps & \emph{AQ} \\
                 3 steps & \emph{KQ} \\
               \end{tabular} \\
  \hline
\end{longtable}

\subsection{$\epsilon$ suit control-asking bid}

\hypertarget{epsilon} The $\epsilon$ asking bid is used to find out
what controls the responder holds in a specific suit. It follows a
$\gamma$ asking bid and terminates only when trumps or no-trumps is
bid, i.e., any other suit bid is an $\epsilon$-ask in that suit.
Responses are in steps:

\begin{longtable}{p{1.5cm}p{9.5cm}}
  \multicolumn{2}{l}{\emph{Responses to a $epsilon$ suit control-asking bid}} \\
  \hline
  \emph{1 step} & No control---\emph{Jxx} or worse. \\
  \emph{2 steps} & Third round control---\emph{Q} or doubleton. \\
  \emph{3 steps} & Second round control---\emph{K} or singleton. \\
  \emph{4 steps} & First round control---\emph{A} or void. \\
  \emph{5 steps} & \emph{AK} or \emph{AQ}. \\
  \hline
\end{longtable}

The $\epsilon$-ask can be repeated if it is important to know whether
the control is based on shortness or strength. The response is again
in steps---the first step showing that the previous response was based
on \emph{shortness} and the second step showing \emph{strength}.

If the first $\epsilon$ bid is at the level of \cl{5} or higher, a
compressed scale of responses is used where the first step shows no
control, second step shows second-round control and the third step
shows first-round control.

\subsection{Bergen raises}

\hypertarget{bergen} After a \he{1} or \sp{1} opening, responses of
\cl{3}, \di{3}, \he{3} and \sp{3} show different types of 4-card
support. The mnemonic \emph{CLAP} (Constructive, Limited and
Preemptive) helps to remember the order of the bids.

\begin{longtable}{p{1.5cm}p{9.5cm}}
  \hline
  \emph{\he{1}--\cl{3}} & Constructive, 7-10\hcp, 4-card \he{}. \\
  \emph{\he{1}--\di{3}} & Limited, 10-12\hcp, 4-card \he{}. \\
  \emph{\he{1}--\he{3}} & Preemptive, 0-6\hcp, 4-card \he{}. \\
  \emph{\sp{1}--\cl{3}} & Constructive, 7-10\hcp, 4-card \sp{}. \\
  \emph{\sp{1}--\di{3}} & Limited, 10-12\hcp, 4-card \sp{}. \\
  \emph{\sp{1}--\he{3}} & \emph{Spare bid}---used to show a strong
                          $12^+$\hcp\ hand with 4-card support and an
                          undisclosed singleton/void. \\
  \emph{\sp{1}--\sp{3}} & Preemptive, 0-6\hcp, 4-card \sp{}. \\
  \hline
\end{longtable}

\subsection{Gerber ace-asking convention}

\hypertarget{gerber} An immediate response of \cl{4} to any no-trump
bid (or overcall) is the \emph{Gerber} ace-asking convention. A jump
rebid of \cl{4} in response to a natural no-trump bid is \emph{Gerber}
as also when a trump suit has not been identified and no-trumps has
been rebid. Gerber should not be used holding a void.

Opener shows the number of aces held in steps as follows:

\begin{longtable}{p{1.5cm}p{9.5cm}}
  \hline
  \di{4} & Zero or four aces. \\
  \he{4} & One ace. \\
  \sp{4} & Two aces. \\
  \nt{4} & Three aces. \\
  \hline
\end{longtable}

\subsection{Jacoby transfers}

\hypertarget{jacoby} After a \nt{1} opening, responder bids \di{2}
with a 5-card or better heart suit and \he{2} with spades. Opener will
bid \he{2} or \sp{2} so that the strong hand becomes
declarer. Responder's rebids are:

\begin{longtable}{p{2.5cm}p{8.5cm}}
  \hline
  \emph{Pass} & A weak hand with $5^+$-card \he{} or \sp{}. \\
  \sp{2} & Invitational with 5-5 in the majors after
           \nt{1}--\di{2}--\he{2}. \\
  \nt{2} & Balanced or semi-balanced hand with a 5-card
           major. Invitational---partner can pass or sign-off in 3 of
           a major or bid \nt{3}. \\
  \gf{\cl{3}},
  \gf{\di{3}} & 4-card suit in addition to 5-card major. \\
  \emph{Raise} & 6-card suit, invitational. \\
  \gf{\he{3}} & (After \sp{2}) 5-5 in the majors with slam
                interest. Stronger than an immediate jump to
                \he{4}. \\
  \gf{\sp{3}} & (After \he{2}) Singleton or void with slam
                interest. \\
  \nt{3} & Balanced or semi-balanced hand. Partner can pass or correct
           to 4 of major. \\
  \he{4} & (After \sp{2}) 5-5 in majors with no slam interest. Partner
           can pass or correct to \sp{4}. \\
  \emph{Double raise} & $6^+$-card major, sign-off. \\
  \nt{4} & Quantitative, inviting slam in major or no-trumps. \\
  \hline
\end{longtable}

\subsubsection{Super-acceptance of a transfer}

\hypertarget{superaccept} Opener could \emph{super-accept} the
transfer with a $4^+$-card holding in the transfer suit. In this case,
he can break the transfer and show any doubletons.  For example, after
\nt{1}--\he{2} (transfer to \sp{2}), opener with a 4-card spade suit
could bid:

\begin{longtable}{p{2.5cm}p{8.5cm}}
  \hline
  \orf{\nt{2}} & 4=3=3=3 shape. \\
  \orf{\cl{3}} & 4=x=x=2 (doubleton club). \\
  \orf{\di{3}} & 4=x=2=x (doubleton diamond). \\
  \orf{\he{3}} & 4=2=x=x (doubleton heart). \\
  \sp{3} & 4 spades, any other distribution. \\
  \hline
\end{longtable}

\subsection{Jacoby 2NT}

\hypertarget{jacoby2nt} A \nt{2} response over an opening of \he{1} or
\sp{1} is conventional and shows $12^+$\hcp\ with $4^+$-card support
of partner's suit and is forcing to game. Opener's rebids are:

\begin{longtable}{p{3.5cm}p{7.5cm}}
  \hline
  \emph{New suit} & Singleton or void in suit bid. \\
  \emph{Rebid of suit at 3-level} & Maximum strength hand. \\
  \emph{Jump shift} & Good $5^+$-card side suit. \\
  \emph{Game in original suit} & Minimum opening, sign-off. \\
  \nt{3} & 12-13 \hcp, medium strength hand. \\
  \hline
\end{longtable}


\subsection{lebensohl Convention (over weak two)}

\hypertarget{lebensohl:weak} The \emph{lebensohl} convention is used
to respond to partner's takeout double of a weak two opening. This
system is geared to show weak, invitational and strong hands without
consuming too much bidding room.

\begin{longtable}{>{\raggedright}p{2cm}p{9cm}}
  \hline
  \orf{\nt{2}} & Artificial, partner must bid \cl{3}. Over partner's
                 forced response, responder bids: \\
         & \begin{tabular}{lp{6.3cm}}
             \emph{Pass} & 0-7\hcp, sign-off in clubs. \\
             \di{3},
             \he{3},
             \sp{3} & 0-7\hcp, correction to longest suit when not
                      clubs, sign-off. \\
             \gf{\emph{Cue-bid}} & $4^+$-cards in unbid major (or one
                                   of the majors if double was over two
                                   of a minor), confirms stopper in
                                   opponent's suit. \\
             \gf{\nt{3}} & Denies $4^+$-cards in majors, confirms
                           stopper in opponent's suit. \\
           \end{tabular} \\
  \cl{3},
  \di{3},
  \he{3},
  \sp{3} & 8-11\hcp, $4^+$-card suit, invitational. \\
  \gf{\emph{Cue-bid}} & $4^+$-cards in unbid major (or one major if
                        double was over 2 of a minor), no stopper in
                        opponent's suit. \\
  \gf{\nt{3}} & Denies $4^+$-cards in majors, no stopper in opponent's
                suit. \\
  \hline
\end{longtable}

\emph{lebensohl} applies even when \emph{RHO} raises opener to the
2-level e.g., over the sequence
(\sp{1})--\emph{Double}--(\sp{2})--???.  In this case, since responder
is not forced to bid, with the weak 0-7\hcp\ hands, responder wil
\emph{pass}. The suit bids after the \nt{2}--\cl{3} sequence now
become invitational hands and the direct raises to three of a suit
become game forces.

\subsection{lebensohl convention (after 1NT)}

\hypertarget{lebensohl:nt} A different \emph{lebensohl} convention is
used by responder after an opponent overcalls partner's opening \nt{1}
opening in order to compete further without necessarily committing to
game. It is initiated after the right-hand opponent makes a suit
overcall at the two-level.

\begin{longtable}{>{\raggedright}p{2.5cm}p{8.5cm}}
  \hline
  \emph{New suit
  at 2-level} & Natural and non-forcing. \\
  \orf{\nt{2}} & A puppet bid requiring opener to bid \cl{3}. After
                 opener's forced \cl{3} bid, \\
              & \begin{tabular}{p{8cm}}
                  --- 3 of a lower ranking suit than overcaller's is
                  natural, to play. \\
                  --- 3 of a higher ranking suit than overcaller's is
                  natural and invitational. \\
                  --- 3 of the opponent's suit is artificial asking
                  opener to bid a 4-card major and showing a stopper
                  in opponent's suit. \\
                  --- 3NT is natural, to play, and shows a stopper in
                  the opponent's suit. \\
                \end{tabular} \\
  \gf{\emph{New suit
  at 3-level}} & Natural and forcing to game. \\
  \orf{\emph{Cue-bid}} & Artificial---asks opener to bid a 4-card
                         major and denies a stopper in opponent's
                         suit. \\
  \nt{3} & Natural, to play, and denies a stopper in opponent's
           suit. \\
  \hline
\end{longtable}

\subsection{Michael's cue-bid}

\hypertarget{michaels} A cue-bid in the opponent's suit after they
have opened the bidding is conventional and shows a two-suited hand
(5-5 or better). Over a minor-suit opening (i.e., \cl{2}/\di{2}) shows
a major two-suiter and over a major-suit (i.e., \he{2}/\sp{2}), it
shows the unbid major and an unspecified minor.

\begin{longtable}{p{2.5cm}p{8.5cm}}
  \hline
  \multicolumn{2}{l}{\emph{\underline{After a cue-bid of \cl{2}/\di{2}}}} \\
  \di{2} & Natural, showing a very good $6^+$-card suit. Tends to deny
           3 cards in either major. Non-forcing. \\
  \he{2},
  \sp{2} & Non-forcing sign-off. With support for both majors,
                 bid \he{2}. \\
  \nt{2} & Natural, invitational. \\
  \orf{\cl{3}},
  \orf{\di{3}} & When not a cue-bid, shows a very strong 6-card
                 suit. \\
  \orf{\emph{Cue-bid}} & Artificial, shows game or slam interest. \\
  \he{3},
  \sp{3} & Preemptive, usually with 4-card suit in accordance with
           the Law of Total Tricks. Non-forcing. \\
  \nt{3} & Natural, sign-off. Shows a big, balanced hand with no
           interest in a major-suit game. Rare. \\
  \multicolumn{2}{l}{\emph{\underline{After a cue-bid of \he{2}/\sp{2}}}} \\
  \sp{2} & Over \he{2}, is a sign-off. \\
  \orf{\nt{2}} & Asks opener to bid his minor suit. Made on a variety
                 of hands but is usually to sign-off in \cl{3} or
                 \di{3} with support in both minors. \\
  \cl{3},
  \di{3} & Natural, $6^+$-card suit---opener probably has the other
           minor. Non-forcing. \\
  \he{3} & Over \sp{2}, natural sign-off. \\
  \sp{3} & Over \he{2}, preemptive with 4-card suit. \\
  \orf{\emph{Cue-bid}} & Artificial showing game or slam interest. \\
  \hline
\end{longtable}

If responder raised the cue-bid, e.g.,
(\he{1})--\he{2}--\emph{Pass}--\he{3}, the Michael's cue-bidder is
expected to bid his cheapest suit (in this case, \sp{3}) with a weak
hand of 0-10\hcp. All other bids show $10^+$\hcp\ and are game
forcing.

\subsection{Roman key-card Blackwood}

\hypertarget{blackwood} A \emph{Roman key-card Blackwood} bid of
\nt{4} is used to enquire about the number of key cards (any ace or
the trump suit king) in partner's hand. It should not be used when you
have a void or two fast losers.

Responses are in steps and differ slightly (when holding all five key
cards) depending on whether opponents have doubled or overcalled over
\nt{4}.

\begin{longtable}{>{\raggedright}p{3.2cm}|p{1cm}>{\raggedright}p{3cm}>{\raggedright\arraybackslash}p{3.2cm}}
  \emph{Holding} & \emph{Silent} & \emph{Double (ROPI)} & \emph{Overcall (DOPI)} \\
  \hline
  \emph{1/4 key cards} & \cl{5} & \emph{Redouble} & \emph{Double} \\
  \emph{0/3 key cards} & \di{5} & \emph{Pass} (\underline{0/3/5} key cards) & \emph{Pass} (\underline{0/3/5} key cards) \\
  \emph{2/5 key cards (no queen of trumps)} & \he{5} & \cl{5} (\underline{two} key cards) & \emph{Cheapest suit} (\underline{two} key cards) \\
  \emph{2/5 key cards with queen of trumps} & \sp{5} & \di{5} (\underline{two} key cards) & \emph{Second-cheapest suit} (\underline{two} key cards) \\

  \hline
\end{longtable}

When holding a void, after a trump suit is agreed, jumping to another
suit at the 4 or 5 level in the void suit initiates a \emph{key card
  exclusion} asking bid. Partner shows his key cards \emph{excluding}
any in the void suit in steps.

\begin{longtable}{p{1.5cm}p{9.5cm}}
  \hline
  \emph{1 step} & 1 or 4 key cards. \\
  \emph{2 steps} & 0 or 3 key cards. \\
  \emph{3 steps} & 2 key cards without trump Q. \\
  \emph{4 steps} & 2 key cards with trump Q. \\
  \hline
\end{longtable}

\subsubsection{Asking for the Trump Queen}

After a \cl{5} or \di{5} response to the \nt{4} \emph{RKCB} bid,
bidding the cheapest non-trump suit asks about the trump
queen. Responses to the queen-ask are:

\begin{longtable}{p{3cm}p{8cm}}
  \hline
  \emph{Trump suit} & No trump queen. \\
  \emph{Non-trump suit} & Trump queen \emph{and} king in suit bid. \\
  \nt{5} & Trump queen without any side-suit kings. \\
  \hline
\end{longtable}

\subsubsection{Asking for Kings}

A \nt{5} rebid following \emph{RKCB} asks for any kings held. It is a
grand slam try, and should only be bid if the partnership holds all 5
key cards and the queen of trumps.

Partner will respond by bidding his cheapest king.

\subsubsection{Modified RKCB (\cl{4} over \he{2}/\sp{2})}

\hypertarget{blackwoodmod} A bid of \cl{4} is used to ask for key
cards after either an opening weak two bid or a weak two response to a
\cl{1} opening. The response structure is based on regular \emph{RKCB}
taking into account the fact that the responder cannot have more than
two key cards.  The responses are slightly different depending on
whether the suit is hearts or spades.

When spades are trumps, the $1^{st}$ step (\di{4}) shows 0 or 2 key
cards, the $2^{nd}$ step (\he{4}) shows one key card without the
\sp{}\emph{Q} and the $3^{rd}$ step (\sp{4}) shows one key card with
the \sp{}\emph{Q}.  If partner signs off in \sp{4} after a 0/2
response, opener bids \nt{4} holidng the \sp{}\emph{Q} allowing
responder to push on to slam if possible.

When hearts is the agreed suit, the $1^{st}$ step (\di{4}) shows 0 or
2 key cards and the $2^{nd}$ step (\he{4}) shows 1 key card. If
partner wants to ask about the \he{}\emph{Q}, he bids \sp{4}. Without
the queen, opener signs off in \he{5} otherwise he bids another suit
showing a side suit king or \nt{5} with no side suit king.

\subsection{Stayman convention}

\hypertarget{stayman} The \emph{Stayman} convention is used to find a
4-4 major suit fit after a \nt{1} opening by bidding \cl{2}. Opener
responds with one of:

\begin{longtable}{p{1.5cm}p{9.5cm}}
  \hline
  \di{2} & No four card major. \\
  \he{2} & 4-card heart suit, may have 4-card spades. \\
  \sp{2} & 4-card spade suit, no 4-card heart suit. \\
  \hline
\end{longtable}

\subsubsection{Puppet Stayman}

\hypertarget{puppetstayman} Used over a \nt{2} bid that may be made
with a 5-card major, \cl{3} is a conventional bid that endeavours to
find a major suit fit. Responses are:

\begin{longtable}{p{1.5cm}p{9.5cm}}
  \hline
  \di{3} & No five card major but at least one four card
           major. Responder with one four card major should bid the
           major that he \emph{does not have} and if there is a 4-4
           fit, opener will bid it. \\
  \he{3} & 5-card heart suit. \\
  \sp{3} & 5-card spade suit. \\
  \nt{3} & No 4 or 5-card major. Responder can bid \cl{4} or \di{4} to
           transfer to \he{4} or \sp{4} respectively when he has a
           six-card major. \\
  \hline
\end{longtable}

\subsection{Unusual 2NT}

\hypertarget{unusualnt} A jump to \nt{2} over opponent's opening of
one of a suit is conventional and shows a two-suited hand in the two
lowest ranking unbid suits. E.g., a \nt{2} overcall of a \cl{1} opener
shows hearts and diamonds. The \emph{unusual \nt{2}} is possible even
when both opponents bid.

For example, (\he{1})--\emph{Pass}--(\sp{1})--\nt{2} shows $5^+$ clubs
and $5^+$ diamonds. There is an overlap with the Michael's cue-bid in
this case since a cue-bid of \he{2} would also show the same shape.

As another example, (\cl{1})--\emph{Pass}--(\nt{1})--\nt{2} would show
$5^+$ hearts and $5^+$ diamonds.

Responder's bids:
\begin{longtable}{p{1.5cm}p{9.5cm}}
  \hline
  \emph{Better suit} & Longer of overcaller's two suits. With equal
                       length in both suits (especially with two
                       doubletons), bid the cheapest suit. \\
  \emph{Jump} & 4-card support for suit, preemptive, non-forcing. \\
  \emph{Cue-bid} & Support for at least one of partner's suits and
                   interest in game or slam. Asks \nt{2} bidder to bid
                   the cheapest suit with 0-10\hcp\ or to make any
                   other bid with $10^+$\hcp. \\
  \emph{$4^{th}$ suit} & Bidding the suit not shown by partner or
                         opponent shows a $6^+$-card suit, a decent
                         hand and no support for either of partner's
                         suits. \\
  \hline
\end{longtable}

All raises by the \nt{2} bidder are natural and non-forcing. All other
bids show a very good hand in terms of points and/or shape but with
the exception of a cue-bid, are non-forcing.

\pagebreak

\section{Miscellaneous}

\emph{High-card Points} (\hcp) are assigned as follows---Ace=4,
King=3, Queen=2 and Jack=1.  Once a trump-fit has been found,
distribution points can be assigned---Void=3, Singleton=2,
Doubleton=1.

Singleton honours should be counted only once (either \hcp\ or
shortness).

\gap

\emph{Suit Quality} (\sq) is calculated as suit length plus number of
honours in the suit. The Jack or Ten should be counted only if a
higher ranking honour is held. For example, a holding of K-J-9-5-4
would have 7\sq\ but J-10-9-5-4 would have 5\sq.

For an overcall, the \sq\ should equal or exceed the number of tricks
bid (e.g., \sq\ of 8 for a 2-level overcall).

When preempting, the \sq\ should equal the level of preempt when
vulnerable and can be one less when non-vulnerable.

\gap

\emph{Rule of 2/3/4} When preempting, the expected number of undertricks should
be no more than two with unfavourable vulnerability, three with equal vulnerability
and four with favourable vulnerability.

For example, a \di{3} preemptive bid is made when expecting to take at least seven
tricks in the case that the opening bidder is vulnerable facing non-vulnerable opponents.

\gap

The \emph{Losing Trick Count} (\ltc) is used only once a trump suit
has been established. Count losers only in the top three cards of the
suit holding---there are never more than 3 losers in a suit. With
three or more cards, A/K/Q are not losers but any lower card is a
loser. With two cards, only A or K are not losers.

Add your and partner's loser count and subtract from 24 to estimate
the number of tricks that can be won.  You can estimate your partner's
\ltc\ as follows:

\begin{tabular}{rp{3cm}}
  \emph{\hcp{}} & \emph{Expected Losers} \\
  \hline
  \emph{7-9} & 8-9 losers (9) \\
  \emph{10-12} & 7-8 losers (8) \\
  \emph{13-15} & 6-7 losers (7) \\
  \emph{16-18} & 5-6 losers (6) \\
  \emph{19-21} & 4-5 losers (5) \\
  \emph{$22^+$} & 4 losers or less \\
  \hline
\end{tabular}

\pagebreak

\section{Bidding Examples}
\setboolean{betweencards}{true} % spaces between cards in hand diagrams
\setboolean{leadingspace}{true}

\subsection{Negative response to \cl{1}}

\hypertarget{ex1c1d}
After a \di{1} response, there is no temptation to get too high on
misfitting hands. For example,

\vhand[West]{4,AK954,AJ4,KQT9}\vhand[East]{KJT753,62,753,54}
\ewauction{1c,1d,1h,1s(1),2c(2),2s(3)}\\ (1) 4-7\hcp, 4+-card
suit. \\ (2) Shows minimum with second 4-card suit (implies 5
hearts). \\ (3) 6-card suit, no fit.

Opener shows discipline and passes recognising misfit and no chance
for game.

\gap
A \di{1} response does not rule out game. With a 2-suited hand,
it is easy to find a game contract when the fit is in the second bid
suit.

\vhand[West]{AK752,AQT43,A5,2}\vhand[East]{4,K852,9642,J754}
\ewauction{1c,1d,1s,1n(1),2h,3h,4h} \\ (1) As he has already
limited his hand, East is not afraid to improve the contract. After
that, all goes smoothly.

Suppose the lead against \he{4} is a low diamond. The best technique
for declarer is to win with the Ace, cash \sp{}A and ruff a spade with
a low trump. Then he plays a club to establish communication between
the two hands.

The opponents will probably continue diamonds. West ruffs the third
round and leads another low spade, ruffing with the \he{}8. If, at
worst, spades are 5-2 and South overruffs, declarer retains the
possibility of ruffing the other spade loser with the \he{}K. The
contract will fail only against very unlucky distribution.

\gap
With a powerful hand, opener would jump rebid his suit and responder
would know there is a game or slam on if he is in the upper range. For
example,

\vhand[West]{AKJ8753,A,K72,AQ}\vhand[East]{642,J73,AJ54,865}
\ewauction{1c,1d,2s(1),3d(2),3s(3),4s,4n,5c(4),6s} \\
(1) 22+\hcp, 5-card suit. \\ (2) 4-7\hcp, showing side-suit before
showing fit in spades. \\ (3) After the new suit bid at the 3-level,
opener knows he will not be left in \sp{3}. \\ (4) One key card.

After a heart lead and assuming trumps don't break worse than 2-1, the
contract can be made without the club finesse by playing \emph{A} and
\emph{K} of diamonds followed by a low diamond towards the
\emph{J}. This works whenever diamonds break 3-3, \di{}\emph{Q} is
held by North or is a doubleton with South.

\gap
With a balanced hand, opener will rebid \nt{}. For example,

\vhand[West]{K63,KJT,A862,AK3}\vhand[East]{AJ742,754,J95,T4}
\ewauction{1c,1d,1n(1),2h(2),2s,2n(3),3n(4)} \\
(1) 17-19\hcp, balanced. \\ (2) Weak transfer to \sp{2}. \\ (3)
Balanced hand, invitational. \\ (4) With 18\hcp\ and three cards in
spades, East tries for game in no trumps.

North leads the \he{}3, South plays the Queen and West the King. After
this favourable opening, West can afford to make a safety play in
spades. He plays King and another, North following suit with low
cards. To make absolutely sure of four tricks, even when North holds
Q10xx, declarer ducks in dummy. He makes game with four spades, two
hearts and three top tricks in the minors.

\gap
Even with a weak two-suiter, Precision enables finding slams with
relatively low point counts following a negative response. For
example,

\vhand[West]{A854,AK943,AKJ8,}\vhand[East]{6,J87652,Q976,64}
\ewauction{1c,1d,1h,3s(1),4c,4h(2),6h} \\ (1) Splinter
showing 4-card or better support in hearts and a singleton or void
in spades. \\ (2) Responder could conceivably also bid \he{5}
with the 6-card suit.

With a combined total of 22 points, although 13 tricks are available
if the opening lead is not ruffed, most pairs will probably stop in
\he{6} after the splinter bid using a sequence similar to the one
above.

\subsection{Positive response in a suit over \cl{1}}

\hypertarget{ex1csuit}
Using Precision, game is always reached after a positive response to a
\cl{1} opening.  The partnership will have a minimum of 24 points if
opener is unbalanced (16 vs 8) or 25 points if he is balanced (17 vs
8). This works well in practice, for example:

\vhand[West]{AKJ86,64,KQT9,K6}\vhand[East]{Q92,875,A543,Q94}
\ewauction{1c,1n,2s,3s,4s} \\
A dull 16\hcp\ \cl{1} opening against an equally dull 8\hcp\ but still
\sp{4} is an odds-on favourite to make.

\gap
Game contracts can be reached on smaller point counts if there are
distributional features. For example,

\vhand[West]{AKJT96,A,QJT9,65}\vhand[East]{Q82,965,K743,743}
\ewauction{1c(1),1d,1s,3s(2),4s} \\ (1) A strong 15\hcp\ with a
good suit should be opened with \cl{1}.\\ (2) As he has already
limited his hand, East is not shy in raising partner's suit with 5\hcp
and inviting game.

As compared to the previous deal, this is a 15\hcp\ vs 5\hcp\ hand
that may be passed out after \sp{1} in standard systems. However, the
game contract is virtually lay-down.

\gap
With a balanced hand, opener will rebid \nt{} over a positive suit
response.  Even with 3-card support for partner's suit, it is
sometimes correct to first bid \nt{} and only later raise partner's
suit. For example, with \hhand{AJT,KQT9,QJ4,KJ7}, if responder bids
\sp{1}, it is correct to rebid \nt{1} showing a balanced minimum
before raising spades. However, with a slightly different hand such as
\hhand{AJT7,KQT,QJ4,KJ7}, the rebid could be \sp{3} or \sp{4} showing
a minimum hand, probably balanced, with 4-card support.

Alternative sequences showing support have slightly different
meanings.  For example, whereas the sequence \cl{1}--\sp{1}--\sp{4}
would show a minimum hand with poor controls, the sequence
\cl{1}--\sp{1}--\nt{1}--any--\sp{4} would show a balanced minimum with
good controls.

The intermediate \nt{1} rebid can also be made when you want to find
out if responder has a distributional hand. For example, when holding
\hhand{AK87,A753,KQ4,A6}, after partner's positive response of \sp{1},
rebid \nt{1} and if partner rebids \cl{2} (four-card suit), you may
have very good play for \sp{7} if partner is holding something like
\hhand{QJ543,82,A8,K954}.  However, you need to know about the four
clubs first.

\gap
With a distributional hand where you have strong support for partner's
suit and the only question for slam is whether his suit has good
quality, \emph{asking bids} ($\gamma$ and $\epsilon$) can be used to
good effect. For example,

\vhand[West]{QJ632,5,AKQ8,KJ9}\vhand[East]{AKT54,987,T4,A53}
\ewauction{1c,1s,2s(1),3h(2),4c(*),4n(3),5h(*),5s(4),6s} \\
(1) $\gamma$ trump-asking bid (possible slam if trumps are strong). \\
(2) 2 honours, 5-card suit. \\ (*) $\epsilon$ control-asking bid in clubs and hearts. \\
(3) Ace or void. \\ (4) No control.

With a sure loser in hearts, opener stops in the small slam.

\subsection{Positive no-trump response to \cl{1}}

\hypertarget{ex1cnt}
With both majors, it is sometimes correct to use \emph{Stayman} even
when holding a 5-card suit.  For example, holding
\hhand{AKQ64,KQ87,A5,95}, it is better to bid \cl{2} over a \nt{1}
response rather than bidding \sp{2}. If responder holds something like
\hhand{JT2,AJ94,543,Q43}, he will certainly raise spades after \sp{2}
and the 4-4 heart fit will not be discovered. In this case although
there are 10 tricks in spades and 11 in hearts (given normal breaks),
sometimes the difference may be 10 tricks in the 4-4 fit versus 9 in
the 5-3 fit.

Similarly, with \hhand{3,AKQ7,AQ,KQJT98}, bid \emph{Stayman}. If
partner bids \di{2} (four hearts), you will bid \he{2} and later ask
for aces. If partner has two aces, you can confidently bid the grand
slam or the small slam if he shows only one ace. If partner holds
something like \hhand{AQ6,JT86,J76,543}, \he{6} from the strong side
is best, while \cl{6} will depend on the diamond finesse.

\subsection{\sp{3} response to \cl{1}}

\hypertarget{ex1c3s}
Opener can place the contract fairly easily given responder's solid
suit and use asking bids to decide if a slam is on. For example,

\vhand[West]{4,AT987,A4,AKQ87}\vhand[East]{AKQJ987,3,K7,T96}
\ewauction{1c,3s(1),4c(2),4h(3),7s(4)} \\
(1) Solid suit. Opener can tell that it is spades by looking
at his own hand. \\ (2) $\beta$-ask for outside controls. \\
(3) One outside control (\di{} or \he{} king). \\ (4) 13 tricks
are on top.

\subsection{Unusual positive response to \cl{1}}

\hypertarget{ex1c3c}
If responder bids an \emph{unusual positive}, slam is most likely on
the cards and with the right cards, grand slams can be reached on very
low point counts.

\vhand[West]{AKQ876,976,AK43,}\vhand[East]{J543,A,T987,AK43}
\ewauction{1c,4d(1),4h(2),4n(3),7s(4)} \\ (1) 4-1-4-4,
$4^+$-controls, $12^+$\hcp \\ (2) $\beta$ asking for controls \\
(3) 5 controls (2 steps) \\ (4) Partner must have two aces and
\cl{}\emph{K}, 13 tricks are visible.

Barring horrendous breaks and a ruff on the opening lead, this
27-point grand slam is lay-down.

\subsection{Intervention after a \cl{1} opening}

\hypertarget{ex1cintervene}
Some examples of bidding after opponents double or overcall after a \cl{1}
opening.

\begin{longtable}{rp{11cm}}
  \multicolumn{2}{l}{\emph{\underline{After a takeout / unusual double: \cl{1}--(Double)}}} \\
  1 & \hhand{J84,AJ82,T5,KT42} \\
    & If the double is an ordinary takeout double either
      \emph{Redouble} or bid \nt{1} showing a balanced 8-13\hcp\ if
      vulnerable.

      If the double shows majors, \emph{Redouble}. If partner doubles
      \sp{1}, you will be delighted to defend. \\
  2 & \hhand{A87,8,KJ8654,Q63} \\
    & Bid \di{2}. Slam is a real possibility despite the double. \\
  3 & \hhand{QT3,,JT9753,QT93} \\
    & Bid \di{1} (5-8\hcp). If partner bids \he{1}, you will bid
      \di{2} showing the long suit. \\
  4 & \hhand{AQ,A863,QJT,JT85} \\
    & Bid \nt{2} showing a balanced $14^+$\hcp\ hand and good stoppers
      in the majors. If the double is real (not a mistake showing
      clubs), the information of length in majors on the right is
      likely to be useful in the play. \\
  5 & \hhand{6,KJT5,A732,JT87} \\
    & Bid \cl{3} showing the 4-4-4-1 hand with a black singleton. \\
\end{longtable}

\begin{longtable}{rp{11cm}}
  \multicolumn{2}{l}{\emph{\underline{After a direct 1-level overcall: \cl{1}--(\sp{1})}}} \\
  6 & \hhand{Q,AQJ32,KJ63,J97} \\
    & Bid \he{2} which is natural and game forcing. \\
  7 & \hhand{4,J8654,T976,KJ6} \\
    & \emph{Double} to show 5-8\hcp.\\
  8 & \hhand{T953,4,A764,AQ92} \\
    & Bid \di{3}, unusual positive showing 4-4-4-1 with a red
      singleton. \\
  9 & \hhand{J,Q652,AQT964,T2} \\
    & Bid \di{2}, natural and forcing. \\
  10 & \hhand{953,AT43,AJ72,95} \\
    & Bid \sp{2}. There is enough to force game but no suit to bid and
      no stopper to bid \nt{}. \\
\end{longtable}

\begin{longtable}{rp{11cm}}
  \multicolumn{2}{l}{\emph{\underline{After an unusual no-trump overcall showing minors: \cl{1}--(\nt{1})}}} \\
  11 & \hhand{T9,AQ64,K862,AQ5} \\
     & \emph{Double} for penalties. If partner bids hearts, explore for
       slam. If not, you can also bid no-trump since the combined hands
       are in the slam zone. \\
  12 & \hhand{AT942,Q4,743,T98} \\
     & Bid \sp{2} (non-forcing). \\
  13 & \hhand{K9743,AQ98,92,87} \\
     & Bid \di{2} showing spades and forcing to game. \\
  14 & \hhand{AQ2,AT82,KJ3,874} \\
     & \emph{Double} showing values with a balanced hand. If partner
       bids \nt{2}, you can show the 4-card hearts on the way to
       \nt{3}. \\
  15 & \hhand{A4,J98,KT64,KT87} \\
     & \emph{Double} (penalty oriented) showing values with a balanced
       hand. There will be a massacre if the final contract is in
       either minor. \\
\end{longtable}

\begin{longtable}{rp{11cm}}
  \multicolumn{2}{l}{\emph{\underline{After a 2-level overcall: \cl{1}--(\he{2})}}} \\
  16 & \hhand{AQT,85,K74,KQT96} \\
     & Bid \cl{3} which is natural and forcing. The main reason for
       not cue-bidding is that this hand will make an excellent dummy
       should partner bid \di{3} or \sp{3} which you will happily
       raise showing slam interest by bypassing \nt{3}. \\
  17 & \hhand{9872,52,AKJ4,T64} \\
     & \emph{Double}. This is more flexible than cue-bidding
       \he{3}. Partner can bid \nt{2} with a stopper and then you
       could bid \cl{3} (\emph{Stayman}). \\
  18 & \hhand{QJ432,A6,JT63,K4} \\
     & Bid \sp{2}. \\
  19 & \hhand{4,KJT94,QJ7,A732} \\
     & \emph{Pass}. You are certain partner will bid again and you
       hope it is a double. The penalty will be a rich one if so. \\
  20 & \hhand{AK64,8765,AKQ7,7} \\
     & Bid \he{3}---game-forcing with no heart stopper and no long
       suit. You can explore slam after getting more information from
       partner. \\
\end{longtable}

\subsection{The \di{1} opening}

\hypertarget{ex1d}
Some examples of bidding after a \di{1} opening.

\begin{longtable}{rp{11cm}}
  \multicolumn{2}{l}{\emph{\underline{Opening bid}}} \\
  1 & \hhand{63,K4,AKJ9,KT984} \\
    & Open \di{1} and if partner bids \he{1}, rebid
      \cl{2}. Alternatively, open \nt{1}. \\
  2 & \hhand{AJ76,2,AQJ62,T72} \\
    & Open \di{1} rebid \sp {1} if partner bids \he{1}. \\
  3 & \hhand{Q76,J3,AQ9,AT982} \\
    & Open \di{1} and rebid \nt{1} over \he{1}/\sp{1}. You cannot bid
      \cl{2} which would show an unbalanced hand. \\
  4 & \hhand{QT9,Q97,Q4,AJ962} \\
    & \emph{Pass} with this weak 11-point hand. \\
  5 & \hhand{65,T,AKQT8,KQT97} \\
    & Open \di{1} and rebid \cl{3} over \he{1}/\sp{1} showing 5-5 in
      the minors. \\
\end{longtable}

\begin{longtable}{rp{11cm}}
  \multicolumn{2}{l}{\emph{\underline{Responses to a \di{1} opening}}} \\
  6 & \hhand{97,AK5,QJ873,KQ5} \\
    & Bid \di{2} showing at least a limit raise. Raise to game over
      \nt{2} or find a forcing bid if opener rebids a minimum. You
      want partner to be declarer in \nt{} with the weak doubleton
      spade. \\
  7 & \hhand{Q95,5,AKQ532,K64} \\
    & Bid \he{3}---a splinter showing the singleton heart and fine
      diamond support. \\
  8 & \hhand{7,AK942,KQJ54,A8} \\
    & Bid \he{1} and use \emph{RKCB} if opener supports
      hearts. Otherwise, jump to \di{3} if opener responds with \nt{1}
      showing the two-suiter and indicating slam interest. \\
  9 & \hhand{6,AK74,42,AKT943} \\
    & Bid \cl{2} and hearts next in the search for the best game
      contract (or slam if opener raises clubs). \\
  10 & \hhand{76,9,AJT642,8532} \\
    & Bid \di{3} (or \di{4} if non-vulnerable) interfering with
      opponent's possible game. \\
\end{longtable}

\begin{longtable}{rp{11cm}}
  \multicolumn{2}{l}{\emph{\underline{Rebids after partner's one-over-one response: \di{1}--\sp{1}--???}}} \\
  11 & \hhand{82,75,AQ52,AKT65} \\
     & Rebid \cl{2}. \\
  12 & \hhand{6,KT,AJT87,KQJ92} \\
     & Rebid \cl{3} showing 5-5 in the minors. \\
  13 & \hhand{KT92,9,AKT64,K65} \\
     & Rebid \sp{3} showing strong support and a singleton / void. \\
  14 & \hhand{AT4,Q76,J964,AK8} \\
     & Rebid \nt{1}. Raising spades is inadvisable with this flat
       hand. \\
  15 & \hhand{KT4,4,QJ974,AKQ4} \\
     & Rebid \cl{2} as the least worst evil---if partner bids again,
       you can show the spade support. \\
\end{longtable}

\subsection{Major suit openings}

\hypertarget{ex1h}
Some examples of bidding after a \he{1} or \sp{1} opening.

\begin{longtable}{rp{11cm}}
  \multicolumn{2}{l}{\emph{\underline{Opening bid}}} \\
  1 & \hhand{AT9765,Q8,K6,K52} \\
    & The quintessential \sp{1} bid. \\
  2 & \hhand{KJT6,AKJT92,K8,9} \\
    & Open \cl{1}---there are 15\hcp, a very good suit and a
      singleton. With unfavourable vulnerability, it may be better to
      bid \he{1} since opponents may intervene at a high level after
      \cl{1}. \\
  3 & \hhand{QJ9654,KT5,K8,Q7} \\
    & Open \sp{1}---this is not a great hand and many may choose to
      pass it or open \sp{2}. \\
  4 & \hhand{QT752,A74,AJ7,A6} \\
    & Open \sp{1}. A case can be made for opening this hand with
      \nt{1} and with \hearts{KJ4} and \clubs{KJ} (same \hcp), it
      would be preferable to open \nt{1}. \\
  5 & \hhand{32,KQ8743,QJ6,AK} \\
    & Open \he{1}. Although there are 15 \hcp, the suit is not good
      enough to play against a singleton and the hand has no
      singletons of its own. \\
\end{longtable}

\begin{longtable}{rp{11cm}}
  \multicolumn{2}{l}{\emph{\underline{Responses to a \sp{1} opening}}} \\
  6 & \hhand{AJT9,KJ8,T97,KJ6} \\
    & Bid \sp{4}---it would be a very unusual hand with partner for
      there to be a slam. Opponents do not know if your hand type is a
      weak distributional hand or this one. \\
  7 & \hhand{AQ982,AT8,4,KT76} \\
    & Bid \di{4} (splinter) with real slam potential. \\
  8 & \hhand{98732,A5,Q,T9743} \\
    & Bid \sp{4}---the textbook example of a game raise. Contrast to
      hand \#6. \\
  9 & \hhand{K832,A65,AKJ9,74} \\
    & Bid \nt{2}---game-forcing raise showing at least 4-card
      support. If partner shows shortness in clubs or hearts, slam is
      a distinct possibility. \\
  10 & \hhand{AJ874,4,Q53,AT95} \\
    & Bid \he{4} (splinter). Another hand with good slam potential if
      partner's hand matches. \\
\end{longtable}

\begin{longtable}{rp{11cm}}
  \multicolumn{2}{l}{\emph{\underline{Responses to a \he{1} opening}}} \\
  11 & \hhand{QT632,K72,A532,T} \\
     & Bid \sp{1}. If partner raises, you can bid game. If partner
       bids \nt{1}, \cl{2} or \di{2}, you will show limit raise values
       with \he{3}. Partner will know you have only 3 hearts since
       ther was no direct raise. \\
  12 & \hhand{AJ763,972,AK753,} \\
     & Bid \sp{1} and if partner raises, you will explore slam. If
       partner bids \cl{2} (likely), you will bid \di{2} (fourth-suit
       forcing). If partner rebids \he{2}, you could bid \he{5}
       (asking about trump quality) or \cl{4} (splinter). This is a
       difficult hand to assess since opposite the first opening hand
       below, a grand slam is on but opposite the second, no game is
       possible.

       \vhand[Opener 1]{8,AKQ863,QJ7,T76}
       \vhand[Opener 2]{86,Q8543,J6,AKQ6} \\
  13 & \hhand{QT,AT98,432,Q965} \\
     & Bid \cl{3}---a constructive \emph{Bergen} raise. \\
  14 & \hhand{Q76,J876,,AJ9853} \\
     & Bid \he{4}. It is certain that the opponents have some high
       card points so this makes them start at a high-level if they
       are going to bid. \\
  15 & \hhand{A94,Q643,JT3,A62} \\
     & Bid \di{3}---a \emph{Bergen} limit raise. \\
\end{longtable}

\subsection{The \cl{2} opening}

\hypertarget{ex2c}
Some examples of bidding after a \cl{2} opening.

\begin{longtable}{rp{11cm}}
  \multicolumn{2}{l}{\emph{\underline{Opening bid}}} \\
  1 & \hhand{KJ62,3,92,AQJ982} \\
    & A good example of a hand that should be opened with a bid of
      \cl{2}. \\
  2 & \hhand{QT6,KQ6,63,AQ843} \\
    & Bid \di{1} not \cl{2}. \\
  3 & \hhand{K3,,AJ82,AQJT974} \\
    & Bid \cl{1}. This hand is too good for a \cl{2} opening. \\
  4 & \hhand{Q86,A6,T8,AKQ874} \\
    & Bid \cl{1} and rebid \cl{2}. Let partner be declarer in \nt{} if
      that is the right spot. \\
  5 & \hhand{62,87,QT,AKQJ982} \\
    & Bid \nt{3}---``Gambling'', showing a solid suit with no ace or
      king outside. \\
\end{longtable}

\begin{longtable}{rp{11cm}}
  \multicolumn{2}{l}{\emph{\underline{Responses to a \cl{2} opening}}} \\
  6 & \hhand{AKT6,J865,T9,976} \\
    & Bid \di{2}. This is a perfect hand to enquire about majors. If
      partner bids a major or \cl{3}, pass (you need at least another
      queen to raise partner's major). If partner bids \nt{2}, correct
      to \cl{3}. \\
  7 & \hhand{KT9832,5,975,KJ7} \\
    & Bid \sp{2}. If partner raises spades, raise to game. Pass if he
      denies spades by rebidding \cl{3} or bid \cl{3} if he rebids
      \nt{2}. \\
  8 & \hhand{K85,KJ95,AT63,93} \\
    & Bid \nt{2} (invitational). If partner accepts game by bidding
      \he{3}, bid \he{4}. If partner accepts with \sp{3}, raise to
      \nt{3}. \\
  9 & \hhand{AJT763,KQ9,T7,Q2} \\
    & Bid \sp{3}. This is forcing to game and shows at least 6
      spades. Pass if partner signs-off in \nt{3}. \\
  10 & \hhand{K73,942,A932,973} \\
    & Bid \cl{3} forcing \emph{LHO} to come in at the three-level. The
      Law of Total Tricks will protect you \ldots \\
\end{longtable}

\begin{longtable}{rp{11cm}}
  \multicolumn{2}{l}{\emph{\underline{Rebids after partner's invitational response in a suit: \cl{2}--\he{2}--???}}} \\
  11 & \hhand{KQJ5,53,4,AQT965} \\
     & Bid \sp{2}. This hand will play better in one of your suits so
       let partner know you have four spades. \\
  12 & \hhand{432,AQ,K7,AT7643} \\
     & \emph{Pass}. There is no reason to think there is a better
       spot. \\
  13 & \hhand{3,AQT8,T8,AKJT84} \\
     & Bid \sp{3} (splinter) or \he{4}. Ten tricks should be on with
       this dummy. \\
  14 & \hhand{96,KJ63,JT,AKQ74} \\
     & Bid \he{3}. Although the hand is a maximum, the shape is not
       inspiring. It may have been preferable to open \di{1} with this
       hand. \\
  15 & \hhand{KJ2,52,3,AKJT962} \\
     & Bid \cl{3}. Although you have a doubleton heart, the clubs are
       good enough to play opposite a void. It must be better to have
       it as trumps. \\
\end{longtable}

\subsection{The \di{2} opening}

\hypertarget{ex2d}
Some examples of bidding after a \di{2} opening.

\begin{longtable}{rp{11cm}}
  \multicolumn{2}{l}{\emph{\underline{Responses to \di{2}}}} \\
  1 & \hhand{QJ543,AT5,KT5,AT} \\
    & Bid \sp{4}. Why mess about? \\
  2 & \hhand{54,A9,AT87643,63} \\
    & \emph{Pass}. You would also pass if one of the low diamonds was
      a low heart since bidding \he{2} may land you in a 3-3 fit. \\
  3 & \hhand{2,JT73,KQ64,K852} \\
    & Bid \he{2}. If partner is 4-3-1-5, he will bid \sp{2} which can
      be corrected to \cl{3}. Do not ask for shape since that may push
      bidding to the 4-level. \\
  4 & \hhand{AJ,A93,AJT97,T87} \\
    & Bid \nt{3}. Diamonds are well under control and there are no
      better prospects for game. \\
  5 & \hhand{AQT65,KJ,987,AQ5} \\
    & Bid \nt{2}---there is a grand slam possible here. Whatever
      partner bids, you will bid diamonds next to ask about controls. \\
\end{longtable}

\begin{longtable}{rp{11cm}}
  \multicolumn{2}{l}{\emph{\underline{Responding over RHO's 2-level suit overcall: \di{2}--(\he{2})--???}}} \\
  6 & \hhand{QT6,JT9,KQJ4,A64} \\
    & \emph{Double}. This can get ugly since opponent is bidding at
      the 2-level with at most seven trumps and without the balance of
      \hcp. \\
  7 & \hhand{9,T64,AQ86543,K8} \\
    & \emph{Pass}. You could double but that would probably drive the
      opponents to spades which is a better spot. \\
  8 & \hhand{84,3,AKT5,987653} \\
    & Bid \cl{4}. This is a preemptive bid to make \emph{LHO} decide
      whether to support at the 4-level. \\
\end{longtable}

\begin{longtable}{rp{11cm}}
  \multicolumn{2}{l}{\emph{\underline{Responding over RHO's 3-level suit overcall / cue-bid: \di{2}--(\di{3})--???}}} \\
  9 & \hhand{952,A95,AJ73,743} \\
    & \emph{Double}. \\
  10 & \hhand{K93,AQT,AT2,JT94} \\
    & A \emph{Double} is probably best with favourable vulnerability.

      However, with unfavourable vulnerability, it is a choice between
      \nt{3} (if you feel lucky) and \cl{5} (more realistic). \\
  11 & \hhand{AT9842,843,T5,74} \\
    & Bid \sp{3} (non-forcing). \\
\end{longtable}

\begin{longtable}{rp{11cm}}
  \multicolumn{2}{l}{\emph{\underline{Responding after RHO's double: \di{2}--(Double)--???}}} \\
  12 & \hhand{K962,K4,AQJ6,862} \\
     & \emph{Redouble}. There could be overtricks here even if partner
       plays in a 4-1 diamond fit. \\
  13 & \hhand{KJ965,J864,J4,Q7} \\
     & Bid \sp{2}---if opponents compete, you can try hearts next. \\
  14 & \hhand{T642,QT63,JT,K75} \\
     & \emph{Pass}. Let partner describe his shape with a redouble or
       bid. \\
  15 & \hhand{A5,KQT,KT9863,Q2} \\
     & \emph{Redouble}. You have a lot of diamonds and good spot
       cards. Even if opponent's have a 4-4 spade fit, they may not
       find it and even if they do it is likely you have a penalty
       double against them in spades. \\
\end{longtable}

\subsection{``Gambling'' and ``Namyats'' openings}

\begin{longtable}{rp{11cm}}
  \multicolumn{2}{l}{\emph{\underline{Responses to a ``Gambling'' \nt{3}}}} \\
  1 & \hhand{32,AK85,AKJ43,JT} \\
    & Bid \di{4} asking partner to show singletons or voids. If he is
      short in spades, you can commit to a club slam. \\
  2 & \hhand{AQJ,9743,T6,AJ86} \\
    & \emph{Pass}. Opponents may be able to run some hearts but the
      odds are in your favour. Even if someone has five hearts, he may
      not be on lead or the suit may be blocked. \\
  3 & \hhand{A92,AK97652,,A85} \\
    & Bid \nt{5}. This asks partner to bid \di{7} with
      \emph{AKQJ}. You certainly want to be in \di{6} although there
      are no guarantees. \\
  4 & \hhand{A,AKQT84,KQJ9,54} \\
    & Bid \cl{6} which should be cold. \\
  5 & \hhand{QJ84,65,T87,JT97} \\
    & Bid \cl{5}. You don't care what partner's suit is (although it
      looks to be diamonds). What you do know is that opponents can
      make a lot of tricks in hearts (or even spades) and this robs
      them of room to find their best spot. \\
\end{longtable}

\begin{longtable}{rp{11cm}}
  \multicolumn{2}{l}{\emph{\underline{Responses to a ``Namyats'' \di{4} opening}}} \\
  6 & \hhand{T,A765,KQ95,AK32} \\
    & Bid \nt{4} (\emph{RKCB}). \\
  7 & \hhand{874,KJT9,KQ65,KJ} \\
    & Bid \sp{4}. Partner cannot have many aces in addition to a solid
      suit (he probably would have opened \cl{1} if so) so slam is out
      of question. \\
  8 & \hhand{972,QJ,AK652,AJT} \\
    & Bid \he{4}, a relay to partner's suit. You plan to cue-bid
      \cl{5} inviting slam and if partner has a cue-bid in hearts, you
      can bid \sp{6}. \\
  9 & \hhand{J752,A92,AKQ53,4} \\
    & Bid \nt{4} (\emph{RKCB}). If partner shows 3 key cards, you will
      bid \sp{7}. This is likely to be lay-down after the opening
      lead. \\
  10 & \hhand{872,AQJ73,,AT742} \\
    & With a solid suit and the heart king, \sp{7} is odds-on. Since
      there is no way to confirm both of these (an asking bid will
      only find the heart king), it is probably best to simply bid
      \sp{6}. The success of the slam may depend on the heart finesse,
      finally. \\
\end{longtable}

\end{document}
