\documentclass[a4paper,article,oneside]{memoir}
\counterwithout{section}{chapter}
\setsecnumdepth{subsection}
\maxtocdepth{subsection}
\usepackage{microtype}
\usepackage{longtable}
\usepackage{nicefrac}
\usepackage{hyperref}
\usepackage{bbding}
\usepackage{xcolor}
\usepackage{grbbridge}
\setboolean{spellten}{true}
\usepackage{url}
\newcommand{\gap}{\vspace{\baselineskip}}
\newcommand{\hcp}{\textsc{hcp}}
\newcommand{\sq}{\textsc{sq}}
\newcommand{\ltc}{\textsc{ltc}}
\newcommand{\forcing}[1]{\fbox{forcing#1}}

\begin{document}

\title{COSL Precision Bidding System}
\author{Sudhir}
\date{v2.21, 2 January 2021}
\maketitle

\tableofcontents

\pagebreak

\section{Opening bids}

All strong hands (with one exception\footnote{Balanced 22-23
  point hands are opened \nt{2}.}) are opened \cl{1} which is forcing
for one round. In general, a major suit opening shows $5^+$-cards and
the higher ranking suit is opened with suits of equal length. A
no-trump opening shows a balanced hand with a possible 5-card minor. A
\di{1} opening normally shows a 3-card holding but could sometimes be
made with a doubleton when bidding \nt{1} or \cl{2} is unattractive
e.g., \hhand{AQJT,KQ,76,J7642} or \hhand{AKT9,AK98,32,432}.

\begin{longtable}{ p{1.5cm}p{9.5cm} }
  \hline
  \cl{1} & $16^+$\hcp\ (unbalanced) or $17^+$\hcp\ (balanced)
           \forcing. Hands with a powerful $6^+$-card suit that can play
           opposite a singleton and have 15\hcp\ with a void or
           singleton should also be opened with \cl{1}, e.g.,
           \hhand{AQJT98,8,KQ7,QJT}.\hyperlink{1c}{\HandCuffRight} \\
  \di{1} & 11-15\hcp, at least 2 cards in \di{}, no 5-card major and
           less than 6 clubs.\hyperlink{1d}{\HandCuffRight} \\
  \he{1}/\sp{} & 11-15\hcp, $5^+$-cards in suit bid.\hyperlink{1major}{\HandCuffRight} \\
  \nt{1} & 14-16\hcp\ in $1^{st}$/$2^{nd}$ positions and 15-17\hcp\ in
           $3^{rd}$/$4^{th}$ position, balanced. May have a five-card
           minor (even a 5-4-2-2 distribution with a five-card minor
           is acceptable with stoppers in the
           doubletons).\hyperlink{1nt}{\HandCuffRight} \\
  \cl{2} & 11-15\hcp, $6^+$-card club suit (7\sq\ hand), may have a
           4 or 5-card major.\hyperlink{2c}{\HandCuffRight} \\
  \di{2} & 11-15\hcp, 5-4-3-1, 4-4-4-1 or 5-4-4-0 shape with short \di{}
           (5-card suit if present is \cl{}), \forcing{}\hyperlink{2d}{\HandCuffRight} \\
  \he{2}/\sp{} & 5-10\hcp, $6^+$-card suit (\sq\ of 8 when vulnerable
                 and 7 non-vulnerable). With 6 or less losers, open
                 \sp{1}/\he{}/\di{}.\hyperlink{2major}{\HandCuffRight} \\
  \nt{2} & 22-23\hcp, balanced hand, may have a 5-card
           major.\hyperlink{2nt}{\HandCuffRight} \\
  \emph{3 of suit} & Pre-emptive, $<10$\hcp, $7^+$-card suit (\sq\ of 9
                     when vulnerable and 8 non-vulnerable). Apply rule of
                     2/3/4.\hyperlink{3preempt}{\HandCuffRight} \\
  \nt{3} & ``Gambling'', solid $7^+$-card minor suit (\emph{AKQ} or
           better) with no outside ace or
           king.\hyperlink{3nt}{\HandCuffRight}  \\
  \cl{4}/\di{} & \emph{Namyats}---long semi-solid major suit (usually
                 $8^+$-cards) with 8 or more tricks. Used in $1^{st}$
                 or $2^{nd}$ seat
                 only.\hyperlink{namyats}{\HandCuffRight} \\
  \he{4}/\sp{} & Pre-emptive with $7\nicefrac{1}{2}$ tricks. \\
  \hline
\end{longtable}

\section{Responses to \cl{1}}

\hypertarget{1c}
The responses to \cl{1} can be negative, constructive or positive. All
positive responses are forcing to game unless both hands are minimum
and no fit is found. There is no response to \cl{1} above \di{4}.

\begin{longtable}{ p{1.5cm}p{9.5cm} }
  \hline
  \multicolumn{2}{l}{\emph{\underline{Negative Response}}} \\
  \di{1} & 0-7\hcp, \forcing. Note that with an ace and a king (3
           controls) either in the same suit or two different
           $4^+$-card suits, a positive response should be
           made.\hyperlink{1c1d}{\HandCuffRight} \\
  \multicolumn{2}{l}{\emph{\underline{Constructive Responses}}} \\
  \he{2}/\sp{} & 4-7\hcp, $6^+$-card suit with two of the top four honours
                 but not \emph{AK}.\hyperlink{1c2major}{\HandCuffRight} \\
  \multicolumn{2}{l}{\emph{\underline{Positive Responses}} \forcing{ to game}} \\
  \he{1}/\sp{}, \cl{2}/\di{} & $8^+$\hcp, $5^+$-cards in suit. There
                               are no restrictions on suit
                               quality.\hyperlink{1csuit}{\HandCuffRight} \\
  \nt{1} & 8-13\hcp, balanced hand with no five-card
           suit.\hyperlink{1c1nt}{\HandCuffRight} \\
  \nt{2} & $14^+$\hcp, balanced hand.
           \forcing{ to \nt{4}}\hyperlink{1c2nt}{\HandCuffRight} \\
  \sp{3} & A solid 7 or 8 card suit (with or without side controls)
           that will play for no losers opposite a singleton, e.g.,
           \emph{AKQJxxx} or
           \emph{AKQxxxxx}.\hyperlink{1cl3sp}{\HandCuffRight} \\
  \multicolumn{2}{l}{\emph{\underline{Unusual Positive---Three-suited hands}}
                     \forcing{ to game}} \\
  \cl{3} & 8-11\hcp\ or $<4$ controls, 4-4-4-1 shape with a black
           singleton (\sp{} or
           \cl{}).\hyperlink{unusualpositive}{\HandCuffRight} \\
  \di{3} & 8-11\hcp\ or $<4$ controls, 4-4-4-1 shape with a red
           singleton (\he{} or
           \di{}).\hyperlink{unusualpositive}{\HandCuffRight} \\
  \he{3}, \nt{3},
  \cl{4}/\di{} & $12^+$\hcp\ \underline{and} $4^+$-controls, 4-4-4-1
                 shape with singleton in suit above the one bid
                 (\sp{}, \cl{}, \di{}, \he{}
                 respectively).\hyperlink{unusualpositive}{\HandCuffRight} \\
  \hline
\end{longtable}

\subsection{Bidding after a negative response \cl{1}--\di{1}}

\hypertarget{1c1d}
Opener rebids 1, 2 or \nt{3} with balanced hands (17-19, 20-21 or
24-26\hcp\ respectively), a non-jump in a suit with 16-21\hcp\ and a
jump in a suit with powerful hands ($22^+$\hcp). The jump rebid may be
made with a lower point-count with greater playing strength.

\hyperlink{ex1c1d}{Bidding Examples\HandCuffRight}

\begin{longtable}{ p{1.5cm}p{9.5cm} }
  \hline
  \multicolumn{2}{l}{\emph{\underline{Balanced Hands}}} \\
  \nt{1} & 17-19\hcp, balanced hand, no 5-card major. Responder's
           rebids are: \\
         & \begin{tabular}{lp{7.5cm}}
             \emph{Pass} & 0-5\hcp, no major suit to escape to. \\
             \cl{2} & 6-7\hcp, \emph{Stayman}.\hyperlink{stayman}{\HandCuffRight} \\
             \di{2}/\he{} & 0-7\hcp, transfer to \he{2}/\sp{2} respectively.
                            Responder will invite with 7\hcp\ and pass
                            with 0-6\hcp\ unless opener
                            \emph{super-accepts}.\hyperlink{superaccept}{\HandCuffRight} \\
             \di{4}/\he{} & \emph{Texas} transfers to \he{4}/\sp{}
                            respectively. \\
           \end{tabular} \\
  \nt{2} & 20-21\hcp, balanced hand, may have a 5-card
           major. Responder's rebids are: \\
         & \begin{tabular}{lp{7.5cm}}
             \emph{Pass} & 0-3\hcp. \\
             \cl{3} & 4-7\hcp, \emph{Puppet Stayman} asking for 5-card majors if
                      any.\hyperlink{puppetstayman}{\HandCuffRight} \\
             \di{3}/\he{} & Weak, transfer to \he{3}/\sp{} respectively. \\
             \nt{3} & 4-5\hcp, sign-off. \\
             \di{4}/\he{} & \emph{Texas} transfers to \he{4}/\sp{}
                            respectively. \\
           \end{tabular} \\
  \nt{3} & 24-26\hcp, balanced hand, may have a 5-card
           major. Responder's rebids are: \\
         & \begin{tabular}{p{1.5cm}p{6.5cm}}
             \emph{Pass} & 0-4\hcp, balanced. \\
             \emph{4 of suit} & 5-7\hcp, $5^+$-cards. Opener bids one
                                above suit (\di{4}/\he{4}/\sp{} or \nt{4}) to show
                                fit and start \emph{Roman key-card
                                Blackwood}.\hyperlink{blackwood}{\HandCuffRight} \\
             \nt{4} & 5-7\hcp, no 5-card suit, quantitative. \\
           \end{tabular} \\
  \multicolumn{2}{l}{\emph{\underline{Unbalanced Hands}}} \\
  \he{1}/\sp{} & 5$^+$-card suit, non-forcing. Can be only four cards
                 if opener started with a 4-4-4-1 shape (with a
                 singleton minor, opener will rebid \he{1}). \\
         & \begin{tabular}{lp{6.7cm}}
             \emph{Pass} & 0-4\hcp, especially when balanced. \\
             \sp{1} & 4-7\hcp, $4^+$-cards, may have three hearts. It is
                      important to bid the spades before supporting
                      hearts since opener may have bid \he{1} holding
                      a 4-4-4-1 distribution with both majors. \\
             \nt{1} & 5-7\hcp, no 5-card suit, no 4-card spade after
                      \he{1}. May have 3-card support. This bid should
                      be avoided as far as possible to prevent the
                      strong hand from coming down. \\
             \cl{2}/\di{} & 5-7\hcp, 5-card suit, denies 3-card support. \\
             \emph{Single raise} & 4-5\hcp\ with $3^+$-card support. \\
             \emph{Double raise} & 6-7\hcp\ with $3^+$-card support. \\
             \emph{Jump shift} & \emph{Splinter} with $4^+$-card
                                 support showing slam interest. E.g.,
                                 bid \cl{4} when holding
                                 \hhand{JT98,93,AJT987,5} after opener
                                 rebids \sp{1}. A splinter of \cl{3}
                                 would be slightly weaker showing a
                                 game-going hand. \\
           \end{tabular} \\
  \cl{2}/\di{} & $5^+$-card suit, may have a 4-card major,
                 non-forcing. Responses have the same structure
                 as over \he{1} and \sp{1}. \\
  \he{2}/\sp{} & Powerful hand with $22^+$\hcp\ and $5^+$-card suit,
                 equivalent of a Standard \cl{2} bid. Responder's
                 rebids are: \\
         & \begin{tabular}{lp{6.7cm}}
             \nt{2} & 0-3\hcp, minimum, no support. \\
             \nt{3} & 4-7\hcp\ maximum, spread values, no support. \\
             \emph{Raise} & 0-3\hcp, minimum, $3^+$-card support. \\
             \emph{Jump raise} & 4-7\hcp, maximum, $3^+$-card support, no
                                 specific values in other suits. \\
             \emph{New suit} & 4-7\hcp, values in suit, does not deny
                               support for partner's suit. \forcing{} \\
           \end{tabular} \\
  \cl{3}/\di{} & Very strong unbalanced hand with a long minor and
                 good playing strength that is too strong for
                 \cl{2}/\di{}. E.g., \hhand{A,AK,KQJT876,QJ5} or
                 \hhand{KQJ5,6,A,AKQT964}. \forcing{} \\
  \he{3}/\sp{} & Extremely powerful hand with a solid suit and at
                 least nine tricks. This bid sets trumps and asks
                 responder to cue bid an ace or void. E.g.,
                 \hhand{AKQJT98,4,KJ3,AK} or \hhand{65,AKQT7543,AKJ,}
                 where a slam is on if responder can cue
                 bid. \forcing{} Responder bids: \\
         & \begin{tabular}{lp{6.7cm}}
             \emph{Suit} & First-round control---ace or void in suit. \\
             \nt{3} & No first round control but has a king or
                      singleton in a non-trump suit. Opener rebids
                      \cl{4} to ask which suit. \\
             \emph{Game raise} & Denies ace, king, singleton or void. \\
          \end{tabular} \\
  \hline
\end{longtable}

\subsection{Bidding after a constructive response \cl{1}--\he{2}/\sp{2}}

\hypertarget{1c2major}
Since responder is showing a strictly limited hand with a long suit,
the opener needs to decide the best contract. If there is no chance
for game or slam, he should pass with a tolerance for responder's
suit.

\begin{longtable}{ p{2cm}p{9cm} }
  \hline
  \emph{Pass} & Game unlikely. \\
  \he{4}/\sp{} & Raise to game is a sign-off. \\
  \emph{New suit} & Natural, \forcing. Responder's rebids are:\\
              & \begin{tabular}{p{1.8cm}p{6.3cm}}
                  \emph{Raise} & $3^+$-card support (or \emph{Qx}). \\
                  \emph{Rebid \he{}/\sp{}} & Minimum, no support. \\
                  \nt{3} & Maximum, no support. \\
                  \emph{Cue bid under} \nt{3} & Maximum with support,
                                                singleton or void in
                                                suit bid. \\
                \end{tabular} \\
  \nt{2} & Support for suit, asking for shortness \forcing{ to
           game.} Responders's rebids are:\\
              & \begin{tabular}{p{1.8cm}p{5cm}}
                  \emph{Rebid} \he{}/\sp{} & Minimum. \\
                  \emph{New suit} & Singleton or void in bid suit. \\
                \end{tabular} \\
  \nt{3} & \emph{AQ} or \emph{KQ} in suit. \\
  \nt{4} & \emph{Roman key-card
           Blackwood}.\hyperlink{blackwood}{\HandCuffRight} \\
  \hline
\end{longtable}

\subsection{Bidding after a positive no-trump response \cl{1}--1NT/2NT}

\hypertarget{1c1nt}
Responder is showing a balanced hand with 8-13\hcp\ (\nt{1}) or
$14^+$\hcp\ (\nt{2}). Opener can either (a) bid his own suit at the
2-level showing a 5-carder, (b) bid his own suit at the 3-level
showing a very strong hand with slam interest, (c) bid \cl{2}
(\emph{Stayman}) or (d) raise no-trumps.

\hyperlink{ex1cnt}{Bidding Examples\HandCuffRight}

\subsubsection{\cl{1}--1NT--\cl{2}}

A \cl{2} rebid after a response of \nt{1} is \emph{Transfer Stayman}
(other rebids are shown subsequently) and responder rebids as below:

\begin{longtable}{ p{1.5cm}p{9.5cm} }
  \hline
  \di{2} & 8-10\hcp, 4 card \he{}, may have 4 card
           \sp{}. Opener's rebids are: \\
         & \begin{tabular}{lp{7cm}}
             \he{2} & \he{} fit assured; responder should then bid
                      \nt{2} with 4-3-3-3 or a second suit at 3-level. \\
             \sp{2} & 4-card \sp{}, no 4-card \he{}. \\
             \nt{2} & No 4-card major. \\
           \end{tabular} \\
  \he{2} & 8-10\hcp, 4 card \sp{}, denies 4-card \he{}. \\
         & \begin{tabular}{lp{7cm}}
             \sp{2} & Spade fit confirmed, relay; responder bids
                      \nt{2} with 4-3-3-3 else second
                      suit at 3-level. \\
             \nt{2} & No 4-card \sp{}, may have 4-card \he{}. \\
           \end{tabular} \\
  \sp{2} & 8-10\hcp, no 4 card major. Opener then bids \nt{2} to ask
           for a further description. Responder's rebids are: \\
         & \begin{tabular}{lp{6cm}}
             \cl{3}/\di{} & 4-3-3-3 with 4-card \cl{}/\di{}. \\
             \he{3} & 4-4 in minors with three hearts. \\
             \sp{3} & 4-4 in minors with three spades. \\
             \nt{3} & 5-card minor. \\
           \end{tabular} \\
  \nt{2} & 11-13\hcp, 4-3-3-3 shape with a 4-card minor. \cl{3} by opener
           is then a relay asking responder to bid his suit or \nt{3}
           with clubs. \\
  \cl{3} & 11-13\hcp, 4-4-3-2 shape with 4 clubs. Opener bids \di{3}
           as a relay and responder bids \he{3} with spades, \sp{3}
           with hearts and \nt{3} with diamonds.\\
  \di{3} & 11-13\hcp, 4-4-3-2 shape with \di{} and \he{}. \\
  \he{3} & 11-13\hcp, 4-4-3-2 shape with \he{} and \sp{}. \\
  \sp{3} & 11-13\hcp, 4-4-3-2 shape with \sp{} and \di{}. \\
  \nt{3} & 11-13\hcp, poor 5-card minor suit. \cl{4} by opener is then a
           relay asking responder to bid his suit. \\
  \hline
\end{longtable}

\subsubsection{Suit bid after \cl{1}--1NT}

A bid of \di{2}/\he{}/\sp{} or \nt{2} (for \cl{}s) over a
response of \nt{1} shows a $5^+$-card suit and asks responder to show
strength and fit in steps as below. This is a \emph{support asking
  bid} and in this context, a minimum is 8-10\hcp\, a maximum is
11-13\hcp\ and support is \emph{Hxx}, \emph{xxxx} or better.

\begin{longtable}{p{1.5cm}p{9.5cm}}
  \hline
  \emph{1 step} & Minimum and no support. \\
  \emph{2 steps} & Minimum with support. \\
  \emph{3 steps} & Maximum and no support. \\
  \emph{4 steps} & Maximum with support. \\
  \hline
\end{longtable}

A jump rebid in a suit, e.g., \cl{1}--\nt{1}--\cl{3} shows a very
strong hand with slam interest, sets trumps and asks for responder's
holding in that suit.  Responses are in steps as below:

\begin{longtable}{ p{1.5cm}p{9.5cm}}
  \hline
  \emph{1 step} & Two or three spot cards. \\
  \emph{2 steps} & Doubleton honour. \\
  \emph{3 steps} & Tripleton honour. \\
  \emph{4 steps} & Two honours doubleton. \\
  \emph{5 steps} & Two honours tripleton. \\
  \emph{6 steps} & Four card support. \\
  \hline
\end{longtable}

Any suit bid after a support asking bid would be an $\epsilon$ control
asking bid in that suit.\hyperlink{epsilon}{\HandCuffRight}

\subsubsection{No-Trump raise after \cl{1}--1NT}


Since a raise to \nt{2} shows a club suit (see above), there are only
two possible raises in no-trumps---\nt{3} and \nt{4}.

\begin{longtable}{ p{1.5cm}p{9.5cm}}
  \hline
  \nt{3} & Minimum balanced hand with no four-card major or interest
           in slam. \\
  \nt{4} & Quantitative raise with a balanced hand and no four-card
           major inviting slam. \\
  \hline
\end{longtable}

\subsubsection{Bidding after \cl{1}--2NT}

\hypertarget{1c2nt}
A \nt{2} response shows $14^+$\hcp\ and immediately puts the
partnership in slam range.  It is, therefore, forcing to
\nt{4}. Responses are:

\begin{longtable}{ p{1.5cm}p{9.5cm} }
  \hline
  \cl{3} & \emph{Baron}: asking responder to show 4-card suits upwards
           (\nt{3} after \cl{3} shows 4-3-3-3 with four cards in \cl{}). \\
  \di{3}/\he{}/\sp{} or \cl{4} & $5^+$-card suit. Subsequent bidding is natural. \\
  \nt{3} & Asks responder to clarify his point range as follows: \\
         & \begin{tabular}{lp{6.5cm}}
             \cl{4} & 14-15\hcp. \\
             \di{4} & 16-17\hcp. \\
             \he{4} & 18-19\hcp. \\
             \sp{4} & 20-21\hcp. \\
             \nt{4} & $22^+$\hcp. \\
           \end{tabular} \\
  \hline
\end{longtable}

\subsection{Bidding after a positive suit response \cl{1}--\he{1}/\sp{1}/\cl{2}/\di{2}}

\hypertarget{1csuit}
Opener rebids no-trumps with a balanced hand. With support for
responder's suit he has the option of initiating a series of
\emph{asking bids}.\footnote{As a rule of thumb, asking bids should
  not be used if two of the outside suits are missing first-round
  controls. This is because once asking bids are triggered, there is
  no way to return to natural bidding.} With an unbalanced hand and no
support for responder's suit, opener bids his suit and further bidding
is natural. All bids short of game are forcing.

With a 4-4-4-1 distribution, if responder bids the singleton suit,
opener should rebid no-trumps. E.g., \nt{1} over \he{1} or \nt{2} over
\di{2}. If responder rebids his suit, opener should rebid \nt{}.
Partner should cater to this possibility and insist on his suit as
trumps only with a $6^+$-card suit.

\hyperlink{ex1suit}{Bidding Examples\HandCuffRight}

\begin{longtable}{ p{2.5cm}p{8.5cm} }
  \hline
  \emph{New suit} & $5^+$-card suit, denies 3-card support for responder's
                    suit. Subsequent bids are natural to find the correct game
                    contract. Responses are: \\
                  & \begin{tabular}{lp{5.7cm}}
                      \emph{New suit} & 4-card suit. \\
                      \emph{Raise} & $3^+$-card support. \\
                      \emph{Rebid suit} & $6^+$-card suit, semi-solid
                                          if minor. \\
                      \emph{Lowest NT} & 5-3-3-2 shape, values in
                                         unbid suits. \\
                    \end{tabular} \\
  \nt{1} & 17-19\hcp, balanced (\nt{2} over \cl{2}/\di{2}).
           No 5-card major, may have 3-card support (shows shape first). \\
  \nt{2} & 20-21\hcp, balanced (\nt{3} over \cl{2}/\di{2}).
           No 5-card major, may have 3-card support. \\
  \emph{Single raise} & $\gamma$-\emph{trump asking bid}---shows
                        an extremely powerful hand with distinct slam
                        possibilities.\hyperlink{gamma}{\HandCuffRight}
                        Any further new suits bid by opener after the $\gamma$
                        response will be $\epsilon$-\emph{control
                        asking bids}.\hyperlink{epsilon}{\HandCuffRight} \\
  \emph{Double raise} & Minimum balanced hand, 4-card fit with good
                        controls. Avoids $\gamma$ sequences.  \\
  \emph{Game raise} & Minimum balanced hand, 4-card fit with poor
                      controls. \\
  \sp{3}, \cl{4}/\di{}/\he{} & \emph{Splinter} bid 4-card fit promised. \\
  \nt{4} & \emph{Roman key-card Blackwood}.\hyperlink{blackwood}{\HandCuffRight} \\
  \hline
\end{longtable}

In general, over a minor suit positive response, an \nt{} bid by
opener is preferred if it is likely that the final contract will be
\nt{3}. This will ensure that the strong hand is declarer. Similarly,
with a 5-card minor suit, opener should consider rebidding \nt{}
rather than his suit since, in most cases, \nt{3} is preferable to 5
in a minor.

\subsection{Bidding after a \sp{3} response}

\hypertarget{1cl3sp}
A \sp{3} response places responder with a minimum 7-card suit headed
by \emph{AKQ} with or without outside controls. The suit should be
obvious on most occasions. Opener's rebids are:

\begin{longtable}{ p{2.5cm}p{8.5cm} }
  \hline
  \nt{3} & To play. Responder should pass unless he has many
           outside controls. \\
  \cl{4} & $\beta$-ask for \emph{outside controls}---responses are in
           the lower (0-3) scale.\hyperlink{beta}{\HandCuffRight}
           Any following suit bid that is not the trump suit is an
           $\epsilon$ control ask.\hyperlink{epsilon}{\HandCuffRight}  \\
  \di{4} & When opener cannot identify the suit---asks responder
           to bid his suit. Diamonds are indicated by a \nt{4}
           response. A subsequent suit bid would be an
           $\epsilon$-ask.\hyperlink{epsilon}{\HandCuffRight} \\
  \he{4}/\sp{} & $5^+$-card suit, to play. Responder should pass
                 with 3-card support or doubleton honour. \\
  \hline
\end{longtable}

\hyperlink{ex1c3s}{Bidding Examples\HandCuffRight}

\subsection{Bidding after an \emph{unusual positive} response}

\hypertarget{unusualpositive}
An unusual positive response shows a 4-4-4-1 distribution. With less
than 4 controls (typically, 8-13\hcp), the singleton is not shown
directly---\cl{3} is bid with a black singleton and \di{3} is bid with
a red singleton. With more than 4 controls (typically $12^+$\hcp), the
singleton is immediately known since responder bids the the suit below
the singleton.

After \cl{3} or \di{3}, opener bids the next higher suit to ask
responder to clarify where his singleton lies.  Responder bids one
step above the relay to show the lower ranking suit and two steps
above to show the higher ranking suit.

\begin{longtable}{ p{5cm}p{6cm}  }
  \multicolumn{2}{l}{\emph{Possible sequences after an unusual positive}}\\
  \hline
  \cl{1}--\cl{3}--\di{3}--\he{3} & 4-4-4-1 (\cl{} singleton) \\
  \cl{1}--\cl{3}--\di{3}--\sp{3} & 1-4-4-4 (\sp{} singleton) \\
  \cl{1}--\di{3}--\he{3}--\sp{3} & 4-4-1-4 (\di{} singleton) \\
  \cl{1}--\di{3}--\he{3}--\nt{3} & 4-1-4-4 (\he{} singleton) \\
  \hline
\end{longtable}

Once the singleton is known, a bid by opener in the singleton suit is
$\beta$ and asks responder for the number of controls held (the lower
scale is used after \cl{3}/\di{} and the upper scale is used after the
stronger responses).\hyperlink{beta}{\HandCuffRight}

\hyperlink{ex1c3c}{Bidding Examples\HandCuffRight}

\subsection{Handling intervention over \cl{1}}

Over a double of \cl{1}, the additional bids of \emph{Redouble} and
\emph{Pass} are used to provide more information. When the double is
conventional (e.g., shows both majors), the bidding is the same except
that a bid of \nt{1} would also confirm stoppers in both majors. All
other bids retain their normal meaning.

\begin{longtable}{ p{1.5cm}p{9.5cm} }
  \hline
  \multicolumn{2}{l}{\emph{\underline{After \cl{1}--(Double)}}} \\
  \emph{Pass} & 0-4\hcp. \\
  \di{1} & 5-7\hcp, artificial. \forcing{} \\
  \emph{Redouble} & $8^+$\hcp, usually balanced. \forcing{ to game}. \\
  \nt{1} & Normal 8-13\hcp, but if the double shows a two-suited hand,
           shows stoppers in both implied suits. \\
  \emph{Others} & Same as over \cl{1} without intervention. \\
  \hline
\end{longtable}

After an overcall in a suit at the one-level, any suit or no-trump bid
is a positive response forcing to game. A \emph{trap pass} can be made
when responder wants to double for penalties---in this case, he will
pass a re-opening double by opener.

When opponents overcall with \nt{1}, the responses are different
depending on whether the overcall is a genuine strong hand or
conventional showing a two-suiter (the \emph{unusual no-trump}). In
the latter case, the \emph{unusual over unusual} approach repurposes
the \cl{2} and \di{2} bids to show a game-going hand with a major
suit.

\begin{longtable}{ p{1.5cm}p{9.5cm} }
  \hline
  \multicolumn{2}{l}{\emph{\underline{After a one-level suit overcall \cl{1}--(\di{1}/\he{1}/\sp{1})}}} \\
  \emph{Pass} & 0-4\hcp\ or a \emph{trap pass}. \\
  \emph{Double} & 5-8\hcp\ unbalanced or $5^+$\hcp, balanced. \\
  \emph{Suit} & Natural, $5^+$-card suit. \forcing{ to game} \\
  \emph{Jump to \cl{3}/\di{3}} & Unusual positive with 4-4-4-1. \\
  \emph{Cue Bid} & Balanced hand with \emph{no stopper} in opponent's
                   suit, \forcing{ to game} \\
  \nt{} & Usual meaning and promises a stopper in opponent's suit. \\
  \multicolumn{2}{l}{\emph{\underline{After a no-trump overcall \cl{1}--(\nt{1}) showing minors}}} \\
  \emph{Double} & Modest high-card points, suitable for penalising one
                  of opponent's suits, usually no 5-card major. \\
  \cl{2} & $5^+$-card heart suit, \forcing{ to game.} \\
  \di{2} &  $5^+$-card spade suit, \forcing{ to game.} \\
              & The above two bids are the so-called \emph{unusual
                over unusual} responses in which cue bids of known
                suits correspond to forcing bids in the unbid suits. \\
  \he{2}/\sp{} & Natural, non-forcing. \\
  \multicolumn{2}{l}{\emph{\underline{After a genuine no-trump overcall of \cl{1}--(\nt{1})}}} \\
  \emph{Pass} & 0-4\hcp. \\
  \emph{Double} & $5^+$\hcp, balanced---for penalties. \\
  \emph{Suit} & 5-8\hcp, $5^+$-card suit. \\
  \hline
\end{longtable}

Over higher level overcalls, bidding is largely natural but responses
over an unusual \nt{2} are still \emph{unusual over unusual} and not
specifically mentioned in the table below.

\begin{longtable}{ p{1.5cm}p{9.5cm} }
  \hline
  \multicolumn{2}{l}{\emph{\underline{After an overcall at 2-level}}} \\
  \emph{Double} & 6-8\hcp, any shape. Any suit rebid by
                         opener would be a one-round force. \\
  \emph{Suit} & Natural and \forcing{ to game.} Note that a jump to
                \di{3} over \cl{2} would be an unusual positive. \\
  \nt{2} & 8-10 or $14^+$\hcp, with stoppers. \\
  \nt{3} & 11-13\hcp\ with stoppers. \\
  \emph{Cue bid} & Values to be in game but no clear-cut action---no
                   long suit, no stopper in overcaller's suit. \\
  \multicolumn{2}{l}{\emph{\underline{After an overcall at 3-level}}} \\
  \emph{Double} & Balanced hand with $8^+$\hcp. \\
  \emph{Suit} & Positive, natural, \forcing{to game} \\
  \nt{3} & 8-11\hcp\ with stoppers. \\
  \multicolumn{2}{l}{\emph{\underline{After an overcall at 4-level}}} \\
  \emph{Double} & Shows values---support for partner if he bids and
                  provides defence if he passes. \\
  \emph{Suit} & Natural. \\
  \hline
\end{longtable}

If the intervention occurs after responder bids, e.g.,
\cl{1}--(\emph{Pass})--\di{1}--(\emph{RHO bids}), opener should rebid
as follows:

\begin{longtable}{ p{1.5cm}p{9.5cm} }
  \hline
  \multicolumn{2}{l}{\emph{\underline{After \cl{1}--(Pass)--\di{1}--(\he{1}/\sp{1})}}} \\
  \emph{Pass} & Balanced minimum (no 5-card suit). \\
  \emph{Double} & Takeout with support for other suits. \\
  \emph{Suit} & Natural, at least 5-cards, non-forcing. \\
  \nt{1} & Upper end of the range with stopper. \\
  \nt{2} & Same as \nt{2} without interference but promises stopper. \\
  \emph{Cue bid} & Strong hand, lacking stopper in overcalled suit. \\
  \multicolumn{2}{l}{\emph{\underline{After \cl{1}--(Pass)--\di{1}--(\nt{1} showing minors)}}} \\
  \emph{Pass} & Balanced minimum (no 5-card suit). \\
  \emph{Double} & Penalty oriented. \\
  \cl{2} & Heart suit with extra values (\emph{unusual over
           unusual}). \\
  \di{2} & Spade suit with extra values (\emph{unusual over
           unusual}). \\
  \he{2}/\sp{} & Natural, non-forcing. \\
  \nt{2} & Upper-end of the \nt{1} rebid range with stoppers. \\
  \multicolumn{2}{l}{\emph{\underline{After \cl{1}--(Pass)--\di{1}--(Double---usually shows diamonds)}}} \\
  \emph{Pass} & Balanced minimum (no 5-card suit). \\
  \nt{1} & Upper end of range with diamond stopper. \\
  \emph{Suit} & Same meaning as without the double. \\
  \multicolumn{2}{l}{\emph{\underline{After an intervention above 1-level}}} \\
  \emph{Pass} & Balanced minimum (no 5-card suit). \\
  \emph{Others} & Little extra weight as compared to without the
                  intervention.

                  \emph{Unusual/unusual} applies over a \nt{2} overcall. \\
  \hline
\end{longtable}

\hyperlink{ex1cintervene}{Bidding Examples\HandCuffRight}

\section{Responses to \di{1}}

\hypertarget{1d}
A \di{1} opening shows either a balanced hand of 11-13\hcp\ or a
genuine diamond suit opening with 11-15\hcp. Although a \di{1}
opening may be made on a doubleton diamond, it is non-forcing and
partner can pass with a weak hand.

\begin{longtable}{ p{1.5cm}p{9.5cm}  }
  \hline
  \emph{Pass} & 0-7\hcp\ and usually, no 4-card major. \\
  \multicolumn{2}{l}{\emph{\underline{New Suit}}} \\
  \he{1}/\sp{} & Usually\footnote{Sometimes, with favourable
                 vulnerabilty and an extremely weak hand, a tactical
                 bid may be made to stop opponent's game. E.g.,
                 holding \hhand{754,J852,985,654}, you could bid
                 \he{1} planning to pass any rebid by opener.}
                 $6^+$\hcp, $4^+$-card suit. \sp{1} denies four cards
                 in hearts but responder will bid \he{1} with both majors.
                 \forcing{} Opener rebids: \\
              & \begin{tabular}{p{1.5cm}p{7cm}}
                  \sp{1} & 4-card \sp{}, denies 4-card
                           \he{}. Responder rebids: \\
                         & \begin{tabular}{lp{5.5cm}}
                             \nt{1} & Sign-off. \\
                             \cl{2} & 4th-suit-\forcing{}
                                      (``do something clever''). \\
                             \sp{2} & 8-9\hcp, 4-card \sp{}. \\
                           \end{tabular} \\
                  \nt{1} & 11-14\hcp, balanced, denies 4-card fit. Can be 3-3-2-5 shape.
                           Responder can rebid \\
                         & \begin{tabular}{ll}
                             \cl{2} & New minor---\forcing{} \\
                             \di{2} & 8-9\hcp, 4-card \sp{} (after initial \he{1}). \\
                           \end{tabular} \\
                  \cl{2} & Unbalanced, usually 5-4 in minors and no 4-card
                           major. Responder can rebid \\
                         & \begin{tabular}{lp{5.2cm}}
                             \di{2} & Weak hand, to play. \\
                             \he{2}/\sp{} & $6^+$-card suit, to play. \\
                             \cl{3} & 8-9\hcp, keep bidding alive. \\
                             \nt{3} & To play. \\
                           \end{tabular} \\
                  \di{2} & 6+-card \di{} suit, non-forcing. \\
                  \he{2}/\sp{} & Raise of responder's suit shows 3-4 card
                                 support (if 3-card, it promises a singleton
                                 in a side suit).

                                 A \emph{reverse} (bidding the
                                 other major) shows 14-15\hcp, $6^+$-\di{}
                                 and $4^+$-cards in the suit bid. \\
                  \nt{2} & 14-15\hcp, good diamonds and stoppers in
                           the unbid major and clubs. \\
                  \he{3}/\sp{} & A double raise shows a maximum
                                 14-15\hcp, 4-card support and
                                 distributional values. \\
                  \di{3} & 14-15\hcp, $6^+$-card \di{} suit, no 4-card major. \\
                  \cl{3} & 14-15\hcp, at least 5-5 in minors with
                           points concentrated in the two suits. \\
                \end{tabular} \\
  \cl{2} & $10^+$\hcp, $5^+$-card suit, no 4-card major,
           \forcing. Opener's rebids are: \\
              & \begin{tabular}{lp{7cm}}
                  \he{2} & 11-14\hcp, \he{} stopper, no \sp{} stopper. \\
                  \sp{2} & 11-14\hcp, \sp{} stopper, no \he{} stopper. \\
                  \nt{2} & 11-14\hcp, stopper in both majors. \\
                  \di{2}/\di{3} & No stopper in majors, genuine \di{}
                                  suit. \\
                  \he{3} & 15\hcp, \forcing{ to game} \he{} stopper, no
                           \sp{} stopper. \\
                  \sp{3} & 15\hcp, \forcing{ to game} \sp{} stopper, no
                           \he{} stopper.  \\
                  \nt{3} & 15\hcp, stopper in both majors. \\
                  \cl{3} & No stopper in majors. \\
                \end{tabular} \\
  \multicolumn{2}{l}{\emph{\underline{Balanced Hands}}} \\
  \nt{1} & 8-10\hcp, balanced, no 4-card major. \\
  \nt{2} & 11-12\hcp, balanced, no 4-card major. Could be a 4-3-3-3
           shape with a weak four card major and tenaces that would
           play better as declarer in \nt{}. E.g.,
           \hhand{AQT,T642,QT9,KT7} or \hhand{9843,KJT,AQ7,JT5} \\
  \nt{3} & 13-15\hcp, balanced, no 4-card major. \\
  \multicolumn{2}{l}{\emph{\underline{Inverted Raises with Diamond Support}}} \\
  \di{2} & 10-12\hcp, $5^+$-card diamond suit. \forcing{ to \nt{2} or \di{3}} \\
  \di{3} & $<10$\hcp, $5^+$-cards in \di{}, usually with a singleton or void. \\
  \di{4} & Pre-emptive, with more shape and trumps than for \di{3},
           i.e., 6 or 7-card diamond suit. \\
  \multicolumn{2}{l}{\emph{\underline{Unbalanced Hands}}} \\
  \he{2}/\sp{} & Weak jump shift (0-7\hcp) with a long suit,
                 non-forcing. Opener's rebids are: \\
              & \begin{tabular}{p{1.1cm}p{7cm}}
                  \nt{2} & 11-12\hcp, no 4-card major. \\
                  \cl{3} & Shows both minors and asks responder to
                           choose between \cl{3}, \di{3} and \nt{3}. \\
                \end{tabular} \\
  \cl{3} & Invitational, with a long club suit. E.g.,
           \hhand{Q4,75,97,AQJ8654} \\
  \he{3}/\sp{}/\cl{4} & \emph{Splinter} bid with $5^+$-card diamond
                        support and no 4-card major. \\
  \he{4}/\sp{} & Single-suited hand with $7^+$-cards and no slam
                 interest. \\
  \hline
\end{longtable}

\hyperlink{ex1d}{Bidding Examples\HandCuffRight}

\subsection{Intervention after a \di{1} opening}

If opponent doubles \di{1}, the responses are:
\begin{longtable}{p{1.5cm}p{9.5cm} }
  \hline
  \emph{Pass} & 0-4\hcp\ or 9-10\hcp. \\
  \emph{New suit} & 5-8\hcp. \\
  \nt{1} & 6-8\hcp, balanced. \\
  \di{2}, \di{3} & $<$8\hcp, 4+-card support, pre-emptive. \\
  \emph{Redouble} & $11^+$\hcp. \\
  \hline
\end{longtable}

After an overcall by opponent up to the \sp{2} level, the responses
are:
\begin{longtable}{p{1.5cm}p{9.5cm} }
  \hline
  \emph{Double} & 8-10\hcp, $4^+$-cards in other major, negative. \\
  \emph{New suit} & 5-card suit if major, 4-card otherwise. \\
  \di{2} & 6-9\hcp, support for major. \\
  \di{3} & 10-11\hcp. \\
  \nt{1} & 8-10\hcp, stopper in opponent's suit, balanced. \\
  \nt{2} & 11-13\hcp, stopper in opponent's suit, balanced. \\
  \hline
\end{longtable}

\section{Responses to \he{1} or \sp{1}}

\hypertarget{1major}
Responses to a major opening include \emph{Bergen} raises,
\emph{Splinter} bids, a forcing \nt{1} and 2/1 game force.

\begin{longtable}{ p{1.5cm}p{9.5cm}  }
  \hline
  \emph{Pass} & 0-7\hcp\ and poor support. \\
  \multicolumn{2}{l}{\emph{\underline{Direct and Bergen Raises}}} \\
  \emph{Single raise} & 7-10\hcp\ with 3-card support, constructive. \\
  \emph{Double raise} & 0-6\hcp\ with 4-card support (pre-emptive
                        \emph{Bergen} raise).\hyperlink{bergen}{\HandCuffRight} \\
  \cl{3} & 7-10\hcp\ with 4-card support (constructive
           \emph{Bergen} raise).\hyperlink{bergen}{\HandCuffRight} \\
  \di{3} & 10-12\hcp\ with 4-card support (limited \emph{Bergen}
           raise).\hyperlink{bergen}{\HandCuffRight} \\
  \emph{Game raise} & Wide variety of hands where responder is fairly
                      sure there is no play for slam. E.g., raise to
                      \sp{4} with \hhand{K65,AQ,K82,J9876} or with
                      \hhand{98732,A5,Q,T9743}. \\
  \multicolumn{2}{l}{\emph{\underline{One-Over-One Response}}} \\
  \sp{1} & $8^+$\hcp, $4^+$-card suit, \forcing. See below for detailed treatment. \\
  \nt{1} & 8-15\hcp, balanced hand with mild support for opener's suit
           or unbalanced hand with insufficient \hcp\ to justify a
           2-over-1 response. \forcing{} Opener's
           rebids are: \\
              & \begin{tabular}{p{2cm}p{6.5cm}}
                  \multicolumn{2}{l}{\emph{\underline{With 11-13\hcp}}} \\
                  \cl{2}/\di{}/\he{} & 11-13\hcp, 4-card suit (or
                                       3-card better minor). \\
                  \emph{Rebid of suit} & 11-13\hcp, 6-card suit. \\
                  \multicolumn{2}{l}{\emph{\underline{With a maximum 14-15\hcp}}} \\
                  \emph{Jump rebid of suit} & 14-15\hcp, 6-card solid
                                              suit. The jump rebid
                                              should be made on the
                                              basis of playing
                                              strength rather than
                                              \hcp. \\
                  \nt{2} & 5-3-3-2 distribution. \\
                  \emph{Jump in new suit} & 5-5 distribution. \\
                \end{tabular} \\
              & A \emph{reverse}, e.g., \he{1}--\nt{1}--\sp{2} would
                show shape rather than \hcp\ (typically 14-15) and
                indicate a 6-5 distribution (or 6-4 with a very strong
                spade holding such as \emph{AKQx}) \\
  \multicolumn{2}{l}{\emph{\underline{Two-Over-One Game Force}}} \\
  \cl{2}/\di{}/\he{} & $12^+$\hcp, $4^+$-card minor or 5-card heart
                       suit (over \sp{1}). Unless the suit is rebid at
                       the three level, all 2-over-1 responses are
                       \forcing{ to game.} Opener's rebids are: \\
              & \begin{tabular}{p{2cm}p{6.5cm}}
                  \he{2} (following \sp{1}) & Natural, $4^+$-card suit. \\
                  \sp{2} (following \he{1}) & 14-15\hcp, reverse. \\
                  \emph{Rebid suit} & Minimum, not necessarily a 6-carder. \\
                  \nt{2} & 11-13\hcp, stoppers in unbid suits \\
                  \emph{Raise} & 11-13\hcp, good support, non-forcing. \\
                  \emph{Jump in new suit} & 14-15\hcp, good support,
                                            control in bid suit. \\
                  \emph{Jump rebid of suit} & 14-15\hcp, very good
                                              6-card suit. The jump
                                              rebid should be made on
                                              the basis of playing
                                              strength rather than
                                              \hcp. \\
                  \nt{3} & 15\hcp, stoppers in unbid suits. \\
                \end{tabular} \\
  \nt{2} & $12^+$\hcp, 4-card support, \forcing{ to game,} \emph{Jacoby
           \nt{2}}.\hyperlink{jacoby2nt}{\HandCuffRight}. \\
  \multicolumn{2}{l}{\emph{\underline{Other Bids at 3-level and Above}}} \\
  \emph{Double jump shift} & \emph{Splinter} bid, singleton or void in bid suit,
                             $4^+$-card support. \forcing{ to game} \\
  \nt{3} & 14-15\hcp, usually 4-card support for opener's major,
           responder lacks a void or singleton, minimum 4 controls. \\
  \hline
\end{longtable}

A response of \sp{1} over \he{1} shows $8^+$\hcp\ with a 4-card suit
and is forcing for one round. Opener's rebids are:

\begin{longtable}{ p{1.5cm}p{9.5cm}  }
  \hline
  \nt{1} & 11-13\hcp, minimum. \\
  \cl{2}/\di{} & 11-15\hcp, $4^+$-card suit, non-forcing. \\
  \sp{2} & 11-13\hcp, 4-card support. \\
  \nt{2} & 14-15\hcp, balanced, stoppers in both minors. \\
  \cl{3}/\di{} & 13-15\hcp, 5-card suit. \\
  \he{3} & 13-15\hcp, $6^+$-card suit. \\
  \sp{3} & 13-15\hcp, 4-card spade support. \\
  \nt{3} & To play with running suit. \\
  \cl{4}/\di{} & \emph{Splinter} bids, \forcing{ to game} \\
  \he{4} & To play---distributional hand. \\
  \sp{4} & To play---maximum hand with 13-15\hcp\ and
           distributional values. \\
  \nt{4} & \emph{Roman key-card Blackwood}\hyperlink{blackwood}{\HandCuffRight}
           with agreement in \sp{}. \\
  \hline
\end{longtable}

\hyperlink{ex1h}{Bidding Examples\HandCuffRight}

\subsection{Intervention after a major suit opening}

If an opponent doubles, responder can bid:
\begin{longtable}{ p{1.5cm}p{9.5cm}  }
  \hline
  \emph{Pass} & 0-4\hcp\ or 9-10\hcp. \\
  \emph{Raise} & $<$8\hcp, pre-emptive, $3^+$-card support following the
                 \emph{Law of Total Tricks}. \\
  \emph{New suit} & 5-8\hcp, $5^+$-card suit. \\
  \nt{1} & 6-8\hcp, balanced. \\
  \nt{2} & \emph{Jacoby \nt{2}}.\hyperlink{jacoby2nt}{\HandCuffRight} \\
  \emph{Redouble} & $12^+$\hcp, support for opener's suit. \\
  \hline
\end{longtable}

If an opponent overcalls:
\begin{longtable}{ p{1.5cm}p{9.5cm}  }
  \hline
  \emph{Pass} & 0-7\hcp\ or $8^+$\hcp\ and waiting for penalty if opener reopens
                with a double. \\
  \emph{Double} & 7-9\hcp, 4-card suit in other major. \\
  \nt{1} & 8-10\hcp\ with stopper in opponent's suit. \\
  \nt{2} & 11-12\hcp\ with stopper in opponent's suit. \\
  \emph{Cue bid} & $15^+$\hcp\ with singleton or void in opponent's
                   suit. \forcing{} \\
  \hline
\end{longtable}

\section{Responses to 1NT}

\hypertarget{1nt}
A \nt{1} opening shows a 14-16\hcp\ balanced hand (15-17\hcp\ in
$3^{rd}$/$4^{th}$ seat) without a 5-card major holding but could be a
6-3-2-2 or 5-4-2-2 hand with a long minor suit in which case opener
must hold stoppers in both doubletons. Responses are as below:

\begin{longtable}{ p{1.5cm}p{9.5cm}  }
  \hline
  \emph{Pass} & 0-7\hcp, poor support. \\
  \cl{2} & 10-11\hcp, \emph{Stayman}.\hyperlink{stayman}{\HandCuffRight} \\
  \di{2}/\he{} & \emph{Jacoby} transfers to \he{} and \sp{}
                 respectively.\hyperlink{jacoby}{\HandCuffRight}. \\
  \sp{2} & $8^+$\hcp---Minor suit \emph{Stayman}, denies 4-card major and asks
           opener for a 4-card minor. Shows a minor 2-suiter (5-4 or
           better). \\
  \emph{3 of suit} & Good suit, \forcing{ to game} \\
  \cl{4} & \emph{Gerber} ace-asking.\hyperlink{gerber}{\HandCuffRight} \\
  \di{4}/\he{} & \emph{Texas} transfers to \he{4} and \sp{4}
                 respectively. Denies slam values. \\
  \nt{4} & 17-18\hcp, balanced, quantitative.\\
  \nt{5} & 22-23\hcp, balanced, \emph{Grand Slam Force}. \\
  \nt{6} & 19-21\hcp, balanced. \\
  \hline
\end{longtable}

\subsection{Intervention after opening 1NT}

If an opponent doubles \nt{1}, responder can bid:
\begin{longtable}{ p{1.5cm}p{9.5cm}  }
  \hline
  \emph{Pass} & Weak or 6-11\hcp, balanced. \\
  \emph{Redouble} & SOS---5-card suit, asking partner to bid \cl{2} and then
                    pass or rectify. \\
  \cl{2} & 4-card suit (or good 3-card clubs). \\
  \di{2} & Short in clubs. \\
  \he{2} & Tolerance for majors (at least 4-3). \\
  \sp{2} & $12^+$\hcp, \forcing{} \\
  \emph{3 of suit} & $6^+$-card suit, invitational. \\
  \hline
\end{longtable}

After an opponent's overcall, responder has the following choices:
\begin{longtable}{p{3cm}p{8cm}  }
  \hline
  \emph{Double} & Penalty double. \\
  \emph{Suit at 2-level} & 0-6\hcp, natural and non-forcing. \\
  \nt{2} & \emph{lebensohl}---partner must bid
           \cl{3}.\hyperlink{lebensohl}{\HandCuffRight}. \\
  \emph{Suit at 3-level} & Natural, \forcing{ to game} \\
  \emph{Cue bid} & Asks opener to bid a 4-card major if he has one,
                   denies a stopper in opponent's suit. \\
  \nt{3} & \emph{lebensohl}---denies stopper in opponent's
           suit.\hyperlink{lebensohl}{\HandCuffRight}. \\
  \hline
\end{longtable}

\section{Responses to \cl{2}}

\hypertarget{2c}
Bidding after a \cl{2} opening (showing a $6^+$-card club suit) is
largely natural except for the conventional \di{2} response that
asks opener to further describe his hand.

\begin{longtable}{ p{1.5cm}p{9.5cm} }
  \hline
  \emph{Pass} & 0-7\hcp, poor hand. \\
  \di{2} & 11+\hcp, conventional and \forcing{} (with club fit,
           may be made with only 8\hcp). Opener's rebids are: \\
              & \begin{tabular}{lp{7.5cm}}
                  \he{2}/\sp{} & 11-13\hcp, 4-card suit. \\
                  \nt{2} & 11-13\hcp, 6-3-2-2 balanced hand with stoppers in two
                           suits. Responder then bids \di{3} to enquire about stoppers and
                           opener's rebids are: \\
                               & \begin{tabular}{ll}
                                   \he{3} & \he{} and \di{} stoppers. \\
                                   \sp{3} & \sp{} and \di{} stoppers. \\
                                   \nt{3} & \he{} and \sp{} stoppers. \\
                                 \end{tabular} \\
                  \cl{3} & 11-13\hcp, 6 clubs with 1 outside stopper. Responder bids
                           \di{3} to enquire about the stopper and opener's rebids are:\\
                               & \begin{tabular}{lp{4.5cm}}
                                   \he{3} & \he{} stopper. \\
                                   \sp{3} & \sp{} stopper. \\
                                   \nt{3} & \di{} stopper.  \\
                                   \di{4} & 5-card suit (6-5 in \cl{} and \di{}) \\
                                 \end{tabular} \\
                  \nt{3} & 14-15\hcp, 6-3-3-2 balanced hand, good club suit. \\
                  \he{3}/\sp{} & 14-15\hcp, 4-card suit. \\
                \end{tabular} \\
  \he{2}/\sp{} & 8-10\hcp, non-forcing, invitational. Opener may
                 pass with a minimum and mild support. \\
  \nt{2} & 10-11\hcp, invitation to \nt{3}. Any rebid other than
                 \cl{3} (sign-off) by opener commits to game.\\
  \cl{3} & Pre-emptive raise based on a club fit. Not forward-going. \\
  \di{3}/\he{}/\sp{} & $6^+$-card suit with game-forcing values ($12^+$\hcp).
                       Opener's rebids are: \\
              & \begin{tabular}{lp{7cm}}
                  \nt{3} & Less than 2-card support. \\
                  \emph{Raise} & Minimum, 3-card support. \\
                  \emph{New suit} & Maximum, 3+-card support, cue bid ace.  \\
                \end{tabular} \\
  \nt{3} & 13-15\hcp, balanced hand with stoppers in the unbid suits
           and no interest in the majors. \\
  \cl{4} & Pre-emptive raise with extra club support or distributional
           values as compared to a raise to \cl{3}. \\
  \he{4}/\sp{} & Very long suit with no interest in slam, to play. \\
  \hline
\end{longtable}

\hyperlink{ex2c}{Bidding Examples\HandCuffRight}

\subsection{Intervention after a \cl{2} opening}

\begin{longtable}{ p{2.5cm}p{8.5cm} }
  \hline
  \emph{Negative double} & Through \sp{3}. \\
  \emph{Redouble} & 10+\hcp. \\
  \emph{Cue bid} & 12+\hcp, singleton or void in opponent's suit. \\
  \emph{New Suit} & Any new suit bid is \forcing. \\
  \hline
\end{longtable}

\section{Responses to \di{2}}

\hypertarget{2d}
A \di{2} opening describes a three suited hand with shortness in
diamonds and the responder can place the contract fairly easily in
most cases. The only positive response is \nt{2} which is forcing to
game.

\begin{longtable}{ p{2cm}p{9cm} }
  \hline
  \emph{Pass} & 6+ diamonds, no interest in other suits or bidding higher. \\
  \he{2}/\sp{}, \cl{3} & Natural, sign-off. \\
  \nt{2} & 11+\hcp, artificial \forcing{ to game} asks opener to
           further describe his hand. Opener's rebids are: \\
              & \begin{tabular}{lp{6.5cm}}
                  \cl{3}/\di{}  & 3=4=1=5 or 4=3=1=5 shape respectively. \\
                  \he{3} & 11-13\hcp\ and 4=4=1=4 shape. \\
                  \sp{3} & 14-15\hcp\ and 4=4=1=4 shape. \\
                  \nt{3} & 14-15\hcp, 4=4=1=4 shape and \di{}A or \di{}K. \\
                  \cl{4} & 11-13\hcp, 4=4=0=5 shape. \\
                  \di{4} & 14-15\hcp, 4=4=0=5 shape. \\
                \end{tabular} \\
              & A further rebid in diamonds (the singleton suit) by
                opener would be $\beta$ asking for
                controls.\hyperlink{beta}{\HandCuffRight} \\
  \di{3} & $\beta$ control asking bid. \\
  \he{3}/\sp{} & 7-9\hcp, pre-emptive, $5^+$-card suit. \\
  \he{4}/\sp{}, \cl{5} & Sign-off, to play. \\
  \hline
\end{longtable}

\hyperlink{ex2d}{Bidding Examples\HandCuffRight}

\subsection{Intervention after a \di{2} opening}

When opponents overcall, all doubles are for penalties. If the
overcall is at the two-level, a response of \nt{2} has the same
meaning as if there was no overcall. If the overcall is at the
three-level, bidding is as per competitive judgment and vulnerability.

When opponents double \di{2} (possibly showing values in diamonds), a
\emph{Redouble} shows a desire to play in diamonds. You have a
surprise holding in diamonds and are prepared to punish the opponents
when they run from the redouble. A \emph{Pass} is a waiting action for
partner to clarify his shape.  Partner will do this by either
\emph{Redoubling} with a 4-4-1-4 or 4-4-0-5 minimum, bidding \he{2}
with a 3-4-1-5 distribution or \sp{2} with 4-3-1-5. All bids other
than \emph{Pass} and \emph{Redouble} retain the same meaning as
without the double.

\section{Responses to \he{2} or \sp{2}}

\hypertarget{2major}
Opener shows a 6+-card major with 8-10\hcp\ and a good suit with a
minimum suit quality of 8 when vulnerable or 7 when non-vulnerable.

\begin{longtable}{ p{2.5cm}p{8.5cm} }
  \hline
  \emph{Pass} & No game, no fit. \\
  \emph{Raise to 3 or 4} & Natural, sign-off. \\
  \emph{New suit} & Natural, \forcing. Opener's rebids are: \\
              & \begin{tabular}{ll}
                  \emph{Raise} & 3-card support. \\
                  \emph{Rebid own suit} & \forcing{} \\
                \end{tabular} \\
  \nt{2} & \forcing{} Opener is asked to bid suit where he
           has a singleton or void or rebid his own suit lacking either. \\
  \hline
\end{longtable}

\section{Responses to 2NT}

Opener is showing a balanced hand with 22-23\hcp\ and no 5-card
suit---this is the only $16^+$\hcp\ hand that is not opened with
\cl{1}. Responses are:

\begin{longtable}{ p{1.5cm}p{9.5cm} }
  \hline
  \emph{Pass} & Very weak, no suit to escape to. \\
  \di{3}/\he{} & Weak hand that doesn't want to play in
                 no-trumps. \emph{Flint} convention---asks for a
                 transfer to \he{}/\sp{} respectively. \\
  \di{4}/\he{} & Transfer to \he{4}/\sp{} respectively. $6^+$-card
                 suit, to play. \\
  \nt{3} & To play. \\
  \cl{3} & \emph{Smolen Stayman}---asking for majors. Game-going with
           slam interest. Responses are: \\
              & \begin{tabular}{lp{7.7cm}}
                  \di{3} & No 4-card major---responder can now rebid: \\
                          & \begin{tabular}{lp{6.2cm}}
                              \he{3} & 4-card \he{} and 5-card \sp{}
                                       (allows for 5-3 fit). \\
                              \sp{3} & 4-card \sp{} and 5-card \he{}
                                       (allows for 5-3 fit). \\
                              \nt{3} & To play. \\
                              \cl{4} & Transfer to diamonds. \\
                            \end{tabular} \\
                  \he{3}/\sp{} & 4-card major. \\
                  \nt{3} & 4 cards in both majors. Responder bids: \\
                          & \begin{tabular}{lp{5.5cm}}
                              \emph{Pass} & To play. \\
                              \cl{4}/\di{}/\he{} & Transfer to
                                                   \di{}/\he{}/\sp{}
                                                   respectively.  \\
                            \end{tabular} \\
                \end{tabular} \\
  \sp{3} & Transfer to clubs with a correction to
           diamonds. \emph{Minor suit slam try}. \\
  \hline
\end{longtable}


\section{Responses to \cl{3}/\di{3}/\he{3}/\sp{3}}

\hypertarget{3preempt}
After a pre-emptive opening of three of a suit, responder's rebids are:

\begin{longtable}{p{1.5cm}p{9.5cm}}
  \hline
  \emph{Raise} & Pre-emptive. \\
  \emph{New suit} & Natural, \forcing{} \\
  \emph{Others} & Natural. \\
  \hline
\end{longtable}

\section{Responses to 3NT}

\hypertarget{3nt}
The ``gambling'' \nt{3} bid shows a solid $7^+$-card minor suit with
\emph{AKQ} or better and no outside ace or king. Responses are:

\begin{longtable}{p{1.5cm}p{9.5cm}}
  \hline
  \emph{Pass} & To play, stoppers in side suits. \\
  \cl{4} & Escape---asks opener to pass or bid \di{4} if that is his
           suit. \\
  \di{4} & Asks opener to bid a singleton or void if he has one.
           Responses: \\
              & \begin{tabular}{lp{6cm}}
                  \he{4}/\sp{} & Singleton or void in the bid suit. \\
                  \nt{4} & Singleton or void in the other minor. \\
                  \cl{5}/\di{} & Shows that minor and denies a
                                 singleton or void. \\
                \end{tabular} \\
  \he{4}/\sp{} & Natural, to play. \\
  \nt{4} & Quantitative, asking opener to bid \nt{6} with extra
           length or an extra trick outside (such as \emph{Qxx}). \\
  \cl{5} & Sign-off and weak. Opener should correct to \di{5} if that
           is his suit. \\
  \di{5} & Also a sign-off but responder indicates he knows opener's
           suit is diamonds and it would be advantageous to play from
           his side. \\
  \nt{5} & Grand Slam try showing no losers outside the trump suit but
           indicating a void in opener's suit. If opener is completely
           solid (e.g., \emph{AKQJ} to seven card), he bids 7
           otherwise he bids 6. \\
  \cl{6} & Asks opener to pass or correct. This could also be a
           tactical bid. \\
  \hline
\end{longtable}

\hyperlink{ex3nt}{Bidding Examples\HandCuffRight}

\section{Responses to \cl{4}/\di{4} (Namyats)}

\hypertarget{namyats}
The \emph{Namyats} convention (\emph{Stayman} spelt backwards) shows a
long (normally eight cards) semi-solid (not missing both ace and king)
major suit with eight or more playing tricks with playing strength
mainly in the trump suit (\cl{4} shows hearts and \di{4} shows
spades). The bid is meant to be constructive rather than pre-emptive
and allows you to distinguish hands that are close to game versus
purely pre-emptive openings. Responses are:

\begin{longtable}{p{1.5cm}p{9.5cm}}
  \hline
  \he{4}/\sp{} & Bidding game in partner's suit is a sign-off with no
                 slam interest (responder wants to be declarer). \\
  \di{4}/\he{} & The next step above opener's bid is a relay asking
                 him to bid his suit (\he{} or \sp{}). After opener
                 bids his suit, any new suit bid by responder is a cue
                 bid. Responder passes if he does not have slam
                 interest but wants his hand to be dummy. \\
  \nt{4} & \emph{Roman key-card Blackwood}.\hyperlink{blackwood}{\HandCuffRight} \\
  \emph{Suit} & $\epsilon$ asking bid. A compressed scale of responses
                is used (note that the steps skip over \nt{} because
                that has a special meaning): \\
               & \begin{tabular}{lp{6.5cm}}
                   \emph{Cheapest NT} & Guarded king in suit. When you
                                        have ample tricks elsewhere,
                                        this allows you to declare an
                                        \nt{} contract from the
                                        correct side. \\
                   \emph{1 step} & No control. \\
                   \emph{2 steps} & Second round control---any
                                    singleton. \\
                   \emph{3 steps} & First round control---void or ace. \\
                                      & A repeat ask is for third
                                        round control---a doubleton or
                                        guarded queen and responses
                                        are (a) No third round
                                        control: $1^{st}$ step, (b)
                                        Doubleton: $2^{nd}$ step and
                                        (c) Guarded Queen: $3^{rd}$
                                        step. \\
                 \end{tabular} \\
  \hline
\end{longtable}


\hyperlink{namyats}{Bidding Examples\HandCuffRight}

\section{Competitive Bidding}

\subsection{Overcalls}

At the one level, overcall with a 5-card suit and 8-15\hcp. At the
two-level, overcall with at least 11-15\hcp. Responses to a suit
overcall are:

\begin{longtable}{p{1.5cm}p{9.5cm}}
  \hline
  \emph{Pass} & $<$8\hcp. \\
  \emph{Raise} & 9-10\hcp\ with 3-card support. \\
  \nt{1} & 9-10\hcp, balanced with a stopper in opponent's suit. \\
  \hline
\end{longtable}

A \nt{1} overcall should be 13-15\hcp\ with a stopper in the opponent's
suit.

A jump suit overcall is a weak pre-emptive bid with a $6^+$-card suit.

A cue bid is \emph{Michael's} showing a 2-suiter in the highest unbid
suit and another.

\subsection{Doubles}

A direct double over opponent's opening is either 13-15\hcp\ (takeout)
or a power double with $16^+$\hcp.

A \emph{takeout} double over opponent's minor opening bid usually
promises a 4-card suit in both majors. Over a major suit opening, it
promise a 4-card suit in the other major. A response is requested even
with a blank hand unless the other opponent bids.

\begin{longtable}{p{1.5cm}p{9.5cm}}
  \hline
  \emph{Pass} & Long and solid holding in opponent's suit \\
  \emph{1-level} & 5-8\hcp. \\
  \emph{2-level} & 7-11\hcp. \\
  \emph{3-level} & 9-13\hcp. \\
  \nt{1} & 8-10\hcp\ with stopper in opponent's suit. \\
  \emph{Jump} & $9^+$\hcp. \\
  \hline
\end{longtable}

If the doubler rebids or raises, it indicates a power hand of $16^+$\hcp.

\gap

Doubles of an opponent's overcall are \emph{negative} doubles
indicating a lack of a biddable 5-card suit and no fit with partner's
bid suit. Interest in one of the unbid suits is strongly indicated.

At the one-level, the \emph{negative} double shows 8-15\hcp; at the
two-level, it shows 11-15\hcp. Over an opponent's overcall in a major,
it promises a 4-card holding in the other major.

\gap

A double in a competitive auction is a \emph{responsive} double
indicating 3-card support in partner's suit.

\gap

A double of a slam contract is a \emph{Lightner} double requesting an
unusual lead from partner.

\section{Gadgets and Conventions}

\subsection{$\beta$ control asking bid}

\hypertarget{beta}
A $\beta$ control asking bid can occur either after a \nt{1} rebid by
the \cl{1} opener over a positive suit response or by cue bidding a
singleton suit after a positive response of \cl{3}, \di{3}, \he{3},
\sp{3}, \cl{4} or \di{4} over \cl{1}.

The number of controls held (\emph{A=2}, \emph{K=1}) are shown in
steps as below:

\begin{longtable}{ p{1.5cm}p{9.5cm} }
  \hline
  \emph{1 step} & 0-2 controls.
                  A relay bid by opener in the cheapest suit over the
                  1-step response will then ask for clarification and
                  again the responses are in steps: \\
                & \begin{tabular}{ll}
                    \emph{1 step} & No controls. \\
                    \emph{2 steps} & 1 control. \\
                    \emph{3 steps} & 2 controls. \\
                  \end{tabular} \\
  \emph{2 steps} & 3 controls. \\
  \emph{3 steps} & 4 controls. \\
  \emph{4 steps} & 5 controls. \\
  \hline
\end{longtable}

When responder has already shown controls as less than four or at
least four, e.g., after an unusual positive, a modified scale of
responses is used as follows:

\begin{longtable}{ p{3cm}p{4cm}p{4cm}}
  \emph{Known to have}  & \emph{0-3 controls} & \emph{$4^+$ controls} \\
  \hline
  \emph{1 step}  & No controls. & 4 controls. \\
  \emph{2 steps} & 1 control.   & 5 controls. \\
  \emph{3 steps} & 2 controls.  & 6 controls. \\
  \emph{4 steps} & 3 controls.  & 7 controls. \\
  \emph{5 steps} &              & 8 controls. \\
  \hline
\end{longtable}

\subsection{$\gamma$ trump asking bid}

\hypertarget{gamma}
The $\gamma$-asking bid is used to find out the quality of responder's
suit. It is typically used to find out whether the suit will provide
enough tricks in no-trumps or as a trump suit.

\begin{longtable}{p{1.5cm}p{9.5cm}}
  \multicolumn{2}{l}{\emph{Responses to a $\gamma$-suit asking bid}} \\
  \hline
  \emph{1 step} & No top honour, $5^+$-card suit. \\
  \emph{2 steps} & 5-card suit, 1 honour. \\
  \emph{3 steps} & 5-card suit, 2 honours. \\
  \emph{4 steps} & 6-card suit, 1 honour. \\
  \emph{5 steps} & 6-card suit, 2 honours. \\
  \emph{6 steps} & AKQ of suit, $5^+$-card suit. \\
  \hline
\end{longtable}

The $\gamma$ bid may be repeated to get clarification on the first
response.

\begin{longtable}{p{3cm}p{8cm}}
  \emph{Honours shown} & \emph{Responses to repeat $\gamma$-ask} \\
  \hline
  \emph{Zero} or \emph{AKQ} & \begin{tabular}{lp{6cm}}
                                1 step & $7^+$-card suit \\
                                2 steps & 6-card suit \\
                                3 steps & 5-card suit \\
                          \end{tabular} \\
  \hline
  \emph{One} & \begin{tabular}{lp{6cm}}
                 1 step & \emph{Ace} \\
                 2 steps & \emph{King} \\
                 3 steps & \emph{Queen} \\
               \end{tabular} \\
  \hline
  \emph{Two} & \begin{tabular}{lp{6cm}}
                 1 step & \emph{AK} \\
                 2 steps & \emph{AQ} \\
                 3 steps & \emph{KQ} \\
               \end{tabular} \\
  \hline
\end{longtable}

\subsection{$\epsilon$ control asking bid}

\hypertarget{epsilon}
The $\epsilon$-asking bid is used to find out what controls the
responder holds in a specific suit. It follows a $\gamma$ asking bid
and terminates only when trumps or no-trumps is bid, i.e., any other
suit bid is an $\epsilon$-ask in that suit.  Responses are in steps:

\begin{longtable}{p{1.5cm}p{9.5cm}}
  \multicolumn{2}{l}{\emph{Responses to a $epsilon$-suit asking bid}} \\
  \hline
  \emph{1 step} & No control---\emph{Jxx} or worse. \\
  \emph{2 steps} & Third round control---\emph{Q} or doubleton. \\
  \emph{3 steps} & Second round control---\emph{K} or singleton. \\
  \emph{4 steps} & First round control---\emph{A} or void. \\
  \emph{5 steps} & \emph{AK} or \emph{AQ}. \\
  \hline
\end{longtable}

The $\epsilon$-ask can be repeated if it is important to know whether
the control is based on shortness or strength. The response is again
in steps---the first step showing that the previous response was based
on \emph{shortness} and the second step showing \emph{strength}.

If the first $\epsilon$ bid is at the level of \cl{5} or higher, a
compressed scale of responses is used where the first step shows no
control, second step shows second-round control and the third step
shows first-round control.

\subsection{Bergen raises}

\hypertarget{bergen}
After a \he{1} or \sp{1} opening, responses of \cl{3}, \di{3}, \he{3}
and \sp{3} show different types of 4-card support. The mnemonic
\emph{CLAP} (Constructive, Limited and Pre-emptive) helps to remember
the order of the bids.

\begin{longtable}{p{1.5cm}p{9.5cm}}
  \hline
  \emph{\he{1}--\cl{3}} & Constructive, 7-10\hcp, 4-card \he{}. \\
  \emph{\he{1}--\di{3}} & Limited, 10-12\hcp, 4-card \he{}. \\
  \emph{\he{1}--\he{3}} & Pre-emptive, 0-6\hcp, 4-card \he{}. \\
  \emph{\sp{1}--\cl{3}} & Constructive, 7-10\hcp, 4-card \sp{}. \\
  \emph{\sp{1}--\di{3}} & Limited, 10-12\hcp, 4-card \sp{}. \\
  \emph{\sp{1}--\he{3}} & \emph{Spare bid}---used to show a strong
                          $12^+$\hcp\ hand with 4-card support and an
                          undisclosed singleton/void. \\
  \emph{\sp{1}--\sp{3}} & Pre-emptive, 0-6\hcp, 4-card \sp{}. \\
  \hline
\end{longtable}

\subsection{Gerber ace-asking convention}

\hypertarget{gerber}
An immediate response of \cl{4} to any no-trump bid (or overcall) is
the \emph{Gerber} ace-asking convention. A jump rebid of \cl{4} in
response to a natural no-trump bid is \emph{Gerber} as also when a
trump suit has not been identified and no-trumps has been
rebid. Gerber should not be used holding a void.

Opener shows number of aces in steps as follows:

\begin{longtable}{p{1.5cm}p{9.5cm}}
  \hline
  \di{4} & Zero or four aces. \\
  \he{4} & One ace. \\
  \sp{4} & Two aces. \\
  \nt{4} & Three aces. \\
  \hline
\end{longtable}

\subsection{Jacoby transfers}

\hypertarget{jacoby}
After a \nt{1} opening, responder bids \di{2} with a 5-card or better
heart suit and \he{2} with spades. Opener will bid \he{2} or \sp{2} so
that the strong hand becomes declarer. Responder's rebids are:

\begin{longtable}{p{2.5cm}p{8.5cm}}
  \hline
  \emph{Pass} & A weak hand with $5^+$-card \he{} or \sp{}. \\
  \sp{2} & Invitational with 5-5 in the majors after
           \nt{1}--\di{2}--\he{2}. \\
  \nt{2} & Balanced or semi-balanced hand with 5-card \he{} or
           \sp{}. Invitational---partner can pass or sign-off in
           3 of a major or bid \nt{3}. \\
  \cl{3}/\di{} & 4-card suit in addition to 5-card major,
                 \forcing{ to game} \\
  \emph{Raise} & 6-card suit, invitational. \\
  \he{3} & (After \sp{2}) 5-5 in the majors with slam
           interest. Stronger than an immediate jump to \he{4}. \\
  \sp{3} & (After \he{2}) Singleton or void with slam interest. \\
  \nt{3} & Balanced or semi-balanced hand. Partner can pass or correct
           to 4 of major. \\
  \he{4} & (After \sp{2}) 5-5 in majors with no slam interest. Partner
           can pass or correct to \sp{4}. \\
  \emph{Double raise} & $6^+$-card major, sign-off. \\
  \nt{4} & Quantitative, inviting slam in major or no-trumps. \\
  \hline
\end{longtable}

\subsubsection{Super-acceptance of a transfer}
\hypertarget{superaccept}
Opener could \emph{super-accept} the transfer with a $4^+$-card
holding in the transfer suit. In this case, he can break the transfer
and show any doubletons.  For example, after \nt{1}--\he{2} (transfer
to \sp{2}), opener with a 4-card spade suit could bid:

\begin{longtable}{p{2.5cm}p{8.5cm}}
  \hline
  \nt{2} & 4=3=3=3 shape. \\
  \cl{3} & 4=x=x=2 (doubleton club). \\
  \di{3} & 4=x=2=x (doubleton diamond). \\
  \he{3} & 4=2=x=x (doubleton heart). \\
  \sp{3} & 4 spades, any other distribution. \\
  \hline
\end{longtable}

\subsection{Jacoby 2NT}

\hypertarget{jacoby2nt}
A \nt{2} response over an opening of \he{1} or \sp{1} is conventional
and shows $12^+$\hcp\ with $4^+$-card support of partner's suit. It is
\forcing{ to game.} Opener's rebids are:

\begin{longtable}{p{3.5cm}p{7.5cm}}
  \hline
  \emph{New suit} & Singleton or void in suit bid. \\
  \emph{Rebid of suit at 3-level} & Maximum strength hand. \\
  \emph{Jump shift} & Good $5^+$-card side suit. \\
  \emph{Game in original suit} & Minimum opening, sign-off. \\
  \nt{3} & 12-13 \hcp, medium strength hand. \\
  \hline
\end{longtable}


\subsection{lebensohl convention}

\hypertarget{lebensohl}
The \emph{lebensohl} convention is used by responder after an opponent
overcalls partner's opening \nt{1} bid in order to compete further
without necessarily committing to game. It is initiated after the
right-hand opponent makes a suit overcall at the two-level.

\begin{longtable}{p{2.5cm}p{8.5cm}}
  \hline
  \emph{2 in new suit} & Natural and non-forcing. \\
  \nt{2} & A puppet bid requiring opener to bid \cl{3}. After opener's
           forced \cl{3} bid, \\
                       & \begin{tabular}{p{8cm}}
                           --- 3 of a lower ranking suit than
                           overcaller's is natural, to play. \\
                           --- 3 of a higher ranking suit than
                           overcaller's is natural and invitational. \\
                           --- 3 of the opponent's suit is artificial
                           asking opener to bid a 4-card major and
                           showing a stopper in opponent's suit. \\
                           --- 3NT is natural, to play, and shows a
                           stopper in the opponent's suit. \\
                         \end{tabular} \\
  \emph{3 in new suit} & Natural, \forcing{ to game} \\
  \emph{Cue bid} & Artificial---asks opener to bid a 4-card major and
                   denies a stopper in opponent's suit. \\
  \nt{3} & Natural, to play, and denies a stopper in opponent's
           suit. \\
  \hline
\end{longtable}

\subsection{Roman key-card Blackwood}

\hypertarget{blackwood}
A \emph{Roman key-card Blackwood} bid of \nt{4} is used to enquire
about the number of key cards (any ace or the trump suit king) in
partner's hand. It should not be used when you have a void or two fast
losers.

Responses are in steps and differ slightly (when holding all five key cards) depending
on whether opponents have doubled or overcalled over \nt{4}.

\begin{longtable}{p{3.2cm}|p{1cm}p{3cm}p{3.2cm}}
  \emph{Holding} & \emph{Silent} & \emph{Double (R0P1)} & \emph{Overcall (D0P1)} \\
  \hline
  \emph{1/4 key cards} & \cl{5} & \emph{Redouble} & \emph{Double} \\
  \emph{0/3 key cards} & \di{5} & \emph{Pass} (0/3/5 key cards) & \emph{Pass} (0/3/5 key cards) \\
  \emph{2/5 key cards
  without queen of trumps} & \he{5} & \cl{5} (two key cards) & \emph{Cheapest suit} (two key cards)
  \\
  \emph{2/5 key cards
  and queen of trumps} & \sp{5} & \di{5} (two key cards) & \emph{Second-cheapest suit} (two key cards)\\

  \hline
\end{longtable}

When holding a void, after a trump suit is agreed, jumping to another
suit at the 4 or 5 level in the void suit initiates a \emph{key card
  exclusion} asking bid. Partner shows his key cards \emph{excluding}
any in the void suit in steps.

\begin{longtable}{p{1.5cm}p{9.5cm}}
  \hline
  \emph{1 step} & 1 or 4 key cards. \\
  \emph{2 steps} & 0 or 3 key cards. \\
  \emph{3 steps} & 2 or 5 key cards without trump Q. \\
  \emph{4 steps} & 2 or 5 key cards with trump Q. \\
  \hline
\end{longtable}

\subsection{Stayman convention}

\hypertarget{stayman}
The \emph{Stayman} convention is used to find a 4-4 major suit fit
after a \nt{1} opening by bidding \cl{2}. Opener responds with one of:

\begin{longtable}{p{1.5cm}p{9.5cm}}
  \hline
  \di{2} & No four card major. \\
  \he{2} & 4-card heart suit, may have 4-card spades. \\
  \sp{2} & 4-card spade suit, no 4-card heart suit. \\
  \hline
\end{longtable}

\subsubsection{Puppet Stayman}

\hypertarget{puppetstayman}
Used over a \nt{2} bid that may be made with a 5-card major, \cl{3} is
a conventional bid that endeavours to find a major suit fit. Responses
are:

\begin{longtable}{p{1.5cm}p{9.5cm}}
  \hline
  \di{3} & No five card major but at least one four card major.
           Responder with one four card major should bid
           the major that he \emph{does not have} and if there is
           a 4-4 fit, opener will bid it. \\
  \he{3} & 5-card heart suit. \\
  \sp{3} & 5-card spade suit. \\
  \nt{3} & No 4 or 5-card major. Responder can bid \cl{4} or \di{4}
           to transfer to \he{4} or \sp{4} respectively when he has
           a six-card major. \\
  \hline
\end{longtable}

\section{Miscellaneous}

\emph{High-card Points} (\hcp) are assigned as follows---Ace=4,
King=3, Queen=2 and Jack=1.  Once a trump-fit has been found,
distribution points can be assigned---Void=3, Singleton=2,
Doubleton=1.

Singleton honours should be counted only once (either \hcp\ or
shortness).

\gap

\emph{Suit Quality} (\sq) is calculated as suit length plus number of
honours in the suit. The Jack or Ten should be counted only if a
higher ranking honour is held. For example, a holding of K-J-9-5-4
would have 7\sq\ but J-10-9-5-4 would have 5\sq.

For an overcall, the \sq\ should equal or exceed the number of tricks
bid (e.g., \sq\ of 8 for a 2-level overcall).

When pre-empting, the \sq\ should equal the level of pre-empt when
vulnerable and can be one less when non-vulnerable.

\gap

The \emph{Losing Trick Count} (\ltc) is used only once a trump suit
has been established. Count losers only in the top three cards of the
suit holding---there are never more than 3 losers in a suit. With
three or more cards, A/K/Q are not losers but any lower card is a
loser. With two cards, only A or K are not losers.

Add your and partner's loser count and subtract from 24 to estimate
the number of tricks that can be won.  You can estimate your partner's
\ltc\ as follows:

\begin{tabular}{rp{3cm}}
  \emph{\hcp{}} & \emph{Expected Losers} \\
  \hline
  \emph{7-9} & 8-9 losers (9) \\
  \emph{10-12} & 7-8 losers (8) \\
  \emph{13-15} & 6-7 losers (7) \\
  \emph{16-18} & 5-6 losers (6) \\
  \emph{19-21} & 4-5 losers (5) \\
  \emph{$22^+$} & 4 losers or less \\
  \hline
\end{tabular}

\section{Bidding Examples}
\setboolean{betweencards}{true} % spaces between cards in hand diagrams
\setboolean{leadingspace}{true}

\subsection{Negative response to \cl{1}}

\hypertarget{ex1c1d}
After a \di{1} response, there is no temptation to get too high on
misfitting hands. For example,

\vhand[West]{4,AK954,AJ4,KQT9}\vhand[East]{KJT753,62,753,54}
\ewauction{1c,1d,1h,1s(1),2c(2),2s(3)}\\ (1) 4-7\hcp, 4+-card
suit. \\ (2) Shows minimum with second 4-card suit (implies 5
hearts). \\ (3) 6-card suit, no fit.

Opener shows discipline and passes recognising misfit and no chance
for game.

\gap
A \di{1} response does not rule out game. With a 2-suited hand,
it is easy to find a game contract when the fit is in the second bid
suit.

\vhand[West]{AK752,AQT43,A5,2}\vhand[East]{4,K852,9642,J754}
\ewauction{1c,1d,1s,1n(1),2h,3h,4h} \\ (1) As he has already
limited his hand, East is not afraid to improve the contract. After
that, all goes smoothly.

Suppose the lead against \he{4} is a low diamond. The best technique
for declarer is to win with the Ace, cash \sp{}A and ruff a spade with
a low trump. Then he plays a club to establish communication between
the two hands.

The opponents will probably continue diamonds. West ruffs the third
round and leads another low spade, ruffing with the \he{}8. If, at
worst, spades are 5-2 and South overruffs, declarer retains the
possibility of ruffing the other spade loser with the \he{}K. The
contract will fail only against very unlucky distribution.

\gap
With a powerful hand, opener would jump rebid his suit and responder
would know there is a game or slam on if he is in the upper range. For
example,

\vhand[West]{AKJ8753,A,K72,AQ}\vhand[East]{642,J73,AJ54,865}
\ewauction{1c,1d,2s(1),3d(2),3s(3),4s,4n,5c(4),6s} \\
(1) 22+\hcp, 5-card suit. \\ (2) 4-7\hcp, showing side-suit before
showing fit in spades. \\ (3) After the new suit bid at the 3-level,
opener knows he will not be left in \sp{3}. \\ (4) One key card.

After a heart lead and assuming trumps don't break worse than 2-1, the
contract can be made without the club finesse by playing \emph{A} and
\emph{K} of diamonds followed by a low diamond towards the
\emph{J}. This works whenever diamonds break 3-3, \di{}\emph{Q} is
held by North or is a doubleton with South.

\gap
With a balanced hand, opener will rebid \nt{}. For example,

\vhand[West]{K63,KJT,A862,AK3}\vhand[East]{AJ742,754,J95,T4}
\ewauction{1c,1d,1n(1),2h(2),2s,2n(3),3n(4)} \\
(1) 17-19\hcp, balanced. \\ (2) Weak transfer to \sp{2}. \\ (3)
Balanced hand, invitational. \\ (4) With 18\hcp\ and three cards in
spades, East tries for game in no trumps.

North leads the \he{}3, South plays the Queen and West the King. After
this favourable opening, West can afford to make a safety play in
spades. He plays King and another, North following suit with low
cards. To make absolutely sure of four tricks, even when North holds
Q10xx, declarer ducks in dummy. He makes game with four spades, two
hearts and three top tricks in the minors.

\gap
Even with a weak two-suiter, Precision enables finding slams with
relatively low point counts following a negative response. For
example,

\vhand[West]{A854,AK943,AKJ8,}\vhand[East]{6,J87652,Q976,64}
\ewauction{1c,1d,1h,3s(1),4c,4h(2),6h} \\ (1) Splinter
showing 4-card or better support in hearts and a singleton or void
in spades. \\ (2) Responder could conceivably also bid \he{5}
with the 6-card suit.

With a combined total of 22 points, although 13 tricks are available
if the opening lead is not ruffed, most pairs will probably stop in
\he{6} after the splinter bid using a sequence similar to the one
above.

\subsection{Positive response in a suit over \cl{1}}

\hypertarget{ex1csuit}
Using Precision, game is always reached after a positive response to a
\cl{1} opening.  The partnership will have a minimum of 24 points if
opener is unbalanced (16 vs 8) or 25 points if he is balanced (17 vs
8). This works well in practice, for example:

\vhand[West]{AKJ86,64,KQT9,K6}\vhand[East]{Q92,875,A543,Q94}
\ewauction{1c,1n,2s,3s,4s} \\
A dull 16\hcp\ \cl{1} opening against an equally dull 8\hcp\ but still
\sp{4} is an odds-on favourite to make.

\gap
Game contracts can be reached on smaller point counts if there are
distributional features. For example,

\vhand[West]{AKJT96,A,QJT9,65}\vhand[East]{Q82,965,K743,743}
\ewauction{1c(1),1d,1s,3s(2),4s} \\ (1) A strong 15\hcp\ with a
good suit should be opened with \cl{1}.\\ (2) As he has already
limited his hand, East is not shy in raising partner's suit with 5\hcp
and inviting game.

As compared to the previous deal, this is a 15\hcp\ vs 5\hcp\ hand
that may be passed out after \sp{1} in standard systems. However, the
game contract is virtually lay-down.

\gap
With a balanced hand, opener will rebid \nt{} over a positive suit
response.  Even with 3-card support for partner's suit, it is
sometimes correct to first bid \nt{} and only later raise partner's
suit. For example, with \hhand{AJT,KQT9,QJ4,KJ7}, if responder bids
\sp{1}, it is correct to rebid \nt{1} showing a balanced minimum
before raising spades. However, with a slightly different hand such as
\hhand{AJT7,KQT,QJ4,KJ7}, the rebid could be \sp{3} or \sp{4} showing
a minimum hand, probably balanced, with 4-card support.

Alternative sequences showing support have slightly different
meanings.  For example, whereas the sequence \cl{1}--\sp{1}--\sp{4}
would show a minimum hand with poor controls, the sequence
\cl{1}--\sp{1}--\nt{1}--any--\sp{4} would show a balanced minimum with
good controls.

The intermediate \nt{1} rebid can also be made when you want to find
out if responder has a distributional hand. For example, when holding
\hhand{AK87,A753,KQ4,A6}, after partner's positive response of \sp{1},
rebid \nt{1} and if partner rebids \cl{2} (four-card suit), you may
have very good play for \sp{7} if partner is holding something like
\hhand{QJ543,82,A8,K954}.  However, you need to know about the four
clubs first.

\gap
With a distributional hand where you have strong support for partner's
suit and the only question for slam is whether his suit has good
quality, \emph{asking bids} ($\gamma$ and $\epsilon$) can be used to
good effect. For example,

\vhand[West]{QJ632,5,AKQ8,KJ9}\vhand[East]{AKT54,987,T4,A53}
\ewauction{1c,1s,2s(1),3h(2),4c(*),4n(3),5h(*),5s(4),6s} \\
(1) $\gamma$ trump asking bid (possible slam if trumps are strong). \\
(2) 2 honours, 5-card suit. \\ (*) $\epsilon$ control asking bid in clubs and hearts. \\
(3) Ace or void. \\ (4) No control.

With a sure loser in hearts, opener stops in the small slam.

\subsection{Positive no-trump response to \cl{1}}

\hypertarget{ex1cnt}
With both majors, it is sometimes correct to use \emph{Stayman} even
when holding a 5-card suit.  For example, holding
\hhand{AKQ64,KQ87,A5,95}, it is better to bid \cl{2} over a \nt{1}
response rather than bidding \sp{2}. If responder holds something like
\hhand{JT2,AJ94,543,Q43}, he will certainly raise spades after \sp{2}
and the 4-4 heart fit will not be discovered. In this case although
there are 10 tricks in spades and 11 in hearts (given normal breaks),
sometimes the difference may be 10 tricks in the 4-4 fit versus 9 in
the 5-3 fit.

Similarly, with \hhand{3,AKQ7,AQ,KQJT98}, bid \emph{Stayman}. If
partner bids \di{2} (four hearts), you will bid \he{2} and later ask
for aces. If partner has two aces, you can confidently bid the grand
slam or the small slam if he shows only one ace. If partner holds
something like \hhand{AQ6,JT86,J76,543}, \he{6} from the strong side
is best, while \cl{6} will depend on the diamond finesse.

\subsection{\sp{3} response to \cl{1}}

\hypertarget{ex1c3s}
Opener can place the contract fairly easily given responder's solid
suit and use asking bids to decide if a slam is on. For example,

\vhand[West]{4,AT987,A4,AKQ87}\vhand[East]{AKQJ987,3,K7,T96}
\ewauction{1c,3s(1),4c(2),4h(3),7s(4)} \\
(1) Solid suit. Opener can tell that it is spades by looking
at his own hand. \\ (2) $\beta$-ask for outside controls. \\
(3) One outside control (\di{} or \he{} king). \\ (4) 13 tricks
are on top.

\subsection{Unusual positive response to \cl{1}}

\hypertarget{ex1c3c}
If responder bids an \emph{unusual positive}, slam is most likely on
the cards and with the right cards, grand slams can be reached on very
low point counts.

\vhand[West]{AKQ876,976,AK43,}\vhand[East]{J543,A,T987,AK43}
\ewauction{1c,4d(1),4h(2),4n(3),7s(4)} \\ (1) 4-1-4-4,
$4^+$-controls, $12^+$\hcp \\ (2) $\beta$ asking for controls \\
(3) 5 controls (2 steps) \\ (4) Partner must have two aces and
\cl{}\emph{K}, 13 tricks are visible.

Barring horrendous breaks and a ruff on the opening lead, this
27-point grand slam is lay-down.

\subsection{Intervention after a \cl{1} opening}

\hypertarget{ex1cintervene}
Some examples of bidding after opponents double or overcall after a \cl{1}
opening.

\begin{longtable}{rp{11cm}}
  \multicolumn{2}{l}{\emph{\underline{After a takeout / unusual double: \cl{1}--(Double)}}} \\
  1 & \hhand{J84,AJ82,T5,KT42} \\
    & If the double is an ordinary takeout double either
      \emph{Redouble} or bid \nt{1} showing a balanced 8-13\hcp\ if
      vulnerable.

      If the double shows majors, \emph{Redouble}. If partner doubles
      \sp{1}, you will be delighted to defend. \\
  2 & \hhand{A87,8,KJ8654,Q63} \\
    & Bid \di{2}. Slam is a real possibility despite the double. \\
  3 & \hhand{QT3,,JT9753,QT93} \\
    & Bid \di{1} (5-8\hcp). If partner bids \he{1}, you will bid
      \di{2} showing the long suit. \\
  4 & \hhand{AQ,A863,QJT,JT85} \\
    & Bid \nt{2} showing a balanced $14^+$\hcp\ hand and good stoppers
      in the majors. If the double is real (not a mistake showing
      clubs), the information of length in majors on the right is
      likely to be useful in the play. \\
  5 & \hhand{6,KJT5,A732,JT87} \\
    & Bid \cl{3} showing the 4-4-4-1 hand with a black singleton. \\
\end{longtable}

\begin{longtable}{rp{11cm}}
  \multicolumn{2}{l}{\emph{\underline{After a direct 1-level overcall: \cl{1}--(\sp{1})}}} \\
  6 & \hhand{Q,AQJ32,KJ63,J97} \\
    & Bid \he{2} which is natural and game forcing. \\
  7 & \hhand{4,J8654,T976,KJ6} \\
    & \emph{Double} to show 5-8\hcp.\\
  8 & \hhand{T953,4,A764,AQ92} \\
    & Bid \di{3}, unusual positive showing 4-4-4-1 with a red
      singleton. \\
  9 & \hhand{J,Q652,AQT964,T2} \\
    & Bid \di{2}, natural and forcing. \\
  10 & \hhand{953,AT43,AJ72,95} \\
    & Bid \sp{2}. There is enough to force game but no suit to bid and
      no stopper to bid \nt{}. \\
\end{longtable}

\begin{longtable}{rp{11cm}}
  \multicolumn{2}{l}{\emph{\underline{After an unusual no-trump overcall showing minors: \cl{1}--(\nt{1})}}} \\
  11 & \hhand{T9,AQ64,K862,AQ5} \\
     & \emph{Double} for penalties. If partner bids hearts, explore for
       slam. If not, you can also bid no-trump since the combined hands
       are in the slam zone. \\
  12 & \hhand{AT942,Q4,743,T98} \\
     & Bid \sp{2} (non-forcing). \\
  13 & \hhand{K9743,AQ98,92,87} \\
     & Bid \di{2} showing spades and forcing to game. \\
  14 & \hhand{AQ2,AT82,KJ3,874} \\
     & \emph{Double} showing values with a balanced hand. If partner
       bids \nt{2}, you can show the 4-card hearts on the way to
       \nt{3}. \\
  15 & \hhand{A4,J98,KT64,KT87} \\
     & \emph{Double} (penalty oriented) showing values with a balanced
       hand. There will be a massacre if the final contract is in
       either minor. \\
\end{longtable}

\begin{longtable}{rp{11cm}}
  \multicolumn{2}{l}{\emph{\underline{After a 2-level overcall: \cl{1}--(\he{2})}}} \\
  16 & \hhand{AQT,85,K74,KQT96} \\
     & Bid \cl{3} which is natural and forcing. The main reason for
       not cue bidding is that this hand will make an excellent dummy
       should partner bid \di{3} or \sp{3} which you will happily
       raise showing slam interest by bypassing \nt{3}. \\
  17 & \hhand{9872,52,AKJ4,T64} \\
     & \emph{Double}. This is more flexible than cue bidding
       \he{3}. Partner can bid \nt{2} with a stopper and then you
       could bid \cl{3} (\emph{Stayman}). \\
  18 & \hhand{QJ432,A6,JT63,K4} \\
     & Bid \sp{2}. \\
  19 & \hhand{4,KJT94,QJ7,A732} \\
     & \emph{Pass}. You are certain partner will bid again and you
       hope it is a double. The penalty will be a rich one if so. \\
  20 & \hhand{AK64,8765,AKQ7,7} \\
     & Bid \he{3}---game-forcing with no heart stopper and no long
       suit. You can explore slam after getting more information from
       partner. \\
\end{longtable}

\subsection{The \di{1} opening}

\hypertarget{ex1d}
Some examples of bidding after a \di{1} opening.

\begin{longtable}{rp{11cm}}
  \multicolumn{2}{l}{\emph{\underline{Opening bid}}} \\
  1 & \hhand{63,K4,AKJ9,KT984} \\
    & Open \di{1} and if partner bids \he{1}, rebid
      \cl{2}. Alternatively, open \nt{1}. \\
  2 & \hhand{AJ76,2,AQJ62,T72} \\
    & Open \di{1} rebid \sp {1} if partner bids \he{1}. \\
  3 & \hhand{Q76,J3,AQ9,AT982} \\
    & Open \di{1} and rebid \nt{1} over \he{1}/\sp{}. You cannot bid
      \cl{2} which would show an unbalanced hand. \\
  4 & \hhand{QT9,Q97,Q4,AJ962} \\
    & \emph{Pass} with this weak 11-point hand. \\
  5 & \hhand{65,T,AKQT8,KQT97} \\
    & Open \di{1} and rebid \cl{3} over \he{1}/\sp{} showing 5-5 in
      the minors. \\
\end{longtable}

\begin{longtable}{rp{11cm}}
  \multicolumn{2}{l}{\emph{\underline{Responses to a \di{1} opening}}} \\
  6 & \hhand{97,AK5,QJ873,KQ5} \\
    & Bid \di{2} showing at least a limit raise. Raise to game over
      \nt{2} or find a forcing bid if opener rebids a minimum. You
      want partner to be declarer in \nt{} with the weak doubleton
      spade. \\
  7 & \hhand{Q95,5,AKQ532,K64} \\
    & Bid \he{3}---a splinter showing the singleton heart and fine
      diamond support. \\
  8 & \hhand{7,AK942,KQJ54,A8} \\
    & Bid \he{1} and use \emph{RKCB} if opener supports
      hearts. Otherwise, jump to \di{3} if opener responds with \nt{1}
      showing the two-suiter and indicating slam interest. \\
  9 & \hhand{6,AK74,42,AKT943} \\
    & Bid \cl{2} and hearts next in the search for the best game
      contract (or slam if opener raises clubs). \\
  10 & \hhand{76,9,AJT642,8532} \\
    & Bid \di{3} (or \di{4} if non-vulnerable) interfering with
      opponent's possible game. \\
\end{longtable}

\begin{longtable}{rp{11cm}}
  \multicolumn{2}{l}{\emph{\underline{Rebids after partner's one-over-one response: \di{1}--\sp{1}--???}}} \\
  11 & \hhand{82,75,AQ52,AKT65} \\
     & Rebid \cl{2}. \\
  12 & \hhand{6,KT,AJT87,KQJ92} \\
     & Rebid \cl{3} showing 5-5 in the minors. \\
  13 & \hhand{KT92,9,AKT64,K65} \\
     & Rebid \sp{3} showing strong support and a singleton / void. \\
  14 & \hhand{AT4,Q76,J964,AK8} \\
     & Rebid \nt{1}. Raising spades is inadvisable with this flat
       hand. \\
  15 & \hhand{KT4,4,QJ974,AKQ4} \\
     & Rebid \cl{2} as the least worst evil---if partner bids again,
       you can show the spade support. \\
\end{longtable}

\subsection{Major suit openings}

\hypertarget{ex1h}
Some examples of bidding after a \he{1} or \sp{1} opening.

\begin{longtable}{rp{11cm}}
  \multicolumn{2}{l}{\emph{\underline{Opening bid}}} \\
  1 & \hhand{AT9765,Q8,K6,K52} \\
    & The quintessential \sp{1} bid. \\
  2 & \hhand{KJT6,AKJT92,K8,9} \\
    & Open \cl{1}---there are 15\hcp, a very good suit and a
      singleton. With unfavourable vulnerability, it may be better to
      bid \he{1} since opponents may intervene at a high level after
      \cl{1}. \\
  3 & \hhand{QJ9654,KT5,K8,Q7} \\
    & Open \sp{1}---this is not a great hand and many may choose to
      pass it or open \sp{2}. \\
  4 & \hhand{QT752,A74,AJ7,A6} \\
    & Open \sp{1}. A case can be made for opening this hand with
      \nt{1} and with \hearts{KJ4} and \clubs{KJ} (same \hcp), it
      would be preferable to open \nt{1}. \\
  5 & \hhand{32,KQ8743,QJ6,AK} \\
    & Open \he{1}. Although there are 15 \hcp, the suit is not good
      enough to play against a singleton and the hand has no
      singletons of its own. \\
\end{longtable}

\begin{longtable}{rp{11cm}}
  \multicolumn{2}{l}{\emph{\underline{Responses to a \sp{1} opening}}} \\
  6 & \hhand{AJT9,KJ8,T97,KJ6} \\
    & Bid \sp{4}---it would be a very unusual hand with partner for
      there to be a slam. Opponents do not know if your hand type is a
      weak distributional hand or this one. \\
  7 & \hhand{AQ982,AT8,4,KT76} \\
    & Bid \di{4} (splinter) with real slam potential. \\
  8 & \hhand{98732,A5,Q,T9743} \\
    & Bid \sp{4}---the textbook example of a game raise. Contrast to
      hand \#6. \\
  9 & \hhand{K832,A65,AKJ9,74} \\
    & Bid \nt{2}---game-forcing raise showing at least 4-card
      support. If partner shows shortness in clubs or hearts, slam is
      a distinct possibility. \\
  10 & \hhand{AJ874,4,Q53,AT95} \\
    & Bid \he{4} (splinter). Another hand with good slam potential if
      partner's hand matches. \\
\end{longtable}

\begin{longtable}{rp{11cm}}
  \multicolumn{2}{l}{\emph{\underline{Responses to a \he{1} opening}}} \\
  11 & \hhand{QT632,K72,A532,T} \\
     & Bid \sp{1}. If partner raises, you can bid game. If partner
       bids \nt{1}, \cl{2} or \di{2}, you will show limit raise values
       with \he{3}. Partner will know you have only 3 hearts since
       ther was no direct raise. \\
  12 & \hhand{AJ763,972,AK753,} \\
     & Bid \sp{1} and if partner raises, you will explore slam. If
       partner bids \cl{2} (likely), you will bid \di{2} (fourth-suit
       forcing). If partner rebids \he{2}, you could bid \he{5}
       (asking about trump quality) or \cl{4} (splinter). This is a
       difficult hand to assess since opposite the first opening hand
       below, a grand slam is on but opposite the second, no game is
       possible.

       \vhand[Opener 1]{8,AKQ863,QJ7,T76}
       \vhand[Opener 2]{86,Q8543,J6,AKQ6} \\
  13 & \hhand{QT,AT98,432,Q965} \\
     & Bid \cl{3}---a constructive \emph{Bergen} raise. \\
  14 & \hhand{Q76,J876,,AJ9853} \\
     & Bid \he{4}. It is certain that the opponents have some high
       card points so this makes them start at a high-level if they
       are going to bid. \\
  15 & \hhand{A94,Q643,JT3,A62} \\
     & Bid \di{3}---a \emph{Bergen} limit raise. \\
\end{longtable}

\subsection{The \cl{2} opening}

\hypertarget{ex2c}
Some examples of bidding after a \cl{2} opening.

\begin{longtable}{rp{11cm}}
  \multicolumn{2}{l}{\emph{\underline{Opening bid}}} \\
  1 & \hhand{KJ62,3,92,AQJ982} \\
    & A good example of a hand that should be opened with a bid of
      \cl{2}. \\
  2 & \hhand{QT6,KQ6,63,AQ843} \\
    & Bid \di{1} not \cl{2}. \\
  3 & \hhand{K3,,AJ82,AQJT974} \\
    & Bid \cl{1}. This hand is too good for a \cl{2} opening. \\
  4 & \hhand{Q86,A6,T8,AKQ874} \\
    & Bid \cl{1} and rebid \cl{2}. Let partner be declarer in \nt{} if
      that is the right spot. \\
  5 & \hhand{62,87,QT,AKQJ982} \\
    & Bid \nt{3}---``Gambling'', showing a solid suit with no ace or
      king outside. \\
\end{longtable}

\begin{longtable}{rp{11cm}}
  \multicolumn{2}{l}{\emph{\underline{Responses to a \cl{2} opening}}} \\
  6 & \hhand{AKT6,J865,T9,976} \\
    & Bid \di{2}. This is a perfect hand to enquire about majors. If
      partner bids a major or \cl{3}, pass (you need at least another
      queen to raise partner's major). If partner bids \nt{2}, correct
      to \cl{3}. \\
  7 & \hhand{KT9832,5,975,KJ7} \\
    & Bid \sp{2}. If partner raises spades, raise to game. Pass if he
      denies spades by rebidding \cl{3} or bid \cl{3} if he rebids
      \nt{2}. \\
  8 & \hhand{K85,KJ95,AT63,93} \\
    & Bid \nt{2} (invitational). If partner accepts game by bidding
      \he{3}, bid \he{4}. If partner accepts with \sp{3}, raise to
      \nt{3}. \\
  9 & \hhand{AJT763,KQ9,T7,Q2} \\
    & Bid \sp{3}. This is forcing to game and shows at least 6
      spades. Pass if partner signs-off in \nt{3}. \\
  10 & \hhand{K73,942,A932,973} \\
    & Bid \cl{3} forcing \emph{LHO} to come in at the three-level. The
      Law of Total Tricks will protect you \ldots \\
\end{longtable}

\begin{longtable}{rp{11cm}}
  \multicolumn{2}{l}{\emph{\underline{Rebids after partner's invitational response in a suit: \cl{2}--\he{2}--???}}} \\
  11 & \hhand{KQJ5,53,4,AQT965} \\
     & Bid \sp{2}. This hand will play better in one of your suits so
       let partner know you have four spades. \\
  12 & \hhand{432,AQ,K7,AT7643} \\
     & \emph{Pass}. There is no reason to think there is a better
       spot. \\
  13 & \hhand{3,AQT8,T8,AKJT84} \\
     & Bid \sp{3} (splinter) or \he{4}. Ten tricks should be on with
       this dummy. \\
  14 & \hhand{96,KJ63,JT,AKQ74} \\
     & Bid \he{3}. Although the hand is a maximum, the shape is not
       inspiring. It may have been preferable to open \di{1} with this
       hand. \\
  15 & \hhand{KJ2,52,3,AKJT962} \\
     & Bid \cl{3}. Although you have a doubleton heart, the clubs are
       good enough to play opposite a void. It must be better to have
       it as trumps. \\
\end{longtable}

\subsection{The \di{2} opening}

\hypertarget{ex2d}
Some examples of bidding after a \di{2} opening.

\begin{longtable}{rp{11cm}}
  \multicolumn{2}{l}{\emph{\underline{Responses to \di{2}}}} \\
  1 & \hhand{QJ543,AT5,KT5,AT} \\
    & Bid \sp{4}. Why mess about? \\
  2 & \hhand{54,A9,AT87643,63} \\
    & \emph{Pass}. You would also pass if one of the low diamonds was
      a low heart since bidding \he{2} may land you in a 3-3 fit. \\
  3 & \hhand{2,JT73,KQ64,K852} \\
    & Bid \he{2}. If partner is 4-3-1-5, he will bid \sp{2} which can
      be corrected to \cl{3}. Do not ask for shape since that may push
      bidding to the 4-level. \\
  4 & \hhand{AJ,A93,AJT97,T87} \\
    & Bid \nt{3}. Diamonds are well under control and there are no
      better prospects for game. \\
  5 & \hhand{AQT65,KJ,987,AQ5} \\
    & Bid \nt{2}---there is a grand slam possible here. Whatever
      partner bids, you will bid diamonds next to ask about controls. \\
\end{longtable}

\begin{longtable}{rp{11cm}}
  \multicolumn{2}{l}{\emph{\underline{Responding over RHO's 2-level suit overcall: \di{2}--(\he{2})--???}}} \\
  6 & \hhand{QT6,JT9,KQJ4,A64} \\
    & \emph{Double}. This can get ugly since opponent is bidding at
      the 2-level with at most seven trumps and without the balance of
      \hcp. \\
  7 & \hhand{9,T64,AQ86543,K8} \\
    & \emph{Pass}. You could double but that would probably drive the
      opponents to spades which is a better spot. \\
  8 & \hhand{84,3,AKT5,987653} \\
    & Bid \cl{4}. This is a pre-emptive bid to make \emph{LHO} decide
      whether to support at the 4-level. \\
\end{longtable}

\begin{longtable}{rp{11cm}}
  \multicolumn{2}{l}{\emph{\underline{Responding over RHO's 3-level suit overcall / cue bid: \di{2}--(\di{3})--???}}} \\
  9 & \hhand{952,A95,AJ73,743} \\
    & \emph{Double}. \\
  10 & \hhand{K93,AQT,AT2,JT94} \\
    & A \emph{Double} is probably best with favourable vulnerability.

      However, with unfavourable vulnerability, it is a choice between
      \nt{3} (if you feel lucky) and \cl{5} (more realistic). \\
  11 & \hhand{AT9842,843,T5,74} \\
    & Bid \sp{3} (non-forcing). \\
\end{longtable}

\begin{longtable}{rp{11cm}}
  \multicolumn{2}{l}{\emph{\underline{Responding after RHO's double: \di{2}--(Double)--???}}} \\
  12 & \hhand{K962,K4,AQJ6,862} \\
     & \emph{Redouble}. There could be overtricks here even if partner
       plays in a 4-1 diamond fit. \\
  13 & \hhand{KJ965,J864,J4,Q7} \\
     & Bid \sp{2}---if opponents compete, you can try hearts next. \\
  14 & \hhand{T642,QT63,JT,K75} \\
     & \emph{Pass}. Let partner describe his shape with a redouble or
       bid. \\
  15 & \hhand{A5,KQT,KT9863,Q2} \\
     & \emph{Redouble}. You have a lot of diamonds and good spot
       cards. Even if opponent's have a 4-4 spade fit, they may not
       find it and even if they do it is likely you have a penalty
       double against them in spades. \\
\end{longtable}

\subsection{``Gambling'' and ``Namyats'' openings}

\begin{longtable}{rp{11cm}}
  \multicolumn{2}{l}{\emph{\underline{Responses to a ``Gambling'' \nt{3}}}} \\
  1 & \hhand{32,AK85,AKJ43,JT} \\
    & Bid \di{4} asking partner to show singletons or voids. If he is
      short in spades, you can commit to a club slam. \\
  2 & \hhand{AQJ,9743,T6,AJ86} \\
    & \emph{Pass}. Opponents may be able to run some hearts but the
      odds are in your favour. Even if someone has five hearts, he may
      not be on lead or the suit may be blocked. \\
  3 & \hhand{A92,AK97652,,A85} \\
    & Bid \nt{5}. This asks partner to bid \di{7} with
      \emph{AKQJ}. You certainly want to be in \di{6} although there
      are no guarantees. \\
  4 & \hhand{A,AKQT84,KQJ9,54} \\
    & Bid \cl{6} which should be cold. \\
  5 & \hhand{QJ84,65,T87,JT97} \\
    & Bid \cl{5}. You don't care what partner's suit is (although it
      looks to be diamonds). What you do know is that opponents can
      make a lot of tricks in hearts (or even spades) and this robs
      them of room to find their best spot. \\
\end{longtable}

\begin{longtable}{rp{11cm}}
  \multicolumn{2}{l}{\emph{\underline{Responses to a ``Namyats'' \di{4} opening}}} \\
  6 & \hhand{T,A765,KQ95,AK32} \\
    & Bid \nt{4} (\emph{RKCB}). \\
  7 & \hhand{874,KJT9,KQ65,KJ} \\
    & Bid \sp{4}. Partner cannot have many aces in addition to a solid
      suit (he probably would have opened \cl{1} if so) so slam is out
      of question. \\
  8 & \hhand{972,QJ,AK652,AJT} \\
    & Bid \he{4}, a relay to partner's suit. You plan to cue bid
      \cl{5} inviting slam and if partner has a cue bid in hearts, you
      can bid \sp{6}. \\
  9 & \hhand{J752,A92,AKQ53,4} \\
    & Bid \nt{4} (\emph{RKCB}). If partner shows 3 key cards, you will
      bid \sp{7}. This is likely to be lay-down after the opening
      lead. \\
  10 & \hhand{872,AQJ73,,AT742} \\
    & With a solid suit and the heart king, \sp{7} is odds-on. Since
      there is no way to confirm both of these (an asking bid will
      only find the heart king), it is probably best to simply bid
      \sp{6}. The success of the slam may depend on the heart finesse,
      finally. \\
\end{longtable}

\end{document}
