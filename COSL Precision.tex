\documentclass[a4paper,article,oneside]{memoir}
\counterwithout{section}{chapter}
\setsecnumdepth{subsection}
\maxtocdepth{subsection}
\usepackage{microtype}
\usepackage{longtable}
\usepackage{nicefrac}
\usepackage{hyperref}
\usepackage{marginnote}
%\usepackage[dvipsnames]{xcolor} % moved to grbbridge.sty
\usepackage{grbbridge}
\setboolean{spellten}{true}
\usepackage{url}
\newcommand{\gap}{\vspace{\baselineskip}}
\newcommand{\hcp}{\textsc{hcp}}
\newcommand{\sq}{\textsc{sq}}
\newcommand{\ltc}{\textsc{ltc}}

\newcommand{\orf}[1]{#1\textcolor{ForestGreen}{\dag}} % One round force
\newcommand{\gf}[1]{#1\textcolor{Orange}{\ddag}} % Game force
\newcommand{\excp}[1]{\textcolor{MidnightBlue}{#1}} % Exception
\newcommand{\hyp}[1]{\hyperlink{#1}{$\hookrightarrow$}} % Hyperlink

\begin{document}

\title{COSL Precision Bidding System}
\author{Sudhir Shenoy}
\date{v2.70, 6 June 2021}
\maketitle

\tableofcontents

\gap\gap

\emph{Note}: In the text, bids that are forcing to game are marked
with a double-dagger (\gf{}) symbol. Bids that are forcing for at
least one round are marked with a dagger (\orf{}) symbol. Descriptions
of exceptions or unintuitive bids are highlighted \excp{like so}.

An right-pointer (\hyp{}) symbol in the text is a \emph{hyperlink} that
jumps to a related section describing follow-up bids or
examples. Depending on the PDF viewer used, this may also be outlined
by a box.

\pagebreak

\section{Opening bids}

All strong hands (\excp{with one exception\footnote{Balanced 22-23
    point hands are opened \Nt{2}.}}) are opened \orf{\Cl{1}} which is
forcing for one round.

In general, a major suit opening shows $5^+$-cards and the higher
ranking suit is opened with suits of equal length. A no-trump opening
shows a balanced hand with a possible 5-card minor. A club suit can be
opened with \Cl{2} with six or more cards in the suit.

A \Di{1} opening would normally be made with at least a 3-card holding
but could \excp{sometimes be made with a doubleton} when bidding
\Nt{1} or \Cl{2} is not attractive e.g., \hhand{AQJT,KQ,76,J7642}
(only 5 clubs) or \hhand{AKT9,AK98,32,432} (both majors, two suits
unstopped). Three suited hands with a singleton or void in diamonds
are opened with an artificial bid of \Di{2}.

All opening bids from \Di{1} through \Di{2} are made with between 11
and 15 \emph{high card points} (\hcp). The strictly limited nature of
these openings means that, in general, partner is not forced to
respond with less than 8\hcp.

\begin{longtable}{>{\raggedright}p{1.5cm}p{9.5cm}}
  \hline
  \orf{\Cl{1}} & $16^+$\hcp\ (unbalanced) or $17^+$\hcp\
                 (balanced). Hands with a powerful $6^+$-card suit
                 that can play opposite a singleton and have 15\hcp\
                 with a void or singleton should also be opened with
                 \Cl{1}, e.g., \hhand{AQJT98,8,KQ7,QJT}.
                 \hyp{1c} \\
  \Di{1} & 11-15\hcp, $3^+$-cards in \Di{} (\excp{could be 2}), no
           5-card major and less than 6 clubs.
           \hyp{1d} \\
  \He{1},
  \Sp{1} & 11-15\hcp, $5^+$-cards in suit bid.
           \hyp{1major} \\
  \Nt{1} & 14-16\hcp\ in $1^{st}$/$2^{nd}$ positions and
           \excp{15-17\hcp\ in $3^{rd}$/$4^{th}$ position},
           balanced. May have a five-card minor and even a 5-4-2-2
           distribution with a five-card minor is acceptable with
           stoppers in the doubletons.
           \hyp{1nt} \\ 
  \Cl{2} & 11-15\hcp, $6^+$-card club suit (7\sq\ hand), may have a 4
           or 5-card major. \hyp{2c} \\
  \orf{\Di{2}} & 11-15\hcp, 5-4-3-1, 4-4-4-1 or 5-4-4-0 shape with
                 short diamonds (a 5-card suit, if present, would be
                 clubs). \hyp{2d} \\
  \He{2},
  \Sp{2} & 6-10\hcp, \emph{exactly} six cards in suit with two of the
           top three or three of the top five honours.
           \hyp{2major} \\
  \Nt{2} & 22-23\hcp, balanced hand, no 5-card
           suit. \hyp{2nt} \\
  \Sp{3},
  \He{3},
  \Di{3},
  \Cl{3} & Preemptive, 0-10\hcp, $7^+$-card suit (\sq\ of 9 when
           vulnerable and 8 non-vulnerable). Apply rule of 2/3/4.
           \hyp{3preempt} \\
  \Nt{3} & ``\emph{Gambling}'', solid $7^+$-card minor suit (\emph{AKQ} or
           better) with \emph{no outside ace or
           king}. \hyp{3nt}  \\
  \gf{\Cl{4}},
  \gf{\Di{4}} & \emph{Namyats}---long semi-solid major suit (usually
                $8^+$-cards) with 8 or more tricks, constructive.
                \excp{Used in $1^{st}$ or $2^{nd}$ seat only}.
                \hyp{namyats} \\
  \He{4},
  \Sp{4} & Preemptive with $7\nicefrac{1}{2}$ tricks. \\
  \hline
\end{longtable}

As in \emph{Standard American}, a \Nt{1} response over an opening of
one of a major is forcing for one round. Like the \emph{2/1 Game
  Force} system a two-over-one response is mostly forcing to game.

In most cases, the bidding in \emph{Precision} does not change with
vulnerability or seat position.

\pagebreak

\section{Responses to \Cl{1}}

\hypertarget{1c} The responses to \orf{\Cl{1}} can be negative, constructive
or positive. All positive responses are forcing to game, i.e., holding
such a hand, you want to be in a game contract opposite any random
8\hcp\ hand.

\begin{longtable}{>{\raggedright}p{2cm}p{9cm}}
  \multicolumn{2}{l}{\emph{\underline{Negative response}}} \\
  \orf{\Di{1}} & 0-7\hcp, no long major suit. \excp{Note that with an
                 ace and a king (3 controls) either in the same suit
                 or two different $4^+$-card suits, a positive
                 response should be made}.
                 \hyp{1c1d} \\
  \multicolumn{2}{l}{\emph{\underline{Constructive responses}}} \\
  \He{2},
  \Sp{2} & 4-7\hcp, $6^+$-card suit with two of the top four honours
           \excp{but not both \emph{A} and \emph{K} when a positive
           suit response should be made}.
           \hyp{1c2major} \\
  \multicolumn{2}{l}{\emph{\underline{Positive responses---forcing to game}}} \\
  \gf{\He{1}},
  \gf{\Sp{1}},
  \gf{\Cl{2}},
  \gf{\Di{2}} & $8^+$\hcp, $5^+$-cards in suit. There are no
                restrictions on suit quality.
                \hyp{1csuit} \\
  \gf{\Nt{1}} & 8-13\hcp, balanced hand with no five-card
                suit. \hyp{1c1nt} \\
  \gf{\Nt{2}} & $14^+$\hcp, balanced hand, \underline{forcing to
                \Nt{4}}. \hyp{1c2nt} \\
  \gf{\Sp{3}} & A solid 7 or 8 card suit (with or without side
                controls) that will play for no losers opposite a
                singleton, e.g., \emph{AKQJxxx} or
                \emph{AKQxxxxx}. \hyp{1cl3sp} \\
  \multicolumn{2}{l}{\emph{\underline{Unusual positive---three-suited hands without a 5-card suit}}} \\
  \gf{\Cl{3}} & 8-11\hcp\ or less than four controls, 4-4-4-1 shape
                with a black singleton (spades or clubs).
                \hyp{unusualpositive} \\
  \gf{\Di{3}} & 8-11\hcp\ or less than four controls, 4-4-4-1 shape
                with a red singleton (hearts or diamonds).
                \hyp{unusualpositive} \\
  \gf{\He{3}},
  \gf{\Nt{3}},
  \gf{\Cl{4}},
  \gf{\Di{4}} & $12^+$\hcp\ \underline{and} $4^+$-controls, 4-4-4-1
                shape with a singleton in the suit above the one bid (\Sp{},
                \Cl{}, \Di{}, \He{} respectively).
                \hyp{unusualpositive} \\
  \hline
\end{longtable}

\subsection{Bidding after a negative response}

\hypertarget{1c1d} Opener rebids no-trumps with balanced hands
(\Nt{1}: 17-19\hcp, \Nt{2}: 20-21\hcp, \Nt{3}: 24-26\hcp), a 5-card
suit with 16-21\hcp\ and jumps in a 5-card suit with powerful hands
($22^+$\hcp). The jump rebid in a suit may be made with a lower
point-count given greater playing strength.

Examples of bidding after a negative response can be found
here. \hyp{ex1c1d}

\begin{longtable}{ p{1.5cm}p{9.5cm}}
  \hline
  \multicolumn{2}{l}{\emph{\underline{Balanced hands}}} \\
  \Nt{1} & 17-19\hcp, balanced, no 5-card major (\excp{18-19\hcp\ in
           $3^{rd}$/$4^{th}$ position}). Responder's rebids are: \\
         & \begin{tabular}{lp{7cm}}
             \emph{Pass} & 0-5\hcp, no major suit to escape to. \\
             \orf{\Cl{2}} & 6-7\hcp, \emph{Stayman}.
                            \hyp{stayman} \\
             \orf{\Di{2}},
             \orf{\He{2}} & 0-7\hcp, transfer to \He{2} and \Sp{2}
                            respectively. Responder will invite with
                            7\hcp\ and pass with 0-6\hcp\ unless
                            opener \emph{super-accepts}.
                            \hyp{superaccept} \\ 
             \orf{\Di{4}},
             \orf{\He{4}} & \emph{Texas} transfers to \He{4} and
                            \Sp{4} respectively. \\
           \end{tabular} \\
  \Nt{2} & 20-21\hcp, balanced, \emph{may have a 5-card
           major}. Responder's rebids are: \\
         & \begin{tabular}{lp{7cm}}
             \emph{Pass} & 0-3\hcp. \\
             \orf{\Cl{3}} & 4-7\hcp, \emph{Puppet Stayman} asking for
                            5-card majors if any.
                            \hyp{puppetstayman} \\
             \orf{\Di{3}},
             \orf{\He{3}} & Weak, transfer to \He{3} and \Sp{3}
                            respectively. \\
             \Nt{3} & 4-5\hcp, sign-off. \\
             \orf{\Di{4}},
             \orf{\He{4}} & Transfer to \He{4} and
                            \Sp{4}---sign-off. \\
           \end{tabular} \\
  \Nt{3} & 24-26\hcp, balanced hand, \emph{may have a 5-card
           major}. Responder's rebids are: \\
         & \begin{tabular}{p{1.5cm}p{6.5cm}}
             \emph{Pass} & 0-4\hcp, balanced. \\
             \orf{\emph{4 of
             suit}} & 5-7\hcp, $5^+$-cards. Opener bids one above suit
                      (\Di{4}, \He{4}, \Sp{4} or \Nt{4}) to show fit
                      and start \emph{Roman key-card
                      Blackwood}.
                      \hyp{blackwood} \\  
             \Nt{4} & 5-7\hcp, no 5-card suit, quantitative. \\
           \end{tabular} \\
  \multicolumn{2}{l}{\emph{\underline{Unbalanced hands}}} \\
  \He{1},
  \Sp{1} & 5$^+$-card suit, non-forcing. Can be only four cards if
           opener started with a 4-4-4-1 shape (with a singleton
           minor, opener will rebid \He{1}). Responder's rebids are: \\ 
         & \begin{tabular}{lp{6.7cm}}
             \emph{Pass} & 0-4\hcp, especially when balanced. \\
             \orf{\Sp{1}} & 4-7\hcp, $4^+$-cards, may have three
                            hearts. It is important to bid the spades
                            before supporting hearts since opener may
                            have bid \He{1} holding a 4-4-4-1
                            distribution with both majors. \\
             \Nt{1} & 5-7\hcp, no 5-card suit, no 4-card spade after
                      \He{1}. May have 3-card support. This bid should
                      be avoided as far as possible to prevent the
                      strong hand from coming down. \\
             \Cl{2},
             \Di{2} & 5-7\hcp, 5-card suit, denies 3-card support. \\
             \emph{Single raise} & 4-5\hcp\ with $3^+$-card support. \\
             \emph{Double raise} & 6-7\hcp\ with $3^+$-card support. \\
             \gf{\emph{Jump shift}} & \emph{Splinter} with $4^+$-card
                                      support showing slam
                                      interest. E.g., bid \Cl{4} when
                                      holding \hhand{JT98,93,AJT987,5}
                                      after opener rebids \Sp{1}. A
                                      splinter of \Cl{3} would be
                                      slightly weaker showing a
                                      game-going hand. \\
           \end{tabular} \\
  \Cl{2},
  \Di{2} & $5^+$-card suit, may have a 4-card major,
           non-forcing. Responses have the same structure as over
           \He{1} and \Sp{1}. \\
  \orf{\He{2}},
  \orf{\Sp{2}} & Powerful hand with $22^+$\hcp\ and $5^+$-card suit,
                 equivalent of a \Cl{2} opener in \emph{Standard
                 American}. Responder's rebids are: \\
         & \begin{tabular}{lp{6.7cm}}
             \Nt{2} & 0-3\hcp, minimum, no support. \\
             \Nt{3} & 4-7\hcp\ maximum, spread values, no support. \\
             \emph{Raise} & 0-3\hcp, minimum, $3^+$-card support. \\
             \emph{Game raise} & 4-7\hcp, maximum, $3^+$-card support,
                                 no specific values in other suits. \\
             \orf{\emph{New suit}} & 4-7\hcp, values in suit, does not
                                     deny support for partner's
                                     suit. \\
           \end{tabular} \\
  \orf{\Cl{3}},
  \orf{\Di{3}} & Very strong unbalanced hand with a long minor and
                 good playing strength that is too strong for either
                 \Cl{2} or \Di{2}. E.g., \hhand{A,AK,KQJT876,QJ5} or
                 \hhand{KQJ5,6,A,AKQT964}. \\
  \gf{\He{3}},
  \gf{\Sp{3}} & Extremely powerful hand with a solid suit and at least
                nine tricks. This bid sets trumps and asks responder
                to cue-bid an ace or void. E.g.,
                \hhand{AKQJT98,4,KJ3,AK} or \hhand{65,AKQT7543,AKJ,}
                where a slam is on if responder can cue-bid. Responder
                bids: \\
         & \begin{tabular}{lp{6.7cm}}
             \orf{\emph{New suit}} & First-round control---ace or void
                                     in suit. \\
             \orf{\Nt{3}} & No first round control but has a king or
                      singleton in a non-trump suit. Opener rebids
                      \Cl{4} to ask which suit. \\
             \emph{Game raise} & Denies ace, king, singleton or void. \\
           \end{tabular} \\
  \hline
\end{longtable}

\subsection{Bidding after a constructive response}

\hypertarget{1c2major} Since responder is showing a strictly limited
hand of 4-7\hcp\ with a long suit, the opener needs to decide on the
best contract. If there is no chance for game or slam, he should pass
with a tolerance for responder's suit.

\begin{longtable}{ p{2cm}p{9cm}}
  \hline
  \emph{Pass} & Game unlikely. \\
  \He{4}, \Sp{4} & Sign-off, to play. \\
  \orf{\emph{New suit}} & Natural, $5^+$-card suit. Responder's rebids
                          are: \\
              & \begin{tabular}{>{\raggedright}p{1.8cm}p{6.3cm}}
                  \emph{Raise} & $3^+$-card support (or \emph{Qx}). \\
                  \emph{Rebid suit} & Minimum, no support. \\
                  \Nt{3} & Maximum, no support. \\
                  \orf{\emph{New
                  suit}} & Maximum, $3^+$-card support and a singleton
                           or void in suit bid. \\
                \end{tabular} \\
  \gf{\Nt{2}} & Showing support for suit and asking for
                shortness. Responders's rebids are: \\
              & \begin{tabular}{p{1.8cm}p{5cm}}
                  \emph{Rebid suit} & Minimum, no singleton or
                                      void. \\
                  \emph{New suit} & Singleton or void in bid suit. \\
                \end{tabular} \\
  \Nt{3} & \emph{AQ} or \emph{KQ} in suit. \\
  \orf{\Cl{4}} & \emph{Roman key-card} ask \excp{with modified
                 responses} since responder cannot have more than 2
                 key cards.
                 \hyp{blackwoodmod} \\
  \hline
\end{longtable}

\subsection{Bidding after a positive no-trump response}

\hypertarget{1c1nt} A no-trump response shows a balanced hand with
8-13\hcp\ (\Nt{1}) or $14^+$\hcp\ (\Nt{2}). After a \Nt{1} response,
opener can either (a) bid his own suit at the 2-level showing a
5-carder, (b) bid his own suit at the 3-level showing a very strong
hand with slam interest, (c) bid \orf{\Cl{2}} (\emph{Transfer Stayman})
or (d) raise no-trumps.

\textcolor{red}{\emph{All bidding sequences are forcing to game}}.
Examples of bidding after a positive no-trump response are available
here. \hyp{ex1cnt}

\subsubsection{\emph{Transfer Stayman}---\Cl{1}--1NT--\Cl{2}}

Responder's rebids after opener's \emph{Transfer Stayman} bid
are:

\begin{longtable}{ p{1.5cm}p{9.5cm}}
  \hline
  \Di{2} & 8-10\hcp, 4 card \He{}, may have 4 card \Sp{}. Opener's
           rebids are: \\
         & \begin{tabular}{lp{7cm}}
             \He{2} & Relay affirming fit in hearts---responder should
                      bid \Nt{2} with 4-3-3-3 or a second suit at
                      3-level. \\
             \Sp{2} & 4-card \Sp{}, no 4-card \He{}. \\
             \Nt{2} & No 4-card major. \\
           \end{tabular} \\
  \He{2} & 8-10\hcp, 4 card \Sp{}, denies 4-card \He{}.
           Opener rebids: \\
         & \begin{tabular}{lp{7cm}}
             \Sp{2} & Relay affirming fit in spades---responder bids
                      \Nt{2} with 4-3-3-3 else second suit at
                      3-level. \\
             \Nt{2} & No 4-card \Sp{}, may have 4-card \He{}. \\
           \end{tabular} \\
  \Sp{2} & 8-10\hcp, no 4 card major. Opener then bids \Nt{2} to ask
           for a further description. Responder's rebids are: \\
         & \begin{tabular}{lp{6cm}}
             \Cl{3},
             \Di{3} & 4-3-3-3 with 4-card suit. \\
             \He{3} & 4-4 in minors with three hearts. \\
             \Sp{3} & 4-4 in minors with three spades. \\
             \Nt{3} & 5-card minor. \\
           \end{tabular} \\
  \Nt{2} & 11-13\hcp, 4-3-3-3 shape. \Cl{3} by opener is then a relay
           asking responder to bid his 4-card suit (\Nt{3} with
           clubs). \\
  \Cl{3} & 11-13\hcp, 4-4-3-2 shape with 4 clubs. Opener bids \Di{3}
           as a relay and responder bids \He{3} with spades, \Sp{3}
           with hearts and \Nt{3} with diamonds.\\
  \Di{3} & 11-13\hcp, 4-4-3-2 shape with \Di{} and \He{}. \\
  \He{3} & 11-13\hcp, 4-4-3-2 shape with \He{} and \Sp{}. \\
  \Sp{3} & 11-13\hcp, 4-4-3-2 shape with \Sp{} and \Di{}. \\
  \Nt{3} & 11-13\hcp, 5-card minor suit. \Cl{4} by opener is then a
           relay asking responder to bid his suit. \\
  \hline
\end{longtable}

\subsubsection{Suit bid after \Cl{1}--1NT}

When opener has a possible trump suit, he bids it asking responder to
show shape and point range. With a very strong hand and a good suit,
he can jump in the suit setting trumps and asking partner for his
holding in that suit.

\begin{longtable}{>{\raggedright}p{1.5cm}p{9.5cm}}
  \hline
  \Di{2},
  \He{2},
  \Sp{2},
  \Nt{2} & $5^+$-card suit, \emph{support-asking bid} (\Nt{2} shows
           clubs). Responder's rebids are (minimum = 8-10\hcp, maximum
           = 11-13\hcp, support = \emph{Hxx}, \emph{xxxx} or
           better): \\
         & \begin{tabular}{ll}
             \emph{1 step} & Minimum and no support. \\
             \emph{2 steps} & Minimum with support. \\
             \emph{3 steps} & Maximum and no support. \\
             \emph{4 steps} & Maximum with support. \\
           \end{tabular} \\
  \Cl{3},
  \Di{3},
  \He{3},
  \Sp{3} & Very strong hand with slam interest, sets
           trumps and asks for responder's holding in
           the suit bid. Responses are in steps: \\
         & \begin{tabular}{ll}
             \emph{1 step} & Two or three spot cards. \\
             \emph{2 steps} & Doubleton honour. \\
             \emph{3 steps} & Tripleton honour. \\
             \emph{4 steps} & Two honours doubleton. \\
             \emph{5 steps} & Two honours tripleton. \\
             \emph{6 steps} & Four card support. \\
           \end{tabular} \\
  \hline
\end{longtable}

\underline{Any suit bid} after a support-asking bid is an $\epsilon$
control-asking bid in that suit. \hyp{epsilon}

\subsubsection{No-Trump raise after \Cl{1}--1NT}

Since \excp{a raise to \Nt{2} shows a club suit} (see above), there
are only two possible raises in no-trumps---\Nt{3} and \Nt{4}.

\begin{longtable}{ p{1.5cm}p{9.5cm}}
  \hline
  \Nt{3} & Minimum balanced hand with no four-card major nor interest
           in slam. \\
  \Nt{4} & Quantitative raise with a balanced hand and no four-card
           major inviting slam if responder has a maximum. \\
  \hline
\end{longtable}

\subsubsection{Bidding after \Cl{1}--2NT}

\hypertarget{1c2nt} A \Nt{2} response shows $14^+$\hcp\ and
immediately puts the partnership in slam range.  It is, therefore,
\underline{forcing to \Nt{4}}. Responses are:

\begin{longtable}{ p{2.5cm}p{8.5cm}}
  \hline
  \Cl{3} & \emph{Baron}---asking responder to show 4-card suits
           upwards (\Nt{3} would show 4-3-3-3 with four clubs). \\
  \Di{3},
  \He{3},
  \Sp{3},
  \Cl{4} & $5^+$-card suit. Subsequent bidding is natural. \\
  \Nt{3} & Asks responder to clarify his point range as follows: \\
         & \begin{tabular}{lp{6.5cm}}
             \Cl{4} & 14-15\hcp. \\
             \Di{4} & 16-17\hcp. \\
             \He{4} & 18-19\hcp. \\
             \Sp{4} & 20-21\hcp. \\
             \Nt{4} & $22^+$\hcp. \\
           \end{tabular} \\
  \hline
\end{longtable}

\subsection{Bidding after a positive suit response}

\hypertarget{1csuit} Opener rebids no-trumps with a balanced
hand. With support for responder's suit he has the option of
initiating a series of \emph{asking bids}.\footnote{As a rule of
  thumb, asking bids should not be used if two of the outside suits
  are missing first-round controls. This is because once asking bids
  are triggered, there is no way to return to natural bidding.} With
an unbalanced hand and no support for responder's suit, opener bids
his suit and further bidding is natural.

With a 4-4-4-1 distribution, if responder bids the singleton suit,
opener should rebid no-trumps. E.g., \Nt{1} over \He{1} or \Nt{2} over
\Di{2}. If responder rebids his suit, opener should rebid \Nt{}.
Partner should cater to this possibility and insist on his suit as
trumps only with a $6^+$-card suit.

\textcolor{red}{\emph{All bids short of game are forcing}}.
Examples of bidding after a positive response in a suit are available
here. \hyp{ex1suit}

\begin{longtable}{ p{2.5cm}p{8.5cm}}
  \hline
  \emph{New suit} & $5^+$-card suit, denies 3-card support for
                    responder's suit. Subsequent bids are natural to
                    find the correct game contract. Responder's rebids
                    are: \\
                  & \begin{tabular}{lp{5.5cm}}
                      \emph{New suit} & $4^+$-card suit. \\
                      \emph{Raise} & $3^+$-card support. \\
                      \emph{Rebid suit} & $6^+$-card suit, semi-solid
                                          if minor. \\
                      \emph{No-trumps} & 5-3-3-2 shape, values in
                                         unbid suits. \\
                    \end{tabular} \\
  \Nt{1} & 17-19\hcp, balanced (\Nt{2} over \Cl{2} or \Di{2}). No
           5-card major, may have 3-card support (shows shape first).

           After \He{1} or \Sp{1}, a \emph{jump rebid} of the suit by
           responder would show a semi-solid $6^+$-card suit. A
           \emph{jump shift} would show a 5-5 two-suited limited
           hand---typically \emph{KQxxx} in both suits with nothing
           outside. \\
  \Nt{2} & 20-21\hcp, balanced (\Nt{3} over \Cl{2} or \Di{2}). No
           5-card major, may have 3-card support. \\
  \emph{Single raise} & $\gamma$ \emph{trump-asking} bid---indicates a
                        powerful hand with distinct slam
                        possibilities. \hyp{gamma}

                        Any further new suits bid by opener after the
                        $\gamma$ response will be $\epsilon$
                        \emph{suit control-asking} bids.
                        \hyp{epsilon} \\ 
  \emph{Double raise} & Minimum balanced hand, 4-card support with
                        good controls. Avoids $\gamma$/$\epsilon$
                        sequences. \\
  \emph{Game raise} & Minimum balanced hand, 4-card fit with poor
                      controls. \\
  \Sp{3},
  \Cl{4},
  \Di{4},
  \He{4} & \emph{Splinter} bid with $4^+$-card support and a singleton
           or void in the bid suit. \\
  \Nt{4} & \emph{Roman key-card Blackwood}.
           \hyp{blackwood} \\
  \hline
\end{longtable}

In general, over a minor suit positive response, a rebid in no-trumps
by opener is preferred if it is likely that the final contract will be
\Nt{3}. This will ensure that the strong hand is declarer. Similarly,
with a 5-card minor suit, opener should consider rebidding no-trumps
rather than his suit since, in most cases, \Nt{3} is preferable to
five of a minor.

\subsection{Bidding after a \Sp{3} response}

\hypertarget{1cl3sp} The \gf{\Sp{3}} response is artificial and shows a solid
$7^+$-card suit headed by \emph{AKQ} with or without outside
controls. The suit should be obvious on most occasions.

\textcolor{red}{\emph{All bidding sequences are forcing to game}}.
Bidding examples are available here. \hyp{ex1c3s}

Opener's rebids are:
\begin{longtable}{p{1cm}p{10cm}}
  \hline
  \Nt{3} & To play. Responder should pass unless he has many outside
           controls. \\
  \Cl{4} & $\beta$-ask for \emph{outside} controls---\excp{responses
           from the 0-3 scale}. \hyp{beta}

           Any following non-trump suit bid is an $\epsilon$
           \emph{suit control} ask.
           \hyp{epsilon} \\ 
  \Di{4} & Asks responder to bid his suit (diamonds are indicated by a
           \Nt{4} response).

           A subsequent bid in a new suit would be an $\epsilon$
           \emph{suit control} ask.
           \hyp{epsilon} \\
  \He{4},
  \Sp{4} & $5^+$-card suit, to play. Responder should pass with 3-card
           support or doubleton honour. \\
  \hline
\end{longtable}

\subsection{Bidding after an \emph{unusual positive} response}

\hypertarget{unusualpositive} An \emph{unusual positive} response
shows a 4-4-4-1 distribution. With less than 4 controls (typically,
8-13\hcp), the singleton is not shown directly---\gf{\Cl{3}} is bid with a
black singleton and \gf{\Di{3}} with a red singleton. With four or more
controls (typically $12^+$\hcp), the singleton is immediately shown by
bidding the suit below the singleton.

After \gf{\Cl{3}} or \gf{\Di{3}}, opener bids the next higher suit to ask
responder to clarify where his singleton lies.  Responder bids one
step above the relay to show the lower ranking suit and two steps
above to show the higher ranking suit.

\textcolor{red}{\emph{All bidding sequences are forcing to game}}.

\begin{longtable}{p{3cm}p{2cm}|p{3cm}p{2cm}}
  \multicolumn{4}{l}{\emph{Possible sequences after an unusual positive of \Cl{3}/\Di{3}}}\\
  \hline
  \Cl{1}--\Cl{3}--\Di{3}--\He{3} & 4-4-4-1 (\Cl{}) & \Cl{1}--\Di{3}--\He{3}--\Sp{3} & 4-4-1-4 (\Di{}) \\
  \Cl{1}--\Cl{3}--\Di{3}--\Sp{3} & 1-4-4-4 (\Sp{}) & \Cl{1}--\Di{3}--\He{3}--\Nt{3} & 4-1-4-4 (\He{}) \\
  \hline
\end{longtable}

Once the singleton is known, opener can bid the singleton suit to
initiate $\beta$ and ask for the number of controls held (\excp{the
  0-3 scale is used after a \gf{\Cl{3}} or \gf{\Di{3}} response and the $4^+$
  scale is used after the stronger
  responses}). \hyp{beta}

Examples of bidding after an unusual positive can be found
here. \hyp{ex1c3c}

\subsection{Handling intervention over \Cl{1}}

Over a double of \orf{\Cl{1}}, the additional bids of \emph{Redouble} and
\emph{Pass} are used to provide more granular information. When the
double is conventional (e.g., shows both majors), the bidding is the
same \excp{except that a bid of \Nt{1} would also confirm stoppers in
  both majors}. All other bids retain their normal meaning.

\begin{longtable}{ p{1.5cm}p{9.5cm}}
  \hline
  \multicolumn{2}{l}{\emph{\underline{After \Cl{1}--(Double)}}} \\
  \emph{Pass} & 0-4\hcp. \\
  \orf{\Di{1}} & 5-7\hcp, artificial. \\
  \gf{\emph{Redouble}} & $8^+$\hcp, usually balanced since \Nt{1}
                         cannot be bid without stoppers in the
                         opponent's suits. \\
  \gf{\Nt{1}} & Normal 8-13\hcp, but if the double shows a two-suited
                hand, shows stoppers in both implied suits. \\
  \emph{Others} & Same as over \Cl{1} without the intervention. \\
  \hline
\end{longtable}

After an overcall in a suit at the one-level, \emph{any suit or
  no-trump bid} is a positive response forcing to game. A \emph{trap
  pass} can be made when responder wants to double for penalties---in
this case, he will pass a re-opening double by opener.

When opponents overcall with \Nt{1}, the responses are different
depending on whether the overcall is a genuine strong hand or is
conventional showing a two-suiter (the \emph{unusual no-trump}). In
the latter case, the \emph{unusual over unusual} approach repurposes
the \gf{\Cl{2}} and \gf{\Di{2}} bids to show a game-going hand with a major
suit.

\begin{longtable}{>{\raggedright}p{2.5cm}p{8.5cm}}
  \hline
  \multicolumn{2}{l}{\emph{\underline{After a one-level suit overcall \Cl{1}--(\Di{1}/\He{1}/\Sp{1})}}} \\
  \emph{Pass} & 0-4\hcp\ or a \emph{trap pass}. \\
  \orf{\emph{Double}} & 5-8\hcp\ unbalanced or $5^+$\hcp, balanced. \\
  \gf{\emph{Suit}} & Natural, $8^+$\hcp, $5^+$-card suit. \\
  \gf{\emph{Jump to \Cl{3}, \Di{3}}} & Unusual positive with
                                       4-4-4-1. \\
  \gf{\emph{Cue-bid}} & $8^+$\hcp, balanced hand with \emph{no
                        stopper} in opponent's suit. \\
  \gf{\Nt{1}},
  \gf{\Nt{2}} & Usual meaning and promises a stopper in opponent's
                suit. \\
  \multicolumn{2}{l}{\emph{\underline{After an artificial no-trump overcall \Cl{1}--(\Nt{1}) (showing minors)}}} \\
  \emph{Double} & Modest high-card points, suitable for penalising one
                  of opponent's suits, usually no 5-card major. \\
  \gf{\Cl{2}},
  \gf{\Di{2}} & $8^+$\hcp, $5^+$-card heart or spade suit
                respectively. \\
              & These two bids are the so-called \emph{unusual over
                unusual} responses in which cue-bids of known suits
                correspond to forcing bids in the unbid suits. \\
  \He{2},
  \Sp{2} & Natural, non-forcing. \\
  \multicolumn{2}{l}{\emph{\underline{After a strong no-trump overcall \Cl{1}--(\Nt{1})}}} \\
  \emph{Pass} & 0-4\hcp. \\
  \emph{Double} & $5^+$\hcp, balanced---for penalties. \\
  \emph{Suit} & 5-8\hcp, $5^+$-card suit. \\
  \hline
\end{longtable}

Over higher level overcalls, bidding is largely natural but responses
over an \emph{unusual \Nt{2}} are still \emph{unusual over unusual}.

\begin{longtable}{ p{1.5cm}p{9.5cm}}
  \hline
  \multicolumn{2}{l}{\emph{\underline{After a suit overcall at 2-level}}} \\
  \orf{\emph{Double}} & 6-8\hcp, any shape. Any suit rebid by opener
                        would be a one-round force. \\
  \gf{\emph{Suit}} & Natural and forcing to game. Note that a jump to
                     \Di{3} over \Cl{2} would be an unusual
                     positive. \\
  \gf{\Nt{2}} & 8-10 or $14^+$\hcp, with stopper in opponent's suit. \\
  \gf{\Nt{3}} & 11-13\hcp\ with stopper in opponent's suit. \\
  \gf{\emph{Cue-bid}} & Values to be in game but no clear-cut
                        action---no long suit, no stopper in
                        overcaller's suit. \\
  \multicolumn{2}{l}{\emph{\underline{After an overcall of \Nt{2} (unusual no-trump)}}} \\
  \orf{\emph{Double}} & Penalty oriented, usually no 5-card major. \\
  \gf{\Cl{3}},
  \gf{\Di{3}} & $8^+$\hcp, $5^+$-card heart or spade suit respectively (\emph{unusual/unusual}). \\  
  \multicolumn{2}{l}{\emph{\underline{After an overcall at 3-level}}} \\
  \orf{\emph{Double}} & Balanced hand with $8^+$\hcp. \\
  \gf{\emph{Suit}} & Positive, natural, game forcing. \\
  \Nt{3} & 8-11\hcp\ with stoppers. \\
  \multicolumn{2}{l}{\emph{\underline{After an overcall at 4-level}}} \\
  \orf{\emph{Double}} & Shows values---support for partner if he bids
                        and provides defence if he passes. \\
  \emph{Suit} & Natural, non-forcing. \\
  \hline
\end{longtable}

\subsubsection{Intervention after a negative response}

If the intervention occurs after partner's negative response of
\orf{\Di{1}}, e.g., \Cl{1}--(\emph{Pass})--\Di{1}--(\emph{RHO doubles /
  bids}), opener should rebid as follows:

\begin{longtable}{ p{1.5cm}p{9.5cm}}
  \hline
  \multicolumn{2}{l}{\emph{\underline{After \Cl{1}--(Pass)--\Di{1}--(Double) (usually showing diamonds)}}} \\
  \emph{Pass} & Balanced minimum (no 5-card suit). \\
  \Nt{1} & Upper end of range (19\hcp) with diamond stopper. \\
  \emph{Suit} & Same meaning as without the double. \\
  \multicolumn{2}{l}{\emph{\underline{After a 1-level suit overcall \Cl{1}--(Pass)--\Di{1}--(\He{1}/\Sp{1})}}} \\
  \emph{Pass} & Balanced minimum (no 5-card suit). \\
  \orf{\emph{Double}} & For takeout with support for other suits. \\
  \emph{Suit} & Natural, at least 5-cards, non-forcing. \\
  \Nt{1} & Upper end of the range (19\hcp) with stopper. \\
  \Nt{2} & Same as \Nt{2} without interference but promises
           stopper. \\
  \orf{\emph{Cue-bid}} & Strong hand, lacking stopper in overcalled
                         suit. \\
  \multicolumn{2}{l}{\emph{\underline{After a no-trump overcall \Cl{1}--(Pass)--\Di{1}--(\Nt{1}) (showing minors)}}} \\
  \emph{Pass} & Balanced minimum (no 5-card suit). \\
  \emph{Double} & Penalty oriented. \\
  \orf{\Cl{2}} & Heart suit with extra values
                 (\emph{unusual/unusual}). \\
  \orf{\Di{2}} & Spade suit with extra values
                 (\emph{unusual/unusual}). \\
  \He{2},
  \Sp{2} & Natural, non-forcing. \\
  \Nt{2} & Upper-end of the \Nt{1} rebid range (19\hcp) with stoppers
           in both minors. \\
  \multicolumn{2}{l}{\emph{\underline{After an intervention above 1-level}}} \\
  \emph{Pass} & Balanced minimum (no 5-card suit). \\
  \emph{Others} & A little extra weight as compared to without the
                  intervention.

                  \excp{\emph{Unusual/unusual} over an unusual \Nt{2}
                  overcall}. \\
  \hline
\end{longtable}

Examples of bidding after opponents intervene can be found
here. \hyp{ex1cintervene}

\pagebreak

\section{Responses to \Di{1}}

\hypertarget{1d} A \Di{1} opening guarantees 11-15\hcp\ and at least
three diamonds although it could occasionally be a
doubleton. Regardless, the bid is \emph{not forcing} and partner can
pass with a poor hand (0-7\hcp\ and no 4-card major).

When opener does not have a genuine diamond suit, it is either because
(a) he is interested in the major suits but does not have a
five-carder, (b) he holds a club suit but cannot bid \Cl{2} because it
is less than six cards or (c) he has a balanced hand that cannot be
opened with \Nt{1}, i.e., 11-13\hcp. The opener's first rebid will
clarify which type of hand he holds---bid or raise a major suit with
(a), bid clubs with (b), or bid no-trumps or a minor with
(c). Opener's rebids will also classify his point range into a minimum
(11-13\hcp) or maximum (14-15\hcp).

The first priority for both partners is to establish a 4-4 major suit
fit if there is one so a bid by either partner that skips a major suit
implies that he does not hold four cards in that suit. If a major suit
fit is found, a real diamond suit holding in opener's hand may never
be mentioned.

Responder will always bid either spades or hearts (\orf{\He{1}},
\orf{\Sp{1}}, \gf{\He{2}}, \gf{\Sp{2}}) if he has a four-card or
longer major so any other bid \emph{denies four cards in either major
  suit}. As a consequence, \emph{Precision} does \emph{not} have the
concept of \emph{$4^{th}$-suit forcing} that is found in standard
systems and a simple rebid by responder in a new suit (which may be a
six-card suit) is always non-forcing. However, \excp{any new suit bid
  at the three-level (with or without a jump) is forcing}.

\subsection{Responder has a major suit}

With a four-card or longer major, responder will bid one, two or four
of a major depending on length and strength.  Responder's bids are:
\begin{longtable}{>{\raggedright}p{1.5cm}p{9.5cm}}
  \hline
  \orf{\He{1}} & $4^+$-cards in hearts, usually
                 $6^+$\hcp\footnote{Sometimes, with favourable
                 vulnerabilty and an extremely weak hand, a tactical
                 bid may be made to interfere with opponent's
                 game. E.g., holding \hhand{754,J852,985,654}, responder
                 could bid \He{1} planning to pass any rebid by
                 opener.}, may also have four spades. Note that with
                 4-4 in the majors and a strong hand, responder will
                 \emph{reverse}, i.e., rebid \Sp{2} after first
                 responding with \He{1}. \\
  \orf{\Sp{1}} & $4^+$-cards in spades, $6^+$\hcp, less than four
                 hearts. \\
  \gf{\He{2}},
  \gf{\Sp{2}} & Strong hand (slam interest) with either a very strong
                suit or a good suit with diamond support. For example,
                bid \He{2} with either \hhand{AJ,KQJT965,K4,K6} or
                \hhand{54,AKT96,KQJ98,A}. The first hand can play
                opposite a void so responder will rebid hearts unless
                opener raises (in which case responder will invoke
                \emph{RKCB}). With the second hand, if opener does not
                raise hearts, responder will bid diamonds. \\
               & After a \emph{jump shift}, the final contract will be
                 in one of---opener's suit, responder's suit or
                 no-trumps. Hence, \excp{a new suit by
                 \emph{responder} shows support for opener's suit with
                 shortness in the suit bid}. E.g.,
                 \Di{1}--\He{2}--\Nt{2}--\Cl{3} shows diamond support
                 and shortness in clubs. Similarly,
                 \Di{1}--(\emph{Pass})--\He{2}--(\emph{Pass})--\Cl{3}--(\emph{Pass})--\Cl{4}
                 would show diamond support and club shortness. \\
  \He{4},
  \Sp{4} & Single-suited hand with $7^+$-cards and no slam interest,
           sign-off. \\
  \hline
\end{longtable}

\subsubsection{Opener's rebids after a \He{1} response}

After a \orf{\He{1}} response, opener's rebids are:
\begin{longtable}{>{\raggedright}p{0.75cm}p{10.25cm}}
  \hline
  \orf{\Sp{1}} & 4-card \Sp{}, denies four cards in hearts. Responder
                 rebids: \\
               & \begin{tabular}{>{\raggedright}p{2cm}p{7.25cm}}
                   \Sp{2} & 8-9\hcp, $4^+$-card spade support. \\
                   \Sp{3} & 10-11\hcp, $4^+$-card spade support,
                            invitational. \\
                   \Sp{4} & Weak hand with long trumps or strong hand
                            with no interest in slam. \\
                   \Nt{1} & Minimum hand, sign-off. \\
                   \Nt{2} & 10-12\hcp, balanced hand, invitational. \\
                   \Cl{2},
                   \Di{2},
                   \He{2} & Minimum hand, $5^+$-cards, attempt to find
                            a better part-score,
                            \excp{non-forcing}. \\
                   \gf{\Cl{3}} & Could be a 3-card suit. Opener should
                                 show 3-card heart support if he has
                                 it with \He{3}. \\
                   \orf{\Di{3}} & $5^+$-card suit, strong hand,
                                  game-going. \\
                   \He{3} & Jump rebid indicates a $6^+$-card suit
                            with 10-12\hcp. Invitational. \\
                   \gf{\emph{Double jump
                   shift}} & \emph{Splinter} bid with singleton or
                             void and $4^+$-card spade support. \\
                 \end{tabular} \\
  \He{2} & Single raise---4-card heart suit, 11-13\hcp. Could be a
           3-card suit with a singleton elsewhere when no other bid is
           available. E.g., \hhand{2,AK9,J9832,AT92}. \\
  \He{3} & Jump raise---4-card heart suit, 14-15\hcp. \\
  \Nt{1} & 11-13\hcp, balanced, denies 4-card major. Can be
           3-3-2-5 shape. Responder can rebid: \\
               & \begin{tabular}{>{\raggedright}p{2cm}p{7.25cm}}
                   \Nt{2} & 11-12\hcp, balanced hand, invitational. \\
                   \Nt{3} & $13^+$\hcp, balanced hand. \\
                   \Cl{2}, \Di{2}, \He{2} & Minimum with $5^+$-card
                                            suit, retreat from
                                            no-trumps. \\
                   \orf{\Sp{2}} & \emph{Reverse}, at least 4-4 in the
                                  major suits, strong hand. \\
                   \orf{\Di{3}} & $5^+$-card diamonds, strong hand. \\
                   \gf{\emph{Jump
                   shift}} & Could be a 3-card suit. Intermediate bid
                             to decide between \Nt{3} and \He{4} (if
                             opener shows delayed support with
                             \He{3}). \\
                 \end{tabular} \\
  \Cl{2} & Unbalanced, usually 5-4 in minors and no 4-card
           major. Responder's rebids: \\
               & \begin{tabular}{>{\raggedright}p{2cm}p{7.25cm}}
                   \Di{2} & Weak hand, to play, preference for
                            diamonds. \\
                   \He{2} & $6^+$-card suit, to play. \\
                   \orf{\Sp{2}} & \emph{Reverse}, 5-4 in the major
                                  suits, strong hand. \\
                   \Cl{3} & 10-12\hcp, at least 3-card support for
                            clubs. \\
                   \orf{\Di{3}} & $5^+$-card suit, strong hand,
                                  invitational. \\
                   \Nt{3} & To play. \\
                   \gf{\emph{Double jump
                   shift}} & \emph{Splinter}---singleton or void in
                             suit bid and club support. \\
                 \end{tabular} \\
  \Di{2} & $6^+$-card diamond suit, no 4-card major, non-forcing. \\
  \Sp{2} & \emph{Reverse}---14-15\hcp, $6^+$-cards in diamonds and
           $4^+$-cards in spades. \\
  \Nt{2} & 14-15\hcp, good diamonds and stoppers in spades
           and clubs. \\
  \Cl{3} & 14-15\hcp, at least 5-5 in minors with values concentrated
           in the two suits. \\
  \Di{3} & 14-15\hcp, $6^+$-card diamond suit, no 4-card major. \\
  \hline
\end{longtable}

\subsubsection{Opener's rebids after a \Sp{1} response}

Opener's rebids after \orf{\Sp{1}} are similar to those after
\orf{\He{1}} but are repeated here for convenience.
\begin{longtable}{>{\raggedright}p{0.75cm}p{10.25cm}}
  \hline
  \Sp{2} & Single raise---4-card spade suit, 11-13\hcp. Could be a
           3-card suit with a singleton elsewhere (see example under
           \He{1}). \\
  \Sp{3} & Jump raise---4-card spade suit, 14-15\hcp. \\
  \Nt{1} & 11-13\hcp, balanced, may have 4-card heart. Responder can
           rebid: \\
         & \begin{tabular}{>{\raggedright}p{2cm}p{7.25cm}}
             \Nt{2} & 11-12\hcp, balanced hand, invitational. \\
             \Nt{3} & $13^+$\hcp, balanced hand. \\
             \Cl{2},
             \Di{2},
             \Sp{2} & $5^+$-card suit, minimum, retreat from
                      no-trumps. \\
             \gf{\emph{Jump
             shift}} & Could be a 3-card suit. Intermediate bid to
                       decide between \Nt{3} and \Sp{4} (if opener
                       shows delayed support with \Sp{3}). \\
           \end{tabular} \\
  \Cl{2} & Unbalanced, usually 5-4 in minors but could have four cards
           in hearts. Responder's rebids: \\
         & \begin{tabular}{>{\raggedright}p{2cm}p{7.25cm}}
             \Di{2} & Weak hand, to play. \\
             \Sp{2} & $6^+$-card suit, to play. \\
             \Cl{3} & 10-12\hcp, at least 3-card support for clubs. \\
             \orf{\Di{3}} & $5^+$-card suit, strong hand. \\
             \Nt{3} & To play. \\
             \gf{\emph{Double jump
             shift}} & \emph{Splinter}---singleton or void in suit bid
                       and club support. \\
           \end{tabular} \\
  \Di{2} & $6^+$-card diamond suit, may have four cards in hearts,
           non-forcing. \\
  \Nt{2} & 14-15\hcp, good diamonds and stoppers in hearts
           and clubs. \\
  \Cl{3} & 14-15\hcp, at least 5-5 in minors with values concentrated
           in the two suits. \\
  \Di{3} & 14-15\hcp, $6^+$-card diamond suit, may have four cards in
           hearts. \\
  \hline
\end{longtable}

\subsection{Responder does not have a major suit}

When responder does not have a major suit, he can bid:
\begin{longtable}{>{\raggedright}p{1.5cm}p{9.5cm}}
  \multicolumn{2}{l}{\emph{\underline{Balanced hands}}} \\
  \Nt{1} & 8-10\hcp, usually balanced hand. \\
  \Nt{2} & 11-12\hcp, balanced hand. Could be a 4-3-3-3 shape with a
           weak four card major and tenaces that would play better as
           declarer in \Nt{}. E.g., \hhand{AQT,T642,QT9,KT7} or
           \hhand{9843,KJT,AQ7,JT5} \\
  \Nt{3} & 13-15\hcp, balanced hand. \\
  \multicolumn{2}{l}{\emph{\underline{Support for diamonds (inverted raises)}}} \\
  \orf{\Di{2}} & $11^+$\hcp, $5^+$-card diamond suit,
                 \underline{forcing to \Nt{2} or \Di{3}}. \\
  \Di{3} & 0-10\hcp, $5^+$-cards in \Di{}, usually with a singleton or
           void. \\
  \Di{4} & Preemptive, with more shape and trumps than for \Di{3},
           i.e., 6 or 7-card diamond suit. \\
  \multicolumn{2}{l}{\emph{\underline{Unbalanced hands}}} \\
  \gf{\Cl{2}} & Usually, $12^+$\hcp\ and a $5^+$-card suit but could be
                four cards. Opener will show strength and stoppers
                targeting \Nt{3}. Bidding may stop in \Cl{3} or \Di{3}
                with missing stoppers. \\
         & \begin{tabular}{lp{7cm}}
             \He{2} & 11-$14^-$\hcp, heart stopper, no spade
                      stopper. \\
             \Sp{2} & 11-$14^-$\hcp, spade stopper, no heart
                      stopper. \\
             \Nt{2} & 11-$14^-$\hcp, stoppers in both majors. \\
             \Di{2},
             \Di{3} & No stopper in majors, genuine diamond suit. Jump
                      shows a maximum 15\hcp. \\
             \He{3} & $14^+$-15\hcp\ with a heart stopper and no
                      spade stopper. \\
             \Sp{3} & $14^+$-15\hcp\ with a spade stopper and no
                      heart stopper. \\
             \Nt{3} & $14^+$-15\hcp\ with stoppers in both majors. \\
             \Cl{3} & No stopper in majors, club support. \\
           \end{tabular} \\
  \Cl{3} & Invitational, with a long club suit. E.g.,
           \hhand{Q4,75,97,AQJ8654} \\
  \orf{\He{3}},
  \orf{\Sp{3}},
  \orf{\Cl{4}} & \emph{Splinter} bid with $5^+$-card diamond support
                 and a singleton or void in the bid suit. \\
  \hline
\end{longtable}

Examples of bidding after a \Di{1} opening are available
here. \hyp{ex1d}

\subsection{Intervention after a \Di{1} opening}

If an opponent doubles \Di{1}, most responses are the same except that
\excp{raises in diamonds are not inverted}. Since opener often uses
\Di{1} as a vehicle to discover a 4-4 major fit, responder, lacking
the strength to redouble should bid a four-card major if he has one.
\begin{longtable}{p{2.5cm}p{8.5cm}}
  \hline
  \emph{Pass} & 0-4\hcp\ or a ``\emph{trap pass}''. \\
  \emph{New suit} & 4-9\hcp, $4^+$-cards, \excp{no longer forcing}. \\
  \Nt{1} & 6-9\hcp, no 4-card major. \\
  \Di{2}, \Di{3} &  0-10\hcp\ with length in diamonds since opener may
                   be short. \\
  \orf{\Nt{2}} & $10^+$\hcp, $5^+$-card diamond suit, \excp{replaces
                 \Di{2} ``inverted raise''}. \\
  \emph{Redouble} & $10^+$\hcp. No specific distribution, may be for
                    penalties. \\
  \emph{Jump shift} & Weak, usually $7^+$-card suit without interest
                      in playing in another suit or \Nt{}. \\
  \hline
\end{longtable}

After an overcall by opponent up to the \Sp{2} level, the responses
are:
\begin{longtable}{p{2.5cm}p{8.5cm}}
  \hline
  \emph{Pass} & Poor hand with nothing to bid or a ``\emph{trap
                pass}''. \\
  \orf{\emph{Double}} & $8^+$\hcp, negative double.
                  \hyp{negative} \\
  \orf{\emph{New suit}} & $10^+$\hcp, natural---if at two-level,
                          values for two-over-one response. \\
  \emph{Raise} & 6-10\hcp, $5^+$-card diamonds (\excp{no longer
                 ``inverted''}). \\
  \orf{\emph{Jump raise}} & $11^+$\hcp, $5^+$-card diamonds (\excp{no
                            longer weak}). \\
  \Nt{1} & 8-10\hcp, stopper in opponent's suit, balanced. \\
  \Nt{2} & 11-12\hcp, stopper in opponent's suit, balanced. \\
  \Nt{3} & $12^+$-15\hcp, stopper in opponent's suit, to play. \\
  \gf{\emph{Cue-bid}} & $13^+$\hcp\ with slam interest. A passed hand
                        may cue-bid with $10^+$\hcp\ as a game try. \\
  \orf{\emph{Jump
  Cue-bid}} & After a one-level overcall, a \emph{jump} cue-bid at the
              three level asks opener to bid \Nt{3} with a stopper in
              opponent's suit. \\
  \hline
\end{longtable}

\pagebreak

\section{Responses to \He{1} or \Sp{1}}

\hypertarget{1major} Responses to a major opening include
\emph{Bergen} raises, \emph{splinter} bids, a forcing \Nt{1} and 2/1
game force.

\begin{longtable}{>{\raggedright}p{2cm}p{9.5cm}}
  \hline
  \emph{Pass} & 0-7\hcp\ and poor support. \\
  \multicolumn{2}{l}{\emph{\underline{Raises with support}}} \\
  \emph{Single raise} & 7-10\hcp\ with 3-card support,
                        constructive. \\
  \emph{Jump raise} & 0-6\hcp\ with 4-card support, preemptive
                      \emph{Bergen} raise.
                      \hyp{bergen} \\
  \Cl{3} & 7-10\hcp\ with 4-card support, constructive \emph{Bergen}
           raise. \hyp{bergen} \\
  \Di{3} & 10-12\hcp\ with 4-card support, limited \emph{Bergen}
           raise.
           \hyp{bergen} \\
  \gf{\Nt{2}} & $12^+$\hcp, 4-card support, slam
                interest. \emph{Jacoby \Nt{2}}.
                \hyp{jacoby2nt}. \\
  \emph{Game raise} & Wide variety of hands where responder is fairly
                      sure there is no play for slam. E.g., raise
                      \Sp{1} to \Sp{4} with \hhand{K65,AQ,K82,J9876}
                      or with \hhand{98732,A5,Q,T9743}. \\
  \multicolumn{2}{l}{\emph{\underline{One-over-one response}}} \\
  \orf{\Sp{1}} & $8^+$\hcp, $4^+$-card suit. See below for detailed
                 treatment. \\
  \orf{\Nt{1}} & 8-15\hcp, balanced hand with mild support for
                 opener's suit or unbalanced hand with insufficient
                 \hcp\ to justify a 2-over-1 response. Opener's rebids
                 are: \\
              & \begin{tabular}{>{\raggedright}p{2cm}p{6cm}}
                  \multicolumn{2}{l}{\emph{\underline{With a minimum 11-13\hcp}}} \\
                  \Cl{2},
                  \Di{2},
                  \He{2} & 11-13\hcp, 4-card suit (\excp{or better
                           3-card minor}). \\
                  \emph{Rebid suit} & 11-13\hcp, 6-card suit. \\
                \end{tabular} \\
              & \begin{tabular}{>{\raggedright}p{2cm}p{6.5cm}}
                  \multicolumn{2}{l}{\emph{\underline{With a maximum 14-15\hcp}}} \\
                  \emph{Jump rebid
                  suit} & 14-15\hcp, 6-card solid suit. The jump rebid
                          should be made on the basis of playing
                          strength rather than \hcp. \\
                  \Nt{2} & 5-3-3-2 distribution. \\
                  \emph{Jump in new suit} & 5-5 distribution. \\
                \end{tabular} \\
              & A \emph{reverse}, e.g., \He{1}--\Nt{1}--\Sp{2} would
                show shape rather than \hcp\ (typically 14-15) and
                indicate a 6-5 distribution (or 6-4 with a very strong
                spade holding such as \emph{AKQx}) \\
  \multicolumn{2}{l}{\emph{\underline{Two-over-one game force}}} \\
  \gf{\Cl{2}},
  \gf{\Di{2}},
  \gf{\He{2}} & $12^+$\hcp, $4^+$-card minor or $5^+$-card heart suit
                (over \Sp{1}). \excp{Unless the suit is rebid at the
                three level}, all 2-over-1 responses are forcing to
                game. \marginnote{A 2-over-1 response by a
                \emph{passed hand} indicates \excp{8-10\hcp\ with a
                good $5^+$-card suit and is non-forcing}.} Opener's
                rebids are: \\
              & \begin{tabular}{>{\raggedright}p{2.5cm}p{6cm}}
                  \He{2} (after \Sp{1}) & Natural, $4^+$-card suit. \\
                  \Sp{2} (after \He{1}) & 14-15\hcp, \emph{reverse}. \\
                  \emph{Rebid suit} & Minimum, \excp{not necessarily a
                                      6-carder}. \\
                  \emph{New Suit} & Natural, second suit. \\
                  \Nt{2} & 11-13\hcp, stoppers in unbid suits \\
                  \emph{Raise} & 11-13\hcp, good support,
                                 non-forcing. \\
                  \emph{Jump in
                  new suit} & 14-15\hcp, good support for responder's
                              suit, singleton or void in bid suit. \\
                  \emph{Jump
                  rebid suit} & 14-15\hcp, very good 6-card suit. The
                                jump rebid should be made on the basis
                                of playing strength rather than
                                \hcp. \\
                  \Nt{3} & 15\hcp\ with stoppers in the unbid
                           suits. \\
                \end{tabular} \\
  \multicolumn{2}{l}{\emph{\underline{Other bids at 3-level and above}}} \\
  \gf{\emph{Double
  jump shift}} & \emph{Splinter} bid, singleton or void in bid suit,
                 $4^+$-card support, slam interest. \\
  \orf{\Nt{3}} & 14-15\hcp, usually 4-card support for opener's major,
                 no void or singleton and a minimum of 4 controls. \\
  \hline
\end{longtable}

A response of \orf{\Sp{1}} over \He{1} shows $8^+$\hcp\ with a 4-card
suit and is forcing for one round. Opener's rebids are:

\begin{longtable}{p{1.5cm}p{9.5cm}}
  \hline
  \multicolumn{2}{l}{\emph{\underline{Balanced hands}}} \\
  \Nt{1} & 11-13\hcp, minimum. \\
  \Nt{2} & 14-15\hcp, balanced, stoppers in both minors. \\
  \Nt{3} & To play with running suit. \\
  \multicolumn{2}{l}{\emph{\underline{Unbalanced hands}}} \\
  \Cl{2},
  \Di{2} & 11-15\hcp, $4^+$-card suit, non-forcing. \\
  \He{2} & 11-13\hcp, $6^+$-cards in hearts, non-forcing. \\
  \Sp{2} & 11-13\hcp, 4-card spade support. \\
  \Cl{3},
  \Di{3} & 13-15\hcp, second 5-card suit. \\
  \He{3} & 13-15\hcp, $6^+$-card heart suit. \\
  \Sp{3} & 13-15\hcp\ with $4^+$-card spade support. \\
  \He{4} & To play---distributional hand. \\
  \Sp{4} & To play---maximum 13-15\hcp\ and distributional values. \\
  \multicolumn{2}{l}{\emph{\underline{Slam tries}}} \\
  \gf{\Cl{4}},
  \gf{\Di{4}} & \emph{Splinter} bids, good spade support (\emph{Qxx},
                \emph{xxxx} or better), slam interest. \\
  \Nt{4} & \emph{Roman key-card Blackwood} with spades as trumps.
           \hyp{blackwood} \\
  \hline
\end{longtable}

Examples of bidding after a major suit opening can be found
here. \hyp{ex1h}

\subsection{Intervention after a major suit opening}

After an opponent's double, responder can bid:
\begin{longtable}{>{\raggedright}p{2.5cm}p{8.5cm}}
  \hline
  \emph{Pass} & 0-4\hcp\ or a ``\emph{trap pass}''. \\
  \emph{Raise},
  \emph{Jump raise},
  \emph{Game raise} & 0-10\hcp, preemptive, $3^+$-card support
                      (\excp{$4^+$-cards for jumps}) following the
                      \emph{Law of Total Tricks}. \\
  \emph{New suit} & 4-9\hcp, $5^+$-card suit, non-forcing. \\
  \emph{Jump shift} & Weak, usually $7^+$-card suit without interest
                      in playing in another suit or \Nt{}. \\
  \Nt{1} & 6-11\hcp, balanced. \\
  \gf{\Nt{2}} & $12^+$\hcp, $4^+$-card support, \emph{Jacoby \Nt{2}}.
                \hyp{jacoby2nt}. \\
  \orf{\emph{Redouble}} & $12^+$\hcp, no specific distribution, may be
                          penalty oriented. \\
  \hline
\end{longtable}

After an opponent's overcall, responder's bids are:
\begin{longtable}{ p{2.5cm}p{8.5cm}}
  \hline
  \emph{Pass} & Poor hand of 0-7\hcp\ or ``\emph{trap pass}''. \\
  \orf{\emph{Double}} & $8^+$\hcp, negative double $4^+$ cards in
                        unbid major.
                        \hyp{negative} \\ 
  \emph{Raise} & Competitive, typically 6-10\hcp, $3^+$-card
                 support. \\
  \emph{Jump
  raise} & 11-12\hcp, limit raise (\excp{replaces \emph{Bergen} raise
           of \Di{3}}). \\
  \orf{\emph{New suit}} & $10^+$\hcp, $5^+$-card suit (values for a
                          2-over-1 response). \\
  \Nt{1} & 8-10\hcp\ with stopper in opponent's suit. \\
  \Nt{2} & 11-12\hcp\ with stopper in opponent's suit. \\
  \Nt{3} & $12^+$-15\hcp, stopper in opponent's suit, to play. \\
  \gf{\emph{Cue-bid}} & $13^+$\hcp, slam interest. A passed hand may
                        cue-bid with $10^+$\hcp\ as a game try.

                        After a \Sp{1} overcall over \He{1}, \excp{a
                        \emph{jump} cue-bid of \Sp{3} asks opener to
                        bid \Nt{3} with a stopper in spades}. \\
  \hline
\end{longtable}

\pagebreak

\section{Responses to 1NT}

\hypertarget{1nt} A \Nt{1} opening shows a 14-16\hcp\ balanced hand
(\excp{15-17\hcp\ in $3^{rd}$ or $4^{th}$ seat}) without a 5-card
major suit (a 5-card minor is possible). Simple arithmetic will almost
always tell responder how high to place the contract. E.g., 0-8\hcp:
1NT or 2 of suit, 9-10\hcp: game invite, 11-17: game force, 18+:
slam. The response structure below is applicable when \Nt{1} is opened
in $1^{st}$ or $2^{nd}$ seat since otherwise, responder is already
limited to less than $11$\hcp.

\begin{longtable}{ p{1.5cm}p{9.5cm}}
  \hline
  \multicolumn{2}{l}{\emph{\underline{Balanced hands---Quantitative raises}}} \\
  \emph{Pass} & 0-8\hcp, balanced hand (or 5-card minor). \\
  \Nt{2} & 9-10\hcp, balanced hand---invitation to \Nt{3}. \\
  \Nt{3} & 11-15\hcp, balanced hand, sign-off. \\
  \Nt{4} & 16-17\hcp, balanced hand, invites slam with a maximum. \\
  \orf{\Nt{5}} & 20-22\hcp, balanced---partner to choose between
                 \Nt{6} and \Nt{7}. \\
  \Nt{6} & 18-19\hcp, balanced, sign-off. \\
  \multicolumn{2}{l}{\emph{\underline{Unbalanced hands}}} \\  
  \orf{\Cl{2}} & \emph{Stayman}. \hyp{stayman}
                 This can be used on both weak and strong hands. \\
              & \begin{tabular}{>{\raggedright}p{2.5cm}p{6cm}}
                  \emph{Weak hand with
                  both majors} & When \He{2} or \Sp{2} is preferable
                                 to \Nt{1}, responder bids \Cl{2} and
                                 passes any major suit response. If
                                 opener denies majors with \Di{2},
                                 responder rebids \He{2} asking opener
                                 to pass or correct to \Sp{2}. \\
                \end{tabular} \\
              & \begin{tabular}{>{\raggedright}p{2.5cm}p{6cm}}
                  \emph{Invitational or game-forcing
                  hands} & To find a 4-4 major suit fit (implies one
                           4-card major). With 5-4 in the majors and a
                           game-going hand, if opener denies majors
                           with \Di{2}, responder jumps in the 4-card
                           suit to \gf{\He{3}}/\gf{\Sp{3}} which shows
                           5-cards in the \emph{other major}
                           (\emph{Smolen} convention). Opener can then
                           bid \Nt{3} with a doubleton or
                           \He{4}/\Sp{4} with 3-card support. \\
                \end{tabular} \\
  \orf{\Di{2}},
  \orf{\He{2}} & \emph{Jacoby} transfers to \He{2} and \Sp{2}
                 respectively. May be made with both strong and weak
                 hands. \hyp{jacoby} \\
  \orf{\Sp{2}} & \emph{Minor suit Stayman}---shows a minor 2-suiter
                 (5-4 or better) and asks opener to bid his 4-card
                 minor. This is used with either very weak hands or
                 strong slam-invitational hands. With intermediate
                 hands, you will want to play \Nt{3} so asking for
                 minors is not helpful. Opener responds: \\
              & \begin{tabular}{lp{7.5cm}}
                  \Cl{3},
                  \Di{3} & 4-card suit. Responder will pass with a
                           weak hand or bid on with slam interest. \\
                  \orf{\Nt{2}} & No four card minor. Responder rebids
                                 \Cl{3} with a weak hand asking opener
                                 to pass or correct to \Di{3}. With a
                                 strong hand, responder bids
                                 \gf{\He{3}} or \gf{\Sp{3}} to show a
                                 singleton. \\
                \end{tabular} \\
  \gf{\emph{3 of suit}} & $10^+$\hcp, good $5^+$-card suit, forcing to
                          game. \\
  \gf{\Di{4}},
  \gf{\He{4}} & \emph{Texas} transfers to \He{4} and \Sp{4}
                respectively. Denies slam values. \\
  \multicolumn{2}{l}{\emph{\underline{Others}}} \\
  \gf{\Cl{4}} & \emph{Gerber} ace-asking.
                \hyp{gerber} \\
  \hline
\end{longtable}

\subsection{Intervention after opening 1NT}

If an opponent doubles \Nt{1} for penalties, we use \emph{Meckwell
  Escapes} to run out to a safer contract but this method does not
work very well if responder has a 4-3-3-3 distribution. \excp{In this
  case, responder can either treat his four-card suit as a five-card
  suit, or treat his four-card suit and his best 3-card suit as a
  two-suiter} and pray that the opponents elect to introduce a suit of
their own.
\begin{longtable}{ p{1.5cm}p{9.5cm}}
  \hline
  \orf{\emph{Pass}} & Forcing, shows either a single-suiter in clubs
                      or diamonds or a two-suiter with both
                      majors. Opener must rebid \Cl{2} after which
                      responder rebids: \\
                    & \begin{tabular}{lp{7.5cm}}
                        \emph{Pass} & 5-card club suit, escape
                                      completed. \\
                        \Di{2} & 5-card diamond suit. \\
                        \orf{\He{2}} & 4-4 in the majors, opener
                                       passes or corrects to
                                       \Sp{2}. \\
              \end{tabular} \\
  \emph{Redouble} & Natural, non-forcing. \\
  \Cl{2} & Clubs and another suit, 4-4 or better. Opener passes or
           corrects to \Di{2} asking for the higher suit. \\
  \Di{2} & Diamonds and a higher ranking suit, 4-4 or better. Opener
           passes or corrects to \He{2} asking for the higher suit. \\
  \He{2},
  \Sp{2} & Natural, 5-card suit. \\
  \emph{3 of suit} & $6^+$-card suit, invitational. \\
  \hline
\end{longtable}


After an opponent's overcall, we use the \emph{lebensohl} convention
and responder has the following choices:
\begin{longtable}{p{3cm}p{8cm}}
  \hline
  \emph{Double} & Penalty double. \\
  \emph{Suit at 2-level} & 0-6\hcp, natural and non-forcing. \\
  \orf{\Nt{2}} & \emph{lebensohl}---partner must bid \Cl{3}.
                 \hyp{lebensohl:nt} \\
  \gf{\emph{Suit at 3-level}} & Natural, forcing to game. \\
  \orf{\emph{Cue-bid}} & Asks opener to bid a 4-card major if he has
                         one, denies a stopper in opponent's suit. \\
  \Nt{3} & \emph{lebensohl}---denies stopper in opponent's suit.
           \hyp{lebensohl:nt} \\
  \hline
\end{longtable}

\pagebreak

\section{Responses to \Cl{2}}

\hypertarget{2c} Bidding after a \Cl{2} opening (11-15\hcp, $6^+$-card
club suit) is largely natural except for the conventional \orf{\Di{2}}
response that asks opener to further describe his hand. Note that a
\excp{jump to \gf{\Di{4}} (not \Nt{4}) is used to trigger key card asking}.

\begin{longtable}{ p{1.5cm}p{9.5cm}}
  \hline
  \emph{Pass} & 0-7\hcp, poor hand. \\
  \multicolumn{2}{l}{\emph{\underline{Weak responses}}} \\
  \Cl{3} & Preemptive raise based on a club fit. Not
           forward-going. \\
  \Cl{4} & Preemptive raise with extra club support or distributional
           values as compared to a raise to \Cl{3}. \\
  \He{2},
  \Sp{2} & 8-10\hcp, non-forcing, invitational. Opener may pass with a
           minimum and mild support. \\
  \Nt{2} & 10-11\hcp, invitation to \Nt{3}. Any rebid other than
           \Cl{3} (sign-off) by opener commits to game.\\
  \He{4},
  \Sp{4} & Very long suit with no interest in slam, to play. \\
  \multicolumn{2}{l}{\emph{\underline{Strong responses}}} \\
  \orf{\Di{2}} & 11+\hcp, conventional (with club fit, may be made
                 with only 8\hcp). Opener's rebids are: \\
              & \begin{tabular}{lp{7cm}}
                  \He{2},
                  \Sp{2} & 11-13\hcp, 4-card suit. \\
                  \Nt{2} & 11-13\hcp, 6-3-2-2 balanced hand with
                           stoppers in two suits. Responder then bids
                           \Di{3} to enquire about stoppers and
                           opener's rebids are: \\
                         & \begin{tabular}{ll}
                             \He{3} & \He{} and \Di{} stoppers. \\
                             \Sp{3} & \Sp{} and \Di{} stoppers. \\
                             \Nt{3} & \He{} and \Sp{} stoppers. \\
                           \end{tabular} \\
                  \Cl{3} & 11-13\hcp, 6 clubs with 1 outside
                           stopper. Responder bids \Di{3} to enquire
                           about the stopper and opener's rebids
                           are: \\ 
                         & \begin{tabular}{lp{4.5cm}}
                             \He{3} & \He{} stopper. \\
                             \Sp{3} & \Sp{} stopper. \\
                             \Nt{3} & \Di{} stopper.  \\
                             \Di{4} & 5-card suit (6-5 in clubs and
                                      diamonds) \\
                           \end{tabular} \\
                  \Nt{3} & 14-15\hcp, 6-3-3-2 balanced hand, good club
                           suit. \\
                  \gf{\He{3}},
                  \gf{\Sp{3}} & 14-15\hcp, 4-card suit. \\
                \end{tabular} \\
  \gf{\Di{3}},
  \gf{\He{3}},
  \gf{\Sp{3}} & $6^+$-card suit with game-forcing values
                ($12^+$\hcp). Opener's rebids are: \\
              & \begin{tabular}{lp{7cm}}
                  \Nt{3} & Less than 2-card support. \\
                  \emph{Raise} & Minimum, 3-card support. \\
                  \gf{\emph{New
                  suit}} & Maximum, 3+-card club support, ace or void
                           in suit bid. \\
                \end{tabular} \\
  \Nt{3} & 13-15\hcp, balanced hand with stoppers in the unbid suits
           and no interest in the majors. \\
  \gf{\Di{4}} & Ace-asking, \emph{Roman key-card Blackwood}
                 \hyp{blackwood} \\
  \hline
\end{longtable}

Examples of bidding after a \Cl{2} opening are available
here. \hyp{ex2c}

\subsection{Intervention after a \Cl{2} opening}

After an opponent's \emph{double}, all bids carry their normal
meaning. The extra bid of \emph{Redouble} shows $10^+$\hcp\ and is
penalty oriented.

Over an opponent's overcall, raises are natural. Others:
\begin{longtable}{ p{2.5cm}p{8.5cm}}
  \hline
  \orf{\emph{Double}} & $8^+$\hcp, negative double.
                        \hyp{negative} \\
  \gf{\emph{Cue-bid}} & 12+\hcp, singleton or void in opponent's
                        suit. \\
  \orf{\emph{New suit}} & Any new suit bid is natural and forcing. \\
  \hline
\end{longtable}

\pagebreak

\section{Responses to \Di{2}}

\hypertarget{2d} A \orf{\Di{2}} opening declares a three suited hand with a
singleton or void in diamonds. The responder should be able to place
the final contract fairly easily in most cases. The only positive
responses are the artificial bids of \gf{\Nt{2}} or \gf{\Di{3}} which are both
forcing to game.

\begin{longtable}{ p{2cm}p{9cm}}
  \hline
  \emph{Pass} & 6+ diamonds, no interest in other suits or bidding
                higher. \\
  \He{2},
  \Sp{2},
  \Cl{3} & Natural, sign-off. \\
  \gf{\Nt{2}} & 11+\hcp, artificial, asks opener to further describe
                his hand. Opener's rebids are: \\
              & \begin{tabular}{lp{6.5cm}}
                  \Cl{3},
                  \Di{3}  & 3=4=1=5 or 4=3=1=5 shape respectively. \\
                  \He{3} & 11-13\hcp\ and 4=4=1=4 shape. \\
                  \Sp{3} & 14-15\hcp\ and 4=4=1=4 shape. \\
                  \Nt{3} & 14-15\hcp, 4=4=1=4 shape and \Di{}\emph{A}
                           or \Di{}\emph{K}. \\
                  \Cl{4} & 11-13\hcp, 4=4=0=5 shape. \\
                  \Di{4} & 14-15\hcp, 4=4=0=5 shape. \\
                \end{tabular} \\
              & \excp{A rebid in diamonds (the singleton suit) by
                responder would be $\beta$ asking for controls}.
                \hyp{beta} \\
  \gf{\Di{3}} & $\beta$ \emph{control-asking} bid. \\
  \He{3},
  \Sp{3} & 7-9\hcp, preemptive, $5^+$-card suit. \\
  \He{4},
  \Sp{4},
  \Cl{5} & Sign-off, to play. \\
  \hline
\end{longtable}

Here are some examples of bidding after a \Di{2} opening. \hyp{ex2d}

\subsection{Intervention after a \Di{2} opening}

When opponents double (possibly showing values in diamonds):
\begin{longtable}{p{2.5cm}p{8.5cm}}
  \hline
  \orf{\emph{Pass}} & Waiting action, forcing. Partner will
                      \emph{redouble} with a minimum hand or bid
                      \He{2} or \Sp{2} (four card suit) with a 3-4-1-5
                      shape. \\
  \emph{Redouble} & Desire to play in diamonds with a long
                    suit. Prepared to punish the opponents when they
                    run from the redouble. \\
  \emph{Others} & Same meaning as without the double. \\
  \hline
\end{longtable}

Responses after an opponent's overcall:
\begin{longtable}{p{2.5cm}p{8.5cm}}
  \hline
  \emph{Double} & For penalties. \\
  \gf{\Nt{2}} & Same as without overcall. \\
  \emph{Others} & Natural, non-forcing. \\
  \hline
\end{longtable}

\pagebreak

\section{Responses to \He{2} or \Sp{2}}

\hypertarget{2major} Opener shows exactly six cards in the suit bid
with two of the top three honours (or three of the top five) and
6-10\hcp. It denies four cards in the unbid major.

Without two-card support, it is generally correct to pass unless
holding $16^+$\hcp\ or a very good suit of one's own. The general
system of responses is called \emph{RONF} (Raise is the Only
Non-Forcing bid) and any new suit bid is forcing asking partner to
raise with 3-card support. A raise is a sign-off and opener should not
bid again.

A \orf{\Nt{2}} response asks opener to show a ``feature'' (an outside
stopper, i.e., an ace or king). If there is a suit fit, the feature
will help in deciding whether to bid game with an invitational
hand. Likewise, without a suit fit, it could help in deciding whether
to bid \Nt{3}.

Summary of responses:
\begin{longtable}{>{\raggedright}p{2cm}p{9cm}}
  \hline
  \emph{Pass} & No game, no fit. \\
  \emph{Raise} & 6-9\hcp, 3-card support, sign-off. \\
  \emph{Game raise} & Variety of hands, sign-off. E.g., raise to
                      \He{4} with either \hhand{AKJ3,Q5,4,A76432} or
                      \hhand{4,KJ43,KJ743,T87}. \\
  \orf{\emph{New
  suit}} & $13^+$\hcp, $5^+$-card suit, asking for 3-card
           support. E.g., with \hhand{AKJ943,2,AQT,T95}, bid \Sp{2}
           over \He{2}. Opener's rebids are: \\
              & \begin{tabular}{ll}
                  \emph{Raise} & $3^+$-card support. \\
                  \emph{Rebid suit} & No support. \\
                \end{tabular} \\
  \orf{\Nt{2}} & Either $16^+$\hcp\ or $13^+$\hcp\ with 3-card
                 support. Opener's rebids are: \\
              & \begin{tabular}{lp{6.5cm}}
                  \emph{Rebid suit} & No outside stoppers. \\
                  \emph{New suit} & Stopper in suit bid. \\
                  \Nt{3} & Very good suit headed by \emph{AQJ},
                           \emph{AKJ}, or exceptionally the
                           \emph{AKQ}. \\
                \end{tabular} \\
  \Nt{3} & $16^+$\hcp, all other suits stopped. Normally indicates a
           solid minor with a singleton or void in opener's
           suit. E.g., bid \Nt{3} with \hhand{K4,2,AKQJ876,K76} over
           an opening \He{2}. \\
  \gf{\Cl{4}} & \emph{Roman key-card} ask with modified responses
                since opener can have at most two key cards.
                \hyp{blackwoodmod} \\
  \hline
\end{longtable}

\pagebreak

\section{Responses to 2NT}

\hypertarget{2nt} Opener is showing a balanced hand with 22-23\hcp\
and no five-card suit. Note that the rebids after a \orf{\Cl{3}} response
differ from those after \Cl{1}--\Di{1}--\Nt{2}--\Cl{3} because here,
responder is not limited to 0-7\hcp\ and slam is a
possibility. Responses are:

\begin{longtable}{ p{1.5cm}p{9.5cm}}
  \hline
  \emph{Pass} & Very weak, no suit to escape to. \\
  \Nt{3} & To play. \\
  \orf{\Cl{3}} & \emph{Smolen Stayman}---asking for majors (implies at
                 least one four-card major). Game-going with slam
                 interest. Responses are: \\
              & \begin{tabular}{lp{7.5cm}}
                  \orf{\Di{3}} & No 4-card major. Responder's
                                 rebids are: \\ 
                               & \begin{tabular}{lp{6cm}}
                                   \He{3} & 4-card \He{} and 5-card
                                            \Sp{} (allows for 5-3
                                            fit). \\
                                   \Sp{3} & 4-card \Sp{} and 5-card
                                            \He{} (allows for 5-3
                                            fit). \\
                                   \Nt{3} & To play (no five-card
                                            major). \\
                                 \end{tabular} \\
                  \He{3},
                  \Sp{3} & 4-card major. \\
                  \Nt{3} & 4 cards in both majors. Responder bids: \\
                               & \begin{tabular}{lp{5.5cm}}
                                   \emph{Pass} & To play. \\
                                   \Di{4},
                                   \He{4} & Transfer to \He{4} or
                                            \Sp{4} respectively. \\
                                 \end{tabular} \\
                \end{tabular} \\
  \orf{\Di{3}},
  \orf{\He{3}} & Transfer to \He{3} or \Sp{3} respectively.

                 With 5-5 in the majors, responder bids \Sp{3} after a
                 transfer to hearts. With 5-4 (four hearts), he bids
                 \He{4} after a transfer to spades. Bidding \Nt{3}
                 asks opener to pass or bid four of the major holding
                 3-card support. \\
  \orf{\Sp{3}} & Minor suit slam try. Opener bids \Cl{4} (corrected to
                 \Di{4} if needed). \\
  \orf{\Cl{4}} & Minor two-suiter. \Di{4} sets diamonds, any other bid
                 agrees clubs. \\
  \orf{\Di{4}},
  \orf{\He{4}} & Transfer to \He{4} or \Sp{4} respectively. $6^+$-card
                 suit, to play. \\
  \hline
\end{longtable}

\pagebreak

\section{Responses to 3-Level Preempts}

\hypertarget{3preempt} A preempt at the 3-level is made with less than
10\hcp\ and at least a 7-card suit with a suit quality of 8 \excp{(or
  9 when vulnerable)} and obeying the rule of 2/3/4.  In general, it
is expected that opener will not have a four-card major and have two
or more good honours in his suit with not more than one useful honour
in a side suit. Unless partner makes a forcing response, the opener is
not expected to bid again.

Responses are based on playing strength (additional tricks that
responder can provide) rather than high card points:
\begin{longtable}{p{1.5cm}p{9.5cm}}
  \hline
  \emph{Pass} & No support, no suit of one's own, at best can provide
                2 or 3 tricks. \\
  \emph{Raise} & Could be preemptive with 3-card support or with good
                 cards in side suits that can provide four tricks or
                 more. \\
  \orf{\emph{New suit}} & $6^+$-card suit, good playing strength since
                          this forces opener to respond. Opener should
                          raise the suit with two or three trumps. \\
  \Nt{3} & Natural, good controls in side suits, good chance of taking
           6-7 tricks in opener's suit even with the known problems of
           entries in dummy. \\
  \hline
\end{longtable}

\section{Responses to 3NT}

\hypertarget{3nt} The ``\emph{gambling}'' \Nt{3} bid is made with a solid $7^+$-card
minor suit (\emph{AKQ} or better) and no outside ace or
king. Responses are:

\begin{longtable}{p{1.5cm}p{9.5cm}}
  \hline
  \emph{Pass} & To play, stoppers in side suits. \\
  \Cl{4} & Escape---asks opener to pass or bid \Di{4} if that is his
           suit. \\
  \orf{\Di{4}} & Asks opener to bid a singleton or void if he has
                 one. Responses: \\
              & \begin{tabular}{lp{6cm}}
                  \orf{\He{4}},
                  \orf{\Sp{4}} & Singleton or void in the bid suit. \\
                  \orf{\Nt{4}} & Singleton or void in the other
                                 minor. \\
                  \Cl{5},
                  \Di{5} & Shows that minor and denies a singleton or
                           void. \\
                \end{tabular} \\
  \He{4},
  \Sp{4} & Natural, to play. \\
  \Nt{4} & Quantitative, asking opener to bid \Nt{6} with extra length
           or an extra trick outside (such as \emph{Qxx}). \\
  \Cl{5} & Sign-off and weak. Opener should correct to \Di{5} if that
           is his suit. \\
  \Di{5} & Also a sign-off but responder indicates he knows opener's
           suit is diamonds and it would be advantageous to play from
           his side. \\
  \orf{\Nt{5}} & Grand Slam try showing no losers outside the trump
                 suit but indicating a void in opener's suit. If
                 opener is completely solid (e.g., \emph{AKQJ} to
                 seven card), he bids 7 otherwise he bids 6. \\
  \Cl{6} & Asks opener to pass or correct to \Di{6}. This could be a
           tactical bid. \\
  \hline
\end{longtable}

Here is an example of bidding after a \Nt{3} opening. \hyp{ex3nt}

\pagebreak

\section{Responses to Namyats}

\hypertarget{namyats} The \emph{Namyats} convention (\emph{Stayman}
spelt backwards) shows a long (normally eight cards) semi-solid (not
missing both ace and king) major suit with eight or more playing
tricks with playing strength concentrated in the trump suit. The bid
is meant to be constructive rather than preemptive and allows you to
distinguish hands that are close to game versus purely preemptive
openings.

\orf{\Cl{4}} shows hearts and \orf{\Di{4}} shows spades. With a weaker
hand, the preemptive bids of \He{4} or \Sp{4} would be used instead.

Responses are:
\begin{longtable}{p{1.5cm}p{9.5cm}}
  \hline
  \He{4},
  \Sp{4} & Bidding game in partner's suit is a sign-off with no slam
           interest (responder wants to be declarer). \\
  \orf{\Di{4}},
  \orf{\He{4}} & The next step above opener's bid is a relay asking
                 him to bid his suit (\He{} or \Sp{}). After opener
                 bids his suit, any new suit bid by responder is a
                 cue-bid. Responder passes if he does not have slam
                 interest but wants his hand to be dummy. \\
  \orf{\Nt{4}} & \emph{Roman key-card Blackwood}.
                 \hyp{blackwood} \\
  \orf{\emph{Suit}} & $\epsilon$ \emph{suit control-asking} bid. A
                      compressed scale of responses is used
                      (\excp{note that the steps skip over \Nt{}
                      because that has a special meaning}): \\
         & \begin{tabular}{lp{6.5cm}}
             \emph{Cheapest
             \Nt{}} & Guarded king in suit. When you have ample tricks
                      elsewhere, this allows you to declare an \Nt{}
                      contract from the correct side. \\
             \emph{1 step} & No first or second round control. \\
             \emph{2 steps} & Second round control---any singleton. \\
             \emph{3 steps} & First round control---void or ace. \\
             \multicolumn{2}{p{9cm}}{A repeat ask in the same suit
             asks for third round control, i.e., a doubleton or
             guarded queen. Responses are---$1^{st}$ step: no control,
             $2^{nd}$ step: doubleton and $3^{rd}$ step: guarded
             queen.} \\
           \end{tabular} \\
  \hline
\end{longtable}

An example of bidding after a \emph{Namyats} opening can be found
here. \hyp{namyats}

\pagebreak

\section{Competitive Bidding}

In general, simple non-jump overcalls are made with 8-15\hcp\ and a
good suit---the higher the level of the overcall, the stronger the
hand in terms of playing strength (better suit quality or length).
Jump overcalls are always weak or preemptive. An overcall in
no-trumps, however, shows the equivalent of an opening no-trump bid
\emph{with a stopper} in the opponent's suit.

\emph{Takeout} doubles are made with an opening hand or better
($12^+$\hcp) and must have tolerance for all the unbid suits. This
restriction can be relaxed holding a powerful hand of $16^+$\hcp\ when
you would double for takeout and bid your suit over partner's forced
response.

A double of \Nt{1} is \underline{not} a takeout double (you cannot
have tolerance for all four suits) and has special meanings described
below.

\subsection{Overcalls}

Overcalls over an opponent's opening of one of a suit carry the
following meanings:
\begin{longtable}{>{\raggedright}p{2.5cm}p{8.5cm}}
  \hline
  \emph{Non-jump
  overcall} & 8-15\hcp, good $5^+$-card suit (11-15\hcp, $6^+$-card
              suit at 2-level). Partner's responses are: \\
            & \begin{tabular}{p{2cm}p{5.5cm}}
                \multicolumn{2}{l}{\emph{\underline{With support for partner's suit}}} \\
                \emph{Single raise} & 8-9\hcp, $3^+$-card support. \\
                \emph{Jump raise} & Weak, 5-8\hcp, $4^+$-card support,
                                    preemptive. \\
                \emph{Game raise} & Hand good enough for game or weak
                                    and preemptive with 5-card
                                    support. \\
                \orf{\emph{Cue-bid}} & $9^+$\hcp, $4^+$-card support
                                       (or $10^+$\hcp\ with 3-card
                                       support).

                                       If the overcaller shows a
                                       minimum by rebidding his suit,
                                       responder can pass with
                                       9-11\hcp, raise with 12-14\hcp\
                                       and cue-bid again with
                                       $15^+$\hcp. \\
              \end{tabular} \\
            & \begin{tabular}{p{2cm}p{5.5cm}}
                \multicolumn{2}{l}{\emph{\underline{Without support for partner's suit}}} \\
                \emph{New suit} & $5^+$-card suit, non-forcing. \\
                \emph{Jump in suit} & 11-13\hcp, $6^+$-card suit,
                                      non-forcing. \\
                \Nt{1} & 8-11\hcp\ with stopper in opponent's suit. \\
                \Nt{2} & 12-15\hcp\ with stopper in opponent's
                         suit. \\
                \Nt{3} & $16^+$\hcp\ with stopper in opponent's
                         suit. \\
                \orf{\emph{Cue-bid}} & Strong hand with interest in
                                       game.

                                       To distinguish from the case
                                       with support for partner's
                                       suit, responder will rebid
                                       \Nt{} or jump in a new suit on
                                       his next bid. \\
              \end{tabular} \\
  \emph{Jump
  overcall} & 0-10\hcp, preemptive with $6^+$-card suit (7-card at
              3-level). \\
  \Nt{1} & 14-16\hcp, stopper in opponent's suit. Responses (including
           \emph{Stayman} and transfers) are the same as over a \Nt{1}
           opening. Stronger balanced hands should open with a takeout
           double followed by a bid in no-trumps. \\
  \orf{\emph{Cue-bid}} & \emph{Michael's cue-bid} showing a two-suiter
                         in the highest ranking unbid suit and
                         another.
                         \hyp{michaels} \\ 
  \orf{\emph{Jump to
  \Nt{2}}} & \emph{Unusual no-trump} showing a two-suiter in the two
             lowest ranking unbid suits.
            \hyp{unusualnt} \\
  \hline
\end{longtable}

\subsection{Takeout doubles}

A double of opponent's opening of one of a suit is either 12-15\hcp\
with tolerance for any unbid suit or a power double with $16^+$\hcp. A
double of a minor suit opening bid usually shows 4-4 in the majors and
a double of a major suit opening normally promises a 4-card suit in
the other major. \excp{A response must be made even with a blank hand unless
the other opponent bids}.

Responder's bids:
\begin{longtable}{>{\raggedright}p{2.5cm}p{8.5cm}}
  \hline
  \emph{Suit} & 0-8\hcp, may be only four cards (forced
                response). Doubler should pass with a minimum, raise
                and invite with support and 15-17\hcp\ or bid a new
                suit or no-trumps with $18^+$\hcp. \\
  \orf{\emph{Free bid
  or jump in suit}} & 9-11\hcp. Doubler should compete with 12-14\hcp,
                      force game with 15-17\hcp\ and bid a new suit or
                      no-trumps with $18^+$\hcp. \\
  \gf{\emph{Cue-bid}} & Strong hand, forcing to game. \\
  \Nt{1} & 6-9\hcp, stopper in opponent's suit, non-forcing. \\
  \Nt{2} & 10-11\hcp, stopper in opponent's suit, invitational. \\
  \Nt{3} & Values for game, no interest in slam. Doubler can check
           aces and try for slam with $18^+$\hcp. \\
  \hline
\end{longtable}

\subsection{Takeout doubles over a weak two opening}

The \emph{lebensohl} convention is used after a double of a weak two
opening (or when \emph{RHO} raises opener's suit to the two level in
an auction such as (\He{1})--\emph{Double}--(\He{2})--???).  Responses
to the double are:
\begin{longtable}{>{\raggedright}p{4cm}p{7cm}}
  \hline
  \emph{Pass} & (Only when \emph{RHO} has bid) Weak, 0-7\hcp. \\
  \emph{2 of suit} & Weak, 0-7\hcp. \\
  \orf{\Nt{2}} or
  \orf{\emph{3 of
  a suit}} & \emph{lebensohl} convention.
             \hyp{lebensohl:weak} \\
  \hline
\end{longtable}

\subsection{Negative doubles}

A \emph{negative} double after an opponent overcalls one of a major
implies possession of four cards in the other major and sufficient
values to justify a response. Responder is also expected to have mild
support either for the unbid minor or for partner's suit. At the two
or three-level, it implies that the hand is playable in either of the
unbid suits and possibly even partner's suit at the level he will be
forced to bid. Sometimes, responder may make a negative double with a
single-suited hand when he has insufficient high-card strength to make
a forcing bid in his suit.

Examples:
\begin{longtable}{l|p{5cm}>{\raggedright\arraybackslash}p{3cm}}
  \emph{Bidding} & \emph{Sample hand} & \\
  \hline
  \Di{1}--(\Cl{2})--\emph{Double}
                 & Reasonable major suit holdings and strength to play
                   at two-level.
                                      & \hhand{KJT5,QJT7,QT9,64}  \\
  \Sp{1}--(\He{2})--\emph{Double}
                 & Reasonable minor suit holdings and sufficient
                   values to play at 3-level.
                                      & \hhand{JT,85,QJT43,AJT9} \\
  \Di{1}--(\Sp{1})--\emph{Double}
                 & Four hearts with at least 8\hcp.
                                      & \hhand{87,KQT3,JT4,QT92} \\
  \Di{1}--(\Sp{1})--\emph{Double}
                 & \He{2} cannot be bid since it would imply values
                   for a two-over-one response. The solution is to use
                   a negative double and rebid hearts.
                                      & \hhand{87,AQJT95,T96,JT} \\
  \hline
\end{longtable}

Note that after the sequence, \Di{1}--(\He{1}), \excp{responder should
  bid \Sp{1} holding four spades} rather than make a negative double.

\hypertarget{negative} Opener's rebids after partner's negative double
are as follows:
\begin{longtable}{p{2.5cm}p{8.5cm}}
  \hline
  \emph{Pass} & For penalties (see below). \\
  \multicolumn{2}{l}{\emph{\underline{With a minimum hand (11-13\hcp)}}} \\
  \emph{New suit} & Shows fit in responder's implied suit (especially
                    if a major). \\
  \emph{Rebid own suit} & Shows $6^+$-card suit (or a strong 5-card
                          suit). \\
  \emph{Cheapest \Nt{}} & No suit worth bidding, stopper in opponent's
                          suit. \\
  \multicolumn{2}{l}{\emph{\underline{With a maximum hand (14-15\hcp)}}} \\
  \emph{Jump shift} & Invitational but non-forcing. \\
  \emph{Jump rebid} & Shows a good $6^+$-card suit, invitational. \\
  \emph{Jump in \Nt{}} & Stopper in opponent's suit, ready source of
                         tricks. \\
  \orf{\emph{Cue-bid}} & Shortness in opponent's suit and by
                         inference, support for doubler's suits. \\
  \hline
\end{longtable}

Opener can also \emph{Pass} for penalties after a negative double
although this is rare since it implies opponents have bid opener's
good suit. For example, after \Di{1}--(\Cl{2})--\emph{Double}, opener
could pass with \hhand{5,A76,AKJ3,QT942} or \hhand{983,Q,AK865,KQJ9}
(but not with \hhand{QT65,9,AKT,KT32} where the correct bid would be
to support one of partner's implied suits with \Sp{2}).

\subsection{Responsive doubles}

A \emph{responsive} double is used when the opponents have bid and
raised a suit and partner has either doubled or overcalled in
between. For example,

(\Di{1})--\emph{Double}--(\Di{2})--\emph{Double} or

(\Sp{1})--\He{2}--(\Sp{2})--\emph{Double}

The last double in each case is a \emph{responsive} double that shows
at least 4-4 in the major suits when opponents have bid a minor or 4-4
in the minor suits when opponents have bid a major (and partner has
doubled in between). When partner has overcalled, it indicates a 4-4
holding in the unbid suits with mild support (at least a doubleton)
for partner's suit.

Responsive doubles are ``on'' up to the level of \Sp{3} and \excp{can be
made over artificial raises} (such as \emph{Bergen} raises) or raises
of weak two bids.

It is not a responsive double (i.e., the double is for penalties) if
partner's overcall was not a natural non-jump suit bid, e.g., a jump
overcall, a \emph{Michael's} cue-bid or an overcall in no-trumps. It
is also not a responsive double if opponents have not bid \emph{and}
raised a suit.

Examples:
\begin{longtable}{p{3.5cm}|p{4cm}>{\raggedright\arraybackslash}p{3cm}}
  \emph{Bidding} & \emph{Meaning} & \emph{Sample hand}\\
  \hline
  (\Cl{1})--\Sp{1}--(\Cl{2})--\emph{Dbl}
                 & 4-4 in the unbid suits, at least doubleton in
                   partner's suit.
                                      & \hhand{92,K9762,AJT3,64}  \\
  (\Di{1})--\emph{Dbl}--(\Di{2})--\emph{Dbl}
                 & 4-4 in the major suits.
                                      & \hhand{AQJ2,QT63,843,65}  \\
  (\He{1})--\emph{Dbl}--(\He{2})--\emph{Dbl}
                 & 4-4 in the minor suits.
                                      & \hhand{Q63,5,KQ43,JT75}  \\
  (\Sp{1})--\emph{Dbl}--(\Cl{2})--\emph{Dbl}
                 & \multicolumn{2}{l}{4-4 in the minor suits since
                   \Cl{2} is a spade raise.} \\
  (\He{2})--\emph{Dbl}--(\He{3})--\emph{Dbl}
                 & \multicolumn{2}{l}{4-4 in the minor suits.} \\
  \hline
  (\He{1})--\Sp{2}--(\He{3})--\emph{Dbl}
                 & \multicolumn{2}{l}{Penalty double after partner's
                   jump overcall.} \\
  (\Di{1})--\Di{2}--(\Di{3})--\emph{Dbl}
                 & \multicolumn{2}{l}{Penalty double after
                   \emph{Michael's} cue-bid.} \\
  (\He{1})--\Nt{1}--(\He{2})--\emph{Dbl}
                 & \multicolumn{2}{l}{Penalty double after strong
                   no-trump overcall.} \\
  (\He{1})--\emph{Dbl}--(\Sp{1})--\emph{Dbl}
                 & \multicolumn{2}{p{6.5cm}}{Penalty double since
                   opponent has not raised but rather bid a suit
                   implied by partner's takeout double.} \\
  (\Di{1})--\emph{Pass}--(\Sp{1})--\emph{Dbl}
  
  (\Di{2})--\emph{Dbl}
                 & \multicolumn{2}{p{6.5cm}}{Penalty double since
                   opponent has not raised his partner's suit.} \\
  \hline
\end{longtable}

If partner has made a takeout double, a 4-4 distribution is enough
since he has support for both suits. However, when partner has made an
overcall, you should have at least a doubleton in partner's suit and
preferably a 5-4 holding in your own suits.

The higher the level of the auction, the greater the strength
required---if partner is forced to bid at the three level, you should
have at least 8\hcp\ (6-7\hcp\ if he can bid at the 2-level).  Playing
strength and suit quality are more important than high card points
when competing.

If you have a ``real'' penalty double of the opponent's contract, you
cannot make an immediate double. You must pass and hope that partner
is strong enough to reopen with a double which you can then pass.

\subsection{\emph{Lightner} doubles of slam contracts}

A double of a slam contract by a defender who is not on lead is a
\emph{Lightner} double requesting an unusual lead from partner. This
is most often bid when you have a void in a side suit, or sometimes
\emph{AQ} or \emph{KQ} in the suit bid by the dummy.

It asks partner not to lead trumps or a suit bid by the
defenders. Instead, he is requested to lead one of:
\begin{itemize}
\item Dummy's side suit if one has been bid.
\item Declarer's side suit if one has been bid.
\item Any other unusual lead---usually his longest suit to get a
  ruff.
\end{itemize}

The \emph{Lightner} double is \excp{not in effect if the opponents are
  sacrificing at the 6 or 7-level}.

There is also a negative inference that can be drawn when partner does
not double. For example, if opponents bid a side suit on the way to
slam and partner does not double, it is an indication that partner
does not want a lead of the side suit.

\subsection{Defence against a no-trump opening}

Doubles and overcalls over a \Nt{1} opening have different meanings
depending on whether the opponents are playing a strong no-trump or a
weak no-trump. For our purposes, a strong no-trump is one whose point
range includes 15\hcp.

\subsubsection{After a strong no-trump opening}

Over a strong no-trump opening, we use the \emph{Meckwell}
convention. This set of responses is \excp{also used when intervening
  over a weak no-trump after having passed once since a penalty double
  is no longer meaningful}.

\begin{longtable}{ p{1.5cm}p{9.5cm}}
  \hline
  \emph{Double} & Single-suited minor or major two-suiter (5-4 or
                  better but normally 5-5). Responder can then bid: \\
                & \begin{tabular}{lp{7.5cm}}
                    \orf{\Cl{2}} & Asks partner to clarify his hand
                                   who rebids \He{2} with a major
                                   two-suiter, or his suit with \Di{2}
                                   or \Cl{3}). \\
                    \emph{Pass} & Good hand, for penalties. \\
                    \emph{Suit} & Good suit, natural. \\
                  \end{tabular} \\
  \Cl{2},
  \Di{2} & Two-suiter---the suit bid and a major suit. Responder can
           bid: \\
                & \begin{tabular}{p{1.5cm}p{7cm}}
                    \emph{Pass} & To play. \\
                    \He{2} & Ask partner to pass or correct to
                             \Sp{2}. \\
                    \emph{New suit} & Natural. \\
                    \emph{Raise} & Preemptive, weak. \\
                    \orf{\Nt{2}} & Game interest, see below. \\
                  \end{tabular} \\
  \He{2},
  \Sp{2} & Natural, $6+$-card suit or an excellent 5-card suit. \\
  \Nt{2} & Minor two-suiter or a big hand. Partner should bid his
           better minor. With a big hand, bidding continues
           naturally. \\
  \hline
\end{longtable}

\Nt{2} after \Cl{2}, \Di{2} (or after
(\Nt{1})--\emph{Double}--(\emph{Pass})--\Cl{2}--(\emph{Pass})--\He{2})
shows game interest and partner rebids as follows:
\begin{longtable}{>{\raggedright}p{2cm}p{9cm}}
  \hline
  \multicolumn{2}{l}{\emph{\underline{After \Cl{2}}}} \\
  \Cl{3} & Minimum hand. \\
  \Di{3},
  \He{3},
  \Sp{3} & Maximum hand, second suit. \\
  \multicolumn{2}{l}{\emph{\underline{After \Di{2}}}} \\
  \Cl{3} & Minimum hand, diamonds and hearts. \\
  \Di{3} & Minimum hand, diamonds and spades. \\
  \He{3},
  \Sp{3} & Maximum hand, second suit. \\
  \multicolumn{2}{l}{\emph{\underline{After (\Nt{1})--Double--\Cl{2}--\He{2}--\Nt{2}}}} \\
  \Cl{3} & Minimum, hearts longer than spades. \\
  \Di{3} & Minimum, spades longer than hearts. \\
  \He{3} & Maximum, hearts longer than spades. \\
  \Sp{3} & Maximum, spades longer than hearts. \\
  \hline
\end{longtable}

\subsubsection{After a weak no-trump opening}

The \emph{Cappelletti} system is employed over a weak no-trump
(\excp{only if you have not passed previously---with a passed hand,
  use the \emph{Meckwell} convention described above}). As opposed to
the \emph{Meckwell} convention, the \emph{Cappelleti} system has the
advantage of retaining a double for penalties.

\begin{longtable}{ p{1.5cm}p{9.5cm}}
  \hline
  \emph{Double} & Strong hand (normally equal strength), for
                  penalties. \\
  \orf{\Cl{2}} & Any one suited hand ($6^+$-cards). Responder bids
                 \Di{2} to ask partner to name his suit. \\
  \orf{\Di{2}} & Two-suiter---both majors. \\
  \He{2},
  \Sp{2} & Two-suiter---hearts/spades and one minor. \Nt{2} asks
           for the minor suit. \\
  \orf{\Nt{2}} & Two-suiter---both minors. \\
  \emph{3 of suit} & Intermediate hands with 13-16\hcp\ and $6^+$-card
                     suit. \\
  \hline
\end{longtable}

After a \Cl{2} overcall, any bid other than \Di{2} is natural and
non-forcing. Responder may pass \Cl{2} with $6^+$ clubs and a void
elsewhere (likely to be partner's suit {\LARGE\SOH}). % Smiley in ascii font package

After (\Nt{1})--\Di{2}--(\emph{Pass})--???, responder can jump in a
major suit to invite game or pass with $6^+$ diamonds and a disaster
in the majors.

\pagebreak

\section{Gadgets and Conventions}

\subsection{$\beta$ control-asking bid}

\hypertarget{beta} A $\beta$ control-asking bid is used in various
situations to find out the number of aces and kings (controls) held by
partner. An ace counts as 2 controls and a king as 1 control so there
are a total of 12 controls.

The number of controls held are shown in steps:
\begin{longtable}{ p{1.5cm}p{9.5cm}}
  \hline
  \emph{1 step} & 0-2 controls. A relay bid by opener in the cheapest
                  suit over the 1-step response will then ask for
                  clarification with responses in steps: \\
                & \begin{tabular}{ll}
                    \emph{1 step} & No controls. \\
                    \emph{2 steps} & 1 control. \\
                    \emph{3 steps} & 2 controls. \\
                  \end{tabular} \\
  \emph{2 steps} & 3 controls. \\
  \emph{3 steps} & 4 controls. \\
  \emph{4 steps} & 5 controls. \\
  \hline
\end{longtable}

When responder has already shown controls as less than four or at
least four, e.g., after an unusual positive, \excp{a modified scale of
  responses is used}:
\begin{longtable}{ p{3cm}p{4cm}p{4cm}}
  \emph{Known to have}  & \emph{0-3 controls} & \emph{$4^+$ controls} \\
  \hline
  \emph{1 step}  & No controls. & 4 controls. \\
  \emph{2 steps} & 1 control.   & 5 controls. \\
  \emph{3 steps} & 2 controls.  & 6 controls. \\
  \emph{4 steps} & 3 controls.  & 7 controls. \\
  \emph{5 steps} &              & 8 controls. \\
  \hline
\end{longtable}

\subsection{$\gamma$ trump-asking bid}

\hypertarget{gamma} The $\gamma$ trump-asking bid is used to determine
the quality of responder's suit, i.e., whether the suit will provide
enough tricks in no-trumps or as a trump suit.

\begin{longtable}{p{1.5cm}p{9.5cm}}
  \multicolumn{2}{l}{\emph{Responses to a $\gamma$ trump-asking bid}} \\
  \hline
  \emph{1 step} & No top honour, $5^+$-card suit. \\
  \emph{2 steps} & 5-card suit, 1 honour. \\
  \emph{3 steps} & 5-card suit, 2 honours. \\
  \emph{4 steps} & 6-card suit, 1 honour. \\
  \emph{5 steps} & 6-card suit, 2 honours. \\
  \emph{6 steps} & \emph{AKQ} of suit, any length. \\
  \hline
\end{longtable}

The $\gamma$ bid may be repeated to get clarification on the first
response.

\begin{longtable}{p{4cm}p{7cm}}
  \emph{Honours already shown} & \emph{Responses to repeat $\gamma$-ask} \\
  \hline
  \emph{Zero} or
  \emph{AKQ} & \begin{tabular}{lp{5cm}}
                 1 step & $7^+$-card suit \\
                 2 steps & 6-card suit \\
                 3 steps & 5-card suit \\
               \end{tabular} \\
  \hline
  \emph{One} & \begin{tabular}{lp{5cm}}
                 1 step & \emph{Ace} \\
                 2 steps & \emph{King} \\
                 3 steps & \emph{Queen} \\
               \end{tabular} \\
  \hline
  \emph{Two} & \begin{tabular}{lp{5cm}}
                 1 step & \emph{AK} \\
                 2 steps & \emph{AQ} \\
                 3 steps & \emph{KQ} \\
               \end{tabular} \\
  \hline
\end{longtable}

\subsection{$\epsilon$ suit control-asking bid}

\hypertarget{epsilon} The $\epsilon$ asking bid is used to find out
what controls the responder holds in a specific suit. It follows a
$\gamma$ asking bid and terminates only when trumps or no-trumps is
bid, i.e., any other suit bid is an $\epsilon$-ask in that suit.
Responses are in steps:

\begin{longtable}{p{1.5cm}p{9.5cm}}
  \multicolumn{2}{l}{\emph{Responses to a $epsilon$ suit control-asking bid}} \\
  \hline
  \emph{1 step} & No control---\emph{Jxx} or worse. \\
  \emph{2 steps} & Third round control---\emph{Q} or doubleton. \\
  \emph{3 steps} & Second round control---\emph{K} or singleton. \\
  \emph{4 steps} & First round control---\emph{A} or void. \\
  \emph{5 steps} & \emph{AK} or \emph{AQ}. \\
  \hline
\end{longtable}

The $\epsilon$-ask can be repeated if it is important to know whether
the control is based on shortness or strength. The response is again
in steps---the first step showing that the previous response was based
on \emph{shortness} and the second step showing \emph{strength}.

If \excp{the first $\epsilon$ bid is at the level of \Cl{5} or higher,
  a compressed scale of responses is used}:
\begin{longtable}{p{1.5cm}p{9.5cm}}
  \multicolumn{2}{l}{\emph{Compressed responses to a $epsilon$ suit control-asking bid}} \\
  \hline
  \emph{1 step} & No control---\emph{Qx} or worse. \\
  \emph{2 steps} & Second round control---\emph{K} or singleton. \\
  \emph{3 steps} & First round control---\emph{A} or void. \\
  \hline
\end{longtable}

\subsection{\emph{Bergen} raises}

\hypertarget{bergen} After a \He{1} or \Sp{1} opening, responses of
\Cl{3}, \Di{3}, \He{3} and \Sp{3} show different types of 4-card
support. The mnemonic \emph{CLAP} (Constructive, Limited and
Preemptive) helps to remember the order of the bids.

\begin{longtable}{p{1.5cm}p{9.5cm}}
  \hline
  \emph{\He{1}--\Cl{3}} & Constructive, 7-10\hcp, 4-card \He{}. \\
  \emph{\He{1}--\Di{3}} & Limited, 10-12\hcp, 4-card \He{}. \\
  \emph{\He{1}--\He{3}} & Preemptive, 0-6\hcp, 4-card \He{}. \\
  \emph{\Sp{1}--\Cl{3}} & Constructive, 7-10\hcp, 4-card \Sp{}. \\
  \emph{\Sp{1}--\Di{3}} & Limited, 10-12\hcp, 4-card \Sp{}. \\
  \emph{\Sp{1}--\He{3}} & \excp{\emph{Spare bid}}---used to show a
                          strong $12^+$\hcp\ hand with 4-card support
                          and an undisclosed singleton/void. \\
  \emph{\Sp{1}--\Sp{3}} & Preemptive, 0-6\hcp, 4-card \Sp{}. \\
  \hline
\end{longtable}

\subsection{\emph{Gerber} ace-asking convention}

\hypertarget{gerber} An immediate response of \orf{\Cl{4}} to any no-trump
bid (or overcall) is the \emph{Gerber} ace-asking convention. A \emph{jump
rebid} of \Cl{4} in response to a natural no-trump bid is \emph{Gerber}
as also when a trump suit has not been identified and no-trumps has
been rebid. Gerber should not be used holding a void.

Opener shows the number of aces held in steps as follows:

\begin{longtable}{p{1.5cm}p{9.5cm}}
  \hline
  \Di{4} & Zero or four aces. \\
  \He{4} & One ace. \\
  \Sp{4} & Two aces. \\
  \Nt{4} & Three aces. \\
  \hline
\end{longtable}

\subsection{\emph{Jacoby} transfers}

\hypertarget{jacoby} After a \Nt{1} opening, responder bids \orf{\Di{2}}
with a 5-card or better heart suit and \orf{\He{2}} with spades. Opener will
bid \He{2} or \Sp{2} so that the strong hand becomes
declarer. Responder's rebids are:

\begin{longtable}{p{2.5cm}p{8.5cm}}
  \hline
  \emph{Pass} & A weak hand with $5^+$-card \He{} or \Sp{}. \\
  \Sp{2} & Invitational with 5-5 in the majors after
           \Nt{1}--\Di{2}--\He{2}. \\
  \Nt{2} & Balanced or semi-balanced hand with a 5-card
           major. Invitational---partner can pass or sign-off in 3 of
           a major or bid \Nt{3}. \\
  \gf{\Cl{3}},
  \gf{\Di{3}} & 4-card suit in addition to 5-card major. \\
  \emph{Raise} & 6-card suit, invitational. \\
  \gf{\He{3}} & (After \Sp{2}) 5-5 in the majors with slam
                interest. Stronger than an immediate jump to
                \He{4}. \\
  \gf{\Sp{3}} & (After \He{2}) Singleton or void with slam
                interest. \\
  \Nt{3} & Balanced or semi-balanced hand. Partner can pass or correct
           to 4 of major. \\
  \He{4} & (After \Sp{2}) 5-5 in majors with no slam interest. Partner
           can pass or correct to \Sp{4}. \\
  \emph{Double raise} & $6^+$-card major, sign-off. \\
  \Nt{4} & \excp{Quantitative}, inviting slam in major or
           no-trumps. \\
  \hline
\end{longtable}

\subsubsection{Super-acceptance of a transfer}

\hypertarget{superaccept} Opener could \emph{super-accept} the
transfer with a $4^+$-card holding in the transfer suit. In this case,
he can break the transfer and show any doubletons.  For example, after
\Nt{1}--\He{2} (transfer to \Sp{2}), opener with a 4-card spade suit
could bid:

\begin{longtable}{p{2.5cm}p{8.5cm}}
  \hline
  \orf{\Nt{2}} & 4=3=3=3 shape. \\
  \orf{\Cl{3}} & 4=x=x=2 (doubleton club). \\
  \orf{\Di{3}} & 4=x=2=x (doubleton diamond). \\
  \orf{\He{3}} & 4=2=x=x (doubleton heart). \\
  \Sp{3} & 4 spades, any other distribution. \\
  \hline
\end{longtable}

\subsection{\emph{Jacoby} 2NT}

\hypertarget{jacoby2nt} A \gf{\Nt{2}} response over an opening of
\He{1} or \Sp{1} is conventional and shows $12^+$\hcp\ with $4^+$-card
support of partner's suit. It is forcing to game and opener's rebids
are:

\begin{longtable}{>{\raggedright}p{2.5cm}p{8.5cm}}
  \hline
  \orf{\emph{New suit}} & Singleton or void in suit bid. \\
  \orf{\emph{Rebid of suit at 3-level}} & Maximum strength hand. \\
  \orf{\emph{Jump shift}} & Good $5^+$-card side suit. \\
  \emph{Game in original suit} & Minimum opening, sign-off. \\
  \orf{\Nt{3}} & 12-13\hcp, medium strength hand. Responder will pass
                 only if he thinks hand will play better in no-trumps
                 in spite of the nine card fit. \\
  \hline
\end{longtable}


\subsection{\emph{lebensohl} convention (over weak two)}

\hypertarget{lebensohl:weak} The \emph{lebensohl} convention is used
to respond to partner's takeout double of a weak two opening. This
system is geared to show weak, invitational and strong hands without
consuming too much bidding room.

\begin{longtable}{>{\raggedright}p{2cm}p{9cm}}
  \hline
  \orf{\Nt{2}} & Artificial, partner must bid \Cl{3}. Over partner's
                 forced response, responder bids: \\
         & \begin{tabular}{lp{6.3cm}}
             \emph{Pass} & 0-7\hcp, sign-off in clubs. \\
             \Di{3},
             \He{3},
             \Sp{3} & 0-7\hcp, correction to longest suit when not
                      clubs, sign-off. \\
             \gf{\emph{Cue-bid}} & $4^+$-cards in unbid major (or one
                                   of the majors if double was over
                                   two of a minor), confirms stopper
                                   in opponent's suit. \\
             \gf{\Nt{3}} & Denies $4^+$-cards in majors, confirms
                           stopper in opponent's suit. \\
           \end{tabular} \\
  \Cl{3},
  \Di{3},
  \He{3},
  \Sp{3} & 8-11\hcp, $4^+$-card suit, invitational. \\
  \gf{\emph{Cue-bid}} & $4^+$-cards in unbid major (or one major if
                        double was over 2 of a minor), no stopper in
                        opponent's suit. \\
  \gf{\Nt{3}} & Denies $4^+$-cards in majors, no stopper in opponent's
                suit. \\
  \hline
\end{longtable}

\emph{lebensohl} applies even when \emph{RHO} raises opener to the
2-level e.g., over the sequence
(\Sp{1})--\emph{Double}--(\Sp{2})--???.  In this case, since responder
is not forced to bid, he will \emph{Pass} with the weak 0-7\hcp\
hands.  \excp{The suit bids after the \Nt{2}--\Cl{3} sequence now
  become invitational and the direct raises to three of a suit become
  game forcing bids}.

\subsection{\emph{lebensohl} convention (after 1NT)}

\hypertarget{lebensohl:nt} A different \emph{lebensohl} convention is
used by responder after an opponent overcalls partner's \Nt{1}
opening in order to compete further without necessarily committing to
game. It is initiated after the right-hand opponent makes a suit
overcall at the two-level.

\begin{longtable}{>{\raggedright}p{2.5cm}p{8.5cm}}
  \hline
  \emph{New suit
  at 2-level} & Natural and non-forcing. \\
  \orf{\Nt{2}} & A puppet bid requiring opener to bid \Cl{3}. After
                 opener's forced \Cl{3} bid, \\
              & \begin{tabular}{p{8cm}}
                  --- 3 of a lower ranking suit than overcaller's is
                  natural, to play. \\
                  --- 3 of a higher ranking suit than overcaller's is
                  natural and invitational. \\
                  --- 3 of the opponent's suit is artificial asking
                  opener to bid a 4-card major and showing a stopper
                  in opponent's suit. \\
                  --- 3NT is natural, to play, and shows a stopper in
                  the opponent's suit. \\
                \end{tabular} \\
  \gf{\emph{New suit at
  3-level}} & Natural and forcing to game. \\
  \orf{\emph{Cue-bid}} & Artificial---asks opener to bid a 4-card
                         major and denies a stopper in opponent's
                         suit. \\
  \Nt{3} & Natural, to play but \excp{denies a stopper in opponent's
           suit}. \\
  \hline
\end{longtable}

\subsection{\emph{Michael's} cue-bid}

\hypertarget{michaels} A cue-bid in the opponent's suit after they
have opened the bidding is conventional and shows a two-suited hand
(5-5 or better). Over a minor-suit opening (i.e., a bid of
\orf{\Cl{2}} or \orf{\Di{2}}) it shows both majors and over a major-suit (i.e., a bid
of \orf{\He{2}} or \orf{\Sp{2}}), it shows the unbid major and an unspecified minor.

\begin{longtable}{p{2.5cm}p{8.5cm}}
  \hline
  \multicolumn{2}{l}{\emph{\underline{After a cue-bid of \Cl{2}/\Di{2}}}} \\
  \Di{2} & Natural, showing a very good $6^+$-card suit. Tends to deny
           3 cards in either major. Non-forcing. \\
  \He{2},
  \Sp{2} & Non-forcing sign-off. With support for both majors,
                 bid \He{2}. \\
  \Nt{2} & Natural, invitational. \\
  \orf{\Cl{3}},
  \orf{\Di{3}} & When not a cue-bid, shows a very strong 6-card
                 suit. \\
  \orf{\emph{Cue-bid}} & Artificial, shows game or slam interest. \\
  \He{3},
  \Sp{3} & Preemptive, usually with 4-card suit in accordance with
           the \emph{Law of Total Tricks}. Non-forcing. \\
  \Nt{3} & Natural, sign-off. Shows a big, balanced hand with no
           interest in a major-suit game. Rare. \\
  \multicolumn{2}{l}{\emph{\underline{After a cue-bid of \He{2}/\Sp{2}}}} \\
  \Sp{2} & Over \He{2}, is a sign-off. \\
  \orf{\Nt{2}} & Asks opener to bid his minor suit. Made on a variety
                 of hands but is usually to sign-off in \Cl{3} or
                 \Di{3} with support in both minors. \\
  \Cl{3},
  \Di{3} & Natural, $6^+$-card suit---opener probably has the other
           minor. Non-forcing. \\
  \He{3} & Over \Sp{2}, natural sign-off. \\
  \Sp{3} & Over \He{2}, preemptive with 4-card suit. \\
  \orf{\emph{Cue-bid}} & Artificial showing game or slam interest. \\
  \hline
\end{longtable}

\excp{If responder raises the cue-bid}, e.g.,
(\He{1})--\He{2}--\emph{Pass}--\He{3}, the Michael's cue-bidder is
expected to bid his cheapest suit (in this case, \Sp{3}) with a weak
hand of 0-10\hcp. All other bids show $10^+$\hcp\ and are game
forcing.

\subsection{Roman key-card \emph{Blackwood}}

\hypertarget{blackwood} A \emph{Roman key-card Blackwood} bid of
\orf{\Nt{4}} is used to enquire about the number of key cards (any ace or
the trump suit king) in partner's hand. It should not be used when you
have a void or two fast losers.

Responses are in steps and differ slightly (when holding all five key
cards) depending on whether opponents have doubled or overcalled over
\Nt{4}.

\begin{longtable}{>{\raggedright}p{3.2cm}|p{1cm}>{\raggedright}p{3cm}>{\raggedright\arraybackslash}p{3.2cm}}
  \emph{Holding} & \emph{Silent} & \emph{Double (ROPI)} & \emph{Overcall (DOPI)} \\
  \hline
  \emph{1/4 key cards} & \Cl{5} & \emph{Redouble} & \emph{Double} \\
  \emph{0/3 key cards} & \Di{5} & \emph{Pass} (0/3\excp{/5} key cards) & \emph{Pass} (0/3\excp{/5} key cards) \\
  \emph{2/5 key cards (no queen of trumps)} & \He{5} & \Cl{5} (\excp{two} key cards) & \emph{Cheapest suit} (\excp{two} key cards) \\
  \emph{2/5 key cards with queen of trumps} & \Sp{5} & \Di{5} (\excp{two} key cards) & \emph{Second-cheapest suit} (\excp{two} key cards) \\

  \hline
\end{longtable}

When holding a void, after a trump suit is agreed, jumping to another
suit at the 4 or 5 level in the void suit initiates a \emph{key card
  exclusion} asking bid. Partner shows his key cards \emph{excluding}
any in the void suit in steps.

\begin{longtable}{p{1.5cm}p{9.5cm}}
  \hline
  \emph{1 step} & 1 or 4 key cards. \\
  \emph{2 steps} & 0 or 3 key cards. \\
  \emph{3 steps} & 2 key cards without trump Q. \\
  \emph{4 steps} & 2 key cards with trump Q. \\
  \hline
\end{longtable}

\subsubsection{Asking for the Trump Queen}

After a \Cl{5} or \Di{5} response to the \emph{RKCB} ask,
bidding the cheapest non-trump suit asks about the trump
queen. Responses to the queen-ask are:

\begin{longtable}{p{3cm}p{8cm}}
  \hline
  \emph{Trump suit} & No trump queen. \\
  \emph{Non-trump suit} & Trump queen \emph{and} king in suit bid. \\
  \Nt{5} & Trump queen without any side-suit kings. \\
  \hline
\end{longtable}

\subsubsection{Asking for Kings}

A \Nt{5} rebid following \emph{RKCB} asks for any kings held. It is a
grand slam try, and should only be bid if the partnership holds all 5
key cards and the queen of trumps.

Partner will respond by bidding his cheapest king.

\subsubsection{Modified RKCB (\Cl{4} over \He{2}/\Sp{2})}

\hypertarget{blackwoodmod} A bid of \Cl{4} is used to ask for key
cards after either an opening weak two bid or a weak two response to a
\Cl{1} opening. The response structure is based on regular \emph{RKCB}
taking into account the fact that the responder cannot have more than
two key cards.  The responses are slightly different depending on
whether the suit is hearts or spades.

When spades are trumps, the $1^{st}$ step (\Di{4}) shows 0 or 2 key
cards, the $2^{nd}$ step (\He{4}) shows one key card without the
\Sp{}\emph{Q} and the $3^{rd}$ step (\Sp{4}) shows one key card with
the \Sp{}\emph{Q}.  If partner signs off in \Sp{4} after a 0/2
response, opener bids \Nt{4} holidng the \Sp{}\emph{Q} allowing
responder to push on to slam if possible.

When hearts is the agreed suit, the $1^{st}$ step (\Di{4}) shows 0 or
2 key cards and the $2^{nd}$ step (\He{4}) shows 1 key card. If
partner wants to ask about the \He{}\emph{Q}, he bids \Sp{4}. Without
the queen, opener signs off in \He{5} otherwise he bids another suit
showing a side suit king or \Nt{5} with no side suit king.

\subsection{\emph{Stayman} convention}

\hypertarget{stayman} The \emph{Stayman} convention is used to find a
4-4 major suit fit after a \Nt{1} opening by bidding \Cl{2}. Opener
responds with one of:

\begin{longtable}{p{1.5cm}p{9.5cm}}
  \hline
  \Di{2} & No four card major. \\
  \He{2} & 4-card heart suit, may have 4-card spades. \\
  \Sp{2} & 4-card spade suit, no 4-card heart suit. \\
  \hline
\end{longtable}

\subsubsection{Puppet \emph{Stayman}}

\hypertarget{puppetstayman} Used over a \Nt{2} bid that may be made
with a 5-card major, \Cl{3} is a conventional bid that endeavours to
find a major suit fit. Responses are:

\begin{longtable}{p{1.5cm}p{9.5cm}}
  \hline
  \Di{3} & No five card major but at least one four card
           major. Responder with one four card major should bid the
           major that he \emph{does not have} and if there is a 4-4
           fit, opener will bid it. \\
  \He{3} & 5-card heart suit. \\
  \Sp{3} & 5-card spade suit. \\
  \Nt{3} & No 4 or 5-card major. Responder can bid \Cl{4} or \Di{4} to
           transfer to \He{4} or \Sp{4} respectively when he has a
           six-card major. \\
  \hline
\end{longtable}

\subsection{Unusual 2NT}

\hypertarget{unusualnt} A jump to \Nt{2} over opponent's opening of
one of a suit is conventional and shows a two-suited hand in the two
lowest ranking unbid suits. E.g., a \Nt{2} overcall of a \Cl{1} opener
shows hearts and diamonds. The \emph{unusual \Nt{2}} is possible even
when both opponents bid.

For example, (\He{1})--\emph{Pass}--(\Sp{1})--\Nt{2} shows $5^+$-clubs
and $5^+$-diamonds. There is an overlap with the Michael's cue-bid in
this case since a cue-bid of \He{2} would also show the same shape.

As another example, (\Cl{1})--\emph{Pass}--(\Nt{1})--\Nt{2} would show
$5^+$-hearts and $5^+$-diamonds.

Responder's bids:
\begin{longtable}{p{1.5cm}p{9.5cm}}
  \hline
  \emph{Better suit} & Longer of overcaller's two suits. With equal
                       length in both suits (especially with two
                       doubletons), bid the cheapest suit. \\
  \emph{Jump} & 4-card support for suit, preemptive, non-forcing. \\
  \orf{\emph{Cue-bid}} & Support for at least one of partner's suits
                         and interest in game or slam. Asks \Nt{2}
                         bidder to bid the cheapest suit with
                         0-10\hcp\ or to make any other bid with
                         $10^+$\hcp. \\
  \emph{$4^{th}$ suit} & Bidding the suit not shown by partner or
                         opponent shows a $6^+$-card suit, a decent
                         hand and no support for either of partner's
                         suits. \\
  \hline
\end{longtable}

All raises by the \Nt{2} bidder are natural and non-forcing. All other
bids show a very good hand in terms of points and/or shape but with
the exception of a cue-bid, are non-forcing.

\pagebreak

\section{Miscellaneous}

\emph{High-card Points} (\hcp) are assigned as follows---Ace=4,
King=3, Queen=2 and Jack=1.  Once a trump-fit has been found,
distribution points can be assigned---Void=3, Singleton=2,
Doubleton=1.

Singleton honours should be counted only once (either \hcp\ or
shortness).

\gap

\emph{Suit Quality} (\sq) is calculated as suit length plus number of
honours in the suit. The Jack or Ten should be counted only if a
higher ranking honour is held. For example, a holding of K-J-9-5-4
would have 7\sq\ but J-10-9-5-4 would have 5\sq.

For an overcall, the \sq\ should equal or exceed the number of tricks
bid (e.g., \sq\ of 8 for a 2-level overcall).

When preempting, the \sq\ should equal the level of preempt when
vulnerable and can be one less when non-vulnerable.

\gap

\emph{Rule of 2/3/4} When preempting, the expected number of
undertricks should be no more than two with unfavourable
vulnerability, three with equal vulnerability and four with favourable
vulnerability.

For example, a \Di{3} preemptive bid is made when expecting to take at
least seven tricks in the case that the opening bidder is vulnerable
facing non-vulnerable opponents.

\gap

The \emph{Losing Trick Count} (\ltc) is used only once a trump suit
has been established. Count losers only in the top three cards of the
suit holding---there are never more than 3 losers in a suit. With
three or more cards, A/K/Q are not losers but any lower card is a
loser. With two cards, only A or K are not losers.

Add your and partner's loser count and subtract from 24 to estimate
the number of tricks that can be won.  You can estimate your partner's
\ltc\ as follows:

\begin{tabular}{rp{3cm}}
  \emph{\hcp{}} & \emph{Expected Losers} \\
  \hline
  \emph{7-9} & 8-9 losers (9) \\
  \emph{10-12} & 7-8 losers (8) \\
  \emph{13-15} & 6-7 losers (7) \\
  \emph{16-18} & 5-6 losers (6) \\
  \emph{19-21} & 4-5 losers (5) \\
  \emph{$22^+$} & 4 losers or less \\
  \hline
\end{tabular}

\pagebreak

\section{Bidding Examples}
\setboolean{betweencards}{true} % spaces between cards in hand diagrams
\setboolean{leadingspace}{true}

\subsection{Negative response to \Cl{1}}

\hypertarget{ex1c1d} After a \Di{1} response, there is no temptation
to get too high on misfitting hands. For example,

\vhand[West]{4,AK954,AJ4,KQT9}\vhand[East]{KJT753,62,753,54}
\ewauction{1c,1d,1h,1s(1),2c(2),2s(3)}\\ (1) 4-7\hcp, 4+-card suit. \\
(2) Shows minimum with second 4-card suit (implies 5 hearts). \\ (3)
6-card suit, no fit.

Opener shows discipline and passes recognising misfit and no chance
for game.

\gap

A \Di{1} response does not rule out game. With a 2-suited hand, it is
easy to find a game contract when the fit is in the second bid suit.

\vhand[West]{AK752,AQT43,A5,2}\vhand[East]{4,K852,9642,J754}
\ewauction{1c,1d,1s,1n(1),2h,3h,4h} \\ (1) As he has already limited
his hand, East is not afraid to improve the contract. After that, all
goes smoothly.

Suppose the lead against \He{4} is a low diamond. The best technique
for declarer is to win with the Ace, cash \Sp{}A and ruff a spade with
a low trump. Then he plays a club to establish communication between
the two hands.

The opponents will probably continue diamonds. West ruffs the third
round and leads another low spade, ruffing with the \He{}8. If, at
worst, spades are 5-2 and South overruffs, declarer retains the
possibility of ruffing the other spade loser with the \He{}K. The
contract will fail only against very unlucky distribution.

\gap

With a powerful hand, opener would jump rebid his suit and responder
would know there is a game or slam on if he is in the upper range. For
example,

\vhand[West]{AKJ8753,A,K72,AQ}\vhand[East]{642,J73,AJ54,865}
\ewauction{1c,1d,2s(1),3d(2),3s(3),4s,4n,5c(4),6s} \\
(1) 22+\hcp, 5-card suit. \\ (2) 4-7\hcp, showing side-suit before
showing fit in spades. \\ (3) After the new suit bid at the 3-level,
opener knows he will not be left in \Sp{3}. \\ (4) One key card.

After a heart lead and assuming trumps don't break worse than 2-1, the
contract can be made without the club finesse by playing \emph{A} and
\emph{K} of diamonds followed by a low diamond towards the
\emph{J}. This works whenever diamonds break 3-3, \Di{}\emph{Q} is
held by North or is a doubleton with South.

\gap

With a balanced hand, opener will rebid \Nt{}. For example,

\vhand[West]{K63,KJT,A862,AK3}\vhand[East]{AJ742,754,J95,T4}
\ewauction{1c,1d,1n(1),2h(2),2s,2n(3),3n(4)} \\
(1) 17-19\hcp, balanced. \\ (2) Weak transfer to \Sp{2}. \\ (3)
Balanced hand, invitational. \\ (4) With 18\hcp\ and three cards in
spades, East tries for game in no trumps.

North leads the \He{}3, South plays the Queen and West the King. After
this favourable opening, West can afford to make a safety play in
spades. He plays King and another, North following suit with low
cards. To make absolutely sure of four tricks, even when North holds
Q10xx, declarer ducks in dummy. He makes game with four spades, two
hearts and three top tricks in the minors.

\gap

Even with a weak two-suiter, Precision enables finding slams with
relatively low point counts following a negative response. For
example,

\vhand[West]{A854,AK943,AKJ8,}\vhand[East]{6,J87652,Q976,64}
\ewauction{1c,1d,1h,3s(1),4c,4h(2),6h} \\ (1) Splinter showing 4-card
or better support in hearts and a singleton or void in spades. \\ (2)
Responder could conceivably also bid \He{5} with the 6-card suit.

With a combined total of 22 points, although 13 tricks are available
if the opening lead is not ruffed, most pairs will probably stop in
\He{6} after the splinter bid using a sequence similar to the one
above.

\subsection{Positive response in a suit over \Cl{1}}

\hypertarget{ex1csuit} Using Precision, game is always reached after a
positive response to a \Cl{1} opening.  The partnership will have a
minimum of 24 points if opener is unbalanced (16 vs 8) or 25 points if
he is balanced (17 vs 8). This works well in practice, for example:

\vhand[West]{AKJ86,64,KQT9,K6}\vhand[East]{Q92,875,A543,Q94}
\ewauction{1c,1n,2s,3s,4s} \\
A dull 16\hcp\ \Cl{1} opening against an equally dull 8\hcp\ but still
\Sp{4} is an odds-on favourite to make.

\gap

Game contracts can be reached on smaller point counts if there are
distributional features. For example,

\vhand[West]{AKJT96,A,QJT9,65}\vhand[East]{Q82,965,K743,743}
\ewauction{1c(1),1d,1s,3s(2),4s} \\ (1) A strong 15\hcp\ with a good
suit should be opened with \Cl{1}.\\ (2) As he has already limited his
hand, East is not shy in raising partner's suit with 5\hcp and
inviting game.

As compared to the previous deal, this is a 15\hcp\ vs 5\hcp\ hand
that may be passed out after \Sp{1} in standard systems. However, the
game contract is virtually lay-down.

\gap

With a balanced hand, opener will rebid \Nt{} over a positive suit
response.  Even with 3-card support for partner's suit, it is
sometimes correct to first bid \Nt{} and only later raise partner's
suit. For example, with \hhand{AJT,KQT9,QJ4,KJ7}, if responder bids
\Sp{1}, it is correct to rebid \Nt{1} showing a balanced minimum
before raising spades. However, with a slightly different hand such as
\hhand{AJT7,KQT,QJ4,KJ7}, the rebid could be \Sp{3} or \Sp{4} showing
a minimum hand, probably balanced, with 4-card support.

Alternative sequences showing support have slightly different
meanings.  For example, whereas the sequence \Cl{1}--\Sp{1}--\Sp{4}
would show a minimum hand with poor controls, the sequence
\Cl{1}--\Sp{1}--\Nt{1}--any--\Sp{4} would show a balanced minimum with
good controls.

The intermediate \Nt{1} rebid can also be made when you want to find
out if responder has a distributional hand. For example, when holding
\hhand{AK87,A753,KQ4,A6}, after partner's positive response of \Sp{1},
rebid \Nt{1} and if partner rebids \Cl{2} (four-card suit), you may
have very good play for \Sp{7} if partner is holding something like
\hhand{QJ543,82,A8,K954}.  However, you need to know about the four
clubs first.

\gap

With a distributional hand where you have strong support for partner's
suit and the only question for slam is whether his suit has good
quality, \emph{asking bids} ($\gamma$ and $\epsilon$) can be used to
good effect. For example,

\vhand[West]{QJ632,5,AKQ8,KJ9}\vhand[East]{AKT54,987,T4,A53}
\ewauction{1c,1s,2s(1),3h(2),4c(*),4n(3),5h(*),5s(4),6s} \\
(1) $\gamma$ trump-asking bid (possible slam if trumps are strong). \\
(2) 2 honours, 5-card suit. \\ (*) $\epsilon$ control-asking bid in
clubs and hearts. \\ (3) Ace or void. \\ (4) No control.

With a sure loser in hearts, opener stops in the small slam.

\subsection{Positive no-trump response to \Cl{1}}

\hypertarget{ex1cnt} With both majors, it is sometimes correct to use
\emph{Stayman} even when holding a 5-card suit.  For example, holding
\hhand{AKQ64,KQ87,A5,95}, it is better to bid \Cl{2} over a \Nt{1}
response rather than bidding \Sp{2}. If responder holds something like
\hhand{JT2,AJ94,543,Q43}, he will certainly raise spades after \Sp{2}
and the 4-4 heart fit will not be discovered. In this case although
there are 10 tricks in spades and 11 in hearts (given normal breaks),
sometimes the difference may be 10 tricks in the 4-4 fit versus 9 in
the 5-3 fit.

Similarly, with \hhand{3,AKQ7,AQ,KQJT98}, bid \emph{Stayman}. If
partner bids \Di{2} (four hearts), you will bid \He{2} and later ask
for aces. If partner has two aces, you can confidently bid the grand
slam or the small slam if he shows only one ace. If partner holds
something like \hhand{AQ6,JT86,J76,543}, \He{6} from the strong side
is best, while \Cl{6} will depend on the diamond finesse.

\subsection{\Sp{3} response to \Cl{1}}

\hypertarget{ex1c3s} Opener can place the contract fairly easily given
responder's solid suit and use asking bids to decide if a slam is
on. For example,

\vhand[West]{4,AT987,A4,AKQ87}\vhand[East]{AKQJ987,3,K7,T96}
\ewauction{1c,3s(1),4c(2),4h(3),7s(4)} \\
(1) Solid suit. Opener can tell that it is spades by looking
at his own hand. \\ (2) $\beta$-ask for outside controls. \\
(3) One outside control (\Di{} or \He{} king). \\ (4) 13 tricks are on
top.

\subsection{Unusual positive response to \Cl{1}}

\hypertarget{ex1c3c} If responder bids an \emph{unusual positive},
slam is most likely on the cards and with the right cards, grand slams
can be reached on very low point counts.

\vhand[West]{AKQ876,976,AK43,}\vhand[East]{J543,A,T987,AK43}
\ewauction{1c,4d(1),4h(2),4n(3),7s(4)} \\ (1) 4-1-4-4,
$4^+$-controls, $12^+$\hcp \\ (2) $\beta$ asking for controls \\
(3) 5 controls (2 steps) \\ (4) Partner must have two aces and
\Cl{}\emph{K}, 13 tricks are visible.

Barring horrendous breaks and a ruff on the opening lead, this
27-point grand slam is lay-down.

\subsection{Intervention after a \Cl{1} opening}

\hypertarget{ex1cintervene} Some examples of bidding after opponents
double or overcall after a \Cl{1} opening.

\begin{longtable}{rp{11cm}}
  \multicolumn{2}{l}{\emph{\underline{After a takeout / unusual double: \Cl{1}--(Double)}}} \\
  1 & \hhand{J84,AJ82,T5,KT42} \\
    & If the double is an ordinary takeout double either
      \emph{Redouble} or bid \Nt{1} showing a balanced 8-13\hcp\ if
      vulnerable.

      If the double shows majors, \emph{Redouble}. If partner doubles
      \Sp{1}, you will be delighted to defend. \\
  2 & \hhand{A87,8,KJ8654,Q63} \\
    & Bid \Di{2}. Slam is a real possibility despite the double. \\
  3 & \hhand{QT3,,JT9753,QT93} \\
    & Bid \Di{1} (5-8\hcp). If partner bids \He{1}, you will bid
      \Di{2} showing the long suit. \\
  4 & \hhand{AQ,A863,QJT,JT85} \\
    & Bid \Nt{2} showing a balanced $14^+$\hcp\ hand and good stoppers
      in the majors. If the double is real (not a mistake showing
      clubs), the information of length in majors on the right is
      likely to be useful in the play. \\
  5 & \hhand{6,KJT5,A732,JT87} \\
    & Bid \Cl{3} showing the 4-4-4-1 hand with a black singleton. \\
\end{longtable}

\begin{longtable}{rp{11cm}}
  \multicolumn{2}{l}{\emph{\underline{After a direct 1-level overcall: \Cl{1}--(\Sp{1})}}} \\
  6 & \hhand{Q,AQJ32,KJ63,J97} \\
    & Bid \He{2} which is natural and game forcing. \\
  7 & \hhand{4,J8654,T976,KJ6} \\
    & \emph{Double} to show 5-8\hcp.\\
  8 & \hhand{T953,4,A764,AQ92} \\
    & Bid \Di{3}, unusual positive showing 4-4-4-1 with a red
      singleton. \\
  9 & \hhand{J,Q652,AQT964,T2} \\
    & Bid \Di{2}, natural and forcing. \\
  10 & \hhand{953,AT43,AJ72,95} \\
    & Bid \Sp{2}. There is enough to force game but no suit to bid and
      no stopper to bid \Nt{}. \\
\end{longtable}

\begin{longtable}{rp{11cm}}
  \multicolumn{2}{l}{\emph{\underline{After an unusual no-trump overcall showing minors: \Cl{1}--(\Nt{1})}}} \\
  11 & \hhand{T9,AQ64,K862,AQ5} \\
     & \emph{Double} for penalties. If partner bids hearts, explore
       for slam. If not, you can also bid no-trump since the combined
       hands are in the slam zone. \\
  12 & \hhand{AT942,Q4,743,T98} \\
     & Bid \Sp{2} (non-forcing). \\
  13 & \hhand{K9743,AQ98,92,87} \\
     & Bid \Di{2} showing spades and forcing to game. \\
  14 & \hhand{AQ2,AT82,KJ3,874} \\
     & \emph{Double} showing values with a balanced hand. If partner
       bids \Nt{2}, you can show the 4-card hearts on the way to
       \Nt{3}. \\
  15 & \hhand{A4,J98,KT64,KT87} \\
     & \emph{Double} (penalty oriented) showing values with a balanced
       hand. There will be a massacre if the final contract is in
       either minor. \\
\end{longtable}

\begin{longtable}{rp{11cm}}
  \multicolumn{2}{l}{\emph{\underline{After a 2-level overcall: \Cl{1}--(\He{2})}}} \\
  16 & \hhand{AQT,85,K74,KQT96} \\
     & Bid \Cl{3} which is natural and forcing. The main reason for
       not cue-bidding is that this hand will make an excellent dummy
       should partner bid \Di{3} or \Sp{3} which you will happily
       raise showing slam interest by bypassing \Nt{3}. \\
  17 & \hhand{9872,52,AKJ4,T64} \\
     & \emph{Double}. This is more flexible than cue-bidding
       \He{3}. Partner can bid \Nt{2} with a stopper and then you
       could bid \Cl{3} (\emph{Stayman}). \\
  18 & \hhand{QJ432,A6,JT63,K4} \\
     & Bid \Sp{2}. \\
  19 & \hhand{4,KJT94,QJ7,A732} \\
     & \emph{Pass}. You are certain partner will bid again and you
       hope it is a double. The penalty will be a rich one if so. \\
  20 & \hhand{AK64,8765,AKQ7,7} \\
     & Bid \He{3}---game-forcing with no heart stopper and no long
       suit. You can explore slam after getting more information from
       partner. \\
\end{longtable}

\subsection{The \Di{1} opening}

\hypertarget{ex1d} Some examples of bidding after a \Di{1} opening.

\begin{longtable}{rp{11cm}}
  \multicolumn{2}{l}{\emph{\underline{Opening bid}}} \\
  1 & \hhand{63,K4,AKJ9,KT984} \\
    & Open \Di{1} and if partner bids \He{1}, rebid
      \Cl{2}. Alternatively, open \Nt{1}. \\
  2 & \hhand{AJ76,2,AQJ62,T72} \\
    & Open \Di{1} rebid \Sp {1} if partner bids \He{1}. \\
  3 & \hhand{Q76,J3,AQ9,AT982} \\
    & Open \Di{1} and rebid \Nt{1} over \He{1}/\Sp{1}. You cannot bid
      \Cl{2} which would show an unbalanced hand. \\
  4 & \hhand{QT9,Q97,Q4,AJ962} \\
    & \emph{Pass} with this weak 11-point hand. \\
  5 & \hhand{65,T,AKQT8,KQT97} \\
    & Open \Di{1} and rebid \Cl{3} over \He{1}/\Sp{1} showing 5-5 in
      the minors. \\
\end{longtable}

\begin{longtable}{rp{11cm}}
  \multicolumn{2}{l}{\emph{\underline{Responses to a \Di{1} opening}}} \\
  6 & \hhand{97,AK5,QJ873,KQ5} \\
    & Bid \Di{2} showing at least a limit raise. Raise to game over
      \Nt{2} or find a forcing bid if opener rebids a minimum. You
      want partner to be declarer in \Nt{} with the weak doubleton
      spade. \\
  7 & \hhand{Q95,5,AKQ532,K64} \\
    & Bid \He{3}---a splinter showing the singleton heart and fine
      diamond support. \\
  8 & \hhand{7,AK942,KQJ54,A8} \\
    & Bid \He{1} and use \emph{RKCB} if opener supports
      hearts. Otherwise, jump to \Di{3} if opener responds with \Nt{1}
      showing the two-suiter and indicating slam interest. \\
  9 & \hhand{6,AK74,42,AKT943} \\
    & Bid \He{1} and clubs next in the search for the best game
      contract or slam if opener raises clubs. \\
  10 & \hhand{76,9,AJT642,8532} \\
    & Bid \Di{3} (or \Di{4} if non-vulnerable) interfering with
      opponent's possible game. \\
\end{longtable}

\begin{longtable}{rp{11cm}}
  \multicolumn{2}{l}{\emph{\underline{Rebids after partner's one-over-one response: \Di{1}--\Sp{1}--???}}} \\
  11 & \hhand{82,75,AQ52,AKT65} \\
     & Rebid \Cl{2}. \\
  12 & \hhand{6,KT,AJT87,KQJ92} \\
     & Rebid \Cl{3} showing 5-5 in the minors. \\
  13 & \hhand{KT92,9,AKT64,K65} \\
     & Rebid \Sp{3} showing strong support and a singleton / void. \\
  14 & \hhand{AT4,Q76,J964,AK8} \\
     & Rebid \Nt{1}. Raising spades is inadvisable with this flat
       hand. \\
  15 & \hhand{KT4,4,QJ974,AKQ4} \\
     & Rebid \Cl{2} as the least worst evil---if partner bids again,
       you can show the spade support. \\
\end{longtable}

\subsection{Major suit openings}

\hypertarget{ex1h} Some examples of bidding after a \He{1} or \Sp{1}
opening.

\begin{longtable}{rp{11cm}}
  \multicolumn{2}{l}{\emph{\underline{Opening bid}}} \\
  1 & \hhand{AT9765,Q8,K6,K52} \\
    & The quintessential \Sp{1} bid. \\
  2 & \hhand{KJT6,AKJT92,K8,9} \\
    & Open \Cl{1}---there are 15\hcp, a very good suit and a
      singleton. With unfavourable vulnerability, it may be better to
      bid \He{1} since opponents may intervene at a high level after
      \Cl{1}. \\
  3 & \hhand{QJ9654,KT5,K8,Q7} \\
    & Open \Sp{1}---this is not a great hand and many may choose to
      pass it or open \Sp{2}. \\
  4 & \hhand{QT752,A74,AJ7,A6} \\
    & Open \Sp{1}. A case can be made for opening this hand with
      \Nt{1} and with \hearts{KJ4} and \clubs{KJ} (same \hcp), it
      would be preferable to open \Nt{1}. \\
  5 & \hhand{32,KQ8743,QJ6,AK} \\
    & Open \He{1}. Although there are 15 \hcp, the suit is not good
      enough to play against a singleton and the hand has no
      singletons of its own. \\
\end{longtable}

\begin{longtable}{rp{11cm}}
  \multicolumn{2}{l}{\emph{\underline{Responses to a \Sp{1} opening}}} \\
  6 & \hhand{AJT9,KJ8,T97,KJ6} \\
    & Bid \Sp{4}---it would be a very unusual hand with partner for
      there to be a slam. Opponents do not know if your hand type is a
      weak distributional hand or this one. \\
  7 & \hhand{AQ982,AT8,4,KT76} \\
    & Bid \Di{4} (splinter) with real slam potential. \\
  8 & \hhand{98732,A5,Q,T9743} \\
    & Bid \Sp{4}---the textbook example of a game raise. Contrast to
      hand \#6. \\
  9 & \hhand{K832,A65,AKJ9,74} \\
    & Bid \Nt{2}---game-forcing raise showing at least 4-card
      support. If partner shows shortness in clubs or hearts, slam is
      a distinct possibility. \\
  10 & \hhand{AJ874,4,Q53,AT95} \\
    & Bid \He{4} (splinter). Another hand with good slam potential if
      partner's hand matches. \\
\end{longtable}

\begin{longtable}{rp{11cm}}
  \multicolumn{2}{l}{\emph{\underline{Responses to a \He{1} opening}}} \\
  11 & \hhand{QT632,K72,A532,T} \\
     & Bid \Sp{1}. If partner raises, you can bid game. If partner
       bids \Nt{1}, \Cl{2} or \Di{2}, you will show limit raise values
       with \He{3}. Partner will know you have only 3 hearts since
       ther was no direct raise. \\
  12 & \hhand{AJ763,972,AK753,} \\
     & Bid \Sp{1} and if partner raises, you will explore slam. If
       partner bids \Cl{2} (likely), you will bid \Di{2} (fourth-suit
       forcing). If partner rebids \He{2}, you could bid \He{5}
       (asking about trump quality) or \Cl{4} (splinter). This is a
       difficult hand to assess since opposite the first opening hand
       below, a grand slam is on but opposite the second, no game is
       possible.

       \vhand[Opener 1]{8,AKQ863,QJ7,T76}
       \vhand[Opener 2]{86,Q8543,J6,AKQ6} \\
  13 & \hhand{QT,AT98,432,Q965} \\
     & Bid \Cl{3}---a constructive \emph{Bergen} raise. \\
  14 & \hhand{Q76,J876,,AJ9853} \\
     & Bid \He{4}. It is certain that the opponents have some high
       card points so this makes them start at a high-level if they
       are going to bid. \\
  15 & \hhand{A94,Q643,JT3,A62} \\
     & Bid \Di{3}---a \emph{Bergen} limit raise. \\
\end{longtable}

\subsection{The \Cl{2} opening}

\hypertarget{ex2c}
Some examples of bidding after a \Cl{2} opening.

\begin{longtable}{rp{11cm}}
  \multicolumn{2}{l}{\emph{\underline{Opening bid}}} \\
  1 & \hhand{KJ62,3,92,AQJ982} \\
    & A good example of a hand that should be opened with a bid of
      \Cl{2}. \\
  2 & \hhand{QT6,KQ6,63,AQ843} \\
    & Bid \Di{1} not \Cl{2}. \\
  3 & \hhand{K3,,AJ82,AQJT974} \\
    & Bid \Cl{1}. This hand is too good for a \Cl{2} opening. \\
  4 & \hhand{Q86,A6,T8,AKQ874} \\
    & Bid \Cl{1} and rebid \Cl{2}. Let partner be declarer in \Nt{} if
      that is the right spot. \\
  5 & \hhand{62,87,QT,AKQJ982} \\
    & Bid \Nt{3} (\emph{gambling}), showing a solid suit with no ace or
      king outside. \\
\end{longtable}

\begin{longtable}{rp{11cm}}
  \multicolumn{2}{l}{\emph{\underline{Responses to a \Cl{2} opening}}} \\
  6 & \hhand{AKT6,J865,T9,976} \\
    & Bid \Di{2}. This is a perfect hand to enquire about majors. If
      partner bids a major or \Cl{3}, pass (you need at least another
      queen to raise partner's major). If partner bids \Nt{2}, correct
      to \Cl{3}. \\
  7 & \hhand{KT9832,5,975,KJ7} \\
    & Bid \Sp{2}. If partner raises spades, raise to game. Pass if he
      denies spades by rebidding \Cl{3} or bid \Cl{3} if he rebids
      \Nt{2}. \\
  8 & \hhand{K85,KJ95,AT63,93} \\
    & Bid \Nt{2} (invitational). If partner accepts game by bidding
      \He{3}, bid \He{4}. If partner accepts with \Sp{3}, raise to
      \Nt{3}. \\
  9 & \hhand{AJT763,KQ9,T7,Q2} \\
    & Bid \Sp{3}. This is forcing to game and shows at least 6
      spades. Pass if partner signs-off in \Nt{3}. \\
  10 & \hhand{K73,942,A932,973} \\
    & Bid \Cl{3} forcing \emph{LHO} to come in at the three-level. The
      \emph{Law of Total Tricks} will protect you \ldots \\
\end{longtable}

\begin{longtable}{rp{11cm}}
  \multicolumn{2}{l}{\emph{\underline{Rebids after partner's invitational response in a suit: \Cl{2}--\He{2}--???}}} \\
  11 & \hhand{KQJ5,53,4,AQT965} \\
     & Bid \Sp{2}. This hand will play better in one of your suits so
       let partner know you have four spades. \\
  12 & \hhand{432,AQ,K7,AT7643} \\
     & \emph{Pass}. There is no reason to think there is a better
       spot. \\
  13 & \hhand{3,AQT8,T8,AKJT84} \\
     & Bid \Sp{3} (splinter) or \He{4}. Ten tricks should be on with
       this dummy. \\
  14 & \hhand{96,KJ63,JT,AKQ74} \\
     & Bid \He{3}. Although the hand is a maximum, the shape is not
       inspiring. It may have been preferable to open \Di{1} with this
       hand. \\
  15 & \hhand{KJ2,52,3,AKJT962} \\
     & Bid \Cl{3}. Although you have a doubleton heart, the clubs are
       good enough to play opposite a void. It must be better to have
       it as trumps. \\
\end{longtable}

\subsection{The \Di{2} opening}

\hypertarget{ex2d}
Some examples of bidding after a \Di{2} opening.

\begin{longtable}{rp{11cm}}
  \multicolumn{2}{l}{\emph{\underline{Responses to \Di{2}}}} \\
  1 & \hhand{QJ543,AT5,KT5,AT} \\
    & Bid \Sp{4}. Why mess about? \\
  2 & \hhand{54,A9,AT87643,63} \\
    & \emph{Pass}. You would also pass if one of the low diamonds was
      a low heart since bidding \He{2} may land you in a 3-3 fit. \\
  3 & \hhand{2,JT73,KQ64,K852} \\
    & Bid \He{2}. If partner is 4-3-1-5, he will bid \Sp{2} which can
      be corrected to \Cl{3}. Do not ask for shape since that may push
      bidding to the 4-level. \\
  4 & \hhand{AJ,A93,AJT97,T87} \\
    & Bid \Nt{3}. Diamonds are well under control and there are no
      better prospects for game. \\
  5 & \hhand{AQT65,KJ,987,AQ5} \\
    & Bid \Nt{2}---there is a grand slam possible here. Whatever
      partner bids, you will bid diamonds next to ask about
      controls. \\
\end{longtable}

\begin{longtable}{rp{11cm}}
  \multicolumn{2}{l}{\emph{\underline{Responding over RHO's 2-level suit overcall: \Di{2}--(\He{2})--???}}} \\
  6 & \hhand{QT6,JT9,KQJ4,A64} \\
    & \emph{Double}. This can get ugly since opponent is bidding at
      the 2-level with at most seven trumps and without the balance of
      \hcp. \\
  7 & \hhand{9,T64,AQ86543,K8} \\
    & \emph{Pass}. You could double but that would probably drive the
      opponents to spades which is a better spot. \\
  8 & \hhand{84,3,AKT5,987653} \\
    & Bid \Cl{4}. This is a preemptive bid to make \emph{LHO} decide
      whether to support at the 4-level. \\
\end{longtable}

\begin{longtable}{rp{11cm}}
  \multicolumn{2}{l}{\emph{\underline{Responding over RHO's 3-level suit overcall / cue-bid: \Di{2}--(\Di{3})--???}}} \\
  9 & \hhand{952,A95,AJ73,743} \\
    & \emph{Double}. \\
  10 & \hhand{K93,AQT,AT2,JT94} \\
    & A \emph{Double} is probably best with favourable vulnerability.

      However, with unfavourable vulnerability, it is a choice between
      \Nt{3} (if you feel lucky) and \Cl{5} (more realistic). \\
  11 & \hhand{AT9842,843,T5,74} \\
    & Bid \Sp{3} (non-forcing). \\
\end{longtable}

\begin{longtable}{rp{11cm}}
  \multicolumn{2}{l}{\emph{\underline{Responding after RHO's double: \Di{2}--(Double)--???}}} \\
  12 & \hhand{K962,K4,AQJ6,862} \\
     & \emph{Redouble}. There could be overtricks here even if partner
       plays in a 4-1 diamond fit. \\
  13 & \hhand{KJ965,J864,J4,Q7} \\
     & Bid \Sp{2}---if opponents compete, you can try hearts next. \\
  14 & \hhand{T642,QT63,JT,K75} \\
     & \emph{Pass}. Let partner describe his shape with a redouble or
       bid. \\
  15 & \hhand{A5,KQT,KT9863,Q2} \\
     & \emph{Redouble}. You have a lot of diamonds and good spot
       cards. Even if opponent's have a 4-4 spade fit, they may not
       find it and even if they do it is likely you have a penalty
       double against them in spades. \\
\end{longtable}

\subsection{``Gambling'' and ``Namyats'' openings}

\begin{longtable}{rp{11cm}}
  \multicolumn{2}{l}{\emph{\underline{Responses to a ``gambling'' \Nt{3}}}} \\
  1 & \hhand{32,AK85,AKJ43,JT} \\
    & Bid \Di{4} asking partner to show singletons or voids. If he is
      short in spades, you can commit to a club slam. \\
  2 & \hhand{AQJ,9743,T6,AJ86} \\
    & \emph{Pass}. Opponents may be able to run some hearts but the
      odds are in your favour. Even if someone has five hearts, he may
      not be on lead or the suit may be blocked. \\
  3 & \hhand{A92,AK97652,,A85} \\
    & Bid \Nt{5}. This asks partner to bid \Di{7} with
      \emph{AKQJ}. You certainly want to be in \Di{6} although there
      are no guarantees. \\
  4 & \hhand{A,AKQT84,KQJ9,54} \\
    & Bid \Cl{6} which should be cold. \\
  5 & \hhand{QJ84,65,T87,JT97} \\
    & Bid \Cl{5}. You don't care what partner's suit is (although it
      looks to be diamonds). What you do know is that opponents can
      make a lot of tricks in hearts (or even spades) and this robs
      them of room to find their best spot. \\
\end{longtable}

\begin{longtable}{rp{11cm}}
  \multicolumn{2}{l}{\emph{\underline{Responses to a ``Namyats'' \Di{4} opening}}} \\
  6 & \hhand{T,A765,KQ95,AK32} \\
    & Bid \Nt{4} (\emph{RKCB}). \\
  7 & \hhand{874,KJT9,KQ65,KJ} \\
    & Bid \Sp{4}. Partner cannot have many aces in addition to a solid
      suit (he probably would have opened \Cl{1} if so) so slam is out
      of question. \\
  8 & \hhand{972,QJ,AK652,AJT} \\
    & Bid \He{4}, a relay to partner's suit. You plan to cue-bid
      \Cl{5} inviting slam and if partner has a cue-bid in hearts, you
      can bid \Sp{6}. \\
  9 & \hhand{J752,A92,AKQ53,4} \\
    & Bid \Nt{4} (\emph{RKCB}). If partner shows 3 key cards, you will
      bid \Sp{7}. This is likely to be lay-down after the opening
      lead. \\
  10 & \hhand{872,AQJ73,,AT742} \\
    & With a solid suit and the heart king, \Sp{7} is odds-on. Since
      there is no way to confirm both of these (an asking bid will
      only find the heart king), it is probably best to simply bid
      \Sp{6}. The success of the slam may depend on the heart finesse,
      finally. \\
\end{longtable}

\end{document}
